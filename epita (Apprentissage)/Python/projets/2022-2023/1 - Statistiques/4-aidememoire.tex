%%%%%%%%%%%%%%%%%%%%%%%%%%%%
%% AIDE MEMOIRE AU CAS OU %%
%%%%%%%%%%%%%%%%%%%%%%%%%%%%
%\newpage

%{\Large \textbf{Aide Mémoire}}

%\vspace{30px}

%\noindent Le travail doit être rendu au format \textbf{\textit{.zip}}, c'est-à-dire une archive \textbf{zip} compressée avec un outil adapté (les logiciels \textit{7zip} ou \textit{Keka} sont gratuits et adaptés).
\noindent Le travail doit être rendu au format \textbf{\textit{.tar.bz2}}, c'est-à-dire une archive \textbf{bz2} compressée avec un outil adapté (voir \TTBF{man 1 tar} et \TTBF{man 1 bz2}).

%\noindent Tout autre format d'archive (rar, 7zip, gz, gzip, bzip, ...) ne sera pas pris en compte, et votre travail ne sera pas corrigé (entraînant la note de 0).
\noindent Tout autre format d'archive (zip, rar, 7zip, gz, gzip, ...) ne sera pas pris en compte, et votre travail ne sera pas corrigé (entraînant la note de 0).

\bigskip

\noindent Pour générer une archive \textit{tar} en y mettant les dossiers \textit{folder1} et \textit{folder2}, vous devez taper :

\TTBF{tar cvf MyTarball.tar folder1 folder2}


\bigskip


\noindent Pour générer une archive \textit{tar} et la compresser avec GZip, vous devez taper :

\TTBF{tar cvzf MyTarball.tar.gz folder1 folder2}


\bigskip


\noindent Pour générer une archive \textit{tar} et la compresser avec BZip2, vous devez taper :

\TTBF{tar cvjf MyTarball.tar.bz2 folder1 folder2}


\bigskip


\noindent Pour lister le contenu d'une archive \textit{tar}, vous devez taper :

\TTBF{tar tf MyTarball.tar.bz2}


\bigskip


\noindent Pour extraire le contenu d'une archive \textit{tar}, vous devez taper :

\TTBF{tar xvf MyTarball.tar.bz2}


\vspace*{1cm}

%\noindent Dans ce sujet précis, vous ferez du code en C et des appels à des scripts shell qui afficheront les résultats dans le terminal (donc des flux de sortie qui pourront être redirigés vers un fichier texte).

%\noindent Dans ce sujet précis, vous ferez du code en script shell, qui affichera les résultats dans le terminal (donc des flux de sortie qui pourront être redirigés vers un fichier texte).

%\noindent Dans ce sujet précis, vous ferez du code en PHP, qui affichera les résultats dans une page HTML. Les valeurs seront affichées dans une \textit{textarea} dont le texte est généré par des outils multiplateformes supportant les retours à la ligne UNIX (\textbf{\textbackslash n}). Il ne faut donc pas inclure de balise \TTBF{"<br />"} pour retourner à la ligne, mais un \TTBF{"\textbackslash n"}.

\noindent Dans ce sujet précis, vous ferez du code en Python, qui affichera les résultats dans le terminal (donc des flux de sortie qui pourront être redirigés vers un fichier texte).

%\vspace*{1cm}

%\noindent Pour réaliser le travail demandé, nous vous fournirons pour chaque exercice au moins 2 fichiers : \TTBF{exoN\_res.php} (le fichier qui sera appelé pour voir le résultat de votre travail), et \TTBF{exoN\_fun.php} (le fichier contenant la fonction que vous devez coder dans chaque exercice).
%Optionnellement, un fichier \TTBF{exoN\_data.php} peut être fourni pour indiquer le format de données en entrée.
%Le \TTBF{N} correspond au numéro de l'exercice.

%\medskip

%\noindent Dans tous les cas, vous ne devez rendre que le fichier \TTBF{exoN\_fun.php} avec au moins la fonction demandée remplie (qui peut faire appel à d'autres fonctions que vous définirez dans le \textbf{même} fichier). Les autres fichiers seront générés par nos soins pour tester vos fonctions.

%\vspace*{1cm}

%\noindent Vous ne devez \textbf{PAS} utiliser la fonction \TTBF{echo} pour écrire !
%Il faut retourner une chaîne de caractères correctement formattée.
