\documentclass[12pt,a4paper]{article}

\usepackage[utf8]{inputenc}
\usepackage[french]{babel}
\usepackage[T1]{fontenc}

\usepackage{amsmath}
\usepackage{amsfonts}
\usepackage{amssymb}

\usepackage[margin=2.5cm,headheight=2cm,headsep=30pt]{geometry}

\usepackage{color}

\definecolor{mygreen}{rgb}{0,0.6,0}		% RGB model
\definecolor{mygray}{rgb}{0.5,0.5,0.5}
\definecolor{mymauve}{rgb}{0.58,0,0.82}


\usepackage{array}
\newcolumntype{P}[1]{>{\raggedright\arraybackslash }m{#1}}

\newcolumntype{L}[1]{>{\raggedright\arraybackslash }b{#1}}
\newcolumntype{C}[1]{>{\centering\arraybackslash }b{#1}}
\newcolumntype{R}[1]{>{\raggedleft\arraybackslash }b{#1}}

\newcolumntype{G}[1]{>{\raggedright\let\newline\\\arraybackslash\hspace{0pt}}m{#1}} % another L
\newcolumntype{M}[1]{>{\centering\let\newline\\\arraybackslash\hspace{0pt}}m{#1}}   % another C
\newcolumntype{D}[1]{>{\raggedleft\let\newline\\\arraybackslash\hspace{0pt}}m{#1}}  % another R


\newcommand{\VersionExo}{Version 1.2}

\begin{document}

\title{Barême Python App1\\
       \large 2023-2024 (\VersionExo)}

\author{Fabrice BOISSIER}

\maketitle

\thispagestyle{empty} % Remove page number


\begin{equation*}
\text{Note finale} = \text{moyenne}(\text{Projet 1}, \text{Projet 2})
\end{equation*}


\medskip

\noindent En cas d'échec majeur par toute la promo d'un projet, le projet le mieux réalisé verra son coefficient augmenté, et celui le moins bien réalisé verra son coefficient réduit.

\noindent Par exemple (mais pas nécesairement) :

\begin{equation*}
\text{Note finale} = \frac{3 \times \text{Projet 1} \; + \; 1 \times \text{Projet 2}}{4}
\end{equation*}

\medskip

\noindent Si certains tests d'exercices sont ratés par trop de personnes dans la promo, les tests en question verront leurs poids réduits.
Voire, dans certains cas, des points seront proportionnellement attribués pour permettre un rééquilibrage entre les meilleurs et les moins avancés (ceci est laissé selon l'appréciation générale de l'enseignant).

\bigskip


\noindent Barême Projet 1 (Statistiques) :

\begin{itemize}
\item Format de rendu : 2 points
\item Exercice 1 : 15 points sur les fonctions + 4 points sur les exceptions
\item Exercice 2 : 15 points sur les fonctions + 4 points sur les exceptions
\item $ \text{Total Projet 1} = \frac{\text{Format} \, + \, \text{Exercice 1} \, + \, \text{Exercice 2}}{2} $
\end{itemize}


\bigskip


\noindent Barême Projet 2 (Statistiques Graphiques) :

\begin{itemize}
\item Format de rendu : 2 points
\item Exercice 1 : 1 point format d'affichage + 2 points sur les données + 0,5 point \textit{main()}
\item Exercice 2 : 1 point sur les données + 1 point graphe barres + 1 point graphe circulaire + 1,5 point \textit{main()}
\item $ \text{Total Projet 2} = \text{Format} \; + \; \text{Exercice 1} \; + \; \text{Exercice 2} $
\end{itemize}

\bigskip

\noindent En cas de rendus multiples sur Teams, celui-ci ajoute automatiquement un numéro à la fin du nom de fichier : cela ne sera pas pris en compte lors de la notation.
Néanmoins, le nom de l'archive doit correspondre à ce qui est demandé concernant le login et le nom du projet.

\bigskip

\noindent Attention, dans tous les cas, si un rattrapage de cette matière doit être produit, il s'agira d'un des deux projets tel quel ou modifié.
Vous devez donc absolument les réaliser.

\bigskip

\noindent \textit{Exceptionnellement, celles et ceux qui auraient été sous Windows pendant les 3 séances de cours peuvent rendre un \textbf{.zip} au lieu d'un \textbf{.tar.bz2}}

\end{document}
