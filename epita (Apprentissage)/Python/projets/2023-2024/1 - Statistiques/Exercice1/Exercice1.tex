%% Exercice 1

%\ExoSpecs{\TTBF{CalculTVA.sh}}{\TTBF{\RenduDir/src/exo1/}}{750}{640}{\TTBF{write}}
\ExoSpecsCustom{\TTBF{MyOwnStats.py}}{\TTBF{\RenduDir/src/module/}}{750}{640}{Fonctions externes autorisées}{\TTBF{math.sqrt}, \TTBF{random.random}, \TTBF{random.randint}}
%{\TTBF{math.floor}, \TTBF{math.ceil}, \TTBF{random.random}}

\vspace*{0.7cm}

\noindent \ExoObjectif{Le but de l'exercice est d'écrire un module python contenant plusieurs fonctions permettant de calculer des statistiques sur une distribution.}

\noindent Les fonctions statistiques seront implémentées dans leur version évaluant une population complète et non pas un échantillon, elles renverront des flottants pour la plupart (sauf celles traitant des rangs ou des valeurs directement).

\bigskip

\noindent Vous devez implémenter les fonctions décrites plus bas, si nécessaire en déclarant vos propres fonctions.
Ce code sera testé en étant lui-même appelé par un autre script, vous devez donc absolument respecter les prototypes décrits ici.

\bigskip

\noindent Vous ne devez pas implémenter de classe dans cet exercice, mais bel et bien un module.
Vous ne devez pas non plus importer la classe de l'exercice 2.

\bigskip

\noindent Pour construire ce module, vous avez le droit d'importer exclusivement 3 fonctions de modules externes :

\begin{itemize}
%\item \TTBF{math.floor} du module \texttt{math}
%\item \TTBF{math.ceil} du module \texttt{math}
\item \TTBF{math.sqrt} du module \texttt{math}
\item \TTBF{random.random} du module \texttt{random}
\item \TTBF{random.randint} du module \texttt{random}
\end{itemize}

\bigskip

\noindent Cependant, vous pouvez parfaitement utiliser les fonctions, types, et classes internes à Python (tels que \TTBF{print()}, \TTBF{int()}, \TTBF{str()}, \TTBF{range()}, \TTBF{list()}, et autres).

\bigskip

\begin{RedBoxTitle}{Titre}
Notez bien qu'aucune fonction de ce module ne doit modifier la distribution donnée en paramètre !
Pas même les fonctions de tri, d'ajout, de suppression, ou de fusion, ni les autres fonctions ayant besoin que la distribution soit triée.
\end{RedBoxTitle}


%\bigskip
\newpage

\noindent Vous devez implémenter les fonctions suivantes :

\bigskip

\lstset{language=python}
\begin{lstlisting}[frame=single,title={Liste des fonctions pour le module de statistiques}]
GenerateDistribution(DistrNbValues)
SortDistribution(Distribution, [asc=True])
PrintDistribution(Distribution)

AddValueDistribution(Distribution, Value)
AddValuePositionDistribution(Distribution, Value, Position)
DeleteValueDistribution(Distribution, Value)
DeletePositionDistribution(Distribution, Position)
MergeDistribution(Distribution, NewDistribution)

Min(Distribution)
Max(Distribution)
Cardinality(Distribution)
Range(Distribution)

Sum(Distribution)
Mean(Distribution)
Median(Distribution)
GetQuartile(Distribution, Quartile)

InterQuartileRange(Distribution)
Variance(Distribution)
StandardDeviation(Distribution)
RelativeStandardDeviation(Distribution)
GiniCoefficient(Distribution) \end{lstlisting}


%\bigskip


\subsubsection*{\TTBF{GenerateDistribution(DistrNbValues)}}

%\noindent Cette fonction crée et renvoie une liste d'entiers aléatoires représentants une distribution, ainsi que les valeurs minimale et maximale.
\noindent Cette fonction crée et renvoie un tuple d'entiers aléatoires représentants une distribution, ainsi que les valeurs minimale et maximale.
%Les valeurs doivent être renvoyées exactement dans cet ordre : la valeur minimum, la valeur maximum, et la liste contenant la distribution.
Les valeurs doivent être renvoyées exactement dans cet ordre : la valeur minimum, la valeur maximum, et le tuple contenant la distribution.
La distribution créée peut contenir plusieurs fois la même valeur, et doit être composée de nombres entiers entre 0 et 1000.
Le paramètre \textit{DistrNbValues} est un entier indiquant le nombre d'entiers que la distribution doit contenir.
%Si \textit{DistrNbValues} est nul, alors vous renverrez une liste vide, avec le minimum et le maximum à $ 0 $.
Si \textit{DistrNbValues} est nul, alors vous renverrez un tuple vide, avec le minimum et le maximum à $ 0 $.
Si \textit{DistrNbValues} est négatif, alors vous renverrez une distribution contenant $ 10 $ entiers.


\subsubsection*{\TTBF{SortDistribution(Distribution, [asc=True])}}

\noindent Cette fonction renvoie la version triée de la distribution donnée en paramètre.
%Le paramètre \textit{Distribution} est la distribution à trier (il doit s'agir d'une liste d'entiers).
Le paramètre \textit{Distribution} est la distribution à trier (il doit s'agir d'un tuple d'entiers).
%Le paramètre \textit{asc} est un booléen indiquant si la liste doit être triée par ordre ascendant ou descendant.
Le paramètre \textit{asc} est un booléen indiquant si le tuple doit être trié par ordre ascendant ou descendant.
Par défaut, le tri doit être ascendant (à la position $ 0 $ doit se trouver l'entier le plus petit de la distribution, et à la dernière position on doit trouver l'entier le plus grand).

\medskip

\noindent Attention : cette fonction ne doit pas modifier le paramètre donné, mais bien renvoyer une copie triée !


\subsubsection*{\TTBF{PrintDistribution(Distribution)}}

\noindent Cette fonction écrit le contenu de la distribution sur la sortie standard.
Le format attendu est simple : un seul élément et sa position doivent être affichés par ligne.

\bigskip

\noindent \TTBF{[\textit{position}] : \textit{valeur}}

\bigskip

\noindent Ce qui donnerait pour la distribution $ [ 10, 5, 42 ] $ :

\lstset{language=sh}
\begin{lstlisting}[frame=single]
$ python3.10 example.py
[0] : 10
[1] : 5
[2] : 42
$
\end{lstlisting}

\bigskip

\noindent Si la distribution est vide, vous n'afficherez rien.


%%%%%%%%%

\subsubsection*{\TTBF{AddValueDistribution(Distribution, Value)}}

\noindent Cette fonction renvoie une distribution contenant la valeur ajoutée à la distribution donnée en paramètre.

\noindent Le paramètre \textit{Value} est la valeur à ajouter.
%La valeur supplémentaire s'ajoute en fin de liste.
La valeur supplémentaire s'ajoute en fin de tuple.
On peut ajouter une valeur déjà existante.

\noindent Si la valeur n'est pas un entier, vous déclencherez une exception \TTBF{ValueNotInteger} sans afficher de message.

\medskip

\noindent Ainsi, pour la distribution $ [ 10, 5, 42 ] $, lorsque l'on ajoute la valeur $ 6 $, on doit obtenir la distribution suivante : $ [ 10, 5, 42, 6 ] $.


\subsubsection*{\TTBF{AddValuePositionDistribution(Distribution, Value, Position)}}

\noindent Cette fonction renvoie une distribution contenant la valeur ajoutée à la position précise dans la distribution donnée en paramètre.

\noindent Le paramètre \textit{Value} est la valeur à ajouter.
On peut ajouter une valeur déjà existante.
%Le paramètre \textit{Position} indique la position dans la liste où ajouter l'élément (l'ancien élément présent à cette position sera décalé à la position suivante : si on ajoute en position $ 0 $, l'élément qui y était sera poussé en position $ 1 $).

\noindent Le paramètre \textit{Position} indique la position dans le tuple où ajouter l'élément (l'ancien élément présent à cette position sera décalé à la position suivante : si on ajoute en position $ 0 $, l'élément qui y était sera poussé en position $ 1 $).
Si la position est négative, on ajoute l'élément en position $ 0 $.
%Si la position est supérieure à la dernière position de la liste, on ajoute l'élément en fin de liste/après la dernière position.
Si la position est supérieure à la dernière position du tuple, on ajoute l'élément en fin de tuple/après la dernière position.

\noindent Si la valeur n'est pas un entier, vous déclencherez une exception \TTBF{ValueNotInteger} sans afficher de message.


\subsubsection*{\TTBF{DeleteValueDistribution(Distribution, Value)}}

\noindent Cette fonction renvoie une copie de la distribution donnée en paramètre dont on a supprimé une occurrence de la valeur indiquée (la première occurrence trouvée depuis la position $ 0 $).
Le paramètre \textit{Value} est la valeur à supprimer.

\noindent Si la valeur n'est pas un entier, vous déclencherez une exception \TTBF{ValueNotInteger} sans afficher de message.
%Si la valeur n'est pas trouvée dans la liste, vous déclencherez une exception \TTBF{ValueNotFoundInDistribution} sans afficher de message.
Si la valeur n'est pas trouvée dans le tuple, vous déclencherez une exception \TTBF{ValueNotFoundInDistribution} sans afficher de message.
Si la distribution ne contient plus aucune valeur au moment de l'appel à cette fonction, vous déclencherez une exception \TTBF{DistributionEmpty} sans afficher de message.
L'exception \TTBF{ValueNotInteger} prime sur l'exception \TTBF{DistributionEmpty} qui elle-même prime sur l'exception \TTBF{ValueNotFoundInDistribution}.


\subsubsection*{\TTBF{DeletePositionDistribution(Distribution, Position)}}

\noindent Cette fonction renvoie une copie de la distribution donnée en paramètre dont on a supprimé la valeur située à la position donnée en paramètre.
Le paramètre \textit{Position} est la position de la valeur à supprimer.
Si la position est négative, vous supprimerez l'élément en position $ 0 $.
%Si la position est supérieure à la dernière position de la liste, vous supprimerez le dernier élément de la liste.
Si la position est supérieure à la dernière position du tuple, vous supprimerez le dernier élément du tuple.
%Si la distribution ne contient plus aucune valeur au moment de l'appel à cette fonction, vous déclencherez une exception \TTBF{DistributionEmpty} sans afficher de message.
Si la distribution donnée en paramètre est vide, vous déclencherez une exception \TTBF{DistributionEmpty} sans rien afficher.


\subsubsection*{\TTBF{MergeDistribution(Distribution, NewDistribution)}}

%\noindent Cette fonction renvoie une liste issue de la fusion des deux distributions données en paramètre.
\noindent Cette fonction renvoie un tuple issu de la fusion des deux distributions données en paramètre.
%Le paramètre \textit{NewDistribution} est la distribution à ajouter après la liste en premier paramètre.
Le paramètre \textit{NewDistribution} est la distribution à ajouter après le tuple en premier paramètre.


%%%%%%%%%

\subsubsection*{\TTBF{Min(Distribution)}}

\noindent Cette fonction renvoie la valeur minimale de la distribution.
%Si la liste est vide, vous déclencherez une exception \TTBF{DistributionEmpty} sans afficher de message.
Si la distribution est vide, vous déclencherez une exception \TTBF{DistributionEmpty} sans rien afficher.


\subsubsection*{\TTBF{Max(Distribution)}}

\noindent Cette fonction renvoie la valeur maximale de la distribution.
%Si la liste est vide, vous déclencherez une exception \TTBF{DistributionEmpty} sans afficher de message.
Si la distribution est vide, vous déclencherez une exception \TTBF{DistributionEmpty} sans rien afficher.


\subsubsection*{\TTBF{Cardinality(Distribution)}}

\noindent Cette fonction calcule et renvoie le \textit{cardinal} de la distribution.


\subsubsection*{\TTBF{Range(Distribution)}}

\noindent Cette fonction calcule et renvoie l'\textit{étendue} de la distribution.
%Si la liste est vide, vous déclencherez une exception \TTBF{DistributionEmpty} sans afficher de message.
Si la distribution est vide, vous déclencherez une exception \TTBF{DistributionEmpty} sans rien afficher.


%%%%%%%%%

\subsubsection*{\TTBF{Sum(Distribution)}}

\noindent Cette fonction calcule et renvoie la \textit{somme} des éléments de la distribution (en float).
%Si la liste est vide, vous déclencherez une exception \TTBF{DistributionEmpty} sans afficher de message.
Si la distribution est vide, vous déclencherez une exception \TTBF{DistributionEmpty} sans rien afficher.


\subsubsection*{\TTBF{Mean(Distribution)}}

\noindent Cette fonction calcule et renvoie la \textit{moyenne} de la distribution (en float).
%Si la liste est vide, vous déclencherez une exception \TTBF{DistributionEmpty} sans afficher de message.
Si la distribution est vide, vous déclencherez une exception \TTBF{DistributionEmpty} sans rien afficher.


\subsubsection*{\TTBF{Median(Distribution)}}

\noindent Cette fonction calcule et renvoie la \textit{médiane} de la distribution (en float).
Si la distribution a un nombre impair d'éléments, vous renverrez l'élément à la position par défaut/obtenue par la partie entière inférieure (par exemple, pour une distribution composée de $ 3 $ éléments, vous renverrez l'élément en position $ 1 $ et non pas en $ 2 $).
%Si la liste est vide, vous déclencherez une exception \TTBF{DistributionEmpty} sans afficher de message.
Si la distribution est vide, vous déclencherez une exception \TTBF{DistributionEmpty} sans rien afficher.

\medskip

\noindent Rappel : cette fonction ne doit pas modifier la distribution donnée en paramètre !


\subsubsection*{\TTBF{GetQuartile(Distribution, Quartile)}}

\noindent Cette fonction calcule et renvoie le \textit{quartile} demandé de la distribution (en float).
Le paramètre \TTBF{Quartile} correspond à un entier entre $ 0 $ et $ 4 $ (inclus).
Les quartiles $ 1 $, $ 2 $, et $ 3 $ correspondent aux quartiles séparant les différentes parties de la distribution.
Le quartile $ 0 $ correspond à la plus petite valeur de la distribution, et le quartile $ 4 $ correspond à la plus grande valeur de la distribution.

\noindent Si le paramètre \TTBF{Quartile} est inférieur à $ 0 $ ou supérieur à $ 4 $, vous déclencherez une exception \TTBF{IncorrectQuartileParameter} sans afficher de message.
%En cas de position non entière, vous prendrez la valeur par défaut en retirant juste la virgule.
En cas de position non entière, vous suivrez les indications du cours à la fin du document.
%Si la liste est vide, vous déclencherez une exception \TTBF{DistributionEmpty} sans afficher de message.
Si la distribution est vide, vous déclencherez une exception \TTBF{DistributionEmpty} sans rien afficher.
L'exception \TTBF{IncorrectQuartileParameter} prime sur l'exception \TTBF{DistributionEmpty}.

\medskip

\noindent Rappel : cette fonction ne doit pas modifier la distribution donnée en paramètre !


%%%%%%%%%

\subsubsection*{\TTBF{InterQuartileRange(Distribution)}}

\noindent Cette fonction calcule et renvoie l'\textit{écart interquartile} de la distribution (en float).
C'est-à-dire la différence entre le quartile $ 3 $ et le quartile $ 1 $.
%Si la liste est vide, vous déclencherez une exception \TTBF{DistributionEmpty} sans afficher de message.
Si la distribution est vide, vous déclencherez une exception \TTBF{DistributionEmpty} sans rien afficher.


\subsubsection*{\TTBF{Variance(Distribution)}}

\noindent Cette fonction calcule et renvoie la \textit{variance} de la distribution (en float).
%Si la liste est vide, vous déclencherez une exception \TTBF{DistributionEmpty} sans afficher de message.
Si la distribution est vide, vous déclencherez une exception \TTBF{DistributionEmpty} sans rien afficher.


\subsubsection*{\TTBF{StandardDeviation(Distribution)}}

\noindent Cette fonction calcule et renvoie l'\textit{écart type} de la distribution (en float).
%Si la liste est vide, vous déclencherez une exception \TTBF{DistributionEmpty} sans afficher de message.
Si la distribution est vide, vous déclencherez une exception \TTBF{DistributionEmpty} sans rien afficher.


\subsubsection*{\TTBF{RelativeStandardDeviation(Distribution)}}

\noindent Cette fonction calcule et renvoie le \textit{coefficient de variation} de la distribution (en float).
%Si la liste est vide, vous déclencherez une exception \TTBF{DistributionEmpty} sans afficher de message.
Si la distribution est vide, vous déclencherez une exception \TTBF{DistributionEmpty} sans rien afficher.


\subsubsection*{\TTBF{GiniCoefficient(Distribution)}}

\noindent Cette fonction calcule et renvoie le \textit{coefficient de Gini} de la distribution (en float).
%Si la liste est vide, vous déclencherez une exception \TTBF{DistributionEmpty} sans afficher de message.
Si la distribution est vide, vous déclencherez une exception \TTBF{DistributionEmpty} sans rien afficher.
