% -*- latex -*-

\part{Files and Directories}  % Files and Directories

%%%%%%%%%%%%%%%%%%%%%%%%%%%%%%%%%%%%%%%%%%%%%%%%%%%%%%%%

% File management in python

\section{Files Processing}

% How to open, read, write, close, seek
\subsection{Open, Read, Write, Close, Seek, Tell}


\begin{frame}[fragile]{Files Operations}

  Files are managed as usual:

  \begin{onlyenv}<2>
    \begin{itemize}
      \item \TTBF{open}
      \item \TTBF{read}
      \item \TTBF{write}
      \item \TTBF{close}
      \item \TTBF{seek}
      \item \TTBF{tell}
    \end{itemize}
  \end{onlyenv}

  \begin{onlyenv}<3>
    \begin{itemize}
      \item \TTBF{open(filename, mode, \textit{[encoding]})}
      \item \TTBF{read}
      \item \TTBF{write}
      \item \TTBF{close}
      \item \TTBF{seek}
      \item \TTBF{tell}
    \end{itemize}
  \end{onlyenv}

  \begin{onlyenv}<4>
    \begin{itemize}
      \item \TTBF{open(filename, mode, \textit{[encoding]})}
      \item \TTBF{read(NbCharacters)}
      \item \TTBF{write}
      \item \TTBF{close}
      \item \TTBF{seek}
      \item \TTBF{tell}
    \end{itemize}
  \end{onlyenv}

  \begin{onlyenv}<5>
    \begin{itemize}
      \item \TTBF{open(filename, mode, \textit{[encoding]})}
      \item \TTBF{read(NbCharacters)}
      \item \TTBF{write(string)}
      \item \TTBF{close}
      \item \TTBF{seek}
      \item \TTBF{tell}
    \end{itemize}
  \end{onlyenv}

  \begin{onlyenv}<6>
    \begin{itemize}
      \item \TTBF{open(filename, mode, \textit{[encoding]})}
      \item \TTBF{read(NbCharacters)}
      \item \TTBF{write(string)}
      \item \TTBF{close()}
      \item \TTBF{seek}
      \item \TTBF{tell}
    \end{itemize}
  \end{onlyenv}

  \begin{onlyenv}<7>
    \begin{itemize}
      \item \TTBF{open(filename, mode, \textit{[encoding]})}
      \item \TTBF{read(NbCharacters)}
      \item \TTBF{write(string)}
      \item \TTBF{close()}
      \item \TTBF{seek(offset, \textit{[whence]})}
      \item \TTBF{tell}
    \end{itemize}
  \end{onlyenv}

  \begin{onlyenv}<8->
    \begin{itemize}
      \item \TTBF{open(filename, mode, \textit{[encoding]})}
      \item \TTBF{read(NbCharacters)}
      \item \TTBF{write(string)}
      \item \TTBF{close()}
      \item \TTBF{seek(offset, \textit{[whence]})}
      \item \TTBF{tell()}
    \end{itemize}
  \end{onlyenv}

  \medskip

  \onslide<9-> Always close files in order to be sure to write changes on the physical support

  \medskip

  \onslide<10-> Files are managed by an internal python's \textbf{class}

\end{frame}

%%%

\begin{frame}<beamer>{Open}

  Open mode:

  \begin{itemize}
    \item<2-> \TTBF{"r"} - open for reading
    \item<3-> \TTBF{"w"} - truncate the file for overwriting
    \item<4-> \TTBF{"a"} - append/begin from the end for writing
    \item<5-> \TTBF{"x"} - create a new file, fail if it already exists
    \item<6-> \TTBF{"r+"} - open for reading and writing
    \item<7-> \TTBF{"b"} - binary mode (read/write with bytes objects)
  \end{itemize}

  \bigskip

  \onslide<8-> Open encoding:

  \begin{itemize}
    \item<9-> \TTBF{encoding="utf-8"} - read/write in \texttt{utf-8}
    \item<10-> Can be omitted for platform encoding \textit{[do this for now]}
    \item<11-> Cannot be used when in binary mode
  \end{itemize}

\end{frame}

%%%

\begin{frame}<beamer>{Text \& Binary mode}

  \onslide<1-> Text mode: \onslide<2-> \TTBF{open(filename, "r+")}

  \begin{itemize}
    \item<3-> For reading/writting text files (CSV, configuration, log, ...)
    \item<4-> When writing in a file: \TTBF{\textbackslash n} is transformed into platform specific line ending (\TTBF{\textbackslash r\textbackslash n}, \TTBF{\textbackslash n\textbackslash r}, ...)
    \item<5-> When reading from a file: platform specific line endings are transformed into simple \TTBF{\textbackslash n}
  \end{itemize}

  \bigskip

  \onslide<6-> Binary mode: \onslide<7-> \TTBF{open(filename, "wb")}

  \begin{itemize}
    \item<8-> For reading/writting binary data (music, video, web page, ...)
    \item<9-> The line ending characters are never transformed when reading/writting
  \end{itemize}

  \bigskip

  \onslide<10-> \textit{For now: do not use binary mode, except if you work on raw data}

\end{frame}

%%%

\begin{frame}<beamer>{Read \& Write}

  Read:

  \begin{itemize}
    \item<1-> \TTBF{read()} - get the whole text
    \item<2-> \TTBF{read(Nb)} - get the next \textit{Nb} characters
    \item<3-> \TTBF{readline()} - get the next line (whole text until next \TTBF{\textbackslash n})
  \end{itemize}

  \bigskip

  \onslide<4-> Write:

  \begin{itemize}
    \item<5-> \TTBF{open(filename, "wb")}
    \item<6-> \TTBF{write(str)} - Write the string and return how many characters were written
  \end{itemize}

\end{frame}

%%%

\begin{frame}<beamer>{Tell \& Seek}

  Tell:

  \begin{itemize}
    \item<1-> \TTBF{tell()} - get current position in the file (in bytes)
    \item<2-> Useful in binary mode
    \item<3-> More difficult in text mode (because of the \textit{encoding)}
  \end{itemize}

  \bigskip

  \onslide<4-> Seek:

  \begin{itemize}
    \item<5-> \TTBF{seek(offset)} - put read/write cursor at the offset
    \item<5-> \TTBF{seek(offset, whence)} - move the cursor relatively
    \item<6-> \TTBF{whence=}
      \begin{itemize}
       \item \TTBF{0} : from the beginning of the file \textit{[default]}
       \item \TTBF{1} : from the current cursor position
       \item \TTBF{2} : from the end of the file
      \end{itemize}
  \end{itemize}

\end{frame}

%%%%%%%%%%%

% Examples of read/write
\subsection{Read \& Write examples}

% Reading files

\begin{frame}[fragile]{Read (1) example}

  Open existing file for reading

  \begin{columns}[onlytextwidth]
    \begin{column}{\textwidth}

      \begin{onlyenv}<1>
        % Balises exception :  %* *)
        \begin{lstlisting}[style=python]





 \end{lstlisting}
      \end{onlyenv}

      \begin{onlyenv}<2>
        % Balises exception :  %* *)
        \begin{lstlisting}[style=python]
f = open("file.txt", "r")




 \end{lstlisting}
      \end{onlyenv}

      \begin{onlyenv}<3>
        % Balises exception :  %* *)
        \begin{lstlisting}[style=python]
f = open("file.txt", "r")
chars = f.read(10)
print(chars)


 \end{lstlisting}
      \end{onlyenv}

      \begin{onlyenv}<4>
        % Balises exception :  %* *)
        \begin{lstlisting}[style=python]
f = open("file.txt", "r")
chars = f.read(10)
print(chars)
line = f.readline()
print(line)
 \end{lstlisting}
      \end{onlyenv}

      \begin{onlyenv}<5->
        % Balises exception :  %* *)
        \begin{lstlisting}[style=python]
f = open("file.txt", "r")
chars = f.read(10)
print(chars)
line = f.readline()
print(line)
f.close() \end{lstlisting}
      \end{onlyenv}

    \end{column}
  \end{columns}

\end{frame}

%%%

\begin{frame}[fragile]{Read (2) example}

  Read line by line a file with python specificities

  \begin{columns}[onlytextwidth]
    \begin{column}{\textwidth}

      \begin{onlyenv}<1>
        % Balises exception :  %* *)
        \begin{lstlisting}[style=python]



 \end{lstlisting}
      \end{onlyenv}

      \begin{onlyenv}<2>
        % Balises exception :  %* *)
        \begin{lstlisting}[style=python]
f = open("file.txt", "r")


 \end{lstlisting}
      \end{onlyenv}

      \begin{onlyenv}<3>
        % Balises exception :  %* *)
        \begin{lstlisting}[style=python]
f = open("file.txt", "r")
for x in f:
  print(x)
 \end{lstlisting}
      \end{onlyenv}

      \begin{onlyenv}<4->
        % Balises exception :  %* *)
        \begin{lstlisting}[style=python]
f = open("file.txt", "r")
for x in f:
  print(x)
f.close() \end{lstlisting}
      \end{onlyenv}

    \end{column}
  \end{columns}

\end{frame}

%%%

% Writing files

\begin{frame}[fragile]{Write example}

  Write at the end of a file (add content)

  \begin{columns}[onlytextwidth]
    \begin{column}{\textwidth}

      \begin{onlyenv}<1>
        % Balises exception :  %* *)
        \begin{lstlisting}[style=python]



 \end{lstlisting}
      \end{onlyenv}

      \begin{onlyenv}<2>
        % Balises exception :  %* *)
        \begin{lstlisting}[style=python]
f = open("file.txt", "a")


 \end{lstlisting}
      \end{onlyenv}

      \begin{onlyenv}<3>
        % Balises exception :  %* *)
        \begin{lstlisting}[style=python]
f = open("file.txt", "a")
nb = f.write("-add content-")

 \end{lstlisting}
      \end{onlyenv}

      \begin{onlyenv}<4>
        % Balises exception :  %* *)
        \begin{lstlisting}[style=python]
f = open("file.txt", "a")
nb = f.write("-add content-")
f.close()
 \end{lstlisting}
      \end{onlyenv}

      \begin{onlyenv}<5->
        % Balises exception :  %* *)
        \begin{lstlisting}[style=python]
f = open("file.txt", "a")
nb = f.write("-add content-")
f.close()
print(nb)  # 13 chars were added \end{lstlisting}
      \end{onlyenv}

    \end{column}
  \end{columns}

\end{frame}

%%%

\begin{frame}<beamer>{Documentation}

  \begin{center}
    \onslide<1-> Check documentation for more informations

    \bigskip

    \onslide<1-> \url{https://docs.python.org/3/tutorial/inputoutput.html}
  \end{center}

\end{frame}


%%%%%%%%%%%%%%%%%%%%%%%%%%%%%%%%%%%%%%


% Test file existence and delete
\section{File existence and management}

\begin{frame}<beamer>{Files and Directories}

  Management of files is done through the \TTBF{os} module

  \begin{itemize}
    \item<2-> \TTBF{os.chdir(path)} - change the current directory
    \item<3-> \TTBF{os.mkdir(dirname)} - create a directory
    \item<4-> \TTBF{os.makedirs(path)} - create the full path (recursively)
    \item<5-> \TTBF{os.remove(filename)} - remove the file
    \item<6-> \TTBF{os.rmdir(dirname)} - remove the directory
    \item<7-> \TTBF{os.listdir(path)} - list files in a directory
    \item<8-> \TTBF{os.scandir(path)} - list files in a directory (generator)
  \end{itemize}

\end{frame}

%%%

\begin{frame}<beamer>{File type}

  Type of file and names are done with \TTBF{path} submodule

  \begin{itemize}
    \item<2-> \TTBF{os.path.exist(path)} - test if \textit{path} is an existing file
    \item<3-> \TTBF{os.path.isfile(path)} - test if \textit{path} is a file
    \item<4-> \TTBF{os.path.isdir(path)} - test if \textit{path} is a directory
    \item<5-> \TTBF{os.path.abspath(path)} - get absolute pathname
    \item<6-> \TTBF{os.path.dirname(path)} - get dirname of pathname
    \item<7-> \TTBF{os.path.basename(path)} - get basename of pathname
    \item<8-> \TTBF{os.path.join(parent, child)} - build a pathname from two paths
  \end{itemize}

  \medskip

  \begin{center}
    \onslide<9-> \url{https://docs.python.org/3/library/os.path.html}
  \end{center}

\end{frame}

%%%

\begin{frame}[fragile]{File browsing example}

  Browse a directory:

  \begin{columns}[onlytextwidth]
    \begin{column}{\textwidth}

      \begin{onlyenv}<1>
        % Balises exception :  %* *)
        \begin{lstlisting}[style=python,basicstyle=\ttfamily\footnotesize]










 \end{lstlisting}
      \end{onlyenv}

      \begin{onlyenv}<2>
        % Balises exception :  %* *)
        \begin{lstlisting}[style=python,basicstyle=\ttfamily\footnotesize]
import os









 \end{lstlisting}
      \end{onlyenv}

      \begin{onlyenv}<3>
        % Balises exception :  %* *)
        \begin{lstlisting}[style=python,basicstyle=\ttfamily\footnotesize]
import os


entries = os.listdir(".")






 \end{lstlisting}
      \end{onlyenv}

      \begin{onlyenv}<4>
        % Balises exception :  %* *)
        \begin{lstlisting}[style=python,basicstyle=\ttfamily\footnotesize]
import os

# Print files & dirs
entries = os.listdir(".")
for entry in entries:
  print(str(entry))




 \end{lstlisting}
      \end{onlyenv}

      \begin{onlyenv}<5>
        % Balises exception :  %* *)
        \begin{lstlisting}[style=python,basicstyle=\ttfamily\footnotesize]
import os

# Print files & dirs
entries = os.listdir(".")
for entry in entries:
  print(str(entry))


files = [f for f in os.listdir() if os.path.isfile(f)]

 \end{lstlisting}
      \end{onlyenv}

      \begin{onlyenv}<6->
        % Balises exception :  %* *)
        \begin{lstlisting}[style=python,basicstyle=\ttfamily\footnotesize]
import os

# Print files & dirs
entries = os.listdir(".")
for entry in entries:
  print(str(entry))

# Print files only
files = [f for f in os.listdir() if os.path.isfile(f)]
for f in files:
  print(str(f)) \end{lstlisting}
      \end{onlyenv}

    \end{column}
  \end{columns}

\end{frame}