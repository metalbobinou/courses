% -*- latex -*-

\part{OOP and Classes}  % OOP and Classes

%%%%%%%%%%%%%%%%%%%%%%%%%%%%%%%%%%%%%%%%%%%%%%%%%%%%%%%%

% How to create a class

\section{Object Oriented Programming}

\begin{frame}<beamer>{OOP}

  Object Oriented Programming (OOP) vocabulary

  \begin{itemize}
    \item<2-> Class: the type and structure
    \item<3-> Attribute: a variable that is a member of the class
    \item<4-> Method: a function/procedure that is a member of the class
  \end{itemize}

  \begin{itemize}
    \item<5-> Object: an instance of a class
    \item<6-> Constructor: the first method called when creating an object
    \item<7-> Destructor: the method called when deleting an object
  \end{itemize}

  \begin{itemize}
    \item<8-> Inheritance: structure of a class reused as a basis for a new one
    \item<9-> Private: when a member is accessible only by the object itself
    \item<10-> Public: when a member is accessible by any object
  \end{itemize}

\end{frame}

%%%

\begin{frame}<beamer>{OOP}

  Object Oriented Programming (OOP) concepts examples

  \begin{itemize}
    \item<1-> Class: the type and structure
    \item<1-> Attribute: a variable that is a member of the class
    \item<1-> Method: a function/procedure that is a member of the class
    \item<1-> Object: an instance of a class
  \end{itemize}

  \begin{itemize}
    \item<2-> Classes: vehicle, car, airplane, boat, ...
    \item<3-> Attributes: speed, passengers, engine, ...
    \item<4-> Methods: accelerate, decelerate, embark, ...
    \item<5-> Objects/Instances:
      \begin{itemize}
        \item<5-> My Peugeot 206 (\textit{BN-340-FT})
        \item<5-> Your Toyota Corolla (\textit{UE-042-FI})
      \end{itemize}
  \end{itemize}

\end{frame}

%%%

\begin{frame}<beamer>{OOP}

  Inheritance explanation

  \begin{itemize}
    \item<2-> Parent Class: the class more general or abstract
    \item<3-> Child Class / Sub Class: a specialized class derived from a parent
    \item<4-> \textit{the child class inherits attributes and methods from its parent}
  \end{itemize}

  \begin{onlyenv}<5->
    \vspace*{1cm}

    Attribute/Method access modifiers
  \end{onlyenv}

  \begin{itemize}
    \item<6-> Private: only the class itself can access it
    \item<7-> Protected: the class itself and its childs can access it
    \item<8-> Public: any class can access it
  \end{itemize}

\end{frame}

%%%

\begin{frame}<beamer>{OOP}

  Inheritance example

  \begin{itemize}
    \item<1-> Parent Class: the class more general or abstract
    \item<1-> Child Class / Sub Class: a specialized class derived from a parent
    \item<1-> \textit{the child class inherits attributes and methods from its parent}
  \end{itemize}

  \begin{onlyenv}<1->
    \vspace*{1cm}
  \end{onlyenv}

  \begin{itemize}
    \item<2-> Class: vehicle
    \item<3-> Child classes: car, airplane, boat, ...
    \item<4-> Inherited Attributes: speed, passengers, ...
  \end{itemize}

\end{frame}


% How to create a class

\section{Classes in Python}

\begin{frame}<beamer>{Classes in Python}

  Specificities of classes in Python:

  \begin{itemize}
    \item<2-> Only one constructor: \lstinline|__init__|
    \item<3-> No destructor (python manages the memory by references)
    \item<4-> Attributes are at least \textit{read-only}, and eventually \textit{writable}
    \item<5-> Writable attributes can be deleted from the object
    \item<6-> Member beginning by an underscore (\lstinline|_|) aren't strictly private, but should be considered internal to the class
    \item<7-> \TTBF{self} keyword is required as the first parameter of each method
  \end{itemize}

\end{frame}

%%%

\begin{frame}[fragile]{Classes in Python}

  \begin{columns}[onlytextwidth]
    \begin{column}{\textwidth}

      \begin{onlyenv}<1>
        % Balises exception :  %* *)
        \begin{lstlisting}[style=python]











 \end{lstlisting}
      \end{onlyenv}

      \begin{onlyenv}<2>
        \begin{lstlisting}[style=python]
class Vehicle:
  """ General vehicles """









 \end{lstlisting}
      \end{onlyenv}

      \begin{onlyenv}<3>
        \begin{lstlisting}[style=python]
class Vehicle:
  """ General vehicles """
  __speed = 0
  Passengers = 0







 \end{lstlisting}
      \end{onlyenv}

      \begin{onlyenv}<4>
        \begin{lstlisting}[style=python]
class Vehicle:
  """ General vehicles """
  __speed = 0
  Passengers = 0
  # Constructor
  def __init__(self):
    self.Passengers = 1




 \end{lstlisting}
      \end{onlyenv}

      \begin{onlyenv}<5>
        \begin{lstlisting}[style=python]
class Vehicle:
  """ General vehicles """
  __speed = 0
  Passengers = 0
  # Constructor
  def __init__(self):
    self.Passengers = 1
  # Method
  def getSpeed(self):
    return (self.__speed)

 \end{lstlisting}
      \end{onlyenv}

      \begin{onlyenv}<6>
        \begin{lstlisting}[style=python]
class Vehicle:
  """ General vehicles """
  __speed = 0
  Passengers = 0
  # Constructor
  def __init__(self):
    self.Passengers = 1
  # Methods
  def getSpeed(self):
    return (self.__speed)
  def Accelerate(self, x):
    self.__speed += x \end{lstlisting}
      \end{onlyenv}


      \begin{onlyenv}<7->
        \begin{lstlisting}[style=python]
class Vehicle:
  """ General vehicles """
  __speed = 0
  Passengers = 0
  # Constructor
  def __init__(self):       # Constructor
    self.Passengers = 1
  # Methods
  def getSpeed(self):       # Accessor
    return (self.__speed)
  def Accelerate(self, x):  # Mutator
    self.__speed += x \end{lstlisting}
      \end{onlyenv}

    \end{column}
  \end{columns}

\end{frame}

%%%

\begin{frame}[fragile]{Classes in Python}

  \begin{columns}[onlytextwidth]
    \begin{column}{\textwidth}

      \begin{onlyenv}<1>
        \begin{lstlisting}[style=python]









 \end{lstlisting}
      \end{onlyenv}

      \begin{onlyenv}<2>
        \begin{lstlisting}[style=python]
class Car(Vehicle):
  """ Cars inherit from Vehicle """
  __CV = 0






 \end{lstlisting}
      \end{onlyenv}

      \begin{onlyenv}<3>
        \begin{lstlisting}[style=python]
class Car(Vehicle):
  """ Cars inherit from Vehicle """
  __CV = 0
  # Constructor
  def __init__(self, CO2, P):
    self.Passengers = 1
    self.__CV = (CO2 / 45) + (P / 40)


 \end{lstlisting}
      \end{onlyenv}

      \begin{onlyenv}<4->
        \begin{lstlisting}[style=python]
class Car(Vehicle):
  """ Cars inherit from Vehicle """
  __CV = 0
  # Constructor
  def __init__(self, CO2, P):
    self.Passengers = 1
    self.__CV = (CO2 / 45) + (P / 40)
  # Method
  def getCV(self):
    return (self.__CV) \end{lstlisting}
      \end{onlyenv}

    \end{column}
  \end{columns}

\end{frame}

%%%

\begin{frame}[fragile]{Classes in Python}

  \begin{columns}[onlytextwidth]
    \begin{column}{\textwidth}

      \begin{onlyenv}<1>
        \begin{lstlisting}[style=python]










 \end{lstlisting}
      \end{onlyenv}

      \begin{onlyenv}<2>
        \begin{lstlisting}[style=python]
Peugeot206Plus = Car(110, 44)  # 110g/km  44kW
AirbusA340 = Vehicle()








 \end{lstlisting}
      \end{onlyenv}

      \begin{onlyenv}<3>
        \begin{lstlisting}[style=python]
Peugeot206Plus = Car(110, 44)  # 110g/km  44kW
AirbusA340 = Vehicle()

Peugeot206Plus.Accelerate(80)
AirbusA340.Accelerate(260)





 \end{lstlisting}
      \end{onlyenv}

      \begin{onlyenv}<4>
        \begin{lstlisting}[style=python]
Peugeot206Plus = Car(110, 44)  # 110g/km  44kW
AirbusA340 = Vehicle()

Peugeot206Plus.Accelerate(80)
AirbusA340.Accelerate(260)

Peugeot206Plus.getSpeed()  # 80
AirbusA340.getSpeed()      # 260

Peugeot206Plus.getCV()
AirbusA340.getCV() \end{lstlisting}
      \end{onlyenv}


      \begin{onlyenv}<5->
        \begin{lstlisting}[style=python]
Peugeot206Plus = Car(110, 44)  # 110g/km  44kW
AirbusA340 = Vehicle()

Peugeot206Plus.Accelerate(80)
AirbusA340.Accelerate(260)

Peugeot206Plus.getSpeed()  # 80
AirbusA340.getSpeed()      # 260

Peugeot206Plus.getCV()
AirbusA340.getCV()  # Error \end{lstlisting}
      \end{onlyenv}

    \end{column}
  \end{columns}

\end{frame}

%%%

\begin{frame}<beamer>{Method Overriding in Python}

  Method Overriding: when a method is redefined in the child class

  \vspace*{1cm}

  \begin{itemize}
    \item<2-> Child method is called if redefined
    \item<3-> Use \TTBF{super()} on current class for calling its parent method: \\
    \TTBF{super(Class, self).Method()}
  \end{itemize}

\end{frame}

%%%

\begin{frame}[fragile]{Method Overriding in Python}

  \begin{columns}[onlytextwidth]
    \begin{column}{0.47\textwidth}

      \begin{onlyenv}<1>
        \begin{lstlisting}[style=python,basicstyle=\ttfamily\footnotesize]












 \end{lstlisting}
      \end{onlyenv}

      \begin{onlyenv}<2>
        \begin{lstlisting}[style=python,basicstyle=\ttfamily\footnotesize]
class Shape:
  def Hello(self):
    print("Shape: Hello!")

  def SayShape(self):
    print("--Shape--")






 \end{lstlisting}
      \end{onlyenv}

      \begin{onlyenv}<3->
        \begin{lstlisting}[style=python,basicstyle=\ttfamily\footnotesize]
class Shape:
  def Hello(self):
    print("Shape: Hello!")

  def SayShape(self):
    print("--Shape--")

class Cube(Shape):
  def Hello(self):
    print("Cube: Hello!")

  def SayCube(self):
    print("--Cube--") \end{lstlisting}
      \end{onlyenv}

    \end{column}


    \begin{column}{0.47\textwidth}

      \begin{onlyenv}<4>
        \begin{lstlisting}[style=python,basicstyle=\ttfamily\footnotesize]












 \end{lstlisting}
      \end{onlyenv}

      \begin{onlyenv}<5>
        \begin{lstlisting}[style=python,basicstyle=\ttfamily\footnotesize]
S = Shape()
C = Cube()










 \end{lstlisting}
      \end{onlyenv}

      \begin{onlyenv}<6>
        \begin{lstlisting}[style=python,basicstyle=\ttfamily\footnotesize]
S = Shape()
C = Cube()


S.Hello()
C.Hello()






 \end{lstlisting}
      \end{onlyenv}

      \begin{onlyenv}<7>
        \begin{lstlisting}[style=python,basicstyle=\ttfamily\footnotesize]
S = Shape()
C = Cube()


S.Hello() # Shape: Hello!
C.Hello() # Cube: Hello!






 \end{lstlisting}
      \end{onlyenv}

      \begin{onlyenv}<8>
        \begin{lstlisting}[style=python,basicstyle=\ttfamily\footnotesize]
S = Shape()
C = Cube()


S.Hello() # Shape: Hello!
C.Hello() # Cube: Hello!

S.SayShape()
C.SayShape()



 \end{lstlisting}
      \end{onlyenv}

      \begin{onlyenv}<9>
        \begin{lstlisting}[style=python,basicstyle=\ttfamily\footnotesize]
S = Shape()
C = Cube()


S.Hello() # Shape: Hello!
C.Hello() # Cube: Hello!

S.SayShape() # --Shape--
C.SayShape() # --Shape--



 \end{lstlisting}
      \end{onlyenv}

      \begin{onlyenv}<10>
        \begin{lstlisting}[style=python,basicstyle=\ttfamily\footnotesize]
S = Shape()
C = Cube()


S.Hello() # Shape: Hello!
C.Hello() # Cube: Hello!

S.SayShape() # --Shape--
C.SayShape() # --Shape--

S.SayCube()
C.SayCube()
 \end{lstlisting}
      \end{onlyenv}

      \begin{onlyenv}<11->
        \begin{lstlisting}[style=python,basicstyle=\ttfamily\footnotesize]
S = Shape()
C = Cube()


S.Hello() # Shape: Hello!
C.Hello() # Cube: Hello!

S.SayShape() # --Shape--
C.SayShape() # --Shape--

S.SayCube() # ERROR
C.SayCube() # --Cube--
 \end{lstlisting}
      \end{onlyenv}

    \end{column}
  \end{columns}

\end{frame}

%%%

\begin{frame}[fragile]{}

  \begin{columns}[onlytextwidth]
    \begin{column}{0.47\textwidth}

      \begin{onlyenv}<1>
        \begin{lstlisting}[style=python,basicstyle=\ttfamily\footnotesize]
class Shape:
  def Hello(self):
    print("Shape: Hello!")

  def SayShape(self):
    print("--Shape--")








 \end{lstlisting}
      \end{onlyenv}

      \begin{onlyenv}<2->
        \begin{lstlisting}[style=python,basicstyle=\ttfamily\footnotesize]
class Shape:
  def Hello(self):
    print("Shape: Hello!")

  def SayShape(self):
    print("--Shape--")

class Cube(Shape):
  def Hello(self):
    print("Cube: Hello!")

  def SayCube(self):
    super(Cube, self).
   SayShape()
    print("--Cube--") \end{lstlisting}
      \end{onlyenv}

    \end{column}


    \begin{column}{0.47\textwidth}

      \begin{onlyenv}<3>
        \begin{lstlisting}[style=python,basicstyle=\ttfamily\footnotesize]














 \end{lstlisting}
      \end{onlyenv}

      \begin{onlyenv}<4>
        \begin{lstlisting}[style=python,basicstyle=\ttfamily\footnotesize]
S = Shape()
C = Cube()












 \end{lstlisting}
      \end{onlyenv}

      \begin{onlyenv}<5>
        \begin{lstlisting}[style=python,basicstyle=\ttfamily\footnotesize]
S = Shape()
C = Cube()


S.Hello()
C.Hello()








 \end{lstlisting}
      \end{onlyenv}

      \begin{onlyenv}<6>
        \begin{lstlisting}[style=python,basicstyle=\ttfamily\footnotesize]
S = Shape()
C = Cube()


S.Hello() # Shape: Hello!
C.Hello() # Cube: Hello!








 \end{lstlisting}
      \end{onlyenv}

      \begin{onlyenv}<7>
        \begin{lstlisting}[style=python,basicstyle=\ttfamily\footnotesize]
S = Shape()
C = Cube()


S.Hello() # Shape: Hello!
C.Hello() # Cube: Hello!

S.SayShape()
C.SayShape()





 \end{lstlisting}
      \end{onlyenv}

      \begin{onlyenv}<8>
        \begin{lstlisting}[style=python,basicstyle=\ttfamily\footnotesize]
S = Shape()
C = Cube()


S.Hello() # Shape: Hello!
C.Hello() # Cube: Hello!

S.SayShape() # --Shape--
C.SayShape() # --Shape--





 \end{lstlisting}
      \end{onlyenv}

      \begin{onlyenv}<9>
        \begin{lstlisting}[style=python,basicstyle=\ttfamily\footnotesize]
S = Shape()
C = Cube()


S.Hello() # Shape: Hello!
C.Hello() # Cube: Hello!

S.SayShape() # --Shape--
C.SayShape() # --Shape--

S.SayCube()
C.SayCube()


 \end{lstlisting}
      \end{onlyenv}

      \begin{onlyenv}<10->
        \begin{lstlisting}[style=python,basicstyle=\ttfamily\footnotesize]
S = Shape()
C = Cube()


S.Hello() # Shape: Hello!
C.Hello() # Cube: Hello!

S.SayShape() # --Shape--
C.SayShape() # --Shape--

S.SayCube() # ERROR
C.SayCube() # --Shape--
            # --Cube--

 \end{lstlisting}
      \end{onlyenv}

    \end{column}
  \end{columns}

\end{frame}

%%%

\begin{frame}<beamer>{Summary of Classes in Python}

  Summary of vocabulary:

  \begin{itemize}
    \item<2-> \TTBF{Vehicle}: class
    \item<3-> \TTBF{Car}: class (inherits from \TTBF{Vehicle})
    \item<4-> \TTBF{Peugeot206Plus}: instance of \TTBF{Car}
    \item<5-> \TTBF{AirbusA340}: instance of \TTBF{Vehicle}
  \end{itemize}

  \begin{itemize}
  \item<6-> \TTBF{Peugeot206Plus} can calculate and give \TTBF{CV}
  \item<7-> \TTBF{AirbusA340}: doesn't have \TTBF{CV} method nor attribute
  \end{itemize}

  \begin{center}
    \onslide<8-> \url{https://docs.python.org/3/tutorial/classes.html}
  \end{center}

\end{frame}
