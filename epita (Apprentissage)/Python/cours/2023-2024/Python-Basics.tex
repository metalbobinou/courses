\documentclass[12pt,a4paper]{beamer}

\usepackage[utf8]{inputenc}
%\usepackage[french]{babel}
\usepackage[english]{babel}
\usepackage[T1]{fontenc}

\usepackage{amsmath}
\usepackage{amsfonts}
\usepackage{amssymb}

\usepackage{icomma} % Put a space after a comma in math "only" if explicitly written

% Preparation des infos generales
\newcommand{\CodeMatiere}{PYTHON}
\newcommand{\TitreMatiere}{Python}
\newcommand{\TitreSeance}{Python Basics}
\newcommand{\DateCours}{12 septembre 2023}
\newcommand{\AnneeScolaire}{2023-2024}
\newcommand{\Organisation}{EPITA}
\newcommand{\Parcours}{Apprentissage}
\newcommand{\NiveauScolaire}{Ing1}
\newcommand{\NomAuteurA}{Fabrice BOISSIER}
\newcommand{\MailAuteurA}{fabrice.boissier@epita.fr}
%\newcommand{\NomAuteurB}{ }
%\newcommand{\MailAuteurB}{ }
\newcommand{\VersionExo}{Version 1}
\newcommand{\DocKeywords}{Python, Introduction, Functions, Interpreter, Basics}
\newcommand{\DocLangue}{en} % "en", "fr", ...


% Ajout de mes classes & definitions
\usepackage{MetalCours-beamer} % Appelle un .sty

%\titlegraphic{\includegraphics[width=.4\linewidth]{images/logos/epita}}
%\titlegraphic{\includegraphics[width=.4\linewidth]{images/logos/epita} \includegraphics[width=.4\linewidth]{images/logos/bachelor_cyber}}
\titlegraphic{\includegraphics[width=.4\linewidth]{images/logos/apprentissage} \\ {\scriptsize \VersionExo}}

% Deactivate navigation
\setbeamertemplate{navigation symbols}{}


\definecolor{mygreen}{rgb}{0,0.6,0}		% RGB model
\definecolor{mygray}{rgb}{0.5,0.5,0.5}
\definecolor{mymauve}{rgb}{0.58,0,0.82}


\usepackage{array}
\newcolumntype{P}[1]{>{\raggedright\arraybackslash }m{#1}}

\newcolumntype{R}[1]{>{\raggedleft\arraybackslash }b{#1}}
\newcolumntype{L}[1]{>{\raggedright\arraybackslash }b{#1}}
\newcolumntype{C}[1]{>{\centering\arraybackslash }b{#1}}

\newcolumntype{G}[1]{>{\raggedleft\let\newline\\\arraybackslash\hspace{0pt}}m{#1}}  % another R
\newcolumntype{D}[1]{>{\raggedright\let\newline\\\arraybackslash\hspace{0pt}}m{#1}} % another L
\newcolumntype{M}[1]{>{\centering\let\newline\\\arraybackslash\hspace{0pt}}m{#1}}   % another C


\begin{document}

%% Titre
%\maketitle
\frame[plain]{\titlepage}

\newpage

%% Copyright
%\pagenumbering{Roman}

\begin{frame}[fragile,plain]{License}

  \begin{itemize}
    \item (\latex source code partially reused from Gabriel Laskar)
    \item Copyright \copyright{} 2022-2024, Fabrice Boissier
  \end{itemize}

  \begin{verbatim}
  Permission is granted to copy, distribute
  and/or modify this document under the terms
  of the GNU Free Documentation License,
  Version 1.2 or any later version published
  by the Free Software Foundation; with the
  Invariant Sections being just ``Copying
  this document'', no Front-Cover Texts, and
  no Back-Cover Texts.
  \end{verbatim}

\end{frame}


%%% Course outline
% -*- latex -*-

\part{Introduction}

\section{Course outline}

\begin{frame}{What are we trying to learn?}{Python language}

  What are the objectives of the course:
  \begin{itemize}
    \item Discovering Python  % Language, CLI, versions, ...
    \item Writing small Python's scripts  % Hello world, print, str, Sys module
    \item Understanding main Python's concepts  % Types, lists (comprehension, ...)
    \item Writing functions and programs in Python  % Declaring functions, imports, modules
    \item Being autonomous while writing code  % Read documentation
  \end{itemize}

\end{frame}

%\section{Course organization}

\begin{frame}{Course organization}

  \begin{itemize}
    \item<1-> Lecture (30 mins \textasciitilde{} 1h per day)
    \item<2-> Labs (full day)
    \item<3-> Submission preparation (15/09/2023 morning)
    \item<4-> Exercices submission (XX/09/2023)
  \end{itemize}

  \medskip

  \begin{center}

    \onslide<4-> tuesday 12\textsuperscript{th} september 2023

    \onslide<4-> wednesday 13\textsuperscript{th} september 2023

    \onslide<4-> friday 15\textsuperscript{th} september 2023 - \textit{[morning only]}

  \end{center}

%  \onslide<4-> Evaluation on exercices only

  \onslide<5-> \begin{block}{Evaluation}
	Evaluation will only be on submitted exercices
  \end{block}

\end{frame}


%%% Language, CLI, versions, ...
% -*- latex -*-

\part{Python Basics}  % Language, CLI, versions, ...

%%%%%%%%%%%%%%%%%%%%%%%%%%%%%%%%%%%%%%%%%%%%%%%%%%%%%%%%

\section{Python Language}

\begin{frame}<beamer>{Discovering Python}

  \begin{itemize}
    \item<1-> Interpreter~\\
      \only<2->{\TTBF{/usr/bin/python}}
    \item<3-> Imperative programming language
    \item<4-> Object Oriented Programming language
    \item<5-> Multiple versions (currently major version 3)~\\
      \only<6->{Beware, version \alert{2.7} is still running on some machines}
  \end{itemize}

\end{frame}


%%% Terminal example

\begin{frame}[fragile]{Calling Python}

  \onslide<1-> Simply like any program

  \begin{columns}[onlytextwidth]
    \begin{column}{\textwidth}
      \begin{onlyenv}<1>
        \begin{lstlisting}[style=sh]
%*\LSTPrompt*)




 \end{lstlisting}
      \end{onlyenv}

      \begin{onlyenv}<2>
        \begin{lstlisting}[style=sh]
%*\LSTPrompt*) python




 \end{lstlisting}
      \end{onlyenv}

      \begin{onlyenv}<3>
        \begin{lstlisting}[style=sh]
%*\LSTPrompt*) python
Python 3.10.4 (main, Apr  2 2022, 09:04:19) [GCC 11.2.0] on linux
Type "help", "copyright", "credits" or "license" for more information.
>>> \end{lstlisting}
      \end{onlyenv}
    \end{column}
  \end{columns}
\end{frame}


%%% Quit terminal example

\begin{frame}[fragile]{Calling Python}

  \onslide<1-> Quit by typing "\TTBF{exit()}" or by pressing \TTBF{Ctrl + D}

  \begin{columns}[onlytextwidth]
    \begin{column}{\textwidth}
      \begin{onlyenv}<1>
        \begin{lstlisting}[style=sh]
%*\LSTPrompt*) python
Python 3.10.4 (main, Apr  2 2022, 09:04:19) [GCC 11.2.0] on linux
Type "help", "copyright", "credits" or "license" for more information.
>>> exit() \end{lstlisting}
      \end{onlyenv}
    \end{column}
  \end{columns}
\end{frame}


%%% Script call example

\begin{frame}[fragile]{Calling Python}

  \onslide<1-> You can call a script by calling python with an argument

  \begin{columns}[onlytextwidth]
    \begin{column}{\textwidth}
      \begin{onlyenv}<1>
        \begin{lstlisting}[style=sh]
%*\LSTPrompt*) python my_script.py
 \end{lstlisting}
      \end{onlyenv}

      \begin{onlyenv}<2>
        \begin{lstlisting}[style=sh]
%*\LSTPrompt*) python my_script.py
Hello World! \end{lstlisting}
      \end{onlyenv}

    \end{column}
  \end{columns}
\end{frame}


%%% Version

%\begin{frame}[fragile]{Versions}
%
%  \onslide<1> Beware of versions !
%
%  \onslide<1> \phantom{We currently use the major version 3 (3.8, 3.10, ...)}
%
%  \begin{columns}[onlytextwidth]
%    \begin{column}{\textwidth}
%      \begin{onlyenv}<1>
%        \begin{lstlisting}[style=sh,morekeywords={3.10.4, python, python3, python3.10}]]
%
%
%
%
%        \end{lstlisting}
%      \end{onlyenv}
%
%    \onslide<1> \phantom{Avoid at all cost the obsolete 2.7 (\TTBF{python27})}
%
%    \end{column}
%  \end{columns}
%\end{frame}

\begin{frame}[fragile]{Versions}

  \onslide<1-> Beware of versions !

  \onslide<1-> We currently use the major version 3 (3.8, 3.10, ...)

  \begin{columns}[onlytextwidth]
    \begin{column}{\textwidth}

      \begin{onlyenv}<1>
        \begin{lstlisting}[style=sh]
%*\LSTPrompt*) python




 \end{lstlisting}
      \end{onlyenv}

      \begin{onlyenv}<2->
        \begin{lstlisting}[style=sh]
%*\LSTPrompt*) python
Python 3.10.4 (main, Apr  2 2022, 09:04:19) [GCC 11.2.0] on linux
Type "help", "copyright", "credits" or "license" for more information.
>>> exit() \end{lstlisting}
      \end{onlyenv}

    \onslide<3-> Do not use the obsolete 2.7 (\TTBF{python2}, \TTBF{python27}, \TTBF{python2.7})

    \end{column}
  \end{columns}
\end{frame}


\begin{frame}[fragile]{Versions}

  \begin{columns}[onlytextwidth]
    \begin{column}{\textwidth}

      \begin{onlyenv}<1>  %,morekeywords={3.10.4, python3.10, python3, python}]
        \begin{lstlisting}[style=sh]
%*\LSTPrompt*) python
Python 3.10.4 (main, Apr  2 2022, 09:04:19)

%*\LSTPrompt*) python3
Python 3.10.4 (main, Apr  2 2022, 09:04:19)

%*\LSTPrompt*) python3.10
Python 3.10.4 (main, Apr  2 2022, 09:04:19)


 \end{lstlisting}
      \end{onlyenv}

      \begin{onlyenv}<2>  %,morekeywords={3.10.4, python3.10, python3, python}]
        \begin{lstlisting}[style=sh]
%*\LSTPrompt*) python
Python 3.10.4 (main, Apr  2 2022, 09:04:19)

%*\LSTPrompt*) python3
Python 3.10.4 (main, Apr  2 2022, 09:04:19)

%*\LSTPrompt*) python3.10
Python 3.10.4 (main, Apr  2 2022, 09:04:19)

%*\LSTPrompt*) python2
Python 2.7.18 (default, Jan  2 2021, 09:22:32) \end{lstlisting}
      \end{onlyenv}

    \end{column}
  \end{columns}
\end{frame}


\begin{frame}[fragile]{Versions}

  \begin{center}

  \centering{
    Please, check right now which default version of Python you have when you type \TTBF{python}
  }

  \end{center}

\end{frame}


%%% Hello world, print, str/int, Sys module
% -*- latex -*-

\part{Python Quick Scripting}  % Hello world, print, str/int, Sys module

%%%%%%%%%%%%%%%%%%%%%%%%%%%%%%%%%%%%%%%%%%%%%%%%%%%%%%%%

\section{First scripts}

%%% Hello World
\pdfbookmark[3]{Hello World!}{ref-30-1-1-HelloWorld}

\begin{frame}[fragile]{Hello World!}

%  \onslide<1-> Your first python code:

  \begin{columns}[onlytextwidth]
    \begin{column}{\textwidth}
      \begin{onlyenv}<1->
        %\begin{lstlisting}[style=python,numbers=left,stepnumber=1]
        \begin{lstlisting}[style=python]
#! /usr/bin/python

print("Hello World!") \end{lstlisting}
      \end{onlyenv}
    \end{column}
  \end{columns}

  \onslide<2-> L1: The interpreter can be ajdusted

  \onslide<3-> L3: \TTBF{print} takes a string as a parameter

  \onslide<4-> L3: Strings are enclosed within quotes (\TTBF{\textquotesingle}) or double quotes (\TTBF{\textquotedbl})

\end{frame}

%%% Declaration, strings & integers
\pdfbookmark[3]{Concatenation \& Print}{ref-30-1-2-ConcatenationPrint}

\begin{frame}[fragile]{Concatenation \& Printing integers}

  \begin{columns}[onlytextwidth]
    \begin{column}{\textwidth}
      \begin{onlyenv}<1->
        \begin{lstlisting}[style=python]
#! /usr/bin/python

var = 42
print("Hello World! " + str(var)) \end{lstlisting}
      \end{onlyenv}
    \end{column}
  \end{columns}

  \onslide<2-> Semicolons (\TTBF{;}) at the end of statements are optional (rarely used)

  \onslide<3-> L3: Declare variables and assign value directly (type not needed)

  \onslide<4-> L4: String concatenation with \TTBF{+}

  \onslide<5-> L4: Conversion from integer or float to string with \TTBF{str()} function

\end{frame}

%%% Read arguments
\pdfbookmark[3]{Arguments (sys.argv)}{ref-30-1-3-Args}

\begin{frame}[fragile]{Read arguments}

  \begin{columns}[onlytextwidth]
    \begin{column}{\textwidth}
      \begin{onlyenv}<1->
        \begin{lstlisting}[style=python,title={args.py}]
#! /usr/bin/python

import sys
print("Hello World! " + sys.argv[0]) \end{lstlisting}
      \end{onlyenv}
    \end{column}
  \end{columns}

  \begin{columns}[onlytextwidth]
    \begin{column}{\textwidth}
      \begin{onlyenv}<1>
        \begin{lstlisting}[style=sh]

 \end{lstlisting}
      \end{onlyenv}

      \begin{onlyenv}<2>
        \begin{lstlisting}[style=sh]
%*\LSTPrompt*) python args.py
 \end{lstlisting}
      \end{onlyenv}

      \begin{onlyenv}<3->
        \begin{lstlisting}[style=sh]
%*\LSTPrompt*) python args.py
Hello World! args.py \end{lstlisting}
      \end{onlyenv}

    \end{column}
  \end{columns}

\end{frame}

\begin{frame}[fragile]{Read arguments}

  \begin{columns}[onlytextwidth]
    \begin{column}{\textwidth}
      \begin{onlyenv}<1->
        \begin{lstlisting}[style=python]
#! /usr/bin/python

import sys
print("Hello World! " + sys.argv[0]) \end{lstlisting}
      \end{onlyenv}
    \end{column}
  \end{columns}

  \onslide<2-> L3: Includes an external module (\TTBF{import module})

  \onslide<3-> L3: Access to arguments is made through \TTBF{sys} module

  \onslide<3-> L4: Access to an attribute is made with a dot (\TTBF{.})

  \onslide<4-> L4: \TTBF{argv} is an array (like in C and others)

  \onslide<5-> L4: Arrays begin at index \TTBF{0}

\end{frame}

%%%%%%%%%%%%%%%%%%

%%% Show types
\section{Quick overview of types}

\pdfbookmark[3]{type() \& debug}{ref-30-2-1-TypesDebug}

\begin{frame}[fragile]{Show types/Quick debug}

  \begin{columns}[onlytextwidth]
    \begin{column}{\textwidth}
      \begin{onlyenv}<1->
        \begin{lstlisting}[style=python,title={types1.py}]
#! /usr/bin/python
type(42)
type("Hello World!") \end{lstlisting}
      \end{onlyenv}
    \end{column}
  \end{columns}

    \begin{columns}[onlytextwidth]
    \begin{column}{\textwidth}
      \begin{onlyenv}<1>
        \begin{lstlisting}[style=sh]
%*\LSTPrompt*) python types1.py

       \end{lstlisting}
      \end{onlyenv}

      \begin{onlyenv}<2->
        \begin{lstlisting}[style=sh]
%*\LSTPrompt*) python types1.py
<class 'int'>
<class 'str'> \end{lstlisting}
      \end{onlyenv}
    \end{column}
  \end{columns}

  \onslide<3-> Function \TTBF{type()} writes the type of the parameter
\end{frame}


\begin{frame}[fragile]{Show types/Quick debug}

  \begin{columns}[onlytextwidth]
    \begin{column}{\textwidth}
      \begin{onlyenv}<1->
        \begin{lstlisting}[style=python,title={types2.py}]
#! /usr/bin/python
import sys
type(sys)
type(print("lol")) \end{lstlisting}
      \end{onlyenv}
    \end{column}
  \end{columns}

    \begin{columns}[onlytextwidth]
    \begin{column}{\textwidth}
      \begin{onlyenv}<1>
        \begin{lstlisting}[style=sh]
%*\LSTPrompt*) python types2.py


        \end{lstlisting}
      \end{onlyenv}

      \begin{onlyenv}<2->
        \begin{lstlisting}[style=sh]
%*\LSTPrompt*) python types2.py
<class 'module'>
lol
<class 'NoneType'> \end{lstlisting}
      \end{onlyenv}
    \end{column}
  \end{columns}
\end{frame}


%%% Syntax and control flow
\section{Overview of syntax and control flow}

\pdfbookmark[3]{Functions}{ref-30-3-1-Functions}

\begin{frame}[fragile]{Functions overview}

  \begin{columns}[onlytextwidth]
    \begin{column}{\textwidth}
      \begin{onlyenv}<1>
        \begin{lstlisting}[style=python,title={functions1.py}]
print("Hello World!")



 \end{lstlisting}
      \end{onlyenv}

      \begin{onlyenv}<2>
        \begin{lstlisting}[style=python,title={functions1.py}]
def MyFunction():
  print("Hello World!")
  return (0)

 \end{lstlisting}
      \end{onlyenv}

      \begin{onlyenv}<3->
        \begin{lstlisting}[style=python,title={functions1.py}]
def MyFunction():
  print("Hello World!")
  return (0)

MyFunction() \end{lstlisting}
      \end{onlyenv}
    \end{column}
  \end{columns}

    \begin{columns}[onlytextwidth]
    \begin{column}{\textwidth}
      \begin{onlyenv}<4>
        \begin{lstlisting}[style=sh]
%*\LSTPrompt*) python functions1.py
 \end{lstlisting}
      \end{onlyenv}

      \begin{onlyenv}<5->
        \begin{lstlisting}[style=sh]
%*\LSTPrompt*) python functions1.py
Hello World! \end{lstlisting}
      \end{onlyenv}
    \end{column}
  \end{columns}
\end{frame}


%%% Functions

\begin{frame}[fragile]{Functions overview}

%  Functions

  \begin{columns}[onlytextwidth]
    \begin{column}{\textwidth}
      \begin{onlyenv}<1->
        \begin{lstlisting}[style=python]
def MyFunction():
  print("Hello World!")
  return (0)

MyFunction() \end{lstlisting}
      \end{onlyenv}
    \end{column}
  \end{columns}

  \onslide<2-> L1: Functions begin by a \TTBF{def} and are followed by their parameters

  \onslide<3-> L1: Definition of functions are terminated by a semicolon (\TTBF{:})

  \onslide<4-> L2: Indentation defines in which scope the line is

  \onslide<5-> L2: \textbf{Indentation is VERY IMPORTANT!}

  \onslide<6-> L3 \& L5: Returning and calling are similar to C

\end{frame}


%%% Conditions

\pdfbookmark[3]{Conditions (if elif else)}{ref-30-3-2-Conditions1}

\begin{frame}[fragile]{Conditions overview}

%  Conditions
  \TTBF{if} ... \TTBF{elif} ... \TTBF{else} ...

  \begin{columns}[onlytextwidth]
    \begin{column}{\textwidth}
      \begin{onlyenv}<1->
        \begin{lstlisting}[style=python]
def MyOtherFunction():
  var = 42
  if (var < 42):
    print("Oh no...")
  elif (var > 42):
    print("WT...")
  else:
    print("OK") \end{lstlisting}
      \end{onlyenv}
    \end{column}
  \end{columns}

%  \onslide<2-> L3 \& L5 \& L6: \TTBF{if} ... \TTBF{elif} ... \TTBF{else} ...
%
%  \onslide<3-> L3 \& L5 \& L6: Conditions are followed by a semicolon (\TTBF{:})
  \onslide<2-> L3 \& L5 \& L6: Conditions are followed by a semicolon (\TTBF{:})

\end{frame}

\pdfbookmark[3]{Conditions (match case)}{ref-30-3-3-Conditions2}

\begin{frame}[fragile]{Conditions overview}

%  Conditions \hspace{3cm} \textbf{(only since Python 3.10)}
  \TTBF{match} ... \TTBF{case} ...  \hspace{3cm}  \textit{(only since Python 3.10)}

  \begin{columns}[onlytextwidth]
    \begin{column}{\textwidth}
      \begin{onlyenv}<1->
        \begin{lstlisting}[style=python]
var = "Hello World!"
match var:
  case ['Hello']:
    print("Beginning")
  case ['World!' | 'Hello World!']:
    print("End")
  case _:
    print("In any other cases") \end{lstlisting}
      \end{onlyenv}
    \end{column}
  \end{columns}

  \onslide<2-> Similar to \textit{switch-case} in C, without needing \textit{return} or \textit{break}

  \onslide<3-> More complex than that (check the documentation after the course)

\end{frame}


%%% Loops

\pdfbookmark[3]{Loops (while)}{ref-30-3-4-Loops1}

\begin{frame}[fragile]{Loops overview}

%  Loops (while)
  \TTBF{while}

  \begin{columns}[onlytextwidth]
    \begin{column}{\textwidth}
      \begin{onlyenv}<1->
        \begin{lstlisting}[style=python]
var = 0
while var < 42:
  print("Hi Nations!")
  var += 1 \end{lstlisting}
      \end{onlyenv}
    \end{column}
  \end{columns}

  \onslide<2-> L2: Regular \TTBF{while} loop

  \onslide<2-> L4: \TTBF{+=} operator acts like \TTBF{var = var + 1}

  \onslide<3-> Don't forget indentation

\end{frame}


\pdfbookmark[3]{Loops (for)}{ref-30-3-5-Loops2}

\begin{frame}[fragile]{Loops overview}

%  Loops (for)
  \TTBF{for} (1)

  \begin{columns}[onlytextwidth]
    \begin{column}{\textwidth}
      \begin{onlyenv}<1->
        \begin{lstlisting}[style=python]
for var in range(0, 10):
  print("Hi Nations!")

my_text = "Yo Countries!"
for char in my_text:
  print(char) \end{lstlisting}
      \end{onlyenv}
    \end{column}
  \end{columns}

  \onslide<2-> L1: \TTBF{range} calculates values from \TTBF{0} to \TTBF{10}

  \onslide<3-> L1: \TTBF{in} iterates through each value of a list

  \onslide<3-> L1: Each value will be put into the variable before \TTBF{in}

  \onslide<4-> L4 \& L5: Strings are considered as characters lists

\end{frame}


\begin{frame}[fragile]{Loops overview}

  \TTBF{for} (2)

  \begin{columns}[onlytextwidth]
    \begin{column}{\textwidth}
      \begin{onlyenv}<1->
        \begin{lstlisting}[style=python]
my_list = [ 5, 2, 3, 1, 4 ]
for var in range(len(my_list)):
  if (var < 2):
    break
  else:
    print("Hi Nations!") \end{lstlisting}
      \end{onlyenv}
    \end{column}
  \end{columns}

  \onslide<2-> L1: Declaration of a list

  \onslide<3-> L1: Never put a dash (\TTBF{-}) in variables name (use an underscore \TTBF{\_})

  \onslide<4-> L2: \TTBF{len} gets the length of a list

  \onslide<5-> L4: \TTBF{break} ends the loop

\end{frame}


%%% Exceptions
\pdfbookmark[3]{Exceptions (try except)}{ref-30-3-5-Exceptions1}

\begin{frame}[fragile]{Exceptions overview}

  \TTBF{try} ... \TTBF{except} ...

  \begin{columns}[onlytextwidth]
    \begin{column}{\textwidth}
      \begin{onlyenv}<1->
        \begin{lstlisting}[style=python,morekeywords={as}]
var = 3
try:
  var = 42 / var
except Exception as exc:
  print("Error: " + str(exc)) \end{lstlisting}
      \end{onlyenv}
    \end{column}
  \end{columns}

  \onslide<2-> L2: Begin the block of code to check with a \TTBF{try}

  \onslide<3-> L4: Catch exceptions with a \TTBF{except}

  \onslide<4-> L4: Catch any exception with \TTBF{Exception} and put it in \TTBF{exc}

  \onslide<5-> L4 \& L5: Write the actions to take if an exception occurs

  \onslide<6-> L5: Don't forget to convert to a string with \TTBF{str()}

\end{frame}


\begin{frame}[fragile]{Exceptions overview}

  \TTBF{try} ... \TTBF{except} ... \TTBF{else} ... \TTBF{finally}

  \begin{columns}[onlytextwidth]
    \begin{column}{\textwidth}
      \begin{onlyenv}<1->
        \begin{lstlisting}[style=python,morekeywords={as}]
try:
  var = 42 / 3
except ZeroDivisionError:
  print("There was a division by zero")
else:
  print("It worked, result: ", var)
finally:
  print("--After everything--")\end{lstlisting}
      \end{onlyenv}
    \end{column}
  \end{columns}

  \onslide<2-> L2: Check for a specific exception (division by zero)

  \onslide<3-> L4: If no exception was raised, it executes the \TTBF{else}

  \onslide<4-> L6: In any case (exception or not), it executes the \TTBF{finally} clause

\end{frame}


\pdfbookmark[3]{Exceptions (raise)}{ref-30-3-6-Exceptions2}

\begin{frame}[fragile]{}

  \TTBF{raise}

  \begin{columns}[onlytextwidth]
    \begin{column}{\textwidth}

      \begin{onlyenv}<1>
        \begin{lstlisting}[style=python,morekeywords={as},basicstyle=\ttfamily\footnotesize]
def FragileFunction():
  var = 42
  if (var == 42):
    raise ValueError("Argf")
  else:
    return (0)







 \end{lstlisting}
      \end{onlyenv}

      \begin{onlyenv}<2->
        \begin{lstlisting}[style=python,morekeywords={as},basicstyle=\ttfamily\footnotesize]
def FragileFunction():
  var = 42
  if (var == 42):
    raise ValueError("Argf")
  else:
    return (0)

def MainFunction():
  print("--Before--")
  try:
    FragileFunction()
  except ValueError as exp:
    print("Exception caught: ", exp)
  print("--After--") \end{lstlisting}
      \end{onlyenv}

    \end{column}
  \end{columns}

\end{frame}


\pdfbookmark[3]{Exceptions (custom exceptions)}{ref-30-3-7-Exceptions3}

\begin{frame}[fragile]{Exceptions overview}

  \TTBF{raise} and custom exceptions

  \begin{itemize}
    \item<2-> \TTBF{raise} instruction triggers an exception
    \item<3-> Use \TTBF{ValueError} in order to customize the message...
    \item<4-> ...or create a class with the name of your exception and \TTBF{Exception} as a parameter
    \item<5-> When raising a custom exception, the message is optional
  \end{itemize}

  \begin{columns}[onlytextwidth]
    \begin{column}{\textwidth}

      \begin{onlyenv}<6>
        \begin{lstlisting}[style=python,morekeywords={as}]




 \end{lstlisting}
      \end{onlyenv}

      \begin{onlyenv}<7>
        \begin{lstlisting}[style=python,morekeywords={as}]
class MyCustomError(Exception):
  pass


 \end{lstlisting}
      \end{onlyenv}

      \begin{onlyenv}<8->
        \begin{lstlisting}[style=python,morekeywords={as}]
class MyCustomError(Exception):
  pass

def FragileFunction():
  raise MyCustomError \end{lstlisting}
      \end{onlyenv}

    \end{column}
  \end{columns}

\end{frame}



\begin{frame}[fragile]{}

  \TTBF{raise} and custom exceptions

    \begin{columns}[onlytextwidth]
    \begin{column}{\textwidth}

      \begin{onlyenv}<1>
        \begin{lstlisting}[style=python,morekeywords={as},basicstyle=\ttfamily\footnotesize]












 \end{lstlisting}
      \end{onlyenv}

      \begin{onlyenv}<2>
        \begin{lstlisting}[style=python,morekeywords={as},basicstyle=\ttfamily\footnotesize]
class MyCustomError(Exception):
  pass










 \end{lstlisting}
      \end{onlyenv}

      \begin{onlyenv}<3>
        \begin{lstlisting}[style=python,morekeywords={as},basicstyle=\ttfamily\footnotesize]
class MyCustomError(Exception):
  pass

def FragileFunction():
  raise MyCustomError("Argf")







 \end{lstlisting}
      \end{onlyenv}

      \begin{onlyenv}<4->
        \begin{lstlisting}[style=python,morekeywords={as},basicstyle=\ttfamily\footnotesize]
class MyCustomError(Exception):
  pass

def FragileFunction():
  raise MyCustomError("Argf")

def MainFunction():
  print("--Before--")
  try:
    FragileFunction()
  except MyCustomError as exp:
    print("Exception caught: ", exp)
  print("--After--") \end{lstlisting}
      \end{onlyenv}

    \end{column}
  \end{columns}

\end{frame}

%%% Declaring functions, imports, modules
% -*- latex -*-

\part{Functions, Imports \& Modules}  % Declaring functions, imports, modules

%%%%%%%%%%%%%%%%%%%%%%%%%%%%%%%%%%%%%%%%%%%%%%%%%%%%%%%%

\section{Functions}

\begin{frame}[fragile]{Functions}

  Usual function definition and call:

  \begin{columns}[onlytextwidth]
    \begin{column}{\textwidth}

      \begin{onlyenv}<1>
        % Balises exception :  %* *)
        \begin{lstlisting}[style=python]





 \end{lstlisting}
      \end{onlyenv}

      \begin{onlyenv}<2>
        \begin{lstlisting}[style=python]
def MyFunction(name):
  print("Hello " + str(name) + "!")
  return (0)


 \end{lstlisting}
      \end{onlyenv}

      \begin{onlyenv}<3>
        \begin{lstlisting}[style=python]
def MyFunction(name):
  print("Hello " + str(name) + "!")
  return (0)

MyFunction("Fabrice")
 \end{lstlisting}
      \end{onlyenv}

      \begin{onlyenv}<4->
        \begin{lstlisting}[style=python]
def MyFunction(name):
  print("Hello " + str(name) + "!")
  return (0)

MyFunction("Fabrice")
MyFunction(42) \end{lstlisting}
      \end{onlyenv}

    \end{column}
  \end{columns}

  \onslide<5-> \textit{MyFunction} is the function name

  \onslide<6-> \textit{name} is a parameter (without a type)

  \onslide<7-> \textit{"Fabrice"} and \textit{42} are arguments

\end{frame}


%%% Parameters
\subsection{Parameters}

\begin{frame}<beamer>{Parameters}

  \begin{itemize}
    \item<1-> Positional arguments
      \begin{itemize}
      \item<2-> \TTBF{def MyFun(param1, param2, param3)}
      \item<3-> \TTBF{MyFun("abc", 42, 1337)}
      \item<4-> \only<4>{\TTBF{MyFun(42, "abc", 1337)}} \only<5->{\TTBF{MyFun(\sout{42}, \sout{"abc"}, 1337)}}
      \end{itemize}
    \item<6-> \only<6>{Keywords arguments} \only<7->{Keywords arguments (with default values)}
      \begin{itemize}
      \item<7-> \TTBF{def MyFun(p1="A", p2=1, p3=9)}
      \item<8-> \TTBF{MyFun("abc", 42, 1337)}
      \item<9-> \TTBF{MyFun(p1="abc", p2=42, p3=1337)}
      \item<10-> \TTBF{MyFun(p3=1337, p2=42, p1="abc")}
      \item<11-> \TTBF{MyFun(p2=42)}
      \end{itemize}
  \end{itemize}

\end{frame}

%%% Positional arguments

\begin{frame}[fragile]{Positional arguments}

  \begin{columns}[onlytextwidth]
    \begin{column}{\textwidth}

      \begin{onlyenv}<1>
        \begin{lstlisting}[style=python]










 \end{lstlisting}
      \end{onlyenv}

      \begin{onlyenv}<2>
        \begin{lstlisting}[style=python]
from datetime import date
def GreetingsPos(Name, Year):
  print("Hi " + Name + "!")
  if (Year <= 0):
    print("(unknown age)")
  else:
    age = date.today().year - Year
    print("(" + str(age) + " years)")


 \end{lstlisting}
      \end{onlyenv}

      \begin{onlyenv}<3>
        \begin{lstlisting}[style=python]
from datetime import date
def GreetingsPos(Name, Year):
  print("Hi " + Name + "!")
  if (Year <= 0):
    print("(unknown age)")
  else:
    age = date.today().year - Year
    print("(" + str(age) + " years)")

GreetingsPos("Fabrice", 1988)
 \end{lstlisting}
      \end{onlyenv}

      \begin{onlyenv}<4>
        \begin{lstlisting}[style=python]
from datetime import date
def GreetingsPos(Name, Year):
  print("Hi " + Name + "!")
  if (Year <= 0):
    print("(unknown age)")
  else:
    age = date.today().year - Year
    print("(" + str(age) + " years)")

GreetingsPos("Fabrice", 1988) # OK
GreetingsPos("Fabrice") \end{lstlisting}
      \end{onlyenv}

      \begin{onlyenv}<5->
        \begin{lstlisting}[style=python]
from datetime import date
def GreetingsPos(Name, Year):
  print("Hi " + Name + "!")
  if (Year <= 0):
    print("(unknown age)")
  else:
    age = date.today().year - Year
    print("(" + str(age) + " years)")

GreetingsPos("Fabrice", 1988) # OK
GreetingsPos("Fabrice")       # error \end{lstlisting}
      \end{onlyenv}

    \end{column}
  \end{columns}

  \onslide<6-> All of the parameters are required if positional arguments

\end{frame}


%%% Keywords arguments / Default parameter values

\begin{frame}[fragile]{Keywords arguments}

  \begin{columns}[onlytextwidth]
    \begin{column}{\textwidth}

      \begin{onlyenv}<1>
        \begin{lstlisting}[style=python]






 \end{lstlisting}
      \end{onlyenv}

      \begin{onlyenv}<2>
        \begin{lstlisting}[style=python]
def GreetingsKey(FName="Unknown", BYear=0):
  GreetingsPos(FName, BYear)




 \end{lstlisting}
      \end{onlyenv}

      \begin{onlyenv}<3>
        \begin{lstlisting}[style=python]
def GreetingsKey(FName="Unknown", BYear=0):
  GreetingsPos(FName, BYear)

GreetingsKey("Fabrice", 1988)


 \end{lstlisting}
      \end{onlyenv}

      \begin{onlyenv}<4>
        \begin{lstlisting}[style=python]
def GreetingsKey(FName="Unknown", BYear=0):
  GreetingsPos(FName, BYear)

GreetingsKey("Fabrice", 1988)  # OK
GreetingsKey("Fabrice")

 \end{lstlisting}
      \end{onlyenv}

      \begin{onlyenv}<5>
        \begin{lstlisting}[style=python]
def GreetingsKey(FName="Unknown", BYear=0):
  GreetingsPos(FName, BYear)

GreetingsKey("Fabrice", 1988)  # OK
GreetingsKey("Fabrice")        # OK
GreetingsKey(BYear=1988, FName="Fabrice")
 \end{lstlisting}
      \end{onlyenv}

      \begin{onlyenv}<6>
        \begin{lstlisting}[style=python]
def GreetingsKey(FName="Unknown", BYear=0):
  GreetingsPos(FName, BYear)

GreetingsKey("Fabrice", 1988)  # OK
GreetingsKey("Fabrice")        # OK
GreetingsKey(BYear=1988, FName="Fabrice") # OK
GreetingsKey("Fabrice", BYear=1988) \end{lstlisting}
      \end{onlyenv}

      \begin{onlyenv}<7->
        \begin{lstlisting}[style=python]
def GreetingsKey(FName="Unknown", BYear=0):
  GreetingsPos(FName, BYear)

GreetingsKey("Fabrice", 1988)  # OK
GreetingsKey("Fabrice")        # OK
GreetingsKey(BYear=1988, FName="Fabrice") # OK
GreetingsKey("Fabrice", BYear=1988) # OK \end{lstlisting}
      \end{onlyenv}

    \end{column}
  \end{columns}

  \onslide<8-> Keywords arguments are more flexibles

\end{frame}


\begin{frame}[fragile]{Keywords arguments}

  \begin{columns}[onlytextwidth]
    \begin{column}{\textwidth}

      \begin{onlyenv}<1>
        \begin{lstlisting}[style=python]







 \end{lstlisting}
      \end{onlyenv}

      \begin{onlyenv}<2>
        \begin{lstlisting}[style=python]
def GreetingsKey2(FName="", BYear=""):
  GreetingsPos(FName, BYear)





 \end{lstlisting}
      \end{onlyenv}

      \begin{onlyenv}<3>
        \begin{lstlisting}[style=python]
def GreetingsKey2(FName="", BYear=""):
  GreetingsPos(FName, BYear)

GreetingsKey2("Fabrice", 1988)



 \end{lstlisting}
      \end{onlyenv}

      \begin{onlyenv}<4>
        \begin{lstlisting}[style=python]
def GreetingsKey2(FName="", BYear=""):
  GreetingsPos(FName, BYear)

GreetingsKey2("Fabrice", 1988)  # OK
GreetingsKey2(BYear=1988, FName="Fabrice")


 \end{lstlisting}
      \end{onlyenv}

      \begin{onlyenv}<5>
        \begin{lstlisting}[style=python]
def GreetingsKey2(FName="", BYear=""):
  GreetingsPos(FName, BYear)

GreetingsKey2("Fabrice", 1988)  # OK
GreetingsKey2(BYear=1988, FName="Fabrice") # OK
GreetingsKey2(BYear=1988)

 \end{lstlisting}
      \end{onlyenv}

      \begin{onlyenv}<6>
        \begin{lstlisting}[style=python]
def GreetingsKey2(FName="", BYear=""):
  GreetingsPos(FName, BYear)

GreetingsKey2("Fabrice", 1988)  # OK
GreetingsKey2(BYear=1988, FName="Fabrice") # OK
GreetingsKey2(BYear=1988)  # OK
GreetingsKey2()
 \end{lstlisting}
      \end{onlyenv}

      \begin{onlyenv}<7>
        \begin{lstlisting}[style=python]
def GreetingsKey2(FName="", BYear=""):
  GreetingsPos(FName, BYear)

GreetingsKey2("Fabrice", 1988)  # OK
GreetingsKey2(BYear=1988, FName="Fabrice") # OK
GreetingsKey2(BYear=1988)  # OK
GreetingsKey2() # error
GreetingsKey2("Fabrice") \end{lstlisting}
      \end{onlyenv}

      \begin{onlyenv}<8->
        \begin{lstlisting}[style=python]
def GreetingsKey2(FName="", BYear=""):
  GreetingsPos(FName, BYear)

GreetingsKey2("Fabrice", 1988)  # OK
GreetingsKey2(BYear=1988, FName="Fabrice") # OK
GreetingsKey2(BYear=1988)  # OK
GreetingsKey2() # error
GreetingsKey2("Fabrice") # error \end{lstlisting}
      \end{onlyenv}

    \end{column}
  \end{columns}

  \onslide<9-> Keywords arguments forces a default value

\end{frame}


%%% Scope of variables
\subsection{Scope of variables}

\begin{frame}<beamer>{Scope of variables}

  Variables are searched within scopes in a specific order:

  \begin{itemize}
    \item<1-> Local (scope of the current function)
    \item<2-> Global (global variables of the program)
    \item<3-> Internal (variables of the interpreter)
  \end{itemize}

  \bigskip

  \onslide<4-> LGI rule

  \medskip

  \onslide<5-> Constants are global variables (uppercase name)

\end{frame}

%%% Return values
\subsection{Return values}

%%% Multiple return of values in functions (return a tuple)

\begin{frame}[fragile]{Multiple return of values}

  \begin{columns}[onlytextwidth]
    \begin{column}{\textwidth}

      \begin{onlyenv}<1>
        \begin{lstlisting}[style=python]










 \end{lstlisting}
      \end{onlyenv}

      \begin{onlyenv}<2>
        \begin{lstlisting}[style=python]
def ConvertTemperature(kelvin):
  celsius = kelvin - 273
  fahrenheit = ((celsius * 9) / 5) + 32
  reaumur = (kelvin * 4) / 5
  return celsius, fahrenheit, reaumur





 \end{lstlisting}
      \end{onlyenv}

      \begin{onlyenv}<3>
        \begin{lstlisting}[style=python]
def ConvertTemperature(kelvin):
  celsius = kelvin - 273
  fahrenheit = ((celsius * 9) / 5) + 32
  reaumur = (kelvin * 4) / 5
  return celsius, fahrenheit, reaumur

temps = ConvertTemperature(42)



 \end{lstlisting}
      \end{onlyenv}

      \begin{onlyenv}<4>
        % Balises exception :  %* *)
        \begin{lstlisting}[style=python]
def ConvertTemperature(kelvin):
  celsius = kelvin - 273
  fahrenheit = ((celsius * 9) / 5) + 32
  reaumur = (kelvin * 4) / 5
  return celsius, fahrenheit, reaumur

temps = ConvertTemperature(42)
print("%*\textdegree*)K : " + str(42))
print("%*\textdegree*)C : " + str(temps[0]))
print("%*\textdegree*)F : " + str(temps[1]))
print("%*\textdegree*)Re: " + str(temps[2])) \end{lstlisting}
      \end{onlyenv}

      \begin{onlyenv}<5->
        \begin{lstlisting}[style=python]
def ConvertTemperature(kelvin):
  celsius = kelvin - 273
  fahrenheit = ((celsius * 9) / 5) + 32
  reaumur = (kelvin * 4) / 5
  return celsius, fahrenheit, reaumur

temps = ConvertTemperature(42)
print("%*\textdegree*)K : " + str(42))       # 42
print("%*\textdegree*)C : " + str(temps[0])) # -231
print("%*\textdegree*)F : " + str(temps[1])) # -383
print("%*\textdegree*)Re: " + str(temps[2])) # 33 \end{lstlisting}
      \end{onlyenv}

    \end{column}
  \end{columns}

  \onslide<6-> Multiple return values uses tuples

\end{frame}


\begin{frame}[fragile]{Multiple return of values}

  Multiple affectations are also possible

  \begin{columns}[onlytextwidth]
    \begin{column}{\textwidth}

      \begin{onlyenv}<1>
        \begin{lstlisting}[style=python]










 \end{lstlisting}
      \end{onlyenv}

      \begin{onlyenv}<2>
        \begin{lstlisting}[style=python]
def ConvertTemperature(kelvin):
  celsius = kelvin - 273
  fahrenheit = ((celsius * 9) / 5) + 32
  reaumur = (kelvin * 4) / 5
  return celsius, fahrenheit, reaumur





 \end{lstlisting}
      \end{onlyenv}

      \begin{onlyenv}<3>
        \begin{lstlisting}[style=python]
def ConvertTemperature(kelvin):
  celsius = kelvin - 273
  fahrenheit = ((celsius * 9) / 5) + 32
  reaumur = (kelvin * 4) / 5
  return celsius, fahrenheit, reaumur

tC, tF, tR = ConvertTemperature(42)



 \end{lstlisting}
      \end{onlyenv}

      \begin{onlyenv}<4>
        \begin{lstlisting}[style=python]
def ConvertTemperature(kelvin):
  celsius = kelvin - 273
  fahrenheit = ((celsius * 9) / 5) + 32
  reaumur = (kelvin * 4) / 5
  return celsius, fahrenheit, reaumur

tC, tF, tR = ConvertTemperature(42)
print("%*\textdegree*)K : " + str(42))
print("%*\textdegree*)C : " + str(tC))
print("%*\textdegree*)F : " + str(tF))
print("%*\textdegree*)Re: " + str(tR)) \end{lstlisting}
      \end{onlyenv}

      \begin{onlyenv}<5->
        \begin{lstlisting}[style=python]
def ConvertTemperature(kelvin):
  celsius = kelvin - 273
  fahrenheit = ((celsius * 9) / 5) + 32
  reaumur = (kelvin * 4) / 5
  return celsius, fahrenheit, reaumur

tC, tF, tR = ConvertTemperature(42)
print("%*\textdegree*)K : " + str(42)) # 42
print("%*\textdegree*)C : " + str(tC)) # -231
print("%*\textdegree*)F : " + str(tF)) # -383
print("%*\textdegree*)Re: " + str(tR)) # 33 \end{lstlisting}
      \end{onlyenv}

    \end{column}
  \end{columns}

\end{frame}


%%%%%%%%%%%%%%%%%%%%%%%%%%%%%%%%%%%%%%%%%%%%%%%%%%%%%%%%

\section{Modules \& Imports}

%%% Modules
\subsection{Modules}

% Module : just put your functions with variables in a .py file
%     triple quotes """ text """ is not a comment, but a "docstring"
%     put one at the beginning for documenting the module, then one per function

\begin{frame}[fragile]{Modules}

  A module contains functions and variables (and classes)

  \medskip

  \onslide<1-> Module name is the filename withtout \TTBF{.py}

  \medskip

  \onslide<2-> Some conventions for a nice documentation:

  \begin{itemize}
    \item<3-> Begin your module with a \textit{docstring}
    \begin{itemize}
      \item triple quotes \lstinline|""" docstring """|
      \item write the description of your module
      \item docstring can be on multiple lines
    \end{itemize}
    \item<4-> Within each function, write a docstring about it
    \item<5-> Eventually, add useful constants
  \end{itemize}

  \bigskip

  \onslide<7-> Check result with \TTBF{help(module)} (after importing it)

\end{frame}


\begin{frame}[fragile]{Modules}

  \TTBF{MyModule.py} \hspace*{2cm} (module name: \TTBF{MyModule})

  \begin{columns}[onlytextwidth]
    \begin{column}{\textwidth}

      \begin{onlyenv}<1>
        \begin{lstlisting}[style=python]









 \end{lstlisting}
      \end{onlyenv}

      \begin{onlyenv}<2>
        \begin{lstlisting}[style=python]
""" Module for explaining modules """








 \end{lstlisting}
      \end{onlyenv}

      \begin{onlyenv}<3>
        \begin{lstlisting}[style=python]
""" Module for explaining modules """
MYCONST=42







 \end{lstlisting}
      \end{onlyenv}

      \begin{onlyenv}<4>
        \begin{lstlisting}[style=python]
""" Module for explaining modules """
MYCONST=42

def MyFunc(test):
  print("Hello World!")


def OtherFunc(var):
  print("Test.")
 \end{lstlisting}
      \end{onlyenv}

      \begin{onlyenv}<5->
        \begin{lstlisting}[style=python]
""" Module for explaining modules """
MYCONST=42

def MyFunc(test):
  """ Function for explaining modules """
  print("Hello World!")

def OtherFunc(var):
  """ Another function """
  print("Test.") \end{lstlisting}
      \end{onlyenv}

    \end{column}
  \end{columns}

\end{frame}


%%% Imports
\subsection{Imports}

\begin{frame}[fragile]{Imports}

  Multiple ways for importing modules:

  \begin{itemize}
    \item<1-> Method 1: \TTBF{import MyModule}
    \item<2-> Method 2: \TTBF{import MyModule as MyM}
    \item<3-> Method 3: \TTBF{from MyModule import MyFunc}
  \end{itemize}

  \bigskip

  \onslide<4-> Variables and constants can be imported too

  \medskip

  \onslide<5-> Beware: with method 1, \textit{everything} is imported

\end{frame}


\begin{frame}[fragile]{Imports}

  Method 1: \TTBF{MyModule.py}

  \begin{columns}[onlytextwidth]
    \begin{column}{\textwidth}

      \begin{onlyenv}<1>
        \begin{lstlisting}[style=python]



 \end{lstlisting}
      \end{onlyenv}

      \begin{onlyenv}<2>
        \begin{lstlisting}[style=python]
import MyModule
# function: def MyFunc(test)

 \end{lstlisting}
      \end{onlyenv}

      \begin{onlyenv}<3->
        \begin{lstlisting}[style=python]
import MyModule
# function: def MyFunc(test)

MyModule.MyFunc(42) \end{lstlisting}
      \end{onlyenv}

    \end{column}
  \end{columns}

\end{frame}


\begin{frame}[fragile]{Imports}

  Method 2: \TTBF{MyModule.py}

  \begin{columns}[onlytextwidth]
    \begin{column}{\textwidth}

      \begin{onlyenv}<1>
        \begin{lstlisting}[style=python]



 \end{lstlisting}
      \end{onlyenv}

      \begin{onlyenv}<2>
        \begin{lstlisting}[style=python]
import MyModule as MyM
# function: def MyFunc(test)

 \end{lstlisting}
      \end{onlyenv}

      \begin{onlyenv}<3->
        \begin{lstlisting}[style=python]
import MyModule as MyM
# function: def MyFunc(test)

MyM.MyFunc(42) \end{lstlisting}
      \end{onlyenv}

    \end{column}
  \end{columns}

\end{frame}


\begin{frame}[fragile]{Imports}

  Method 3: \TTBF{MyModule.py}

  \begin{columns}[onlytextwidth]
    \begin{column}{\textwidth}

      \begin{onlyenv}<1>
        \begin{lstlisting}[style=python]



 \end{lstlisting}
      \end{onlyenv}

      \begin{onlyenv}<2>
        \begin{lstlisting}[style=python]
from MyModule import MyFunc
# function: def MyFunc(test)

 \end{lstlisting}
      \end{onlyenv}

      \begin{onlyenv}<3->
        \begin{lstlisting}[style=python]
from MyModule import MyFunc
# function: def MyFunc(test)

MyFunc(42) \end{lstlisting}
      \end{onlyenv}

    \end{column}
  \end{columns}

\end{frame}


%%% Packages
\subsection{Packages}

\begin{frame}[fragile]{Packages}

  \begin{center}

  Packages contain multiple modules and dependencies

  \bigskip

  \onslide<2-> \textit{(see documentation and tutorials about how to build one)}

  \end{center}

\end{frame}


%%% OOP, Classes
% -*- latex -*-

\part{OOP and Classes}  % OOP and Classes

%%%%%%%%%%%%%%%%%%%%%%%%%%%%%%%%%%%%%%%%%%%%%%%%%%%%%%%%

% How to create a class

\section{Object Oriented Programming}

\begin{frame}<beamer>{OOP}

  Object Oriented Programming (OOP) vocabulary

  \begin{itemize}
    \item<2-> Class: the type and structure
    \item<3-> Attribute: a variable that is a member of the class
    \item<4-> Method: a function/procedure that is a member of the class
  \end{itemize}

  \begin{itemize}
    \item<5-> Object: an instance of a class
    \item<6-> Constructor: the first method called when creating an object
    \item<7-> Destructor: the method called when deleting an object
  \end{itemize}

  \begin{itemize}
    \item<8-> Inheritance: structure of a class reused as a basis for a new one
    \item<9-> Private: when a member is accessible only by the object itself
    \item<10-> Public: when a member is accessible by any object
  \end{itemize}

\end{frame}

%%%

\begin{frame}<beamer>{OOP}

  Object Oriented Programming (OOP) concepts examples

  \begin{itemize}
    \item<1-> Class: the type and structure
    \item<1-> Attribute: a variable that is a member of the class
    \item<1-> Method: a function/procedure that is a member of the class
    \item<1-> Object: an instance of a class
  \end{itemize}

  \begin{itemize}
    \item<2-> Classes: vehicle, car, airplane, boat, ...
    \item<3-> Attributes: speed, passengers, engine, ...
    \item<4-> Methods: accelerate, decelerate, embark, ...
    \item<5-> Objects/Instances:
      \begin{itemize}
        \item<5-> My Peugeot 206 (\textit{BN-340-FT})
        \item<5-> Your Toyota Corolla (\textit{UE-042-FI})
      \end{itemize}
  \end{itemize}

\end{frame}

%%%

\begin{frame}<beamer>{OOP}

  Inheritance explanation

  \begin{itemize}
    \item<2-> Parent Class: the class more general or abstract
    \item<3-> Child Class / Sub Class: a specialized class derived from a parent
    \item<4-> \textit{the child class inherits attributes and methods from its parent}
  \end{itemize}

  \begin{onlyenv}<5->
    \vspace*{1cm}

    Attribute/Method access modifiers
  \end{onlyenv}

  \begin{itemize}
    \item<6-> Private: only the class itself can access it
    \item<7-> Protected: the class itself and its childs can access it
    \item<8-> Public: any class can access it
  \end{itemize}

\end{frame}

%%%

\begin{frame}<beamer>{OOP}

  Inheritance example

  \begin{itemize}
    \item<1-> Parent Class: the class more general or abstract
    \item<1-> Child Class / Sub Class: a specialized class derived from a parent
    \item<1-> \textit{the child class inherits attributes and methods from its parent}
  \end{itemize}

  \begin{onlyenv}<1->
    \vspace*{1cm}
  \end{onlyenv}

  \begin{itemize}
    \item<2-> Class: vehicle
    \item<3-> Child classes: car, airplane, boat, ...
    \item<4-> Inherited Attributes: speed, passengers, ...
  \end{itemize}

\end{frame}


% How to create a class

\section{Classes in Python}

\begin{frame}<beamer>{Classes in Python}

  Specificities of classes in Python:

  \begin{itemize}
    \item<2-> Only one constructor: \lstinline|__init__|
    \item<3-> No destructor (python manages the memory by references)
    \item<4-> Attributes are at least \textit{read-only}, and eventually \textit{writable}
    \item<5-> Writable attributes can be deleted from the object
    \item<6-> Member beginning by an underscore (\lstinline|_|) aren't strictly private, but should be considered internal to the class
    \item<7-> \TTBF{self} keyword is required as the first parameter of each method
  \end{itemize}

\end{frame}

%%%

\begin{frame}[fragile]{Classes in Python}

  \begin{columns}[onlytextwidth]
    \begin{column}{\textwidth}

      \begin{onlyenv}<1>
        % Balises exception :  %* *)
        \begin{lstlisting}[style=python]











 \end{lstlisting}
      \end{onlyenv}

      \begin{onlyenv}<2>
        \begin{lstlisting}[style=python]
class Vehicle:
  """ General vehicles """









 \end{lstlisting}
      \end{onlyenv}

      \begin{onlyenv}<3>
        \begin{lstlisting}[style=python]
class Vehicle:
  """ General vehicles """
  __speed = 0
  Passengers = 0







 \end{lstlisting}
      \end{onlyenv}

      \begin{onlyenv}<4>
        \begin{lstlisting}[style=python]
class Vehicle:
  """ General vehicles """
  __speed = 0
  Passengers = 0
  # Constructor
  def __init__(self):
    self.Passengers = 1




 \end{lstlisting}
      \end{onlyenv}

      \begin{onlyenv}<5>
        \begin{lstlisting}[style=python]
class Vehicle:
  """ General vehicles """
  __speed = 0
  Passengers = 0
  # Constructor
  def __init__(self):
    self.Passengers = 1
  # Method
  def getSpeed(self):
    return (self.__speed)

 \end{lstlisting}
      \end{onlyenv}

      \begin{onlyenv}<6>
        \begin{lstlisting}[style=python]
class Vehicle:
  """ General vehicles """
  __speed = 0
  Passengers = 0
  # Constructor
  def __init__(self):
    self.Passengers = 1
  # Methods
  def getSpeed(self):
    return (self.__speed)
  def Accelerate(self, x):
    self.__speed += x \end{lstlisting}
      \end{onlyenv}


      \begin{onlyenv}<7->
        \begin{lstlisting}[style=python]
class Vehicle:
  """ General vehicles """
  __speed = 0
  Passengers = 0
  # Constructor
  def __init__(self):       # Constructor
    self.Passengers = 1
  # Methods
  def getSpeed(self):       # Accessor
    return (self.__speed)
  def Accelerate(self, x):  # Mutator
    self.__speed += x \end{lstlisting}
      \end{onlyenv}

    \end{column}
  \end{columns}

\end{frame}

%%%

\begin{frame}[fragile]{Classes in Python}

  \begin{columns}[onlytextwidth]
    \begin{column}{\textwidth}

      \begin{onlyenv}<1>
        \begin{lstlisting}[style=python]









 \end{lstlisting}
      \end{onlyenv}

      \begin{onlyenv}<2>
        \begin{lstlisting}[style=python]
class Car(Vehicle):
  """ Cars inherit from Vehicle """
  __CV = 0






 \end{lstlisting}
      \end{onlyenv}

      \begin{onlyenv}<3>
        \begin{lstlisting}[style=python]
class Car(Vehicle):
  """ Cars inherit from Vehicle """
  __CV = 0
  # Constructor
  def __init__(self, CO2, P):
    self.Passengers = 1
    self.__CV = (CO2 / 45) + (P / 40)


 \end{lstlisting}
      \end{onlyenv}

      \begin{onlyenv}<4->
        \begin{lstlisting}[style=python]
class Car(Vehicle):
  """ Cars inherit from Vehicle """
  __CV = 0
  # Constructor
  def __init__(self, CO2, P):
    self.Passengers = 1
    self.__CV = (CO2 / 45) + (P / 40)
  # Method
  def getCV(self):
    return (self.__CV) \end{lstlisting}
      \end{onlyenv}

    \end{column}
  \end{columns}

\end{frame}

%%%

\begin{frame}[fragile]{Classes in Python}

  \begin{columns}[onlytextwidth]
    \begin{column}{\textwidth}

      \begin{onlyenv}<1>
        \begin{lstlisting}[style=python]










 \end{lstlisting}
      \end{onlyenv}

      \begin{onlyenv}<2>
        \begin{lstlisting}[style=python]
Peugeot206Plus = Car(110, 44)  # 110g/km  44kW
AirbusA340 = Vehicle()








 \end{lstlisting}
      \end{onlyenv}

      \begin{onlyenv}<3>
        \begin{lstlisting}[style=python]
Peugeot206Plus = Car(110, 44)  # 110g/km  44kW
AirbusA340 = Vehicle()

Peugeot206Plus.Accelerate(80)
AirbusA340.Accelerate(260)





 \end{lstlisting}
      \end{onlyenv}

      \begin{onlyenv}<4>
        \begin{lstlisting}[style=python]
Peugeot206Plus = Car(110, 44)  # 110g/km  44kW
AirbusA340 = Vehicle()

Peugeot206Plus.Accelerate(80)
AirbusA340.Accelerate(260)

Peugeot206Plus.getSpeed()  # 80
AirbusA340.getSpeed()      # 260

Peugeot206Plus.getCV()
AirbusA340.getCV() \end{lstlisting}
      \end{onlyenv}


      \begin{onlyenv}<5->
        \begin{lstlisting}[style=python]
Peugeot206Plus = Car(110, 44)  # 110g/km  44kW
AirbusA340 = Vehicle()

Peugeot206Plus.Accelerate(80)
AirbusA340.Accelerate(260)

Peugeot206Plus.getSpeed()  # 80
AirbusA340.getSpeed()      # 260

Peugeot206Plus.getCV()
AirbusA340.getCV()  # Error \end{lstlisting}
      \end{onlyenv}

    \end{column}
  \end{columns}

\end{frame}

%%%

\begin{frame}<beamer>{Method Overriding in Python}

  Method Overriding: when a method is redefined in the child class

  \vspace*{1cm}

  \begin{itemize}
    \item<2-> Child method is called if redefined
    \item<3-> Use \TTBF{super()} on current class for calling its parent method: \\
    \TTBF{super(Class, self).Method()}
  \end{itemize}

\end{frame}

%%%

\begin{frame}[fragile]{Method Overriding in Python}

  \begin{columns}[onlytextwidth]
    \begin{column}{0.47\textwidth}

      \begin{onlyenv}<1>
        \begin{lstlisting}[style=python,basicstyle=\ttfamily\footnotesize]












 \end{lstlisting}
      \end{onlyenv}

      \begin{onlyenv}<2>
        \begin{lstlisting}[style=python,basicstyle=\ttfamily\footnotesize]
class Shape:
  def Hello(self):
    print("Shape: Hello!")

  def SayShape(self):
    print("--Shape--")






 \end{lstlisting}
      \end{onlyenv}

      \begin{onlyenv}<3->
        \begin{lstlisting}[style=python,basicstyle=\ttfamily\footnotesize]
class Shape:
  def Hello(self):
    print("Shape: Hello!")

  def SayShape(self):
    print("--Shape--")

class Cube(Shape):
  def Hello(self):
    print("Cube: Hello!")

  def SayCube(self):
    print("--Cube--") \end{lstlisting}
      \end{onlyenv}

    \end{column}


    \begin{column}{0.47\textwidth}

      \begin{onlyenv}<4>
        \begin{lstlisting}[style=python,basicstyle=\ttfamily\footnotesize]












 \end{lstlisting}
      \end{onlyenv}

      \begin{onlyenv}<5>
        \begin{lstlisting}[style=python,basicstyle=\ttfamily\footnotesize]
S = Shape()
C = Cube()










 \end{lstlisting}
      \end{onlyenv}

      \begin{onlyenv}<6>
        \begin{lstlisting}[style=python,basicstyle=\ttfamily\footnotesize]
S = Shape()
C = Cube()


S.Hello()
C.Hello()






 \end{lstlisting}
      \end{onlyenv}

      \begin{onlyenv}<7>
        \begin{lstlisting}[style=python,basicstyle=\ttfamily\footnotesize]
S = Shape()
C = Cube()


S.Hello() # Shape: Hello!
C.Hello() # Cube: Hello!






 \end{lstlisting}
      \end{onlyenv}

      \begin{onlyenv}<8>
        \begin{lstlisting}[style=python,basicstyle=\ttfamily\footnotesize]
S = Shape()
C = Cube()


S.Hello() # Shape: Hello!
C.Hello() # Cube: Hello!

S.SayShape()
C.SayShape()



 \end{lstlisting}
      \end{onlyenv}

      \begin{onlyenv}<9>
        \begin{lstlisting}[style=python,basicstyle=\ttfamily\footnotesize]
S = Shape()
C = Cube()


S.Hello() # Shape: Hello!
C.Hello() # Cube: Hello!

S.SayShape() # --Shape--
C.SayShape() # --Shape--



 \end{lstlisting}
      \end{onlyenv}

      \begin{onlyenv}<10>
        \begin{lstlisting}[style=python,basicstyle=\ttfamily\footnotesize]
S = Shape()
C = Cube()


S.Hello() # Shape: Hello!
C.Hello() # Cube: Hello!

S.SayShape() # --Shape--
C.SayShape() # --Shape--

S.SayCube()
C.SayCube()
 \end{lstlisting}
      \end{onlyenv}

      \begin{onlyenv}<11->
        \begin{lstlisting}[style=python,basicstyle=\ttfamily\footnotesize]
S = Shape()
C = Cube()


S.Hello() # Shape: Hello!
C.Hello() # Cube: Hello!

S.SayShape() # --Shape--
C.SayShape() # --Shape--

S.SayCube() # ERROR
C.SayCube() # --Cube--
 \end{lstlisting}
      \end{onlyenv}

    \end{column}
  \end{columns}

\end{frame}

%%%

\begin{frame}[fragile]{}

  \begin{columns}[onlytextwidth]
    \begin{column}{0.47\textwidth}

      \begin{onlyenv}<1>
        \begin{lstlisting}[style=python,basicstyle=\ttfamily\footnotesize]
class Shape:
  def Hello(self):
    print("Shape: Hello!")

  def SayShape(self):
    print("--Shape--")








 \end{lstlisting}
      \end{onlyenv}

      \begin{onlyenv}<2->
        \begin{lstlisting}[style=python,basicstyle=\ttfamily\footnotesize]
class Shape:
  def Hello(self):
    print("Shape: Hello!")

  def SayShape(self):
    print("--Shape--")

class Cube(Shape):
  def Hello(self):
    print("Cube: Hello!")

  def SayCube(self):
    super(Cube, self).
   SayShape()
    print("--Cube--") \end{lstlisting}
      \end{onlyenv}

    \end{column}


    \begin{column}{0.47\textwidth}

      \begin{onlyenv}<3>
        \begin{lstlisting}[style=python,basicstyle=\ttfamily\footnotesize]














 \end{lstlisting}
      \end{onlyenv}

      \begin{onlyenv}<4>
        \begin{lstlisting}[style=python,basicstyle=\ttfamily\footnotesize]
S = Shape()
C = Cube()












 \end{lstlisting}
      \end{onlyenv}

      \begin{onlyenv}<5>
        \begin{lstlisting}[style=python,basicstyle=\ttfamily\footnotesize]
S = Shape()
C = Cube()


S.Hello()
C.Hello()








 \end{lstlisting}
      \end{onlyenv}

      \begin{onlyenv}<6>
        \begin{lstlisting}[style=python,basicstyle=\ttfamily\footnotesize]
S = Shape()
C = Cube()


S.Hello() # Shape: Hello!
C.Hello() # Cube: Hello!








 \end{lstlisting}
      \end{onlyenv}

      \begin{onlyenv}<7>
        \begin{lstlisting}[style=python,basicstyle=\ttfamily\footnotesize]
S = Shape()
C = Cube()


S.Hello() # Shape: Hello!
C.Hello() # Cube: Hello!

S.SayShape()
C.SayShape()





 \end{lstlisting}
      \end{onlyenv}

      \begin{onlyenv}<8>
        \begin{lstlisting}[style=python,basicstyle=\ttfamily\footnotesize]
S = Shape()
C = Cube()


S.Hello() # Shape: Hello!
C.Hello() # Cube: Hello!

S.SayShape() # --Shape--
C.SayShape() # --Shape--





 \end{lstlisting}
      \end{onlyenv}

      \begin{onlyenv}<9>
        \begin{lstlisting}[style=python,basicstyle=\ttfamily\footnotesize]
S = Shape()
C = Cube()


S.Hello() # Shape: Hello!
C.Hello() # Cube: Hello!

S.SayShape() # --Shape--
C.SayShape() # --Shape--

S.SayCube()
C.SayCube()


 \end{lstlisting}
      \end{onlyenv}

      \begin{onlyenv}<10->
        \begin{lstlisting}[style=python,basicstyle=\ttfamily\footnotesize]
S = Shape()
C = Cube()


S.Hello() # Shape: Hello!
C.Hello() # Cube: Hello!

S.SayShape() # --Shape--
C.SayShape() # --Shape--

S.SayCube() # ERROR
C.SayCube() # --Shape--
            # --Cube--

 \end{lstlisting}
      \end{onlyenv}

    \end{column}
  \end{columns}

\end{frame}

%%%

\begin{frame}<beamer>{Summary of Classes in Python}

  Summary of vocabulary:

  \begin{itemize}
    \item<2-> \TTBF{Vehicle}: class
    \item<3-> \TTBF{Car}: class (inherits from \TTBF{Vehicle})
    \item<4-> \TTBF{Peugeot206Plus}: instance of \TTBF{Car}
    \item<5-> \TTBF{AirbusA340}: instance of \TTBF{Vehicle}
  \end{itemize}

  \begin{itemize}
  \item<6-> \TTBF{Peugeot206Plus} can calculate and give \TTBF{CV}
  \item<7-> \TTBF{AirbusA340}: doesn't have \TTBF{CV} method nor attribute
  \end{itemize}

  \begin{center}
    \onslide<8-> \url{https://docs.python.org/3/tutorial/classes.html}
  \end{center}

\end{frame}


%%% Files, Directories
% -*- latex -*-

\part{Files and Directories}  % Files and Directories

%%%%%%%%%%%%%%%%%%%%%%%%%%%%%%%%%%%%%%%%%%%%%%%%%%%%%%%%

% File management in python

\section{Files Processing}

% How to open, read, write, close, seek
\subsection{Open, Read, Write, Close, Seek, Tell}


\begin{frame}[fragile]{Files Operations}

  Files are managed as usual:

  \begin{onlyenv}<2>
    \begin{itemize}
      \item \TTBF{open}
      \item \TTBF{read}
      \item \TTBF{write}
      \item \TTBF{close}
      \item \TTBF{seek}
      \item \TTBF{tell}
    \end{itemize}
  \end{onlyenv}

  \begin{onlyenv}<3>
    \begin{itemize}
      \item \TTBF{open(filename, mode, \textit{[encoding]})}
      \item \TTBF{read}
      \item \TTBF{write}
      \item \TTBF{close}
      \item \TTBF{seek}
      \item \TTBF{tell}
    \end{itemize}
  \end{onlyenv}

  \begin{onlyenv}<4>
    \begin{itemize}
      \item \TTBF{open(filename, mode, \textit{[encoding]})}
      \item \TTBF{read(NbCharacters)}
      \item \TTBF{write}
      \item \TTBF{close}
      \item \TTBF{seek}
      \item \TTBF{tell}
    \end{itemize}
  \end{onlyenv}

  \begin{onlyenv}<5>
    \begin{itemize}
      \item \TTBF{open(filename, mode, \textit{[encoding]})}
      \item \TTBF{read(NbCharacters)}
      \item \TTBF{write(string)}
      \item \TTBF{close}
      \item \TTBF{seek}
      \item \TTBF{tell}
    \end{itemize}
  \end{onlyenv}

  \begin{onlyenv}<6>
    \begin{itemize}
      \item \TTBF{open(filename, mode, \textit{[encoding]})}
      \item \TTBF{read(NbCharacters)}
      \item \TTBF{write(string)}
      \item \TTBF{close()}
      \item \TTBF{seek}
      \item \TTBF{tell}
    \end{itemize}
  \end{onlyenv}

  \begin{onlyenv}<7>
    \begin{itemize}
      \item \TTBF{open(filename, mode, \textit{[encoding]})}
      \item \TTBF{read(NbCharacters)}
      \item \TTBF{write(string)}
      \item \TTBF{close()}
      \item \TTBF{seek(offset, \textit{[whence]})}
      \item \TTBF{tell}
    \end{itemize}
  \end{onlyenv}

  \begin{onlyenv}<8->
    \begin{itemize}
      \item \TTBF{open(filename, mode, \textit{[encoding]})}
      \item \TTBF{read(NbCharacters)}
      \item \TTBF{write(string)}
      \item \TTBF{close()}
      \item \TTBF{seek(offset, \textit{[whence]})}
      \item \TTBF{tell()}
    \end{itemize}
  \end{onlyenv}

  \medskip

  \onslide<9-> Always close files in order to be sure to write changes on the physical support

  \medskip

  \onslide<10-> Files are managed by an internal python's \textbf{class}

\end{frame}

%%%

\begin{frame}<beamer>{Open}

  Open mode:

  \begin{itemize}
    \item<2-> \TTBF{"r"} - open for reading
    \item<3-> \TTBF{"w"} - truncate the file for overwriting
    \item<4-> \TTBF{"a"} - append/begin from the end for writing
    \item<5-> \TTBF{"x"} - create a new file, fail if it already exists
    \item<6-> \TTBF{"r+"} - open for reading and writing
    \item<7-> \TTBF{"b"} - binary mode (read/write with bytes objects)
  \end{itemize}

  \bigskip

  \onslide<8-> Open encoding:

  \begin{itemize}
    \item<9-> \TTBF{encoding="utf-8"} - read/write in \texttt{utf-8}
    \item<10-> Can be omitted for platform encoding \textit{[do this for now]}
    \item<11-> Cannot be used when in binary mode
  \end{itemize}

\end{frame}

%%%

\begin{frame}<beamer>{Text \& Binary mode}

  \onslide<1-> Text mode: \onslide<2-> \TTBF{open(filename, "r+")}

  \begin{itemize}
    \item<3-> For reading/writting text files (CSV, configuration, log, ...)
    \item<4-> When writing in a file: \TTBF{\textbackslash n} is transformed into platform specific line ending (\TTBF{\textbackslash r\textbackslash n}, \TTBF{\textbackslash n\textbackslash r}, ...)
    \item<5-> When reading from a file: platform specific line endings are transformed into simple \TTBF{\textbackslash n}
  \end{itemize}

  \bigskip

  \onslide<6-> Binary mode: \onslide<7-> \TTBF{open(filename, "wb")}

  \begin{itemize}
    \item<8-> For reading/writting binary data (music, video, web page, ...)
    \item<9-> The line ending characters are never transformed when reading/writting
  \end{itemize}

  \bigskip

  \onslide<10-> \textit{For now: do not use binary mode, except if you work on raw data}

\end{frame}

%%%

\begin{frame}<beamer>{Read \& Write}

  Read:

  \begin{itemize}
    \item<1-> \TTBF{read()} - get the whole text
    \item<2-> \TTBF{read(Nb)} - get the next \textit{Nb} characters
    \item<3-> \TTBF{readline()} - get the next line (whole text until next \TTBF{\textbackslash n})
  \end{itemize}

  \bigskip

  \onslide<4-> Write:

  \begin{itemize}
    \item<5-> \TTBF{open(filename, "wb")}
    \item<6-> \TTBF{write(str)} - Write the string and return how many characters were written
  \end{itemize}

\end{frame}

%%%

\begin{frame}<beamer>{Tell \& Seek}

  Tell:

  \begin{itemize}
    \item<1-> \TTBF{tell()} - get current position in the file (in bytes)
    \item<2-> Useful in binary mode
    \item<3-> More difficult in text mode (because of the \textit{encoding)}
  \end{itemize}

  \bigskip

  \onslide<4-> Seek:

  \begin{itemize}
    \item<5-> \TTBF{seek(offset)} - put read/write cursor at the offset
    \item<5-> \TTBF{seek(offset, whence)} - move the cursor relatively
    \item<6-> \TTBF{whence=}
      \begin{itemize}
       \item \TTBF{0} : from the beginning of the file \textit{[default]}
       \item \TTBF{1} : from the current cursor position
       \item \TTBF{2} : from the end of the file
      \end{itemize}
  \end{itemize}

\end{frame}

%%%%%%%%%%%

% Examples of read/write
\subsection{Read \& Write examples}

% Reading files

\begin{frame}[fragile]{Read (1) example}

  Open existing file for reading

  \begin{columns}[onlytextwidth]
    \begin{column}{\textwidth}

      \begin{onlyenv}<1>
        % Balises exception :  %* *)
        \begin{lstlisting}[style=python]





 \end{lstlisting}
      \end{onlyenv}

      \begin{onlyenv}<2>
        % Balises exception :  %* *)
        \begin{lstlisting}[style=python]
f = open("file.txt", "r")




 \end{lstlisting}
      \end{onlyenv}

      \begin{onlyenv}<3>
        % Balises exception :  %* *)
        \begin{lstlisting}[style=python]
f = open("file.txt", "r")
chars = f.read(10)
print(chars)


 \end{lstlisting}
      \end{onlyenv}

      \begin{onlyenv}<4>
        % Balises exception :  %* *)
        \begin{lstlisting}[style=python]
f = open("file.txt", "r")
chars = f.read(10)
print(chars)
line = f.readline()
print(line)
 \end{lstlisting}
      \end{onlyenv}

      \begin{onlyenv}<5->
        % Balises exception :  %* *)
        \begin{lstlisting}[style=python]
f = open("file.txt", "r")
chars = f.read(10)
print(chars)
line = f.readline()
print(line)
f.close() \end{lstlisting}
      \end{onlyenv}

    \end{column}
  \end{columns}

\end{frame}

%%%

\begin{frame}[fragile]{Read (2) example}

  Read line by line a file with python specificities

  \begin{columns}[onlytextwidth]
    \begin{column}{\textwidth}

      \begin{onlyenv}<1>
        % Balises exception :  %* *)
        \begin{lstlisting}[style=python]



 \end{lstlisting}
      \end{onlyenv}

      \begin{onlyenv}<2>
        % Balises exception :  %* *)
        \begin{lstlisting}[style=python]
f = open("file.txt", "r")


 \end{lstlisting}
      \end{onlyenv}

      \begin{onlyenv}<3>
        % Balises exception :  %* *)
        \begin{lstlisting}[style=python]
f = open("file.txt", "r")
for x in f:
  print(x)
 \end{lstlisting}
      \end{onlyenv}

      \begin{onlyenv}<4->
        % Balises exception :  %* *)
        \begin{lstlisting}[style=python]
f = open("file.txt", "r")
for x in f:
  print(x)
f.close() \end{lstlisting}
      \end{onlyenv}

    \end{column}
  \end{columns}

\end{frame}

%%%

% Writing files

\begin{frame}[fragile]{Write example}

  Write at the end of a file (add content)

  \begin{columns}[onlytextwidth]
    \begin{column}{\textwidth}

      \begin{onlyenv}<1>
        % Balises exception :  %* *)
        \begin{lstlisting}[style=python]



 \end{lstlisting}
      \end{onlyenv}

      \begin{onlyenv}<2>
        % Balises exception :  %* *)
        \begin{lstlisting}[style=python]
f = open("file.txt", "a")


 \end{lstlisting}
      \end{onlyenv}

      \begin{onlyenv}<3>
        % Balises exception :  %* *)
        \begin{lstlisting}[style=python]
f = open("file.txt", "a")
nb = f.write("-add content-")

 \end{lstlisting}
      \end{onlyenv}

      \begin{onlyenv}<4>
        % Balises exception :  %* *)
        \begin{lstlisting}[style=python]
f = open("file.txt", "a")
nb = f.write("-add content-")
f.close()
 \end{lstlisting}
      \end{onlyenv}

      \begin{onlyenv}<5->
        % Balises exception :  %* *)
        \begin{lstlisting}[style=python]
f = open("file.txt", "a")
nb = f.write("-add content-")
f.close()
print(nb)  # 13 chars were added \end{lstlisting}
      \end{onlyenv}

    \end{column}
  \end{columns}

\end{frame}

%%%

\begin{frame}<beamer>{Documentation}

  \begin{center}
    \onslide<1-> Check documentation for more informations

    \bigskip

    \onslide<1-> \url{https://docs.python.org/3/tutorial/inputoutput.html}
  \end{center}

\end{frame}


%%%%%%%%%%%%%%%%%%%%%%%%%%%%%%%%%%%%%%


% Test file existence and delete
\section{File existence and management}

\begin{frame}<beamer>{Files and Directories}

  Management of files is done through the \TTBF{os} module

  \begin{itemize}
    \item<2-> \TTBF{os.chdir(path)} - change the current directory
    \item<3-> \TTBF{os.mkdir(dirname)} - create a directory
    \item<4-> \TTBF{os.makedirs(path)} - create the full path (recursively)
    \item<5-> \TTBF{os.remove(filename)} - remove the file
    \item<6-> \TTBF{os.rmdir(dirname)} - remove the directory
    \item<7-> \TTBF{os.listdir(path)} - list files in a directory
    \item<8-> \TTBF{os.scandir(path)} - list files in a directory (generator)
  \end{itemize}

\end{frame}

%%%

\begin{frame}<beamer>{File type}

  Type of file and names are done with \TTBF{path} submodule

  \begin{itemize}
    \item<2-> \TTBF{os.path.exist(path)} - test if \textit{path} is an existing file
    \item<3-> \TTBF{os.path.isfile(path)} - test if \textit{path} is a file
    \item<4-> \TTBF{os.path.isdir(path)} - test if \textit{path} is a directory
    \item<5-> \TTBF{os.path.abspath(path)} - get absolute pathname
    \item<6-> \TTBF{os.path.dirname(path)} - get dirname of pathname
    \item<7-> \TTBF{os.path.basename(path)} - get basename of pathname
    \item<8-> \TTBF{os.path.join(parent, child)} - build a pathname from two paths
  \end{itemize}

  \medskip

  \begin{center}
    \onslide<9-> \url{https://docs.python.org/3/library/os.path.html}
  \end{center}

\end{frame}

%%%

\begin{frame}[fragile]{File browsing example}

  Browse a directory:

  \begin{columns}[onlytextwidth]
    \begin{column}{\textwidth}

      \begin{onlyenv}<1>
        % Balises exception :  %* *)
        \begin{lstlisting}[style=python,basicstyle=\ttfamily\footnotesize]










 \end{lstlisting}
      \end{onlyenv}

      \begin{onlyenv}<2>
        % Balises exception :  %* *)
        \begin{lstlisting}[style=python,basicstyle=\ttfamily\footnotesize]
import os









 \end{lstlisting}
      \end{onlyenv}

      \begin{onlyenv}<3>
        % Balises exception :  %* *)
        \begin{lstlisting}[style=python,basicstyle=\ttfamily\footnotesize]
import os


entries = os.listdir(".")






 \end{lstlisting}
      \end{onlyenv}

      \begin{onlyenv}<4>
        % Balises exception :  %* *)
        \begin{lstlisting}[style=python,basicstyle=\ttfamily\footnotesize]
import os

# Print files & dirs
entries = os.listdir(".")
for entry in entries:
  print(str(entry))




 \end{lstlisting}
      \end{onlyenv}

      \begin{onlyenv}<5>
        % Balises exception :  %* *)
        \begin{lstlisting}[style=python,basicstyle=\ttfamily\footnotesize]
import os

# Print files & dirs
entries = os.listdir(".")
for entry in entries:
  print(str(entry))


files = [f for f in os.listdir() if os.path.isfile(f)]

 \end{lstlisting}
      \end{onlyenv}

      \begin{onlyenv}<6->
        % Balises exception :  %* *)
        \begin{lstlisting}[style=python,basicstyle=\ttfamily\footnotesize]
import os

# Print files & dirs
entries = os.listdir(".")
for entry in entries:
  print(str(entry))

# Print files only
files = [f for f in os.listdir() if os.path.isfile(f)]
for f in files:
  print(str(f)) \end{lstlisting}
      \end{onlyenv}

    \end{column}
  \end{columns}

\end{frame}

%%% Types, lists (comprehension, ...)
% -*- latex -*-

\part{Python's Main Concepts}  % Types, lists (comprehension, ...)

%%%%%%%%%%%%%%%%%%%%%%%%%%%%%%%%%%%%%%%%%%%%%%%%%%%%%%%%

\begin{frame}<beamer>{Built-in types}

  \begin{itemize}
    \item Numerics
    \item Sequences
    \item Mappings
    \item Classes
    \item Instances
    \item Exceptions
  \end{itemize}

\end{frame}


%%% Numeric types
\section{Numeric types}

\begin{frame}<beamer>{Numeric types}

  \begin{onlyenv}<1>
    \begin{itemize}
      \item<1> int
      \item<1> float
      \item<1> complex
    \end{itemize}
  \end{onlyenv}

  \begin{onlyenv}<2->
    \begin{itemize}
      \item<2-> int
      \begin{itemize}
        \item<2-> booleans \textit{- (subtype of int)}
      \end{itemize}
      \item<2-> float
      \item<2-> complex
      \begin{itemize}
        \item<3-> \textit{(pair of floats for real and imaginary parts)}
      \end{itemize}
    \end{itemize}
  \end{onlyenv}

\end{frame}


\begin{frame}<beamer>{Numeric types}

  \begin{center}

%  Numeric types examples:
%
%  \bigskip

  \begin{tabular}{| l | c  c  c |}
    \hline
    \textbf{Type} & \multicolumn{3}{ c |}{\textbf{Examples}} \\
    \hline
    int 		& 0 		& 42 		& 1337 	\\
    float 	& 0.0 	& 1.8e30 	& 1. 	\\
    complex 	& 3j 	& 2J 		& 5+7j 	\\
    \hline
  \end{tabular}

%  \bigskip
  \vspace{1.5cm}

  \textquotedbl 1.8e30\textquotedbl{} is equivalent to \textquotedbl $ 1,8 \times {10}^{30} $\textquotedbl

  \medskip

  \textquotedbl 5j\textquotedbl{} is equivalent in french to \textquotedbl 5i\textquotedbl
  \end{center}

\end{frame}


%%% Numeric operations

\begin{frame}<beamer>{Numeric operations}

  \begin{center}

%  Main integers operations:
%
%  \bigskip

  \begin{tabular}{| c | l |}
    \hline
    \textbf{Operation} & \textbf{Result} \\
    \hline
    $ \text{x} \: + \: \text{y} $ 	& sum of \textit{x} and \textit{y} \\
    $ \text{x} \: - \: \text{y} $ 	& difference of \textit{x} and \textit{y} \\
    $ \text{x} \: * \: \text{y} $ 	& product of \textit{x} and \textit{y} \\
    $ \text{x} \: / \: \text{y} $ 	& quotient of \textit{x} and \textit{y} \\
    $ \text{x} \: // \: \text{y} $ 	& floored quotient of \textit{x} and \textit{y} \\
    $ \text{x} \: \% \: \text{y} $ 	& remainder of $ \text{x} / \text{y} $ \\
    $ \text{pow}(\text{x}, \text{y}) $ 	& \textit{x} to the power \textit{y} \\
    $ \text{x} \: ** \: \text{y} $ 		& \textit{x} to the power \textit{y} \\
    $ - \text{x} $ 		& \textit{x} negated \\
    $ + \text{x} $ 		& \textit{x} unchanged \\
    $ \text{abs(x)} $ 	& absolute value or magnitude of \textit{x} \\
    \hline
  \end{tabular}

  \medskip

  \textquotedbl /\textquotedbl{} gives a float as a result

  \textquotedbl //\textquotedbl{} gives an integer as a result (euclidean division)

  \end{center}

\end{frame}


\begin{frame}[fragile]{Numeric operations}

  \begin{center}

%  Useful integers operations:
%
%  \bigskip

  \begin{tabular}{| c | l |}
    \hline
    \textbf{Operation} & \textbf{Result} \\
    \hline
    $ \text{abs(x)} $ 	& absolute value or magnitude of \textit{x} \\
    $ \text{floor(x)} $ 	& the largest integer not greater than \textit{x} \\
    $ \text{ceil(x)} $ 	& the smallest integer greater than or equal to \textit{x} \\
    \hline
  \end{tabular}

  \medskip

  \begin{lstlisting}[style=python,morekeywords={floor,ceil}]
import math

Absolue = abs(-42.7)  # 42.7
ParDefaut1 = math.floor(3.2)  # 3
ParDefaut2 = math.floor(3.7)  # 3
ParExces1 = math.ceil(5.2)  # 6
ParExces2 = math.ceil(5.7)  # 6 \end{lstlisting}

  \end{center}

\end{frame}


\begin{frame}[fragile]{Numeric operations}

  \begin{center}

  Constructors and type conversion:

  \bigskip

  \begin{tabular}{| l | p{5cm} |}
    \hline
    \textbf{Operation} & \textbf{Result} \\
    \hline
    $ \text{int(x)} $ 	& \textit{x} converted to integer \\
    $ \text{float(x)} $ 	& \textit{x} converted to floating point \\
    $ \text{complex(re, im)} $ 	& a complex number with real part \textit{re}, imaginary part \textit{im}. (\textit{im} defaults to zero) \\
    \hline
  \end{tabular}

  \medskip

  \begin{lstlisting}[style=python,morekeywords={int,float}]
entier = %*\color{blue}int*)(42.7)     # 42
flottant = %*\color{blue}float*)(3)    # 3.0
complexe1 = complex(4)     # 4+0j
complexe2 = complex(5, 6)  # 5+6j \end{lstlisting}

  \end{center}

\end{frame}


%%% Other numeric types

\begin{frame}<beamer>{Other numeric types}

  Booleans values:
  \begin{itemize}
    \item<1-> \TTBF{False} (equivalent to \TTBF{int(0)})
    \item<1-> \TTBF{True} (equivalent to \TTBF{int(1)})
  \end{itemize}

  \bigskip

%  \begin{onlyenv}<2->
    \begin{center}
    More details on types

    \medskip

    \url{https://docs.python.org/3/library/stdtypes.html}
    \end{center}
%  \end{onlyenv}

\end{frame}


%%% Sequence types
\section{Sequence types}

\begin{frame}<beamer>{Sequence types}

  \begin{itemize}
    \item<1-> list
    \item<1-> tuple
    \item<1-> range
    \item<2-> str \textit{- (text sequence type)}
    \item<3-> \textit{binary sequence types}
  \end{itemize}

  \bigskip

  \begin{center}
    \begin{onlyenv}<4->

    index begins at 0

    \end{onlyenv}
  \end{center}

\end{frame}


%%% Sequence operations
\subsection{Sequence operations}

\subsubsection{Common operations}

\begin{frame}<beamer>{Sequence operations}

  \begin{center}

    Sequences do have common operations:

    \begin{itemize}
      \item Access
      \item Search
      \item Concatenation
      \item min/max/len
      \item ...
    \end{itemize}

  \end{center}

\end{frame}


\begin{frame}<beamer>{Sequence operations}

  \begin{center}

%  Common sequence operations:
%
%  \bigskip

  \begin{tabular}{| c | l |}
    \hline
    \textbf{Operation} & \textbf{Result} \\
    \hline
    $ \text{x} \: \text{in} \: \text{s} $ 		& \TTBF{True} if an item of \textit{s} is equal to \textit{x}, else \TTBF{False} \\
    $ \text{x} \: \text{not in} \: \text{s} $ 	& \TTBF{False} if an item of \textit{s} is equal to \textit{x}, else \TTBF{True} \\
    $ \text{s} \: + \: \text{t} $ 	& the concatenation of \textit{s} and \textit{t} \\
    $ \text{s} \: * \: \text{n} $ 	& equivalent to adding \textit{s} to itself \textit{n} times \\
    s[i] 		& access to the \textit{i}th item of \textit{s} (first index: 0) \\
    s[i:j] 		& slice of \textit{s} from \textit{i} (included) to \textit{j} (excluded) \\
    s[i:j:k] 	& slice of \textit{s} from \textit{i} to \textit{j} (excluded) with step \textit{k} \\
    len(x) 	& length of \textit{s} \\
    min(x) 	& smallest item of \textit{s} \\
    max(x) 	& largest item of \textit{s} \\
    \hline
  \end{tabular}

%    $ \text{s}[\text{i}] $ 					& access to the \textit{i}th item of \textit{s} (first index: \textit{0}) \\
%    $ \text{s}[\text{i}:\text{j}] $ 			& slice of \textit{s} from \textit{i} to \textit{j} \\
%    $ \text{s}[\text{i}:\text{j}:\text{k}] $ 	& slice of \textit{s} from \textit{i} to \textit{j} with step \textit{k} \\


  \medskip

  \textit{s} and \textit{t} are sequences of the same type

  \textit{n}, \textit{i}, \textit{j} and \textit{k} are integers

  \textit{x} is an arbitrary object

  \end{center}

\end{frame}


\begin{frame}[fragile]{Sequence operations}

  \begin{center}

  Access to items:

  \medskip

  \begin{columns}[onlytextwidth]
    \begin{column}{\textwidth}

      \begin{onlyenv}<1>
        % Balises exception :  %* *)
        \begin{lstlisting}[style=python,morekeywords={for, in, range, list}]
mylist = [ 42, 1337, 42, 666, 15 ]






 \end{lstlisting}
      \end{onlyenv}

      \begin{onlyenv}<2>
        % Balises exception :  %* *)
        \begin{lstlisting}[style=python,morekeywords={for, in, range, list}]
mylist = [ 42, 1337, 42, 666, 15 ]
print(len(mylist)) # length: 5





 \end{lstlisting}
      \end{onlyenv}

      \begin{onlyenv}<3>
        % Balises exception :  %* *)
        \begin{lstlisting}[style=python,morekeywords={for, in, range, list}]
mylist = [ 42, 1337, 42, 666, 15 ]
print(len(mylist)) # length: 5
print(min(mylist)) # smallest: 15
print(max(mylist)) # biggest: 1337



 \end{lstlisting}
      \end{onlyenv}

      \begin{onlyenv}<4>
        % Balises exception :  %* *)
        \begin{lstlisting}[style=python,morekeywords={for, in, range, list}]
mylist = [ 42, 1337, 42, 666, 15 ]
print(len(mylist)) # length: 5
print(min(mylist)) # smallest: 15
print(max(mylist)) # biggest: 1337
print(42 in mylist)  # test presence 42: True
print(mylist[3])       # 4th item: 666

 \end{lstlisting}
      \end{onlyenv}

      \begin{onlyenv}<5>
        % Balises exception :  %* *)
        \begin{lstlisting}[style=python,morekeywords={for, in, range, list}]
mylist = [ 42, 1337, 42, 666, 15 ]
print(len(mylist)) # length: 5
print(min(mylist)) # smallest: 15
print(max(mylist)) # biggest: 1337
print(42 in mylist)  # test presence 42: True
print(mylist[3])       # 4th item: 666
print(mylist[1:3])     # slice: 1337, 42
 \end{lstlisting}
      \end{onlyenv}

      \begin{onlyenv}<6->
        % Balises exception :  %* *)
        \begin{lstlisting}[style=python,morekeywords={for, in, range, list}]
mylist = [ 42, 1337, 42, 666, 15 ]
print(len(mylist)) # length: 5
print(min(mylist)) # smallest: 15
print(max(mylist)) # biggest: 1337
print(42 in mylist)  # test presence 42: True
print(mylist[3])       # 4th item: 666
print(mylist[1:3])     # slice: 1337, 42
print(mylist[0:4:2])   # slice: 42, 42 \end{lstlisting}
      \end{onlyenv}

    \end{column}
  \end{columns}

  \end{center}

\end{frame}


\begin{frame}<beamer>{Sequence operations}

  \begin{center}

%  Common sequence operations:
%
%  \bigskip

  \begin{tabular}{| c | p{8.5cm} |}
    \hline
    \textbf{Operation} & \textbf{Result} \\
    \hline
    s[i] 		& access to the \textit{i}th item of \textit{s} [first index: 0] \\
    s[i:j] 		& slice beginning at \textit{i} ending before \textit{j} [ends at \textit{j - 1}] \\
    s[i:j:k] 	& slice beginning at \textit{i} ending before \textit{j} by steps of \textit{k} \\
    s.count(x) 		& total number of occurrences of \textit{x} in \textit{s} \\
    s.index(x) 			& index of the first occurrence of \textit{x} in \textit{s} \\
    s.index(x, i) 		& index of the first occurrence of \textit{x} in \textit{s} at or after index \textit{i} \\
    s.index(x, i, j) 	& index of the first occurrence of \textit{x} in \textit{s} at or after index \textit{i} and before index \textit{j} \\
    \hline
  \end{tabular}

%    $ \text{s}[\text{i}] $ 					& access to the \textit{i}th item of \textit{s} (first index: \textit{0}) \\
%    $ \text{s}[\text{i}:\text{j}] $ 			& slice of \textit{s} from \textit{i} to \textit{j} \\
%    $ \text{s}[\text{i}:\text{j}:\text{k}] $ 	& slice of \textit{s} from \textit{i} to \textit{j} with step \textit{k} \\


  \medskip

  \textit{s} and \textit{t} are sequences of the same type

  \textit{n}, \textit{i}, \textit{j} and \textit{k} are integers

  \textit{x} is an arbitrary object

  \end{center}

\end{frame}


%%% Slices
\subsubsection{Slices}

\begin{frame}<beamer>{Sequence operations}

  \begin{center}

  Slicing tips (1):

  \bigskip

  \begin{itemize}
    \item<1-> s[:3] is equivalent to s[0:3] \\
      { \footnotesize (prints everything from start until 3) } \\
    \item<2-> s[5:] is equivalent to s[5:{\footnotesize\textit{(last index + 1)}}] \\
      { \footnotesize (prints everything from 5) } \\
    \item<3-> s[:] is equivalent to s[0:{\footnotesize\textit{(last index + 1)}}] \\
      { \footnotesize (prints everything) } \\
    \item<4-> s[::2] is equivalent to s[0:{\footnotesize\textit{(last index + 1)}}:2] \\
      { \footnotesize (prints everything by steps of 2) } \\
  \end{itemize}

  \bigskip

  \begin{onlyenv}<5->
    \textit{(last index + 1)} is equivalent to \textit{len(s)} \\
  \end{onlyenv}

  \end{center}

\end{frame}



%%% Slicing (1) : positive values

\begin{frame}[fragile]{Sequence operations}

  \begin{center}

  Slicing (1): Positive indexes

  \smallskip

  \begin{columns}[onlytextwidth]
    \begin{column}{\textwidth}

      \begin{onlyenv}<1>
        % Balises exception :  %* *)
        \begin{lstlisting}[style=python,morekeywords={for, in, range, list}]
mylist = [ 42, 37, 42, 66, 15 ]
#                      [0] [1] [2] [3] [4]
print(mylist)






#                      [0] [1] [2] [3] [4] \end{lstlisting}
      \end{onlyenv}

      \begin{onlyenv}<2>
        % Balises exception :  %* *)
        \begin{lstlisting}[style=python,morekeywords={for, in, range, list}]
mylist = [ 42, 37, 42, 66, 15 ]
#                      [0] [1] [2] [3] [4]
print(mylist)        # 42, 37, 42, 66, 15
print(mylist[:])





#                      [0] [1] [2] [3] [4] \end{lstlisting}
      \end{onlyenv}

      \begin{onlyenv}<3>
        % Balises exception :  %* *)
        \begin{lstlisting}[style=python,morekeywords={for, in, range, list}]
mylist = [ 42, 37, 42, 66, 15 ]
#                      [0] [1] [2] [3] [4]
print(mylist)        # 42, 37, 42, 66, 15
print(mylist[:])     # 42, 37, 42, 66, 15
print(mylist[1:4])




#                      [0] [1] [2] [3] [4] \end{lstlisting}
      \end{onlyenv}

      \begin{onlyenv}<4>
        % Balises exception :  %* *)
        \begin{lstlisting}[style=python,morekeywords={for, in, range, list}]
mylist = [ 42, 37, 42, 66, 15 ]
#                      [0] [1] [2] [3] [4]
print(mylist)        # 42, 37, 42, 66, 15
print(mylist[:])     # 42, 37, 42, 66, 15
print(mylist[1:4])   #     37, 42, 66
print(mylist[2:])



#                      [0] [1] [2] [3] [4] \end{lstlisting}
      \end{onlyenv}

      \begin{onlyenv}<5>
        % Balises exception :  %* *)
        \begin{lstlisting}[style=python,morekeywords={for, in, range, list}]
mylist = [ 42, 37, 42, 66, 15 ]
#                      [0] [1] [2] [3] [4]
print(mylist)        # 42, 37, 42, 66, 15
print(mylist[:])     # 42, 37, 42, 66, 15
print(mylist[1:4])   #     37, 42, 66
print(mylist[2:])    #         42, 66, 15
print(mylist[:3])


#                      [0] [1] [2] [3] [4] \end{lstlisting}
      \end{onlyenv}

      \begin{onlyenv}<6>
        % Balises exception :  %* *)
        \begin{lstlisting}[style=python,morekeywords={for, in, range, list}]
mylist = [ 42, 37, 42, 66, 15 ]
#                      [0] [1] [2] [3] [4]
print(mylist)        # 42, 37, 42, 66, 15
print(mylist[:])     # 42, 37, 42, 66, 15
print(mylist[1:4])   #     37, 42, 66
print(mylist[2:])    #         42, 66, 15
print(mylist[:3])    # 42, 37, 42
print(mylist[::2])

#                      [0] [1] [2] [3] [4] \end{lstlisting}
      \end{onlyenv}

      \begin{onlyenv}<7>
        % Balises exception :  %* *)
        \begin{lstlisting}[style=python,morekeywords={for, in, range, list}]
mylist = [ 42, 37, 42, 66, 15 ]
#                      [0] [1] [2] [3] [4]
print(mylist)        # 42, 37, 42, 66, 15
print(mylist[:])     # 42, 37, 42, 66, 15
print(mylist[1:4])   #     37, 42, 66
print(mylist[2:])    #         42, 66, 15
print(mylist[:3])    # 42, 37, 42
print(mylist[::2])   # 42,     42,     15
print(mylist[1:5:3])
#                      [0] [1] [2] [3] [4] \end{lstlisting}
      \end{onlyenv}

      \begin{onlyenv}<8->
        % Balises exception :  %* *)
        \begin{lstlisting}[style=python,morekeywords={for, in, range, list}]
mylist = [ 42, 37, 42, 66, 15 ]
#                      [0] [1] [2] [3] [4]
print(mylist)        # 42, 37, 42, 66, 15
print(mylist[:])     # 42, 37, 42, 66, 15
print(mylist[1:4])   #     37, 42, 66
print(mylist[2:])    #         42, 66, 15
print(mylist[:3])    # 42, 37, 42
print(mylist[::2])   # 42,     42,     15
print(mylist[1:5:3]) #     37,         15
#                      [0] [1] [2] [3] [4] \end{lstlisting}
      \end{onlyenv}

    \end{column}
  \end{columns}

  \end{center}

\end{frame}


\begin{frame}<beamer>{Sequence operations}

  \begin{center}

  Slicing tips (2):

  \bigskip

  \begin{onlyenv}<1->
    you can go back from the last item thanks to negative values \\
  \end{onlyenv}

  \bigskip

  \begin{itemize}
    \item<2-> s[-1] is equivalent to s[{\footnotesize\textit{(last index + 1)}}] \textit{(last item)} \\
    \item<3-> s[-4:] prints the last 4 items \\
    \item<4-> s[:-1] prints all of the items, except the last one \\
    \item<5-> s[-4:-1] prints the last 4 items, except the last one \textit{(stop at -1)} \\
    \item<6-> s[2:-1] prints all the items from index 2, except the last one \\
    \item<7-> s[-4:3] prints all the items from index -4 to index 2 \textit{(stop at 3)} \\
  \end{itemize}

  \bigskip

  \begin{onlyenv}<8->
    don't forget that the second index is not selected/it is the limit \\
  \end{onlyenv}

  \end{center}

\end{frame}




%%% Slicing (2): negative values
\begin{frame}[fragile]{Sequence operations}

  \begin{center}

  Slicing (2): Negative indexes

  \bigskip

  \begin{columns}[onlytextwidth]
    \begin{column}{\textwidth}

      \begin{onlyenv}<1>
        % Balises exception :  %* *)
        \begin{lstlisting}[style=python,morekeywords={for, in, range, list}]
mylist = [ 42, 37, 42, 66, 15 ]
#                      [0] [1] [2] [3] [4]
print(mylist[:])     # 42, 37, 42, 66, 15





#                      [0] [1] [2] [3] [4] \end{lstlisting}
      \end{onlyenv}

      \begin{onlyenv}<2>
        % Balises exception :  %* *)
        \begin{lstlisting}[style=python,morekeywords={for, in, range, list}]
mylist = [ 42, 37, 42, 66, 15 ]
#                      [0] [1] [2] [3] [4]
print(mylist[:])     # 42, 37, 42, 66, 15
print(mylist[-1])




#                      [0] [1] [2] [3] [4] \end{lstlisting}
      \end{onlyenv}

      \begin{onlyenv}<3>
        % Balises exception :  %* *)
        \begin{lstlisting}[style=python,morekeywords={for, in, range, list}]
mylist = [ 42, 37, 42, 66, 15 ]
#                      [0] [1] [2] [3] [4]
print(mylist[:])     # 42, 37, 42, 66, 15
print(mylist[-1])    #                 15
print(mylist[:-1])



#                     [-5][-4][-3][-2][-1] \end{lstlisting}
      \end{onlyenv}

      \begin{onlyenv}<4>
        % Balises exception :  %* *)
        \begin{lstlisting}[style=python,morekeywords={for, in, range, list}]
mylist = [ 42, 37, 42, 66, 15 ]
#                      [0] [1] [2] [3] [4]
print(mylist[:])     # 42, 37, 42, 66, 15
print(mylist[-1])    #                 15
print(mylist[:-1])   # 42, 37, 42, 66
print(mylist[2:-1])


#                     [-5][-4][-3][-2][-1] \end{lstlisting}
      \end{onlyenv}

      \begin{onlyenv}<5>
        % Balises exception :  %* *)
        \begin{lstlisting}[style=python,morekeywords={for, in, range, list}]
mylist = [ 42, 37, 42, 66, 15 ]
#                      [0] [1] [2] [3] [4]
print(mylist[:])     # 42, 37, 42, 66, 15
print(mylist[-1])    #                 15
print(mylist[:-1])   # 42, 37, 42, 66
print(mylist[2:-1])  #         42, 66
print(mylist[-4:-1])

#                     [-5][-4][-3][-2][-1] \end{lstlisting}
      \end{onlyenv}

      \begin{onlyenv}<6>
        % Balises exception :  %* *)
        \begin{lstlisting}[style=python,morekeywords={for, in, range, list}]
mylist = [ 42, 37, 42, 66, 15 ]
#                      [0] [1] [2] [3] [4]
print(mylist[:])     # 42, 37, 42, 66, 15
print(mylist[-1])    #                 15
print(mylist[:-1])   # 42, 37, 42, 66
print(mylist[2:-1])  #         42, 66
print(mylist[-4:-1]) #     37, 42, 66
print(mylist[-4:4])
#                     [-5][-4][-3][-2][-1] \end{lstlisting}
      \end{onlyenv}

      \begin{onlyenv}<7->
        % Balises exception :  %* *)
        \begin{lstlisting}[style=python,morekeywords={for, in, range, list}]
mylist = [ 42, 37, 42, 66, 15 ]
#                      [0] [1] [2] [3] [4]
print(mylist[:])     # 42, 37, 42, 66, 15
print(mylist[-1])    #                 15
print(mylist[:-1])   # 42, 37, 42, 66
print(mylist[2:-1])  #         42, 66
print(mylist[-4:-1]) #     37, 42, 66
print(mylist[-4:4])  #     37, 42, 66
#                     [-5][-4][-3][-2][-1] \end{lstlisting}
      \end{onlyenv}

    \end{column}
  \end{columns}

  \end{center}

\end{frame}


%%% Shallow copy and Deep copy
\subsubsection{References, Shallow copy, Deep copy}


\begin{frame}<beamer>{Sequence operations}

  \onslide<1-> Beware of \textbf{references}!

  \bigskip

  \onslide<2-> Python does not copy everything

  \onslide<3-> It "\textit{creates bindings between a target and an object}"

  \bigskip

  \onslide<4-> Modifying a sequence will change the assigned values in other sequences where it is referenced

\end{frame}


\begin{frame}[fragile]{Sequence operations}

  \onslide<1-> \TTBF{id()} is a builtin that gives the identifier of an object

  \onslide<2-> {\footnotesize \textit{(in a specific context, it gives the memory address of the object)} }

  \begin{center}

  \begin{columns}[onlytextwidth]
    \begin{column}{\textwidth}

      \begin{onlyenv}<3>
        % Balises exception :  %* *)
        \begin{lstlisting}[style=python,morekeywords={for, in, range, list}]









 \end{lstlisting}
      \end{onlyenv}

      \begin{onlyenv}<4>
        % Balises exception :  %* *)
        \begin{lstlisting}[style=python,morekeywords={for, in, range, list}]
L1 = [ 42, 37, 42 ]     #  [0] [1] [2] [3] [4]
L2 = L1                 #  42, 37, 42







                        #  [0] [1] [2] [3] [4] \end{lstlisting}
      \end{onlyenv}

      \begin{onlyenv}<5>
        % Balises exception :  %* *)
        \begin{lstlisting}[style=python,morekeywords={for, in, range, list}]
L1 = [ 42, 37, 42 ]     #  [0] [1] [2] [3] [4]
L2 = L1                 #  42, 37, 42
print(id(L1))
print(id(L2))
print(L1 == L2)
print(id(L1) == id(L2))



                        #  [0] [1] [2] [3] [4] \end{lstlisting}
      \end{onlyenv}

      \begin{onlyenv}<6>
        % Balises exception :  %* *)
        \begin{lstlisting}[style=python,morekeywords={for, in, range, list}]
L1 = [ 42, 37, 42 ]     #  [0] [1] [2] [3] [4]
L2 = L1                 #  42, 37, 42
print(id(L1))
print(id(L2))
print(L1 == L2)         #  True
print(id(L1) == id(L2)) #  True



                        #  [0] [1] [2] [3] [4] \end{lstlisting}
      \end{onlyenv}

      \begin{onlyenv}<7>
        % Balises exception :  %* *)
        \begin{lstlisting}[style=python,morekeywords={for, in, range, list}]
L1 = [ 42, 37, 42 ]     #  [0] [1] [2] [3] [4]
L2 = L1                 #  42, 37, 42
print(id(L1))
print(id(L2))
print(L1 == L2)         #  True
print(id(L1) == id(L2)) #  True
L2[0] = 88
print(L1)
print(L2)
                        #  [0] [1] [2] [3] [4] \end{lstlisting}
      \end{onlyenv}

      \begin{onlyenv}<8->
        % Balises exception :  %* *)
        \begin{lstlisting}[style=python,morekeywords={for, in, range, list}]
L1 = [ 42, 37, 42 ]     #  [0] [1] [2] [3] [4]
L2 = L1                 #  42, 37, 42
print(id(L1))
print(id(L2))
print(L1 == L2)         #  True
print(id(L1) == id(L2)) #  True
L2[0] = 88
print(L1)               #  88, 37, 42
print(L2)               #  88, 37, 42
                        #  [0] [1] [2] [3] [4] \end{lstlisting}
      \end{onlyenv}

    \end{column}
  \end{columns}

  \end{center}

\end{frame}


\begin{frame}[fragile]{Sequence operations}

  \onslide<1-> Lists are not copied, only their references are used \phantom{\TTBF{id()}}

  \onslide<1-> { \footnotesize  \phantom{\textit{(list)}} }

  \begin{center}

  \begin{columns}[onlytextwidth]
    \begin{column}{\textwidth}

      \begin{onlyenv}<1->
        % Balises exception :  %* *)
        \begin{lstlisting}[style=python,morekeywords={for, in, range, list}]
L1 = [ 42, 37, 42 ]     #  [0] [1] [2] [3] [4]
L2 = L1                 #  42, 37, 42
print(id(L1))
print(id(L2))
print(L1 == L2)         #  True
print(id(L1) == id(L2)) #  True
L2[0] = 88
print(L1)               #  88, 37, 42
print(L2)               #  88, 37, 42
                        #  [0] [1] [2] [3] [4] \end{lstlisting}
      \end{onlyenv}

    \end{column}
  \end{columns}

  \end{center}

\end{frame}



\begin{frame}[fragile]{Sequence operations}

  \begin{center}

  \begin{columns}[onlytextwidth]
    \begin{column}{\textwidth}

      \begin{onlyenv}<1>
        % Balises exception :  %* *)
        \begin{lstlisting}[style=python,morekeywords={for, in, range, list}]
L1 = [ 2, 7, [4, 3] ]   #  [0] [1] [2] [3] [4]









                        #  [0] [1] [2] [3] [4] \end{lstlisting}
      \end{onlyenv}

      \begin{onlyenv}<2>
        % Balises exception :  %* *)
        \begin{lstlisting}[style=python,morekeywords={for, in, range, list}]
L1 = [ 2, 7, [4, 3] ]   #  [0] [1] [2] [3] [4]
L2 = L1.copy()          #   2   7 [4,3]
print(L1 == L2)
print(id(L1) == id(L2))






                        #  [0] [1] [2] [3] [4] \end{lstlisting}
      \end{onlyenv}

      \begin{onlyenv}<3>
        % Balises exception :  %* *)
        \begin{lstlisting}[style=python,morekeywords={for, in, range, list}]
L1 = [ 2, 7, [4, 3] ]   #  [0] [1] [2] [3] [4]
L2 = L1.copy()          #   2   7 [4,3]
print(L1 == L2)         #  True
print(id(L1) == id(L2)) #  False
L2[0] = 8





                        #  [0] [1] [2] [3] [4] \end{lstlisting}
      \end{onlyenv}

      \begin{onlyenv}<4>
        % Balises exception :  %* *)
        \begin{lstlisting}[style=python,morekeywords={for, in, range, list}]
L1 = [ 2, 7, [4, 3] ]   #  [0] [1] [2] [3] [4]
L2 = L1.copy()          #   2   7 [4,3]
print(L1 == L2)         #  True
print(id(L1) == id(L2)) #  False
L2[0] = 8
print(L1)
print(L2)



                        #  [0] [1] [2] [3] [4] \end{lstlisting}
      \end{onlyenv}

      \begin{onlyenv}<5>
        % Balises exception :  %* *)
        \begin{lstlisting}[style=python,morekeywords={for, in, range, list}]
L1 = [ 2, 7, [4, 3] ]   #  [0] [1] [2] [3] [4]
L2 = L1.copy()          #   2   7 [4,3]
print(L1 == L2)         #  True
print(id(L1) == id(L2)) #  False
L2[0] = 8
print(L1)               #   2   7 [4,3]
print(L2)               #   8   7 [4,3]
L2[2][0] = 9
print(L1)
print(L2)
                        #  [0] [1] [2] [3] [4] \end{lstlisting}
      \end{onlyenv}

      \begin{onlyenv}<6->
        % Balises exception :  %* *)
        \begin{lstlisting}[style=python,morekeywords={for, in, range, list}]
L1 = [ 2, 7, [4, 3] ]   #  [0] [1] [2] [3] [4]
L2 = L1.copy()          #   2   7 [4,3]
print(L1 == L2)         #  True
print(id(L1) == id(L2)) #  False
L2[0] = 8
print(L1)               #   2   7 [4,3]
print(L2)               #   8   7 [4,3]
L2[2][0] = 9
print(L1)               #   2   7 [9,3]
print(L2)               #   8   7 [9,3]
                        #  [0] [1] [2] [3] [4] \end{lstlisting}
      \end{onlyenv}

    \end{column}
  \end{columns}

  \end{center}

\end{frame}


\begin{frame}<beamer>{Sequence operations}

  \onslide<1-> Module \textit{copy}

  \medskip

  \begin{itemize}
    \item<2-> \textbf{Shallow copy}: creates a new sequence and inserts references to each contained object of the original sequence
    \item<3-> \textbf{Deep copy}: creates a new sequence and recursively inserts copies of each object of the original sequence
  \end{itemize}

  \bigskip

  \onslide<4-> \TTBF{NewList1 = list(MyList)} works as a shallow copy

  \onslide<5-> \TTBF{NewList2 = MyList[:]} works as a shallow copy

  \onslide<6-> \TTBF{NewList3 = MyList.copy()} works as a shallow copy

  %\bigskip

  %\onslide<6-> \TTBF{list.copy()} and \TTBF{list[:]} works like shallow copies at different levels
\end{frame}


\begin{frame}[fragile]{Sequence operations}

  \onslide<1-> Major difficulties concern containers containing containers

  \begin{center}

  \begin{columns}[onlytextwidth]
    \begin{column}{\textwidth}

      \begin{onlyenv}<1>
        % Balises exception :  %* *)
        \begin{lstlisting}[style=python,morekeywords={for, in, range, list}]
import copy

L1 = [ [1,1], [2,2], [3,3] ]






 \end{lstlisting}
      \end{onlyenv}

      \begin{onlyenv}<2>
        % Balises exception :  %* *)
        \begin{lstlisting}[style=python,morekeywords={for, in, range, list}]
import copy

L1 = [ [1,1], [2,2], [3,3] ]
L2 = copy.copy(L1)     # shallow copy
L3 = copy.deepcopy(L1) # deep copy




 \end{lstlisting}
      \end{onlyenv}

      \begin{onlyenv}<3>
        % Balises exception :  %* *)
        \begin{lstlisting}[style=python,morekeywords={for, in, range, list}]
import copy

L1 = [ [1,1], [2,2], [3,3] ]
L2 = copy.copy(L1)     # shallow copy
L3 = copy.deepcopy(L1) # deep copy
L1[1][1] = 9
print(L2)
print(L3)

 \end{lstlisting}
      \end{onlyenv}

      \begin{onlyenv}<4>
        % Balises exception :  %* *)
        \begin{lstlisting}[style=python,morekeywords={for, in, range, list}]
import copy

L1 = [ [1,1], [2,2], [3,3] ]
L2 = copy.copy(L1)     # shallow copy
L3 = copy.deepcopy(L1) # deep copy
L1[1][1] = 9
print(L2)   # [ [1,1], [2,9], [3,3] ]
print(L3)   # [ [1,1], [2,2], [3,3] ]

 \end{lstlisting}
      \end{onlyenv}

      \begin{onlyenv}<5>
        % Balises exception :  %* *)
        \begin{lstlisting}[style=python,morekeywords={for, in, range, list}]
import copy

L1 = [ [1,1], [2,2], [3,3] ]
L2 = copy.copy(L1)     # shallow copy
L3 = copy.deepcopy(L1) # deep copy
L1[1][1] = 9
print(L2)   # [ [1,1], [2,9], [3,3] ]
print(L3)   # [ [1,1], [2,2], [3,3] ]
L1[1] = [0,0]
print(L2) \end{lstlisting}
      \end{onlyenv}

      \begin{onlyenv}<6->
        % Balises exception :  %* *)
        \begin{lstlisting}[style=python,morekeywords={for, in, range, list}]
import copy

L1 = [ [1,1], [2,2], [3,3] ]
L2 = copy.copy(L1)     # shallow copy
L3 = copy.deepcopy(L1) # deep copy
L1[1][1] = 9
print(L2)   # [ [1,1], [2,9], [3,3] ]
print(L3)   # [ [1,1], [2,2], [3,3] ]
L1[1] = [0,0]
print(L2)   # [ [1,1], [2,9], [3,3] ] \end{lstlisting}
      \end{onlyenv}

    \end{column}
  \end{columns}

  \end{center}

\end{frame}



%%% Lists
\subsection{Lists}

\begin{frame}<beamer>{Lists}

  \begin{itemize}
    \item<1-> mutable (can be modified after creation)
    \begin{itemize}
      \item<2-> \textit{beware of shallow and deep copies}
    \end{itemize}
    \item<3-> ordered (only the functions of reorganization change the order)
    \item<4-> same value can be recorded multiple times
    \item<5-> different types of items are allowed in the same list
  \end{itemize}

\end{frame}


\begin{frame}[fragile]{Lists}

  \begin{center}

  Creation of a list:

  \medskip

  \begin{columns}[onlytextwidth]
    \begin{column}{\textwidth}

      \begin{onlyenv}<1>
        % Balises exception :  %* *)
        \begin{lstlisting}[style=python,morekeywords={for, in, range, list}]
vide = []





 \end{lstlisting}
      \end{onlyenv}

      \begin{onlyenv}<2>
        % Balises exception :  %* *)
        \begin{lstlisting}[style=python,morekeywords={for, in, range, list}]
vide = []
initialisee1 = [ 42 ]
initialisee2 = [ 42, 1337, 666 ]



 \end{lstlisting}
      \end{onlyenv}

      \begin{onlyenv}<3>
        % Balises exception :  %* *)
        \begin{lstlisting}[style=python,morekeywords={for, in, range, list}]
vide = []
initialisee1 = [ 42 ]
initialisee2 = [ 42, 1337, 666 ]
comprehension = [ x for x in range(0, 10) ]


 \end{lstlisting}
      \end{onlyenv}

      \begin{onlyenv}<4->
        % Balises exception :  %* *)
        \begin{lstlisting}[style=python,morekeywords={for, in, range, list}]
vide = []
initialisee1 = [ 42 ]
initialisee2 = [ 42, 1337, 666 ]
comprehension = [ x for x in range(0, 10) ]
constructeurVide = list()
constructeur1 = list("abc")
constructeur2 = list( (58, "abc", 58) ) \end{lstlisting}
      \end{onlyenv}

    \end{column}
  \end{columns}

  \medskip

  \begin{itemize}
    \item<5-> list initialized by comprehension or with a constructor can use any \textit{iterable} object
  \end{itemize}

  \end{center}

\end{frame}


%%% List operations

\begin{frame}<beamer>{Lists operations}

  \begin{center}

  \begin{tabular}{| c | l |}
    \hline
    \textbf{Operation} & \textbf{Result} \\
    \hline
    list.append(x) 		& add an item \textit{x} to the end of the list \\
    list.insert(pos, x) 	& insert an item \textit{x} at position \textit{pos} \\
     					& {\footnotesize(items are pushed back/+1 applied to indexes) } \\
    list.extend(L) 		& concatenates lists : \textit{L} added at the end of \textit{list} \\
     					& {\footnotesize(any iterable objects can be concatenated) } \\
    list.pop(pos) 		& remove the item at position \textit{pos} \\
     					& {\footnotesize(if no parameter given, it removes the last one) } \\
    list.remove(x) 		& remove the first item \textit{x} found in the list \\
    list.copy() 		& copy the list \\
    list.reverse() 		& reverse the order of the list \\
    list.sort() 		& sort alphabetically the list \\
     					& {\footnotesize(add the parameter \textit{reverse = True} to reverse) } \\
    \hline
  \end{tabular}

%    $ \text{s}[\text{i}] $ 					& access to the \textit{i}th item of \textit{s} (first index: \textit{0}) \\
%    $ \text{s}[\text{i}:\text{j}] $ 			& slice of \textit{s} from \textit{i} to \textit{j} \\
%    $ \text{s}[\text{i}:\text{j}:\text{k}] $ 	& slice of \textit{s} from \textit{i} to \textit{j} with step \textit{k} \\

  \medskip

  \textit{x} is an arbitrary object, \textit{pos} is an integer, and \textit{L} is a list

  \end{center}

\end{frame}


\begin{frame}[fragile]{Lists}

  \begin{center}

  \begin{columns}[onlytextwidth]
    \begin{column}{\textwidth}

      \begin{onlyenv}<1>
        % Balises exception :  %* *)
        \begin{lstlisting}[style=python,morekeywords={for, in, range, list}]
mylist = [ 42, 37, 42 ] # [0] [1] [2] [3] [4]
mylist.append(66)








       \end{lstlisting}
      \end{onlyenv}

      \begin{onlyenv}<2>
        % Balises exception :  %* *)
        \begin{lstlisting}[style=python,morekeywords={for, in, range, list}]
mylist = [ 42, 37, 42 ] # [0] [1] [2] [3] [4]
mylist.append(66)       # 42, 37, 42, 66
mylist.insert(1, 99)







       \end{lstlisting}
      \end{onlyenv}

      \begin{onlyenv}<3>
        % Balises exception :  %* *)
        \begin{lstlisting}[style=python,morekeywords={for, in, range, list}]
mylist = [ 42, 37, 42 ] # [0] [1] [2] [3] [4]
mylist.append(66)       # 42, 37, 42, 66
mylist.insert(1, 99)    # 42, 99, 37, 42, 66
mylist.remove(42)






       \end{lstlisting}
      \end{onlyenv}

      \begin{onlyenv}<4>
        % Balises exception :  %* *)
        \begin{lstlisting}[style=python,morekeywords={for, in, range, list}]
mylist = [ 42, 37, 42 ] # [0] [1] [2] [3] [4]
mylist.append(66)       # 42, 37, 42, 66
mylist.insert(1, 99)    # 42, 99, 37, 42, 66
mylist.remove(42)       # 99, 37, 42, 66
mylist.reverse()





       \end{lstlisting}
      \end{onlyenv}

      \begin{onlyenv}<5>
        % Balises exception :  %* *)
        \begin{lstlisting}[style=python,morekeywords={for, in, range, list}]
mylist = [ 42, 37, 42 ] # [0] [1] [2] [3] [4]
mylist.append(66)       # 42, 37, 42, 66
mylist.insert(1, 99)    # 42, 99, 37, 42, 66
mylist.remove(42)       # 99, 37, 42, 66
mylist.reverse()        # 66, 42, 37, 99
mylist.sort()




       \end{lstlisting}
      \end{onlyenv}

      \begin{onlyenv}<6>
        % Balises exception :  %* *)
        \begin{lstlisting}[style=python,morekeywords={for, in, range, list}]
mylist = [ 42, 37, 42 ] # [0] [1] [2] [3] [4]
mylist.append(66)       # 42, 37, 42, 66
mylist.insert(1, 99)    # 42, 99, 37, 42, 66
mylist.remove(42)       # 99, 37, 42, 66
mylist.reverse()        # 66, 42, 37, 99
mylist.sort()           # 37, 42, 66, 99
mylist.pop()



       \end{lstlisting}
      \end{onlyenv}

      \begin{onlyenv}<7>
        % Balises exception :  %* *)
        \begin{lstlisting}[style=python,morekeywords={for, in, range, list}]
mylist = [ 42, 37, 42 ] # [0] [1] [2] [3] [4]
mylist.append(66)       # 42, 37, 42, 66
mylist.insert(1, 99)    # 42, 99, 37, 42, 66
mylist.remove(42)       # 99, 37, 42, 66
mylist.reverse()        # 66, 42, 37, 99
mylist.sort()           # 37, 42, 66, 99
mylist.pop()            # 37, 42, 66
mylist.pop(1)


       \end{lstlisting}
      \end{onlyenv}

      \begin{onlyenv}<8>
        % Balises exception :  %* *)
        \begin{lstlisting}[style=python,morekeywords={for, in, range, list}]
mylist = [ 42, 37, 42 ] # [0] [1] [2] [3] [4]
mylist.append(66)       # 42, 37, 42, 66
mylist.insert(1, 99)    # 42, 99, 37, 42, 66
mylist.remove(42)       # 99, 37, 42, 66
mylist.reverse()        # 66, 42, 37, 99
mylist.sort()           # 37, 42, 66, 99
mylist.pop()            # 37, 42, 66
mylist.pop(1)           # 37, 66
l2 = mylist.copy()
l2.append(88)
mylist.extend(l2) \end{lstlisting}
      \end{onlyenv}

      \begin{onlyenv}<9->
        % Balises exception :  %* *)
        \begin{lstlisting}[style=python,morekeywords={for, in, range, list}]
mylist = [ 42, 37, 42 ] # [0] [1] [2] [3] [4]
mylist.append(66)       # 42, 37, 42, 66
mylist.insert(1, 99)    # 42, 99, 37, 42, 66
mylist.remove(42)       # 99, 37, 42, 66
mylist.reverse()        # 66, 42, 37, 99
mylist.sort()           # 37, 42, 66, 99
mylist.pop()            # 37, 42, 66
mylist.pop(1)           # 37, 66
l2 = mylist.copy()  # 37, 66
l2.append(88)       # 37, 66, 88
mylist.extend(l2)       # 37, 66, 37, 66, 88 \end{lstlisting}
      \end{onlyenv}

    \end{column}
  \end{columns}

  \end{center}

\end{frame}


%%% Tuples
\subsection{Tuples}

\begin{frame}<beamer>{Tuples}

  \begin{itemize}
    \item<1-> immutable (when created, cannot be modified)
    \item<2-> ordered
    \item<3-> same value can be recorded multiple times
    \item<4-> different types of items are allowed in the same tuple
  \end{itemize}

\end{frame}


\begin{frame}[fragile]{Tuples}

  \begin{center}

  Creation of a tuple:

  \medskip

  \begin{columns}[onlytextwidth]
    \begin{column}{\textwidth}

      \begin{onlyenv}<1>
        % Balises exception :  %* *)
        \begin{lstlisting}[style=python,morekeywords={for, in, range, list}]
vide = ()




 \end{lstlisting}
      \end{onlyenv}

      \begin{onlyenv}<2>
        % Balises exception :  %* *)
        \begin{lstlisting}[style=python,morekeywords={for, in, range, list}]
vide = ()
initialisee1 = ( 42, )
initialisee2 = ( 42, 1337, 666 )


 \end{lstlisting}
      \end{onlyenv}

      \begin{onlyenv}<3->
        % Balises exception :  %* *)
        \begin{lstlisting}[style=python,morekeywords={for, in, range, list}]
vide = ()
initialisee1 = ( 42, )
initialisee2 = ( 42, 1337, 666 )
constructeurVide = tuple()
constructeur1 = tuple("abc",)
constructeur2 = tuple( (58, "abc", 58) ) \end{lstlisting}
      \end{onlyenv}

    \end{column}
  \end{columns}

  \medskip

  \begin{itemize}
    \item<4-> Tuple with only $ 1 $ element must have a comma (\TTBF{,})
    \item<5-> Tuples can be initialized with any iterable object
  \end{itemize}

  \end{center}

\end{frame}



\begin{frame}[fragile]{Tuples}

  \onslide<1-> Creating a tuple requires commas OR the iterable property

  \onslide<2-> Parenthesis are mandatory when a tuple is used as an argument or when creating an empty tuple

  \medskip

  \begin{center}

  \begin{columns}[onlytextwidth]
    \begin{column}{\textwidth}

      \begin{onlyenv}<3>
        % Balises exception :  %* *)
        \begin{lstlisting}[style=python,morekeywords={for, in, range, list}]




 \end{lstlisting}
      \end{onlyenv}

      \begin{onlyenv}<4>
        % Balises exception :  %* *)
        \begin{lstlisting}[style=python,morekeywords={for, in, range, list}]
my_tuple = tuple('a', 2)     # ('a', 2)



 \end{lstlisting}
      \end{onlyenv}

      \begin{onlyenv}<5>
        % Balises exception :  %* *)
        \begin{lstlisting}[style=python,morekeywords={for, in, range, list}]
my_tuple = tuple('a', 2)     # ('a', 2)
nb_tuple = tuple([1, 2, 3])


 \end{lstlisting}
      \end{onlyenv}

      \begin{onlyenv}<6>
        % Balises exception :  %* *)
        \begin{lstlisting}[style=python,morekeywords={for, in, range, list}]
my_tuple = tuple('a', 2)     # ('a', 2)
nb_tuple = tuple([1, 2, 3])  # (1, 2, 3)
str_tuple = tuple('abc')

 \end{lstlisting}
      \end{onlyenv}

      \begin{onlyenv}<7>
        % Balises exception :  %* *)
        \begin{lstlisting}[style=python,morekeywords={for, in, range, list}]
my_tuple = tuple('a', 2)     # ('a', 2)
nb_tuple = tuple([1, 2, 3])  # (1, 2, 3)
str_tuple = tuple('abc')     # ('a', 'b', 'c')
f('a', 'b', 'c')
f( ('a','b','c') ) \end{lstlisting}
      \end{onlyenv}

      \begin{onlyenv}<8->
        % Balises exception :  %* *)
        \begin{lstlisting}[style=python,morekeywords={for, in, range, list}]
my_tuple = tuple('a', 2)     # ('a', 2)
nb_tuple = tuple([1, 2, 3])  # (1, 2, 3)
str_tuple = tuple('abc')     # ('a', 'b', 'c')
f('a', 'b', 'c')   # Calls f with 3 arguments
f( ('a','b','c') ) # Calls f with 1 argument \end{lstlisting}
      \end{onlyenv}

    \end{column}
  \end{columns}

  \end{center}

\end{frame}



%%% Ranges
\subsection{Ranges}

\begin{frame}<beamer>{Ranges}

  \begin{itemize}
    \item<1-> immutable (when created, cannot be modified)
    \item<2-> ordered
    \item<3-> same value can be recorded multiple times
    \item<4-> only for integers numbers
    \item<5-> always takes the same amount of memory
    \begin{itemize}
      \item<6-> only 3 parameters are used (\TTBF{start}, \TTBF{stop}, \TTBF{step})
    \end{itemize}
  \end{itemize}

\end{frame}


\begin{frame}[fragile]{Ranges}

  \onslide<1-> Ranges are defined at least by the \textit{stop} boundary

  \onslide<2-> \textit{Start} defines the first value to take, by default it is 0

  \onslide<2-> \textit{Step} defines the step between each value, by default it is 1

  \medskip

  \begin{center}

  \begin{columns}[onlytextwidth]
    \begin{column}{\textwidth}

      \begin{onlyenv}<3>
        % Balises exception :  %* *)
        \begin{lstlisting}[style=python,morekeywords={for, in, range, list}]


 \end{lstlisting}
      \end{onlyenv}

      \begin{onlyenv}<4>
        % Balises exception :  %* *)
        \begin{lstlisting}[style=python,morekeywords={for, in, range, list}]
list(range(6))

 \end{lstlisting}
      \end{onlyenv}

      \begin{onlyenv}<5>
        % Balises exception :  %* *)
        \begin{lstlisting}[style=python,morekeywords={for, in, range, list}]
list(range(6))       # 0 1 2 3 4 5
list(range(1, 6))
  \end{lstlisting}
      \end{onlyenv}

      \begin{onlyenv}<6>
        % Balises exception :  %* *)
        \begin{lstlisting}[style=python,morekeywords={for, in, range, list}]
list(range(6))       # 0 1 2 3 4 5
list(range(1, 6))    #   1 2 3 4 5
list(range(0, 6, 2)) \end{lstlisting}
      \end{onlyenv}

      \begin{onlyenv}<7->
        % Balises exception :  %* *)
        \begin{lstlisting}[style=python,morekeywords={for, in, range, list}]
list(range(6))       # 0 1 2 3 4 5
list(range(1, 6))    #   1 2 3 4 5
list(range(0, 6, 2)) # 0   2   4 \end{lstlisting}
      \end{onlyenv}

    \end{column}
  \end{columns}

  \end{center}

\end{frame}


\begin{frame}[fragile]{Ranges}

  \onslide<1-> Main range formulae: $ r[i] = \text{start} + \text{step} \cdot i $

  \onslide<2-> Constraint: $ i \geq 0 $ and $ r[i] < \text{stop} $

  \onslide<3-> Ranges can go backward with negative steps!

  \onslide<4-> Constraint (negative step): $ i \geq 0 $ and $ r[i] > \text{stop} $

  \medskip

  \begin{center}

  \begin{columns}[onlytextwidth]
    \begin{column}{\textwidth}

      \begin{onlyenv}<5>
        % Balises exception :  %* *)
        \begin{lstlisting}[style=python,morekeywords={for, in, range, list}]





 \end{lstlisting}
      \end{onlyenv}

      \begin{onlyenv}<6>
        % Balises exception :  %* *)
        \begin{lstlisting}[style=python,morekeywords={for, in, range, list}]
list(range(-5, 0))




 \end{lstlisting}
      \end{onlyenv}

      \begin{onlyenv}<7>
        % Balises exception :  %* *)
        \begin{lstlisting}[style=python,morekeywords={for, in, range, list}]
list(range(-5, 0))     # -5 -4 -3 -2 -1
list(range(0, -5, -1))



 \end{lstlisting}
      \end{onlyenv}

      \begin{onlyenv}<8>
        % Balises exception :  %* *)
        \begin{lstlisting}[style=python,morekeywords={for, in, range, list}]
list(range(-5, 0))     # -5 -4 -3 -2 -1
list(range(0, -5, -1)) #  0 -1 -2 -3 -4
list(range(5,  0, -1))


 \end{lstlisting}
      \end{onlyenv}

      \begin{onlyenv}<9>
        % Balises exception :  %* *)
        \begin{lstlisting}[style=python,morekeywords={for, in, range, list}]
list(range(-5, 0))     # -5 -4 -3 -2 -1
list(range(0, -5, -1)) #  0 -1 -2 -3 -4
list(range(5,  0, -1)) #  5  4  3  2  1
list(range(0))

 \end{lstlisting}
      \end{onlyenv}

      \begin{onlyenv}<10>
        % Balises exception :  %* *)
        \begin{lstlisting}[style=python,morekeywords={for, in, range, list}]
list(range(-5, 0))     # -5 -4 -3 -2 -1
list(range(0, -5, -1)) #  0 -1 -2 -3 -4
list(range(5,  0, -1)) #  5  4  3  2  1
list(range(0))         # []  step+  0 == stop
list(range(1,  0))
 \end{lstlisting}
      \end{onlyenv}

      \begin{onlyenv}<11>
        % Balises exception :  %* *)
        \begin{lstlisting}[style=python,morekeywords={for, in, range, list}]
list(range(-5, 0))     # -5 -4 -3 -2 -1
list(range(0, -5, -1)) #  0 -1 -2 -3 -4
list(range(5,  0, -1)) #  5  4  3  2  1
list(range(0))         # []  step+  0 == stop
list(range(1,  0))     # []  step+  1 > stop
list(range(0, -1)) \end{lstlisting}
      \end{onlyenv}

      \begin{onlyenv}<12->
        % Balises exception :  %* *)
        \begin{lstlisting}[style=python,morekeywords={for, in, range, list}]
list(range(-5, 0))     # -5 -4 -3 -2 -1
list(range(0, -5, -1)) #  0 -1 -2 -3 -4
list(range(5,  0, -1)) #  5  4  3  2  1
list(range(0))         # []  step+  0 == stop
list(range(1,  0))     # []  step+  1 > stop
list(range(0, -1))     # []  step+  0 > stop \end{lstlisting}
      \end{onlyenv}

    \end{column}
  \end{columns}

  \end{center}

\end{frame}


\begin{frame}[fragile]{Ranges}

  \onslide<1-> Ranges are a type, and can be compared with \TTBF{==} and \TTBF{!=}

  \onslide<2-> They are equal if they represent the same sequence of values

  \onslide<3-> (\TTBF{start}, \TTBF{stop}, and \TTBF{step} might be differents)

  \medskip

  \begin{center}

  \begin{columns}[onlytextwidth]
    \begin{column}{\textwidth}

      \begin{onlyenv}<4>
        % Balises exception :  %* *)
        \begin{lstlisting}[style=python,morekeywords={for, in, range, list}]
r0 = range(0)
r1 = range(0, 5, 1)
r2 = range(0, 3, 2)
r3 = range(0, 4, 2)


 \end{lstlisting}
      \end{onlyenv}

      \begin{onlyenv}<5>
        % Balises exception :  %* *)
        \begin{lstlisting}[style=python,morekeywords={for, in, range, list}]
r0 = range(0)          # []
r1 = range(0, 5, 1)    # 0 1 2 3 4
r2 = range(0, 3, 2)    # 0     3
r3 = range(0, 4, 2)    # 0     3


 \end{lstlisting}
      \end{onlyenv}

      \begin{onlyenv}<6>
        % Balises exception :  %* *)
        \begin{lstlisting}[style=python,morekeywords={for, in, range, list}]
r0 = range(0)          # []
r1 = range(0, 5, 1)    # 0 1 2 3 4
r2 = range(0, 3, 2)    # 0     3
r3 = range(0, 4, 2)    # 0     3
print(r0 == r1)

 \end{lstlisting}
      \end{onlyenv}

      \begin{onlyenv}<7>
        % Balises exception :  %* *)
        \begin{lstlisting}[style=python,morekeywords={for, in, range, list}]
r0 = range(0)          # []
r1 = range(0, 5, 1)    # 0 1 2 3 4
r2 = range(0, 3, 2)    # 0     3
r3 = range(0, 4, 2)    # 0     3
print(r0 == r1)    # False
print(r1 == r2)
 \end{lstlisting}
      \end{onlyenv}

      \begin{onlyenv}<8>
        % Balises exception :  %* *)
        \begin{lstlisting}[style=python,morekeywords={for, in, range, list}]
r0 = range(0)          # []
r1 = range(0, 5, 1)    # 0 1 2 3 4
r2 = range(0, 3, 2)    # 0     3
r3 = range(0, 4, 2)    # 0     3
print(r0 == r1)    # False
print(r1 == r2)    # False
print(r2 == r3) \end{lstlisting}
      \end{onlyenv}

      \begin{onlyenv}<9->
        % Balises exception :  %* *)
        \begin{lstlisting}[style=python,morekeywords={for, in, range, list}]
r0 = range(0)          # []
r1 = range(0, 5, 1)    # 0 1 2 3 4
r2 = range(0, 3, 2)    # 0     3
r3 = range(0, 4, 2)    # 0     3
print(r0 == r1)    # False
print(r1 == r2)    # False
print(r2 == r3)    # True \end{lstlisting}
      \end{onlyenv}

    \end{column}
  \end{columns}

  \end{center}

\end{frame}


%%% Mapping types
\section{Mapping types}

\begin{frame}<beamer>{Mapping types}

  \begin{itemize}
    \item<1-> dict \textit{(dictionnary)}
  \end{itemize}

  \begin{itemize}
    \item<2-> "\textit{maps hashable values to arbitrary objects}"
    \begin{itemize}
      \item<3-> keys = hashable values
      \item<4-> keys must be hashable (a list can't be a key)
      \item<5-> if two values are equal, they refer to the same key
    \end{itemize}
    \item<6-> mutable
  \end{itemize}

%  \TTBF{key: value}

\end{frame}

%% Dict
\subsection{Dictionnaries}

\begin{frame}[fragile]{Dictionnaries}

  \begin{center}

  Creation of a dict:

  \medskip

  \begin{columns}[onlytextwidth]
    \begin{column}{\textwidth}

      \begin{onlyenv}<1>
        % Balises exception :  %* *)
        \begin{lstlisting}[style=python,morekeywords={for, in, range, list},basicstyle=\ttfamily\small]





 \end{lstlisting}
      \end{onlyenv}

      \begin{onlyenv}<2>
        % Balises exception :  %* *)
        \begin{lstlisting}[style=python,morekeywords={for, in, range, list},basicstyle=\ttfamily\small]
vide = dict()
dict1 = {"one": 1, "two": 2, "three": 3}



 \end{lstlisting}
      \end{onlyenv}

      \begin{onlyenv}<3>
        % Balises exception :  %* *)
        \begin{lstlisting}[style=python,morekeywords={for, in, range, list},basicstyle=\ttfamily\small]
vide = dict()
dict1 = {"one": 1, "two": 2, "three": 3}
dict2 = dict(one=1, two=2, three=3)


 \end{lstlisting}
      \end{onlyenv}

      \begin{onlyenv}<4>
        % Balises exception :  %* *)
        \begin{lstlisting}[style=python,morekeywords={for, in, range, list},basicstyle=\ttfamily\small]
vide = dict()
dict1 = {"one": 1, "two": 2, "three": 3}
dict2 = dict(one=1, two=2, three=3)
dict3 = dict({'three': 3, 'one': 1, 'two': 2})

 \end{lstlisting}
      \end{onlyenv}

      \begin{onlyenv}<5>
        % Balises exception :  %* *)
        \begin{lstlisting}[style=python,morekeywords={for, in, range, list},basicstyle=\ttfamily\small]
vide = dict()
dict1 = {"one": 1, "two": 2, "three": 3}
dict2 = dict(one=1, two=2, three=3)
dict3 = dict({'three': 3, 'one': 1, 'two': 2})
dict4 = dict({'one': 1, 'three': 3}, two=2)
 \end{lstlisting}
      \end{onlyenv}

      \begin{onlyenv}<6->
        % Balises exception :  %* *)
        \begin{lstlisting}[style=python,morekeywords={for, in, range, list},basicstyle=\ttfamily\small]
vide = dict()
dict1 = {"one": 1, "two": 2, "three": 3}
dict2 = dict(one=1, two=2, three=3)
dict3 = dict({'three': 3, 'one': 1, 'two': 2})
dict4 = dict({'one': 1, 'three': 3}, two=2)
dict5 = dict([('two',2),('one',1), ('three',3)]) \end{lstlisting}
      \end{onlyenv}

    \end{column}
  \end{columns}

  \end{center}

\end{frame}


%%% Dicts operations

\begin{frame}<beamer>{Dicts operations}

  \begin{center}

  \begin{tabular}{| c | l |}
    \hline
    \textbf{Operation} & \textbf{Result} \\
    \hline
    list(d) 			& return a list of all the keys of \textit{d} \\
    len(d) 				& number of items in \textit{d} \\
    {\footnotesize \textit{d.get(key, default)}} & {\small like \TTBF{d[key]}, but return \textit{default} if \textit{key} not found} \\
%     					& return \textit{default} if \textit{key} not found \\
    d.keys() 			& return the dictionnary's keys \\
    iter(d) 			& return an iterator over the keys of \textit{d} \\
    d[key] = value 		& set \textit{value} at key \textit{key} \\
    del d[key] 			& Remove the value at \TTBF{d[key]} \\
     					& {\footnotesize(Raise a KeyError if \textit{key} not found) } \\
    d.pop(key) 			& remove \textit{key} and return the associated value \\
    d.clear() 			& remove all items from the dictionnary \\
    d.copy() 			& shallow copy of the dictionnary \textit{d} \\
    \hline
  \end{tabular}

  \end{center}

\end{frame}


\begin{frame}[fragile]{Dicts operations}

  \begin{columns}[onlytextwidth]
    \begin{column}{\textwidth}

      \begin{onlyenv}<1>
        % Balises exception :  %* *)
        \begin{lstlisting}[style=python,morekeywords={for, in, range, list}]
dict1 = {"one": 1, "two": 2, "three": 3}
dict1["one"]
dict1["one"] + dict1["two"]
len(dict1)




 \end{lstlisting}
      \end{onlyenv}

      \begin{onlyenv}<2>
        % Balises exception :  %* *)
        \begin{lstlisting}[style=python,morekeywords={for, in, range, list}]
dict1 = {"one": 1, "two": 2, "three": 3}
dict1["one"]       # 1
dict1["one"] + dict1["two"] # 3
len(dict1)         # 3
list(dict1)



 \end{lstlisting}
      \end{onlyenv}

      \begin{onlyenv}<3>
        % Balises exception :  %* *)
        \begin{lstlisting}[style=python,morekeywords={for, in, range, list}]
dict1 = {"one": 1, "two": 2, "three": 3}
dict1["one"]       # 1
dict1["one"] + dict1["two"] # 3
len(dict1)         # 3
list(dict1)        # ['one','two','three']



 \end{lstlisting}
      \end{onlyenv}

      \begin{onlyenv}<4>
        % Balises exception :  %* *)
        \begin{lstlisting}[style=python,morekeywords={for, in, range, list}]
dict1 = {"one": 1, "two": 2, "three": 3}
dict1["one"]       # 1
dict1["one"] + dict1["two"] # 3
len(dict1)         # 3
list(dict1)        # ['one','two','three']
dict1.get("one")
dict1.get("A", 42)

 \end{lstlisting}
      \end{onlyenv}

      \begin{onlyenv}<5>
        % Balises exception :  %* *)
        \begin{lstlisting}[style=python,morekeywords={for, in, range, list}]
dict1 = {"one": 1, "two": 2, "three": 3}
dict1["one"]       # 1
dict1["one"] + dict1["two"] # 3
len(dict1)         # 3
list(dict1)        # ['one','two','three']
dict1.get("one")   # 1
dict1.get("A", 42) # 42
dict1["A"] = 8
dict1.get("A", 42)\end{lstlisting}
      \end{onlyenv}

      \begin{onlyenv}<6->
        % Balises exception :  %* *)
        \begin{lstlisting}[style=python,morekeywords={for, in, range, list}]
dict1 = {"one": 1, "two": 2, "three": 3}
dict1["one"]       # 1
dict1["one"] + dict1["two"] # 3
len(dict1)         # 3
list(dict1)        # ['one','two','three']
dict1.get("one")   # 1
dict1.get("A", 42) # 42
dict1["A"] = 8
dict1.get("A", 42) # 8 \end{lstlisting}
      \end{onlyenv}

    \end{column}
  \end{columns}

\end{frame}


\begin{frame}<beamer>{Dicts operations}

  \begin{center}

  \begin{tabular}{| c | l |}
    \hline
    \textbf{Operation} & \textbf{Result} \\
    \hline
    d | e 				& new dict with merged keys and values from \textit{d} and \textit{e} \\
     					& (values of \textit{e} erase those from \textit{d} if same key) \\[0.5cm]
    d |= e 				& like \TTBF{d | e} except that \textit{d} is updated \\
     					& (values of \textit{e} erase those from \textit{d} if same key) \\
    \hline
  \end{tabular}

  \end{center}

\end{frame}


\begin{frame}[fragile]{Dicts operations}

  \begin{columns}[onlytextwidth]
    \begin{column}{\textwidth}

      \begin{onlyenv}<1>
        % Balises exception :  %* *)
        \begin{lstlisting}[style=python,morekeywords={for, in, range, list}]
dict1 = {"one": 1, "two": 2}
dict2 = {"one": 4, "six": 6}




 \end{lstlisting}
      \end{onlyenv}

      \begin{onlyenv}<2>
        % Balises exception :  %* *)
        \begin{lstlisting}[style=python,morekeywords={for, in, range, list}]
dict1 = {"one": 1, "two": 2}
dict2 = {"one": 4, "six": 6}
dict1["one"] + dict2["one"]



 \end{lstlisting}
      \end{onlyenv}

      \begin{onlyenv}<3>
        % Balises exception :  %* *)
        \begin{lstlisting}[style=python,morekeywords={for, in, range, list}]
dict1 = {"one": 1, "two": 2}
dict2 = {"one": 4, "six": 6}
dict1["one"] + dict2["one"]   # 5
NewD = dict1 | dict2


 \end{lstlisting}
      \end{onlyenv}

      \begin{onlyenv}<4>
        % Balises exception :  %* *)
        \begin{lstlisting}[style=python,morekeywords={for, in, range, list}]
dict1 = {"one": 1, "two": 2}
dict2 = {"one": 4, "six": 6}
dict1["one"] + dict2["one"]   # 5
NewD = dict1 | dict2
NewD["two"]
NewD["six"]
NewD["one"] \end{lstlisting}
      \end{onlyenv}

      \begin{onlyenv}<5->
        % Balises exception :  %* *)
        \begin{lstlisting}[style=python,morekeywords={for, in, range, list}]
dict1 = {"one": 1, "two": 2}
dict2 = {"one": 4, "six": 6}
dict1["one"] + dict2["one"]   # 5
NewD = dict1 | dict2
NewD["two"]   # 2
NewD["six"]   # 6
NewD["one"]   # 4 \end{lstlisting}
      \end{onlyenv}

    \end{column}
  \end{columns}

\end{frame}


%%% Other types
\section{Other types}

\begin{frame}<beamer>{Other Built-in types}

  \begin{itemize}
    \item<1-> Set (and FrozenSet)
    \item<1-> Text sequence type - \TTBF{str}
    \item<1-> Binary sequence type
      \begin{itemize}
      \item<2-> \textit{(we won't see it)}
      \end{itemize}
  \end{itemize}

  \begin{center}

  \bigskip

  More details on types

  \medskip

  \url{https://docs.python.org/3/library/stdtypes.html}

  \end{center}

\end{frame}

%%% Set and FrozenSet
\subsection{Set and FrozenSet}

\begin{frame}<beamer>{Set}

  \begin{itemize}
    \item<1-> mutable
    \begin{itemize}
      \item<2-> \textit{FrozenSet are immutables}
    \end{itemize}
    \item<3-> unordered
    \begin{itemize}
      \item<4-> \textit{sequences cannot be used (index, slicing, ...)}
    \end{itemize}
    \item<5-> distinct values only
    \item<6-> only hashable objects can be added
  \end{itemize}

\end{frame}


\begin{frame}[fragile]{Set}

  \begin{columns}[onlytextwidth]
    \begin{column}{\textwidth}

      \begin{onlyenv}<1>
        % Balises exception :  %* *)
        \begin{lstlisting}[style=python,morekeywords={for, in, range, list}]
set1 = {'world', 'hello'}
set2 = set( [4, 42, 6] )
set3 = {x for x in 'abracadabra'
         if x not in 'abc'}
set3 \end{lstlisting}
      \end{onlyenv}

      \begin{onlyenv}<2>
        % Balises exception :  %* *)
        \begin{lstlisting}[style=python,morekeywords={for, in, range, list}]
set1 = {'world', 'hello'}  # {'hello', 'world'}
set2 = set( [4, 42, 6] )
set3 = {x for x in 'abracadabra'
         if x not in 'abc'}
set3 \end{lstlisting}
      \end{onlyenv}

      \begin{onlyenv}<3>
        % Balises exception :  %* *)
        \begin{lstlisting}[style=python,morekeywords={for, in, range, list}]
set1 = {'world', 'hello'}  # {'hello', 'world'}
set2 = set( [4, 42, 6] )   # {42, 4, 6}
set3 = {x for x in 'abracadabra'
         if x not in 'abc'}
set3 \end{lstlisting}
      \end{onlyenv}

      \begin{onlyenv}<4->
        % Balises exception :  %* *)
        \begin{lstlisting}[style=python,morekeywords={for, in, range, list}]
set1 = {'world', 'hello'}  # {'hello', 'world'}
set2 = set( [4, 42, 6] )   # {42, 4, 6}
set3 = {x for x in 'abracadabra'
         if x not in 'abc'}
set3                       # {'d', 'r'} \end{lstlisting}
      \end{onlyenv}

    \end{column}
  \end{columns}

  \bigskip

  \onslide<5-> Main usages: membership testing, removing duplicates from a sequence, mathematical operations (union, intersection, ...)


\end{frame}


\begin{frame}<beamer>{Set operations}

  \begin{center}

  \begin{tabular}{| c | l |}
    \hline
    \textbf{Operation} & \textbf{Result} \\
    \hline
    add(e) 				& add an element to a set \\
    remove(e) 			& remove an element from a set \\
    len(s) 				& number of elements in set \textit{s} \\
    x in s 				& test \textit{x} for membership in \textit{s} \\
    copy() 				& shallow copy of set \\
    isdisjoint(s2) 		& test if no element are in common with \textit{s2} \\
    issubset(s2) 		& test if every element is in the set \textit{s2} \\
	issuperset(s2) 		& test if every element of \textit{s2} is in the set \\
	                    & new set with elements: \\
	union(s2) 			& - from the set and \textit{s2} \\
	intersection(s2) 	& - common to the set and \textit{s2} \\
	difference(s2) 		& - in the set that are not in \textit{s2} \\
	{\scriptsize symmetric\_difference(s2)} 	& - in either the set or \textit{s2} but not both \\
    \hline
  \end{tabular}

  \end{center}

\end{frame}


\begin{frame}[fragile]{Set operations}

  \begin{columns}[onlytextwidth]
    \begin{column}{\textwidth}

      \begin{onlyenv}<1>
        % Balises exception :  %* *)
        \begin{lstlisting}[style=python,morekeywords={for, in, range, list}]
s1 = { 1, 2, 3, 4 }
s2 = { 3, 4, 5, 6 }
s3 = { 1, 2 }
s4 = { 1, 2, 3, 4, 5}





 \end{lstlisting}
      \end{onlyenv}

      \begin{onlyenv}<2>
        % Balises exception :  %* *)
        \begin{lstlisting}[style=python,morekeywords={for, in, range, list}]
s1 = { 1, 2, 3, 4 }
s2 = { 3, 4, 5, 6 }
s3 = { 1, 2 }
s4 = { 1, 2, 3, 4, 5}
s2.isdisjoint(s1)




 \end{lstlisting}
      \end{onlyenv}

      \begin{onlyenv}<3>
        % Balises exception :  %* *)
        \begin{lstlisting}[style=python,morekeywords={for, in, range, list}]
s1 = { 1, 2, 3, 4 }
s2 = { 3, 4, 5, 6 }
s3 = { 1, 2 }
s4 = { 1, 2, 3, 4, 5}
s2.isdisjoint(s1)     # False
s2.isdisjoint(s3)



 \end{lstlisting}
      \end{onlyenv}

      \begin{onlyenv}<4>
        % Balises exception :  %* *)
        \begin{lstlisting}[style=python,morekeywords={for, in, range, list}]
s1 = { 1, 2, 3, 4 }
s2 = { 3, 4, 5, 6 }
s3 = { 1, 2 }
s4 = { 1, 2, 3, 4, 5}
s2.isdisjoint(s1)     # False
s2.isdisjoint(s3)     # True
s3.issubset(s1)


 \end{lstlisting}
      \end{onlyenv}

      \begin{onlyenv}<5>
        % Balises exception :  %* *)
        \begin{lstlisting}[style=python,morekeywords={for, in, range, list}]
s1 = { 1, 2, 3, 4 }
s2 = { 3, 4, 5, 6 }
s3 = { 1, 2 }
s4 = { 1, 2, 3, 4, 5}
s2.isdisjoint(s1)     # False
s2.isdisjoint(s3)     # True
s3.issubset(s1)       # True
s3.issuperset(s1)

 \end{lstlisting}
      \end{onlyenv}

      \begin{onlyenv}<6>
        % Balises exception :  %* *)
        \begin{lstlisting}[style=python,morekeywords={for, in, range, list}]
s1 = { 1, 2, 3, 4 }
s2 = { 3, 4, 5, 6 }
s3 = { 1, 2 }
s4 = { 1, 2, 3, 4, 5}
s2.isdisjoint(s1)     # False
s2.isdisjoint(s3)     # True
s3.issubset(s1)       # True
s3.issuperset(s1)     # False
s4.issubset(s1)
 \end{lstlisting}
      \end{onlyenv}

      \begin{onlyenv}<7>
        % Balises exception :  %* *)
        \begin{lstlisting}[style=python,morekeywords={for, in, range, list}]
s1 = { 1, 2, 3, 4 }
s2 = { 3, 4, 5, 6 }
s3 = { 1, 2 }
s4 = { 1, 2, 3, 4, 5}
s2.isdisjoint(s1)     # False
s2.isdisjoint(s3)     # True
s3.issubset(s1)       # True
s3.issuperset(s1)     # False
s4.issubset(s1)       # False
s4.issuperset(s1)\end{lstlisting}
      \end{onlyenv}

      \begin{onlyenv}<8->
        % Balises exception :  %* *)
        \begin{lstlisting}[style=python,morekeywords={for, in, range, list}]
s1 = { 1, 2, 3, 4 }
s2 = { 3, 4, 5, 6 }
s3 = { 1, 2 }
s4 = { 1, 2, 3, 4, 5}
s2.isdisjoint(s1)     # False
s2.isdisjoint(s3)     # True
s3.issubset(s1)       # True
s3.issuperset(s1)     # False
s4.issubset(s1)       # False
s4.issuperset(s1)     # True \end{lstlisting}
      \end{onlyenv}

    \end{column}
  \end{columns}

\end{frame}


%%% Text sequence type
\subsection{Text sequence type}

\begin{frame}<beamer>{Text sequence type}

  \begin{itemize}
    \item<1-> immutable (when created, cannot be modified)
    \item<2-> ordered
    \item<3-> same value can be recorded multiple times
    \item<4-> only \TTBF{str} objets can be used inside
  \end{itemize}

\end{frame}


\subsubsection{Literals}

\begin{frame}<beamer>{Text sequence type}

  \onslide<1-> Three ways to write literals:

  \medskip

  \begin{itemize}
    \item<1-> Single quotes: \lstinline|'Text'|
    \begin{itemize}
      \item<2-> \textit{(may embed double quotes \lstinline|'The "only" Text'| )}
    \end{itemize}

    \medskip

    \item<3-> Double quotes: \lstinline|"Text"|
    \begin{itemize}
      \item<4-> \textit{(may embed single quotes \lstinline|"The 'only' Text"| )}
    \end{itemize}

    \medskip

    \item<5-> Triple quotes: \lstinline|'''Text'''| or \lstinline|"""Text"""|
    \begin{itemize}
      \item<6-> \textit{(triple quotes can be on multiple lines)}
    \end{itemize}
  \end{itemize}

\end{frame}


\begin{frame}<beamer>{Text sequence type}

  \onslide<1-> As strings are immutables, you must:

  \medskip

  \begin{itemize}
    \item<2-> concatenate strings into a new string
    \item<3-> extract substrings with slices
  \end{itemize}

  \bigskip

  \onslide<4-> \TTBF{str()} constructor converts native types into strings

\end{frame}


\begin{frame}[fragile]{Text sequence type}

  \begin{columns}[onlytextwidth]
    \begin{column}{\textwidth}
      \begin{onlyenv}<1>
        \begin{lstlisting}[style=python]
A = str('My String')
B = str("A sentence")
C = str('''The giant
text with a lot of
lines''')


 \end{lstlisting}
      \end{onlyenv}

      \begin{onlyenv}<2>
        \begin{lstlisting}[style=python]
A = str('My String')
B = str("A sentence")
C = str('''The giant
text with a lot of
lines''')
D = B[0:2] + A[3:10]

print(D) \end{lstlisting}
      \end{onlyenv}


      \begin{onlyenv}<3->
        \begin{lstlisting}[style=python]
A = str('My String')
B = str("A sentence")
C = str('''The giant
text with a lot of
lines''')
D = B[0:2] + A[3:10]

print(D)        #  "A String" \end{lstlisting}
      \end{onlyenv}
    \end{column}
  \end{columns}

\end{frame}


\subsubsection{Types of characters}

\begin{frame}<beamer>{Text sequence type}

  \begin{center}

  \begin{tabular}{| c | l |}
    \hline
    \textbf{Operation} & \textbf{Result} \\
    \hline
    {\small \texttt{str.}\TTBF{isprintable()}} 	& \TTBF{True} if only printable characters and \\
     											& more than one char \\
    {\small \texttt{str.}\TTBF{isascii()}} 		& \TTBF{True} if only ASCII characters and \\
     											& more than one char \\
    {\small \texttt{str.}\TTBF{isalpha()}} 		& \TTBF{True} if only alphabetic characters and \\
     											& more than one char \\
    {\small \texttt{str.}\TTBF{isalnum()}} 		& \TTBF{True} if only alphanumeric characters \\
     											& and more than one char \\
    {\small \texttt{str.}\TTBF{isdigit()}} 		& \TTBF{True} if only numbers characters and \\
     											& more than one char \\
    {\small \texttt{str.}\TTBF{isnumeric()}} 		& \TTBF{True} if only numeric characters and \\
     											& more than one char \\
    \hline
  \end{tabular}

  \end{center}

\end{frame}


\begin{frame}[fragile]{Text sequence type}

  \begin{columns}[onlytextwidth]
    \begin{column}{\textwidth}

      \begin{onlyenv}<1->
        \begin{lstlisting}[style=python]
"abc42".isprintable()  #  True
"abc42".isascii()      #  True
"abc42".isalpha()      # False
"abc".isalpha()        #  True
"abc42".isalnum()      #  True
"42".isdigit()         #  True
"42.2".isdigit()       # False
"-42".isdigit()        # False
"-42".isnumeric()      # False \end{lstlisting}
      \end{onlyenv}

    \end{column}
  \end{columns}

\end{frame}


\subsubsection{Capitalization}

\begin{frame}<beamer>{Text sequence type}

  \begin{center}

  \begin{tabular}{| c | l |}
    \hline
    \textbf{Operation} & \textbf{Result} \\
    \hline
    \texttt{str.}\TTBF{lower()} 			& Put the string into lowercase \\
    \texttt{str.}\TTBF{upper()} 			& Put the string into uppercase \\
    \texttt{str.}\TTBF{capitalize()} 	& $\text{1}^{\text{st}}$ character is capitalized, other are \\
    	 									& lowercased \\
    \texttt{str.}\TTBF{title()} 			& Same as capitalized, but on each word \\
    \texttt{str.}\TTBF{swapcase()} 		& Reverse upper to lower case characters, \\
     									& and lower to upper \\
    \texttt{str.}\TTBF{isspace()} 		& \TTBF{True} if there are only whitespaces \\
     									& and at least one character \\
    \texttt{str.}\TTBF{isupper()} 		& \TTBF{True} if all characters are uppercase \\
     									& and at least one character \\
    \hline
  \end{tabular}

  \medskip

  Don't forget: a string is immutable!

  All of the string methods \textit{always} return a copy of the string

  \end{center}

\end{frame}


\begin{frame}[fragile]{Text sequence type}

  \begin{columns}[onlytextwidth]
    \begin{column}{\textwidth}

      \begin{onlyenv}<1>
        \begin{lstlisting}[style=python]
A = str('tHis is A TExT wIth')
print(A.capitalize())





 \end{lstlisting}
      \end{onlyenv}

      \begin{onlyenv}<2>
        \begin{lstlisting}[style=python]
A = str('tHis is A TExT wIth')
print(A.capitalize()) # "This is a text with"
print(A.title())




 \end{lstlisting}
      \end{onlyenv}

      \begin{onlyenv}<3>
        \begin{lstlisting}[style=python]
A = str('tHis is A TExT wIth')
print(A.capitalize()) # "This is a text with"
print(A.title())      # "THis Is A TExT WIth"
print(A.swapcase())



 \end{lstlisting}
      \end{onlyenv}

      \begin{onlyenv}<4>
        \begin{lstlisting}[style=python]
A = str('tHis is A TExT wIth')
print(A.capitalize()) # "This is a text with"
print(A.title())      # "THis Is A TExT WIth"
print(A.swapcase())   # "ThIS IS a teXt WiTH"



 \end{lstlisting}
      \end{onlyenv}

      \begin{onlyenv}<5>
        \begin{lstlisting}[style=python]
A = str('tHis is A TExT wIth')
print(A.capitalize()) # "This is a text with"
print(A.title())      # "THis Is A TExT WIth"
print(A.swapcase())   # "ThIS IS a teXt WiTH"

B = str("   ")
print(B.isspace())
print(B.isupper()) \end{lstlisting}
      \end{onlyenv}

      \begin{onlyenv}<6->
        \begin{lstlisting}[style=python]
A = str('tHis is A TExT wIth')
print(A.capitalize()) # "This is a text with"
print(A.title())      # "THis Is A TExT WIth"
print(A.swapcase())   # "ThIS IS a teXt WiTH"

B = str("   ")
print(B.isspace())   # True
print(B.isupper())   # False \end{lstlisting}
      \end{onlyenv}

    \end{column}
  \end{columns}

\end{frame}


\subsubsection{Prefixes, Suffixes, Stripping}

\begin{frame}<beamer>{Text sequence type}

  \begin{center}
  \begin{tabular}{| c | l |}
    \hline
    \textbf{Operation} & \textbf{Result} \\
    \hline
    \texttt{str.}\TTBF{removeprefix(p)} 	& If the string begins with prefix \textit{p}, \\
     									& then it is removed \\[0.25cm]

    \texttt{str.}\TTBF{removesuffix(s)} 	& If the string ends with suffix \textit{s}, \\
     									& then it is removed \\[0.25cm]

    \texttt{str.}\TTBF{lstrip(\textit{[chars]})} 	& Remove leading characters present \\
     									& in \textit{chars}. If no argument, remove \\
     									& whitespaces. \\[0.25cm]

    \texttt{str.}\TTBF{rstrip(\textit{[chars]})} 	& Remove trailing characters present \\
     									& in \textit{chars}. If no argument, remove \\
     									& whitespaces. \\[0.25cm]

    \texttt{str.}\TTBF{strip(\textit{[chars]})} 	& Apply \TTBF{lstrip()} and \TTBF{rstrip()} \\
    \hline
  \end{tabular}
  \end{center}

\end{frame}


\begin{frame}[fragile]{Text sequence type}

  \begin{columns}[onlytextwidth]
    \begin{column}{\textwidth}

      \begin{onlyenv}<1>
        \begin{lstlisting}[style=python,basicstyle=\ttfamily\small,keepspaces=true,columns=fullflexible]
'TestHook'.removeprefix('Test')









 \end{lstlisting}
      \end{onlyenv}

      \begin{onlyenv}<2>
        \begin{lstlisting}[style=python,basicstyle=\ttfamily\small,keepspaces=true,columns=fullflexible]
'TestHook'.removeprefix('Test')      # Hook
'BaseTestCase'.removeprefix('Test')








 \end{lstlisting}
      \end{onlyenv}

      \begin{onlyenv}<3>
        \begin{lstlisting}[style=python,basicstyle=\ttfamily\small,keepspaces=true,columns=fullflexible]
'TestHook'.removeprefix('Test')      # Hook
'BaseTestCase'.removeprefix('Test')  # BaseTestCase

'Arthur: three!'.removeprefix('Arthur: ')






 \end{lstlisting}
      \end{onlyenv}

      \begin{onlyenv}<4>
        \begin{lstlisting}[style=python,basicstyle=\ttfamily\small,keepspaces=true,columns=fullflexible]
'TestHook'.removeprefix('Test')      # Hook
'BaseTestCase'.removeprefix('Test')  # BaseTestCase

'Arthur: three!'.removeprefix('Arthur: ')  # three!
'Arthur: three!'.lstrip('Arthur: ')





 \end{lstlisting}
      \end{onlyenv}

      \begin{onlyenv}<5>
        \begin{lstlisting}[style=python,basicstyle=\ttfamily\small,keepspaces=true,columns=fullflexible]
'TestHook'.removeprefix('Test')      # Hook
'BaseTestCase'.removeprefix('Test')  # BaseTestCase

'Arthur: three!'.removeprefix('Arthur: ')  # three!
'Arthur: three!'.lstrip('Arthur: ')        # ee!

'www.example.com'.lstrip('cmowz.')



 \end{lstlisting}
      \end{onlyenv}

      \begin{onlyenv}<6>
        \begin{lstlisting}[style=python,basicstyle=\ttfamily\small,keepspaces=true,columns=fullflexible]
'TestHook'.removeprefix('Test')      # Hook
'BaseTestCase'.removeprefix('Test')  # BaseTestCase

'Arthur: three!'.removeprefix('Arthur: ')  # three!
'Arthur: three!'.lstrip('Arthur: ')        # ee!

'www.example.com'.lstrip('cmowz.')  # 'example.com'
'   spacious   '.lstrip()


 \end{lstlisting}
      \end{onlyenv}

      \begin{onlyenv}<7>
        \begin{lstlisting}[style=python,basicstyle=\ttfamily\small,keepspaces=true,columns=fullflexible]
'TestHook'.removeprefix('Test')      # Hook
'BaseTestCase'.removeprefix('Test')  # BaseTestCase

'Arthur: three!'.removeprefix('Arthur: ')  # three!
'Arthur: three!'.lstrip('Arthur: ')        # ee!

'www.example.com'.lstrip('cmowz.')  # 'example.com'
'   spacious   '.lstrip()           # 'spacious   '

'Monty Python'.rstrip(' Python')
 \end{lstlisting}
      \end{onlyenv}

      \begin{onlyenv}<8>
        \begin{lstlisting}[style=python,basicstyle=\ttfamily\small,keepspaces=true,columns=fullflexible]
'TestHook'.removeprefix('Test')      # Hook
'BaseTestCase'.removeprefix('Test')  # BaseTestCase

'Arthur: three!'.removeprefix('Arthur: ')  # three!
'Arthur: three!'.lstrip('Arthur: ')        # ee!

'www.example.com'.lstrip('cmowz.')  # 'example.com'
'   spacious   '.lstrip()           # 'spacious   '

'Monty Python'.rstrip(' Python')        # M
'Monty Python'.removesuffix(' Python') \end{lstlisting}
      \end{onlyenv}

      \begin{onlyenv}<9->
        \begin{lstlisting}[style=python,basicstyle=\ttfamily\small,keepspaces=true,columns=fullflexible]
'TestHook'.removeprefix('Test')      # Hook
'BaseTestCase'.removeprefix('Test')  # BaseTestCase

'Arthur: three!'.removeprefix('Arthur: ')  # three!
'Arthur: three!'.lstrip('Arthur: ')        # ee!

'www.example.com'.lstrip('cmowz.')  # 'example.com'
'   spacious   '.lstrip()           # 'spacious   '

'Monty Python'.rstrip(' Python')        # M
'Monty Python'.removesuffix(' Python')  # Monty \end{lstlisting}
      \end{onlyenv}

    \end{column}
  \end{columns}

\end{frame}


\subsubsection{Find, Index, Replace}

\begin{frame}<beamer>{Text sequence type}

  \begin{center}

  \begin{tabular}{| c | l |}
    \hline
    \textbf{Operation} & \textbf{Result} \\
    \hline
    {\footnotesize \texttt{str.}\TTBF{find(sub)}} 			& {\normalsize Find $1^{\text{st}}$ index of substring \textit{sub} within} \\
    {\scriptsize \textit{\TTBF{find(sub,start,end)}}}	 		& {\normalsize the slice \TTBF{[start:end]}, or return \TTBF{-1}} \\[0.5cm]

    {\footnotesize \texttt{str.}\TTBF{index(sub)}} 				& {\normalsize Like \TTBF{find}, but raise \TTBF{ValueError}} \\
    {\scriptsize \textit{\TTBF{index(sub,start,end)}}} 		& {\normalsize instead of returning \TTBF{-1}} \\[0.5cm]

    {\footnotesize \texttt{str.}\TTBF{replace(old,new)}} 		& {\normalsize Replace all (or the \textit{count} firsts) of the} \\
    {\scriptsize \textit{\TTBF{replace(old,new,count)}}} 		& {\normalsize occurrencies of \textit{old} by \textit{new}} \\
    \hline
  \end{tabular}

  \end{center}

\end{frame}


\begin{frame}[fragile]{Text sequence type}

  \begin{columns}[onlytextwidth]
    \begin{column}{\textwidth}

      \begin{onlyenv}<1>
        \begin{lstlisting}[style=python]
"ballo" in "C'est ballo"







 \end{lstlisting}
      \end{onlyenv}

      \begin{onlyenv}<2>
        \begin{lstlisting}[style=python]
"ballo" in "C'est ballo"      # True
"C'est ballo".find("ballo")






 \end{lstlisting}
      \end{onlyenv}

      \begin{onlyenv}<3>
        \begin{lstlisting}[style=python]
"ballo" in "C'est ballo"      # True
"C'est ballo".find("ballo")   #  6
"alalala".find("la", 2, 5)





 \end{lstlisting}
      \end{onlyenv}

      \begin{onlyenv}<4>
        \begin{lstlisting}[style=python]
"ballo" in "C'est ballo"      # True
"C'est ballo".find("ballo")   #  6
"alalala".find("la", 2, 5)    #  3
"ah".find("b")




 \end{lstlisting}
      \end{onlyenv}

      \begin{onlyenv}<5>
        \begin{lstlisting}[style=python]
"ballo" in "C'est ballo"      # True
"C'est ballo".find("ballo")   #  6
"alalala".find("la", 2, 5)    #  3
"ah".find("b")                # -1
"alalala".index("la", 2, 5)



 \end{lstlisting}
      \end{onlyenv}

      \begin{onlyenv}<6>
        \begin{lstlisting}[style=python]
"ballo" in "C'est ballo"      # True
"C'est ballo".find("ballo")   #  6
"alalala".find("la", 2, 5)    #  3
"ah".find("b")                # -1
"alalala".index("la", 2, 5)   #  3
"ah".index("b")


 \end{lstlisting}
      \end{onlyenv}

      \begin{onlyenv}<7>
        \begin{lstlisting}[style=python]
"ballo" in "C'est ballo"      # True
"C'est ballo".find("ballo")   #  6
"alalala".find("la", 2, 5)    #  3
"ah".find("b")                # -1
"alalala".index("la", 2, 5)   #  3
"ah".index("b")               # [Exception]
"ahehih".replace("ih", "oh")

 \end{lstlisting}
      \end{onlyenv}

      \begin{onlyenv}<8>
        \begin{lstlisting}[style=python]
"ballo" in "C'est ballo"      # True
"C'est ballo".find("ballo")   #  6
"alalala".find("la", 2, 5)    #  3
"ah".find("b")                # -1
"alalala".index("la", 2, 5)   #  3
"ah".index("b")               # [Exception]
"ahehih".replace("ih", "oh")  #  "ahehoh"
"hohoho".replace("ho", "ha")
 \end{lstlisting}
      \end{onlyenv}

      \begin{onlyenv}<9>
        \begin{lstlisting}[style=python]
"ballo" in "C'est ballo"      # True
"C'est ballo".find("ballo")   #  6
"alalala".find("la", 2, 5)    #  3
"ah".find("b")                # -1
"alalala".index("la", 2, 5)   #  3
"ah".index("b")               # [Exception]
"ahehih".replace("ih", "oh")  #  "ahehoh"
"hohoho".replace("ho", "ha")  #  "hahaha"
"hohoho".replace("ho", "ha",2) \end{lstlisting}
      \end{onlyenv}

      \begin{onlyenv}<10->
        \begin{lstlisting}[style=python]
"ballo" in "C'est ballo"      # True
"C'est ballo".find("ballo")   #  6
"alalala".find("la", 2, 5)    #  3
"ah".find("b")                # -1
"alalala".index("la", 2, 5)   #  3
"ah".index("b")               # [Exception]
"ahehih".replace("ih", "oh")  #  "ahehoh"
"hohoho".replace("ho", "ha")  #  "hahaha"
"hohoho".replace("ho", "ha",2)#  "hahaho" \end{lstlisting}
      \end{onlyenv}

    \end{column}
  \end{columns}

\end{frame}



\subsubsection{Splitting}

\begin{frame}<beamer>{Text sequence type}

  \begin{center}

  \begin{tabular}{| c | l |}
    \hline
    \textbf{Operation} & \textbf{Result} \\
    \hline
    {\footnotesize \texttt{str.}\TTBF{join(iterable)}} 		& {\normalsize Separate each element of \textit{iterate}} \\
     														& {\normalsize with the calling string} \\[0.25cm]

    {\footnotesize \texttt{str.}\TTBF{partition(sep)}} 		& {\normalsize Separate the string in a 3-tuple:} \\
     														& {\normalsize - the substring before the $1^{\text{st}}$ \textit{sep}} \\
     														& {\normalsize - \textit{sep}} \\
     														& {\normalsize - the substring after the $1^{\text{st}}$ \textit{sep}} \\[0.25cm]

    {\footnotesize \texttt{str.}\TTBF{split(sep,maxsplit)}} 	& {\normalsize Split the string into a list of words} \\
    {\scriptsize \textit{\TTBF{split(sep=None,maxsplit=-1)}}}	& {\normalsize Split \textit{maxsplit} times ($-1$ for all)} \\
     														& {\small \textit{(Whitespaces are treated differently)}} \\[0.25cm]

    {\footnotesize \texttt{str.}\TTBF{splitlines(keepends)}} 	& {\normalsize Split a string by lines} \\
    {\scriptsize \textit{\TTBF{splitlines(keepends=False)}}}	& {\small \textit{(see manual for line boundaries)}} \\
    \hline
  \end{tabular}

  \end{center}

\end{frame}


\begin{frame}[fragile]{Text sequence type}

  \begin{columns}[onlytextwidth]
    \begin{column}{\textwidth}

      \begin{onlyenv}<1>
        \begin{lstlisting}[style=python]
"|".join("alo alo")







 \end{lstlisting}
      \end{onlyenv}

      \begin{onlyenv}<2>
        \begin{lstlisting}[style=python]
"|".join("alo alo")      # 'a|l|o| |a|l|o'
"alo alo".partition(" ")






 \end{lstlisting}
      \end{onlyenv}

      \begin{onlyenv}<3>
        \begin{lstlisting}[style=python]
"|".join("alo alo")      # 'a|l|o| |a|l|o'
"alo alo".partition(" ") # ('alo', ' ', 'alo')
"a b c".partition(" ")





 \end{lstlisting}
      \end{onlyenv}

      \begin{onlyenv}<4>
        \begin{lstlisting}[style=python]
"|".join("alo alo")      # 'a|l|o| |a|l|o'
"alo alo".partition(" ") # ('alo', ' ', 'alo')
"a b c".partition(" ")   # ('a', ' ', 'b c')
"a b c".split()




 \end{lstlisting}
      \end{onlyenv}

      \begin{onlyenv}<5>
        \begin{lstlisting}[style=python]
"|".join("alo alo")      # 'a|l|o| |a|l|o'
"alo alo".partition(" ") # ('alo', ' ', 'alo')
"a b c".partition(" ")   # ('a', ' ', 'b c')
"a b c".split()          # ['a', 'b', 'c']
"abcbdbe".split()



 \end{lstlisting}
      \end{onlyenv}

      \begin{onlyenv}<6>
        \begin{lstlisting}[style=python]
"|".join("alo alo")      # 'a|l|o| |a|l|o'
"alo alo".partition(" ") # ('alo', ' ', 'alo')
"a b c".partition(" ")   # ('a', ' ', 'b c')
"a b c".split()          # ['a', 'b', 'c']
"abcbdbe".split()        # ['abcbdbe']
"abcbdbe".split("b", 1)


 \end{lstlisting}
      \end{onlyenv}

      \begin{onlyenv}<7>
        \begin{lstlisting}[style=python]
"|".join("alo alo")      # 'a|l|o| |a|l|o'
"alo alo".partition(" ") # ('alo', ' ', 'alo')
"a b c".partition(" ")   # ('a', ' ', 'b c')
"a b c".split()          # ['a', 'b', 'c']
"abcbdbe".split()        # ['abcbdbe']
"abcbdbe".split("b", 1)  # ['a', 'cbdbe']
"abcbdbe".split("b", -1)

 \end{lstlisting}
      \end{onlyenv}

      \begin{onlyenv}<8>
        \begin{lstlisting}[style=python]
"|".join("alo alo")      # 'a|l|o| |a|l|o'
"alo alo".partition(" ") # ('alo', ' ', 'alo')
"a b c".partition(" ")   # ('a', ' ', 'b c')
"a b c".split()          # ['a', 'b', 'c']
"abcbdbe".split()        # ['abcbdbe']
"abcbdbe".split("b", 1)  # ['a', 'cbdbe']
"abcbdbe".split("b", -1) # ['a','c','d','e']
'::1:2:'.split(':')
 \end{lstlisting}
      \end{onlyenv}

      \begin{onlyenv}<9>
        \begin{lstlisting}[style=python]
"|".join("alo alo")      # 'a|l|o| |a|l|o'
"alo alo".partition(" ") # ('alo', ' ', 'alo')
"a b c".partition(" ")   # ('a', ' ', 'b c')
"a b c".split()          # ['a', 'b', 'c']
"abcbdbe".split()        # ['abcbdbe']
"abcbdbe".split("b", 1)  # ['a', 'cbdbe']
"abcbdbe".split("b", -1) # ['a','c','d','e']
'::1:2:'.split(':')      # ['','','1','2', '']
"   a   b c   ".split() \end{lstlisting}
      \end{onlyenv}

      \begin{onlyenv}<10->
        \begin{lstlisting}[style=python]
"|".join("alo alo")      # 'a|l|o| |a|l|o'
"alo alo".partition(" ") # ('alo', ' ', 'alo')
"a b c".partition(" ")   # ('a', ' ', 'b c')
"a b c".split()          # ['a', 'b', 'c']
"abcbdbe".split()        # ['abcbdbe']
"abcbdbe".split("b", 1)  # ['a', 'cbdbe']
"abcbdbe".split("b", -1) # ['a','c','d','e']
'::1:2:'.split(':')      # ['','','1','2', '']
"   a   b c   ".split()  # ['a', 'b', 'c']  \end{lstlisting}
      \end{onlyenv}

    \end{column}
  \end{columns}

\end{frame}


\subsubsection{Centering, Filling, Justifying}

\begin{frame}<beamer>{Text sequence type}

  \begin{center}

  \begin{tabular}{| c | l |}
    \hline
    \textbf{Operation} & \textbf{Result} \\
    \hline
    {\small \texttt{str.}\TTBF{center(width)}} 	& Center the string with leading and \\
     											& trailing whitespaces and make its \\
     											& length equal to \textit{fill} \\[0.25cm]

    {\small \texttt{str.}\TTBF{zfill(width)}} 	& Put leading $ 0 $ to fill the string and \\
     											& make its length equal to \textit{fill} \\
     											& (eventually add $ - $ as a prefix) \\[0.25cm]

    {\small \texttt{str.}\TTBF{ljust(width)}} 	& Left justify for length \textit{width}, and \\
    {\scriptsize \textit{\TTBF{ljust(width,fillchar)}}} 	& fill with \textit{fillchar} characters \\[0.25cm]

    {\small \texttt{str.}\TTBF{rjust(width)}} 	& Right justify like \TTBF{str.ljust()} \\
    {\scriptsize \textit{\TTBF{rjust(width,fillchar)}}} 	& \\
    \hline
  \end{tabular}

  \end{center}

\end{frame}


\begin{frame}[fragile]{Text sequence type}

  \begin{columns}[onlytextwidth]
    \begin{column}{\textwidth}

      \begin{onlyenv}<1>
        \begin{lstlisting}[style=python]
"abc".center(7)






 \end{lstlisting}
      \end{onlyenv}

      \begin{onlyenv}<2>
        \begin{lstlisting}[style=python]
"abc".center(7)   # '  abc  '
"abc".center(2)





 \end{lstlisting}
      \end{onlyenv}

      \begin{onlyenv}<3>
        \begin{lstlisting}[style=python]
"abc".center(7)   # '  abc  '
"abc".center(2)   # 'abc'
"abc".center(4)




 \end{lstlisting}
      \end{onlyenv}

      \begin{onlyenv}<4>
        \begin{lstlisting}[style=python]
"abc".center(7)   # '  abc  '
"abc".center(2)   # 'abc'
"abc".center(4)   # 'abc '
"  abc".center(7)



 \end{lstlisting}
      \end{onlyenv}

      \begin{onlyenv}<5>
        \begin{lstlisting}[style=python]
"abc".center(7)   # '  abc  '
"abc".center(2)   # 'abc'
"abc".center(4)   # 'abc '
"  abc".center(7) # '   abc '
"abc".rjust(7)
"abc".ljust(7)

 \end{lstlisting}
      \end{onlyenv}

      \begin{onlyenv}<6>
        \begin{lstlisting}[style=python]
"abc".center(7)   # '  abc  '
"abc".center(2)   # 'abc'
"abc".center(4)   # 'abc '
"  abc".center(7) # '   abc '
"abc".rjust(7)    # '    abc'
"abc".ljust(7)    # 'abc    '
"42".zfill(5)
 \end{lstlisting}
      \end{onlyenv}

      \begin{onlyenv}<7>
        \begin{lstlisting}[style=python]
"abc".center(7)   # '  abc  '
"abc".center(2)   # 'abc'
"abc".center(4)   # 'abc '
"  abc".center(7) # '   abc '
"abc".rjust(7)    # '    abc'
"abc".ljust(7)    # 'abc    '
"42".zfill(5)     # '00042'
"42.3".zfill(5) \end{lstlisting}
      \end{onlyenv}

      \begin{onlyenv}<8->
        \begin{lstlisting}[style=python]
"abc".center(7)   # '  abc  '
"abc".center(2)   # 'abc'
"abc".center(4)   # 'abc '
"  abc".center(7) # '   abc '
"abc".rjust(7)    # '    abc'
"abc".ljust(7)    # 'abc    '
"42".zfill(5)     # '00042'
"42.3".zfill(5)   # '042.3' \end{lstlisting}
      \end{onlyenv}

    \end{column}
  \end{columns}

\end{frame}



%%% End
\begin{frame}<beamer>{}

  \begin{center}
  \textit{Thank you for your attention}
  \end{center}

\end{frame}


\end{document}
