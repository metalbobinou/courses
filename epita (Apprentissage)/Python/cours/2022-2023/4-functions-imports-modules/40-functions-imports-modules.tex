% -*- latex -*-

\part{Functions, Imports \& Modules}  % Declaring functions, imports, modules

%%%%%%%%%%%%%%%%%%%%%%%%%%%%%%%%%%%%%%%%%%%%%%%%%%%%%%%%

\section{Functions}

\begin{frame}[fragile]{Functions}

  Usual function definition and call:

  \begin{columns}[onlytextwidth]
    \begin{column}{\textwidth}

      \begin{onlyenv}<1>
        % Balises exception :  %* *)
        \begin{lstlisting}[style=python]





 \end{lstlisting}
      \end{onlyenv}

      \begin{onlyenv}<2>
        \begin{lstlisting}[style=python]
def MyFunction(name):
  print("Hello " + str(name) + "!")
  return (0)


 \end{lstlisting}
      \end{onlyenv}

      \begin{onlyenv}<3>
        \begin{lstlisting}[style=python]
def MyFunction(name):
  print("Hello " + str(name) + "!")
  return (0)

MyFunction("Fabrice")
 \end{lstlisting}
      \end{onlyenv}

      \begin{onlyenv}<4->
        \begin{lstlisting}[style=python]
def MyFunction(name):
  print("Hello " + str(name) + "!")
  return (0)

MyFunction("Fabrice")
MyFunction(42) \end{lstlisting}
      \end{onlyenv}

    \end{column}
  \end{columns}

  \onslide<5-> \textit{MyFunction} is the function name

  \onslide<6-> \textit{name} is a parameter (without a type)

  \onslide<7-> \textit{"Fabrice"} and \textit{42} are arguments

\end{frame}


%%% Parameters
\subsection{Parameters}

\begin{frame}<beamer>{Parameters}

  \begin{itemize}
    \item<1-> Positional arguments
      \begin{itemize}
      \item<2-> \TTBF{def MyFun(param1, param2, param3)}
      \item<3-> \TTBF{MyFun("abc", 42, 1337)}
      \item<4-> \only<4>{\TTBF{MyFun(42, "abc", 1337)}} \only<5->{\TTBF{MyFun(\sout{42}, \sout{"abc"}, 1337)}}
      \end{itemize}
    \item<6-> \only<6>{Keywords arguments} \only<7->{Keywords arguments (with default values)}
      \begin{itemize}
      \item<7-> \TTBF{def MyFun(p1="A", p2=1, p3=9)}
      \item<8-> \TTBF{MyFun("abc", 42, 1337)}
      \item<9-> \TTBF{MyFun(p1="abc", p2=42, p3=1337)}
      \item<10-> \TTBF{MyFun(p3=1337, p2=42, p1="abc")}
      \item<11-> \TTBF{MyFun(p2=42)}
      \end{itemize}
  \end{itemize}

\end{frame}

%%% Positional arguments

\begin{frame}[fragile]{Positional arguments}

  \begin{columns}[onlytextwidth]
    \begin{column}{\textwidth}

      \begin{onlyenv}<1>
        \begin{lstlisting}[style=python]










 \end{lstlisting}
      \end{onlyenv}

      \begin{onlyenv}<2>
        \begin{lstlisting}[style=python]
from datetime import date
def GreetingsPos(Name, Year):
  print("Hi " + Name + "!")
  if (Year <= 0):
    print("(unknown age)")
  else:
    age = date.today().year - Year
    print("(" + str(age) + " years)")


 \end{lstlisting}
      \end{onlyenv}

      \begin{onlyenv}<3>
        \begin{lstlisting}[style=python]
from datetime import date
def GreetingsPos(Name, Year):
  print("Hi " + Name + "!")
  if (Year <= 0):
    print("(unknown age)")
  else:
    age = date.today().year - Year
    print("(" + str(age) + " years)")

GreetingsPos("Fabrice", 1988)
 \end{lstlisting}
      \end{onlyenv}

      \begin{onlyenv}<4>
        \begin{lstlisting}[style=python]
from datetime import date
def GreetingsPos(Name, Year):
  print("Hi " + Name + "!")
  if (Year <= 0):
    print("(unknown age)")
  else:
    age = date.today().year - Year
    print("(" + str(age) + " years)")

GreetingsPos("Fabrice", 1988) # OK
GreetingsPos("Fabrice") \end{lstlisting}
      \end{onlyenv}

      \begin{onlyenv}<5->
        \begin{lstlisting}[style=python]
from datetime import date
def GreetingsPos(Name, Year):
  print("Hi " + Name + "!")
  if (Year <= 0):
    print("(unknown age)")
  else:
    age = date.today().year - Year
    print("(" + str(age) + " years)")

GreetingsPos("Fabrice", 1988) # OK
GreetingsPos("Fabrice")       # error \end{lstlisting}
      \end{onlyenv}

    \end{column}
  \end{columns}

  \onslide<6-> All of the parameters are required if positional arguments

\end{frame}


%%% Keywords arguments / Default parameter values

\begin{frame}[fragile]{Keywords arguments}

  \begin{columns}[onlytextwidth]
    \begin{column}{\textwidth}

      \begin{onlyenv}<1>
        \begin{lstlisting}[style=python]






 \end{lstlisting}
      \end{onlyenv}

      \begin{onlyenv}<2>
        \begin{lstlisting}[style=python]
def GreetingsKey(FName="Unknown", BYear=0):
  GreetingsPos(FName, BYear)




 \end{lstlisting}
      \end{onlyenv}

      \begin{onlyenv}<3>
        \begin{lstlisting}[style=python]
def GreetingsKey(FName="Unknown", BYear=0):
  GreetingsPos(FName, BYear)

GreetingsKey("Fabrice", 1988)


 \end{lstlisting}
      \end{onlyenv}

      \begin{onlyenv}<4>
        \begin{lstlisting}[style=python]
def GreetingsKey(FName="Unknown", BYear=0):
  GreetingsPos(FName, BYear)

GreetingsKey("Fabrice", 1988)  # OK
GreetingsKey("Fabrice")

 \end{lstlisting}
      \end{onlyenv}

      \begin{onlyenv}<5>
        \begin{lstlisting}[style=python]
def GreetingsKey(FName="Unknown", BYear=0):
  GreetingsPos(FName, BYear)

GreetingsKey("Fabrice", 1988)  # OK
GreetingsKey("Fabrice")        # OK
GreetingsKey(BYear=1988, FName="Fabrice")
 \end{lstlisting}
      \end{onlyenv}

      \begin{onlyenv}<6>
        \begin{lstlisting}[style=python]
def GreetingsKey(FName="Unknown", BYear=0):
  GreetingsPos(FName, BYear)

GreetingsKey("Fabrice", 1988)  # OK
GreetingsKey("Fabrice")        # OK
GreetingsKey(BYear=1988, FName="Fabrice") # OK
GreetingsKey("Fabrice", BYear=1988) \end{lstlisting}
      \end{onlyenv}

      \begin{onlyenv}<7->
        \begin{lstlisting}[style=python]
def GreetingsKey(FName="Unknown", BYear=0):
  GreetingsPos(FName, BYear)

GreetingsKey("Fabrice", 1988)  # OK
GreetingsKey("Fabrice")        # OK
GreetingsKey(BYear=1988, FName="Fabrice") # OK
GreetingsKey("Fabrice", BYear=1988) # OK \end{lstlisting}
      \end{onlyenv}

    \end{column}
  \end{columns}

  \onslide<8-> Keywords arguments are more flexibles

\end{frame}


\begin{frame}[fragile]{Keywords arguments}

  \begin{columns}[onlytextwidth]
    \begin{column}{\textwidth}

      \begin{onlyenv}<1>
        \begin{lstlisting}[style=python]







 \end{lstlisting}
      \end{onlyenv}

      \begin{onlyenv}<2>
        \begin{lstlisting}[style=python]
def GreetingsKey2(FName="", BYear=""):
  GreetingsPos(FName, BYear)





 \end{lstlisting}
      \end{onlyenv}

      \begin{onlyenv}<3>
        \begin{lstlisting}[style=python]
def GreetingsKey2(FName="", BYear=""):
  GreetingsPos(FName, BYear)

GreetingsKey2("Fabrice", 1988)



 \end{lstlisting}
      \end{onlyenv}

      \begin{onlyenv}<4>
        \begin{lstlisting}[style=python]
def GreetingsKey2(FName="", BYear=""):
  GreetingsPos(FName, BYear)

GreetingsKey2("Fabrice", 1988)  # OK
GreetingsKey2(BYear=1988, FName="Fabrice")


 \end{lstlisting}
      \end{onlyenv}

      \begin{onlyenv}<5>
        \begin{lstlisting}[style=python]
def GreetingsKey2(FName="", BYear=""):
  GreetingsPos(FName, BYear)

GreetingsKey2("Fabrice", 1988)  # OK
GreetingsKey2(BYear=1988, FName="Fabrice") # OK
GreetingsKey2(BYear=1988)

 \end{lstlisting}
      \end{onlyenv}

      \begin{onlyenv}<6>
        \begin{lstlisting}[style=python]
def GreetingsKey2(FName="", BYear=""):
  GreetingsPos(FName, BYear)

GreetingsKey2("Fabrice", 1988)  # OK
GreetingsKey2(BYear=1988, FName="Fabrice") # OK
GreetingsKey2(BYear=1988)  # OK
GreetingsKey2()
 \end{lstlisting}
      \end{onlyenv}

      \begin{onlyenv}<7>
        \begin{lstlisting}[style=python]
def GreetingsKey2(FName="", BYear=""):
  GreetingsPos(FName, BYear)

GreetingsKey2("Fabrice", 1988)  # OK
GreetingsKey2(BYear=1988, FName="Fabrice") # OK
GreetingsKey2(BYear=1988)  # OK
GreetingsKey2() # error
GreetingsKey2("Fabrice") \end{lstlisting}
      \end{onlyenv}

      \begin{onlyenv}<8->
        \begin{lstlisting}[style=python]
def GreetingsKey2(FName="", BYear=""):
  GreetingsPos(FName, BYear)

GreetingsKey2("Fabrice", 1988)  # OK
GreetingsKey2(BYear=1988, FName="Fabrice") # OK
GreetingsKey2(BYear=1988)  # OK
GreetingsKey2() # error
GreetingsKey2("Fabrice") # error \end{lstlisting}
      \end{onlyenv}

    \end{column}
  \end{columns}

  \onslide<9-> Keywords arguments forces a default value

\end{frame}


%%% Scope of variables
\subsection{Scope of variables}

\begin{frame}<beamer>{Scope of variables}

  Variables are searched within scopes in a specific order:

  \begin{itemize}
    \item<1-> Local (scope of the current function)
    \item<2-> Global (global variables of the program)
    \item<3-> Internal (variables of the interpreter)
  \end{itemize}

  \bigskip

  \onslide<4-> LGI rule

  \medskip

  \onslide<5-> Constants are global variables (uppercase name)

\end{frame}

%%% Return values
\subsection{Return values}

%%% Multiple return of values in functions (return a tuple)

\begin{frame}[fragile]{Multiple return of values}

  \begin{columns}[onlytextwidth]
    \begin{column}{\textwidth}

      \begin{onlyenv}<1>
        \begin{lstlisting}[style=python]










 \end{lstlisting}
      \end{onlyenv}

      \begin{onlyenv}<2>
        \begin{lstlisting}[style=python]
def ConvertTemperature(kelvin):
  celsius = kelvin - 273
  fahrenheit = ((celsius * 9) / 5) + 32
  reaumur = (kelvin * 4) / 5
  return celsius, fahrenheit, reaumur





 \end{lstlisting}
      \end{onlyenv}

      \begin{onlyenv}<3>
        \begin{lstlisting}[style=python]
def ConvertTemperature(kelvin):
  celsius = kelvin - 273
  fahrenheit = ((celsius * 9) / 5) + 32
  reaumur = (kelvin * 4) / 5
  return celsius, fahrenheit, reaumur

temps = ConvertTemperature(42)



 \end{lstlisting}
      \end{onlyenv}

      \begin{onlyenv}<4>
        % Balises exception :  %* *)
        \begin{lstlisting}[style=python]
def ConvertTemperature(kelvin):
  celsius = kelvin - 273
  fahrenheit = ((celsius * 9) / 5) + 32
  reaumur = (kelvin * 4) / 5
  return celsius, fahrenheit, reaumur

temps = ConvertTemperature(42)
print("%*\textdegree*)K : " + str(42))
print("%*\textdegree*)C : " + str(temps[0]))
print("%*\textdegree*)F : " + str(temps[1]))
print("%*\textdegree*)Re: " + str(temps[2])) \end{lstlisting}
      \end{onlyenv}

      \begin{onlyenv}<5->
        \begin{lstlisting}[style=python]
def ConvertTemperature(kelvin):
  celsius = kelvin - 273
  fahrenheit = ((celsius * 9) / 5) + 32
  reaumur = (kelvin * 4) / 5
  return celsius, fahrenheit, reaumur

temps = ConvertTemperature(42)
print("%*\textdegree*)K : " + str(42))       # 42
print("%*\textdegree*)C : " + str(temps[0])) # -231
print("%*\textdegree*)F : " + str(temps[1])) # -383
print("%*\textdegree*)Re: " + str(temps[2])) # 33 \end{lstlisting}
      \end{onlyenv}

    \end{column}
  \end{columns}

  \onslide<6-> Multiple return values uses tuples

\end{frame}


\begin{frame}[fragile]{Multiple return of values}

  Multiple affectations are also possible

  \begin{columns}[onlytextwidth]
    \begin{column}{\textwidth}

      \begin{onlyenv}<1>
        \begin{lstlisting}[style=python]










 \end{lstlisting}
      \end{onlyenv}

      \begin{onlyenv}<2>
        \begin{lstlisting}[style=python]
def ConvertTemperature(kelvin):
  celsius = kelvin - 273
  fahrenheit = ((celsius * 9) / 5) + 32
  reaumur = (kelvin * 4) / 5
  return celsius, fahrenheit, reaumur





 \end{lstlisting}
      \end{onlyenv}

      \begin{onlyenv}<3>
        \begin{lstlisting}[style=python]
def ConvertTemperature(kelvin):
  celsius = kelvin - 273
  fahrenheit = ((celsius * 9) / 5) + 32
  reaumur = (kelvin * 4) / 5
  return celsius, fahrenheit, reaumur

tC, tF, tR = ConvertTemperature(42)



 \end{lstlisting}
      \end{onlyenv}

      \begin{onlyenv}<4>
        \begin{lstlisting}[style=python]
def ConvertTemperature(kelvin):
  celsius = kelvin - 273
  fahrenheit = ((celsius * 9) / 5) + 32
  reaumur = (kelvin * 4) / 5
  return celsius, fahrenheit, reaumur

tC, tF, tR = ConvertTemperature(42)
print("%*\textdegree*)K : " + str(42))
print("%*\textdegree*)C : " + str(tC))
print("%*\textdegree*)F : " + str(tF))
print("%*\textdegree*)Re: " + str(tR)) \end{lstlisting}
      \end{onlyenv}

      \begin{onlyenv}<5->
        \begin{lstlisting}[style=python]
def ConvertTemperature(kelvin):
  celsius = kelvin - 273
  fahrenheit = ((celsius * 9) / 5) + 32
  reaumur = (kelvin * 4) / 5
  return celsius, fahrenheit, reaumur

tC, tF, tR = ConvertTemperature(42)
print("%*\textdegree*)K : " + str(42)) # 42
print("%*\textdegree*)C : " + str(tC)) # -231
print("%*\textdegree*)F : " + str(tF)) # -383
print("%*\textdegree*)Re: " + str(tR)) # 33 \end{lstlisting}
      \end{onlyenv}

    \end{column}
  \end{columns}

\end{frame}


%%%%%%%%%%%%%%%%%%%%%%%%%%%%%%%%%%%%%%%%%%%%%%%%%%%%%%%%

\section{Modules \& Imports}

%%% Modules
\subsection{Modules}

% Module : just put your functions with variables in a .py file
%     triple quotes """ text """ is not a comment, but a "docstring"
%     put one at the beginning for documenting the module, then one per function

\begin{frame}[fragile]{Modules}

  A module contains functions and variables (and classes)

  \medskip

  \onslide<1-> Module name is the filename withtout \TTBF{.py}

  \medskip

  \onslide<2-> Some conventions for a nice documentation:

  \begin{itemize}
    \item<3-> Begin your module with a \textit{docstring}
    \begin{itemize}
      \item triple quotes \lstinline|""" docstring """|
      \item write the description of your module
      \item docstring can be on multiple lines
    \end{itemize}
    \item<4-> Within each function, write a docstring about it
    \item<5-> Eventually, add useful constants
  \end{itemize}

  \bigskip

  \onslide<7-> Check result with \TTBF{help(module)} (after importing it)

\end{frame}


\begin{frame}[fragile]{Modules}

  \TTBF{MyModule.py} \hspace*{2cm} (module name: \TTBF{MyModule})

  \begin{columns}[onlytextwidth]
    \begin{column}{\textwidth}

      \begin{onlyenv}<1>
        \begin{lstlisting}[style=python]









 \end{lstlisting}
      \end{onlyenv}

      \begin{onlyenv}<2>
        \begin{lstlisting}[style=python]
""" Module for explaining modules """








 \end{lstlisting}
      \end{onlyenv}

      \begin{onlyenv}<3>
        \begin{lstlisting}[style=python]
""" Module for explaining modules """
MYCONST=42







 \end{lstlisting}
      \end{onlyenv}

      \begin{onlyenv}<4>
        \begin{lstlisting}[style=python]
""" Module for explaining modules """
MYCONST=42

def MyFunc(test):
  print("Hello World!")


def OtherFunc(var):
  print("Test.")
 \end{lstlisting}
      \end{onlyenv}

      \begin{onlyenv}<5->
        \begin{lstlisting}[style=python]
""" Module for explaining modules """
MYCONST=42

def MyFunc(test):
  """ Function for explaining modules """
  print("Hello World!")

def OtherFunc(var):
  """ Another function """
  print("Test.") \end{lstlisting}
      \end{onlyenv}

    \end{column}
  \end{columns}

\end{frame}


%%% Imports
\subsection{Imports}

\begin{frame}[fragile]{Imports}

  Multiple ways for importing modules:

  \begin{itemize}
    \item<1-> Method 1: \TTBF{import MyModule}
    \item<2-> Method 2: \TTBF{import MyModule as MyM}
    \item<3-> Method 3: \TTBF{from MyModule import MyFunc}
  \end{itemize}

  \bigskip

  \onslide<4-> Variables and constants can be imported too

  \medskip

  \onslide<5-> Beware: with method 1, \textit{everything} is imported

\end{frame}


\begin{frame}[fragile]{Imports}

  Method 1: \TTBF{MyModule.py}

  \begin{columns}[onlytextwidth]
    \begin{column}{\textwidth}

      \begin{onlyenv}<1>
        \begin{lstlisting}[style=python]



 \end{lstlisting}
      \end{onlyenv}

      \begin{onlyenv}<2>
        \begin{lstlisting}[style=python]
import MyModule
# function: def MyFunc(test)

 \end{lstlisting}
      \end{onlyenv}

      \begin{onlyenv}<3->
        \begin{lstlisting}[style=python]
import MyModule
# function: def MyFunc(test)

MyModule.MyFunc(42) \end{lstlisting}
      \end{onlyenv}

    \end{column}
  \end{columns}

\end{frame}


\begin{frame}[fragile]{Imports}

  Method 2: \TTBF{MyModule.py}

  \begin{columns}[onlytextwidth]
    \begin{column}{\textwidth}

      \begin{onlyenv}<1>
        \begin{lstlisting}[style=python]



 \end{lstlisting}
      \end{onlyenv}

      \begin{onlyenv}<2>
        \begin{lstlisting}[style=python]
import MyModule as MyM
# function: def MyFunc(test)

 \end{lstlisting}
      \end{onlyenv}

      \begin{onlyenv}<3->
        \begin{lstlisting}[style=python]
import MyModule as MyM
# function: def MyFunc(test)

MyM.MyFunc(42) \end{lstlisting}
      \end{onlyenv}

    \end{column}
  \end{columns}

\end{frame}


\begin{frame}[fragile]{Imports}

  Method 3: \TTBF{MyModule.py}

  \begin{columns}[onlytextwidth]
    \begin{column}{\textwidth}

      \begin{onlyenv}<1>
        \begin{lstlisting}[style=python]



 \end{lstlisting}
      \end{onlyenv}

      \begin{onlyenv}<2>
        \begin{lstlisting}[style=python]
from MyModule import MyFunc
# function: def MyFunc(test)

 \end{lstlisting}
      \end{onlyenv}

      \begin{onlyenv}<3->
        \begin{lstlisting}[style=python]
from MyModule import MyFunc
# function: def MyFunc(test)

MyFunc(42) \end{lstlisting}
      \end{onlyenv}

    \end{column}
  \end{columns}

\end{frame}


%%% Packages
\subsection{Packages}

\begin{frame}[fragile]{Packages}

  \begin{center}

  Packages contain multiple modules and dependencies

  \bigskip

  \onslide<2-> \textit{(see documentation and tutorials about how to build one)}

  \end{center}

\end{frame}
