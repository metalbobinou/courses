% -*- latex -*-

\part{Python Quick Scripting}  % Hello world, print, str/int, Sys module

%%%%%%%%%%%%%%%%%%%%%%%%%%%%%%%%%%%%%%%%%%%%%%%%%%%%%%%%

\section{First scripts}

%%% Hello World

\begin{frame}[fragile]{Hello World!}

%  \onslide<1-> Your first python code:

  \begin{columns}[onlytextwidth]
    \begin{column}{\textwidth}
      \begin{onlyenv}<1->
        %\begin{lstlisting}[style=python,numbers=left,stepnumber=1]
        \begin{lstlisting}[style=python]
#! /usr/bin/python

print("Hello World!") \end{lstlisting}
      \end{onlyenv}
    \end{column}
  \end{columns}

  \onslide<2-> L1: The interpreter can be ajdusted

  \onslide<3-> L3: \TTBF{print} takes a string as a parameter

  \onslide<4-> L3: Strings are enclosed within quotes (\TTBF{\textquotesingle}) or double quotes (\TTBF{\textquotedbl})

\end{frame}

%%% Declaration, strings & integers

\begin{frame}[fragile]{Concatenation \& Printing integers}

  \begin{columns}[onlytextwidth]
    \begin{column}{\textwidth}
      \begin{onlyenv}<1->
        \begin{lstlisting}[style=python]
#! /usr/bin/python

var = 42
print("Hello World! " + str(var)) \end{lstlisting}
      \end{onlyenv}
    \end{column}
  \end{columns}

  \onslide<2-> Semicolons (\TTBF{;}) at the end of statements are optional (rarely used)

  \onslide<3-> L3: Declare variables and assign value directly (type not needed)

  \onslide<4-> L4: String concatenation with \TTBF{+}

  \onslide<5-> L4: Conversion from integer or float to string with \TTBF{str()} function

\end{frame}

%%% Read arguments

\begin{frame}[fragile]{Read arguments}

  \begin{columns}[onlytextwidth]
    \begin{column}{\textwidth}
      \begin{onlyenv}<1->
        \begin{lstlisting}[style=python,title={args.py}]
#! /usr/bin/python

import sys
print("Hello World! " + sys.argv[0]) \end{lstlisting}
      \end{onlyenv}
    \end{column}
  \end{columns}

  \begin{columns}[onlytextwidth]
    \begin{column}{\textwidth}
      \begin{onlyenv}<1>
        \begin{lstlisting}[style=sh]

 \end{lstlisting}
      \end{onlyenv}

      \begin{onlyenv}<2>
        \begin{lstlisting}[style=sh]
%*\LSTPrompt*) python args.py
 \end{lstlisting}
      \end{onlyenv}

      \begin{onlyenv}<3->
        \begin{lstlisting}[style=sh]
%*\LSTPrompt*) python args.py
Hello World! args.py \end{lstlisting}
      \end{onlyenv}

    \end{column}
  \end{columns}

\end{frame}

\begin{frame}[fragile]{Read arguments}

  \begin{columns}[onlytextwidth]
    \begin{column}{\textwidth}
      \begin{onlyenv}<1->
        \begin{lstlisting}[style=python]
#! /usr/bin/python

import sys
print("Hello World! " + sys.argv[0]) \end{lstlisting}
      \end{onlyenv}
    \end{column}
  \end{columns}

  \onslide<2-> L3: Includes an external module (\TTBF{import module})

  \onslide<3-> L3: Access to arguments is made through \TTBF{sys} module

  \onslide<3-> L4: Access to an attribute is made with a dot (\TTBF{.})

  \onslide<4-> L4: \TTBF{argv} is an array (like in C and others)

  \onslide<5-> L4: Arrays begin at index \TTBF{0}

\end{frame}


%%% Show types
\section{Quick overview of types}

\begin{frame}[fragile]{Show types/Quick debug}

  \begin{columns}[onlytextwidth]
    \begin{column}{\textwidth}
      \begin{onlyenv}<1->
        \begin{lstlisting}[style=python,title={types1.py}]
#! /usr/bin/python
type(42)
type("Hello World!") \end{lstlisting}
      \end{onlyenv}
    \end{column}
  \end{columns}

    \begin{columns}[onlytextwidth]
    \begin{column}{\textwidth}
      \begin{onlyenv}<1>
        \begin{lstlisting}[style=sh]
%*\LSTPrompt*) python types1.py

       \end{lstlisting}
      \end{onlyenv}

      \begin{onlyenv}<2->
        \begin{lstlisting}[style=sh]
%*\LSTPrompt*) python types1.py
<class 'int'>
<class 'str'> \end{lstlisting}
      \end{onlyenv}
    \end{column}
  \end{columns}

  \onslide<3-> Function \TTBF{type()} writes the type of the parameter
\end{frame}


\begin{frame}[fragile]{Show types/Quick debug}

  \begin{columns}[onlytextwidth]
    \begin{column}{\textwidth}
      \begin{onlyenv}<1->
        \begin{lstlisting}[style=python,title={types2.py}]
#! /usr/bin/python
import sys
type(sys)
type(print("lol")) \end{lstlisting}
      \end{onlyenv}
    \end{column}
  \end{columns}

    \begin{columns}[onlytextwidth]
    \begin{column}{\textwidth}
      \begin{onlyenv}<1>
        \begin{lstlisting}[style=sh]
%*\LSTPrompt*) python types2.py


        \end{lstlisting}
      \end{onlyenv}

      \begin{onlyenv}<2->
        \begin{lstlisting}[style=sh]
%*\LSTPrompt*) python types2.py
<class 'module'>
lol
<class 'NoneType'> \end{lstlisting}
      \end{onlyenv}
    \end{column}
  \end{columns}
\end{frame}


%%% Syntax and control flow
\section{Overview of syntax and control flow}

\begin{frame}[fragile]{Functions overview}

  \begin{columns}[onlytextwidth]
    \begin{column}{\textwidth}
      \begin{onlyenv}<1>
        \begin{lstlisting}[style=python,title={functions1.py}]
print("Hello World!")



 \end{lstlisting}
      \end{onlyenv}

      \begin{onlyenv}<2>
        \begin{lstlisting}[style=python,title={functions1.py}]
def MyFunction():
  print("Hello World!")
  return (0)

 \end{lstlisting}
      \end{onlyenv}

      \begin{onlyenv}<3->
        \begin{lstlisting}[style=python,title={functions1.py}]
def MyFunction():
  print("Hello World!")
  return (0)

MyFunction() \end{lstlisting}
      \end{onlyenv}
    \end{column}
  \end{columns}

    \begin{columns}[onlytextwidth]
    \begin{column}{\textwidth}
      \begin{onlyenv}<4>
        \begin{lstlisting}[style=sh]
%*\LSTPrompt*) python functions1.py
 \end{lstlisting}
      \end{onlyenv}

      \begin{onlyenv}<5->
        \begin{lstlisting}[style=sh]
%*\LSTPrompt*) python functions1.py
Hello World! \end{lstlisting}
      \end{onlyenv}
    \end{column}
  \end{columns}
\end{frame}


%%% Functions

\begin{frame}[fragile]{Functions overview}

%  Functions

  \begin{columns}[onlytextwidth]
    \begin{column}{\textwidth}
      \begin{onlyenv}<1->
        \begin{lstlisting}[style=python]
def MyFunction():
  print("Hello World!")
  return (0)

MyFunction() \end{lstlisting}
      \end{onlyenv}
    \end{column}
  \end{columns}

  \onslide<2-> L1: Functions begin by a \TTBF{def} and are followed by their parameters

  \onslide<3-> L1: Definition of functions are terminated by a semicolon (\TTBF{:})

  \onslide<4-> L2: Indentation defines in which scope the line is

  \onslide<5-> L2: \textbf{Indentation is VERY IMPORTANT!}

  \onslide<6-> L3 \& L5: Returning and calling are similar to C

\end{frame}


%%% Conditions

\begin{frame}[fragile]{Conditions overview}

%  Conditions
  \TTBF{if} ... \TTBF{elif} ... \TTBF{else} ...

  \begin{columns}[onlytextwidth]
    \begin{column}{\textwidth}
      \begin{onlyenv}<1->
        \begin{lstlisting}[style=python]
def MyOtherFunction():
  var = 42
  if (var < 42):
    print("Oh no...")
  elif (var > 42):
    print("WT...")
  else:
    print("OK") \end{lstlisting}
      \end{onlyenv}
    \end{column}
  \end{columns}

%  \onslide<2-> L3 \& L5 \& L6: \TTBF{if} ... \TTBF{elif} ... \TTBF{else} ...
%
%  \onslide<3-> L3 \& L5 \& L6: Conditions are followed by a semicolon (\TTBF{:})
  \onslide<2-> L3 \& L5 \& L6: Conditions are followed by a semicolon (\TTBF{:})

\end{frame}


\begin{frame}[fragile]{Conditions overview}

%  Conditions \hspace{3cm} \textbf{(only since Python 3.10)}
  \TTBF{match} ... \TTBF{case} ...  \hspace{3cm}  \textit{(only since Python 3.10)}

  \begin{columns}[onlytextwidth]
    \begin{column}{\textwidth}
      \begin{onlyenv}<1->
        \begin{lstlisting}[style=python]
var = "Hello World!"
match var:
  case ['Hello']:
    print("Beginning")
  case ['World!' | 'Hello World!']:
    print("End")
  case _:
    print("In any other cases") \end{lstlisting}
      \end{onlyenv}
    \end{column}
  \end{columns}

  \onslide<2-> Similar to \textit{switch-case} in C, without needing \textit{return} or \textit{break}

  \onslide<3-> More complex than that (check the documentation after the course)

\end{frame}


%%% Loops

\begin{frame}[fragile]{Loops overview}

%  Loops (while)
  \TTBF{while}

  \begin{columns}[onlytextwidth]
    \begin{column}{\textwidth}
      \begin{onlyenv}<1->
        \begin{lstlisting}[style=python]
var = 0
while var < 42:
  print("Hi Nations!")
  var += 1 \end{lstlisting}
      \end{onlyenv}
    \end{column}
  \end{columns}

  \onslide<2-> L2: Regular \TTBF{while} loop

  \onslide<2-> L4: \TTBF{+=} operator acts like \TTBF{var = var + 1}

  \onslide<3-> Don't forget indentation

\end{frame}


\begin{frame}[fragile]{Loops overview}

%  Loops (for)
  \TTBF{for} (1)

  \begin{columns}[onlytextwidth]
    \begin{column}{\textwidth}
      \begin{onlyenv}<1->
        \begin{lstlisting}[style=python]
for var in range(0, 10):
  print("Hi Nations!")

my_text = "Yo Countries!"
for char in my_text:
  print(char) \end{lstlisting}
      \end{onlyenv}
    \end{column}
  \end{columns}

  \onslide<2-> L1: \TTBF{range} calculates values from \TTBF{0} to \TTBF{10}

  \onslide<3-> L1: \TTBF{in} iterates through each value of a list

  \onslide<3-> L1: Each value will be put into the variable before \TTBF{in}

  \onslide<4-> L4 \& L5: Strings are considered as characters lists

\end{frame}


\begin{frame}[fragile]{Loops overview}

  \TTBF{for} (2)

  \begin{columns}[onlytextwidth]
    \begin{column}{\textwidth}
      \begin{onlyenv}<1->
        \begin{lstlisting}[style=python]
my_list = [ 5, 2, 3, 1, 4 ]
for var in range(len(my_list)):
  if (var < 2):
    break
  else:
    print("Hi Nations!") \end{lstlisting}
      \end{onlyenv}
    \end{column}
  \end{columns}

  \onslide<2-> L1: Declaration of a list

  \onslide<3-> L1: Never put a dash (\TTBF{-}) in variables name (use an underscore \TTBF{\_})

  \onslide<4-> L2: \TTBF{len} gets the length of a list

  \onslide<5-> L4: \TTBF{break} ends the loop

\end{frame}


%%% Exceptions

\begin{frame}[fragile]{Exceptions overview}

  \TTBF{try} ... \TTBF{except} ...

  \begin{columns}[onlytextwidth]
    \begin{column}{\textwidth}
      \begin{onlyenv}<1->
        \begin{lstlisting}[style=python,morekeywords={as}]
var = 3
try:
  var = 42 / var
except Exception as exc:
  print("Error: " + str(exc)) \end{lstlisting}
      \end{onlyenv}
    \end{column}
  \end{columns}

  \onslide<2-> L3: Begin the block of code to check with a \TTBF{try}

  \onslide<3-> L6: Catch exceptions with a \TTBF{except}

  \onslide<4-> L6: Catch any exception with \TTBF{Exception} and put it in \TTBF{exc}

  \onslide<5-> L6 \& L7: Write the actions to take if an exception occurs

  \onslide<6-> L7: Don't forget to convert to a string with \TTBF{str()}

\end{frame}


\begin{frame}[fragile]{Exceptions overview}

  \TTBF{try} ... \TTBF{except} ... \TTBF{else} ... \TTBF{finally}

  \begin{columns}[onlytextwidth]
    \begin{column}{\textwidth}
      \begin{onlyenv}<1->
        \begin{lstlisting}[style=python,morekeywords={as}]
try:
  var = 42 / 3
except ZeroDivisionError:
  print("There was a division by zero")
else:
  print("It worked, result: ", var)
finally:
  print("--After everything--")\end{lstlisting}
      \end{onlyenv}
    \end{column}
  \end{columns}

  \onslide<2-> L4: Check for a specific exception (division by zero)

  \onslide<3-> L6: If no exception was raised, it executes the \TTBF{else}

  \onslide<4-> L8: In any case (exception or not), it executes the \TTBF{finally} clause

\end{frame}


\begin{frame}[fragile]{}

  \begin{columns}[onlytextwidth]
    \begin{column}{\textwidth}
      \begin{onlyenv}<1->
        \begin{lstlisting}[style=python,morekeywords={as}]
def FragileFunction():
  var = 42
  if (var == 42):
    raise ValueError("Argf")
  else:
    return (0)

def MainFunction():
  print("--Before--")
  try:
    FragileFunction()
  except ValueError as exp:
    print("Exception caught: ", exp)
  print("--After--") \end{lstlisting}
      \end{onlyenv}
    \end{column}
  \end{columns}

\end{frame}

\begin{frame}[fragile]{}

    \begin{columns}[onlytextwidth]
    \begin{column}{\textwidth}
      \begin{onlyenv}<1->
        \begin{lstlisting}[style=python,morekeywords={as}]
class MyCustomError(Exception):
  pass

def FragileFunction():
  raise MyCustomError("Argf")

def MainFunction():
  print("--Before--")
  try:
    FragileFunction()
  except MyCustomError as exp:
    print("Exception caught: ", exp)
  print("--After--") \end{lstlisting}
      \end{onlyenv}
    \end{column}
  \end{columns}

\end{frame}

\begin{frame}[fragile]{Exceptions overview}

  \TTBF{raise} and custom exceptions

  \begin{itemize}
    \item<2-> \TTBF{raise} instruction triggers an exception
    \item<3-> Use \TTBF{ValueError} in order to customize the message...
    \item<4-> ...or create a class with the name of your exception and \TTBF{Exception} as a parameter
    \item<5-> When raising a custom exception, the message is optional
  \end{itemize}

  \begin{columns}[onlytextwidth]
    \begin{column}{\textwidth}
      \begin{onlyenv}<6->
        \begin{lstlisting}[style=python,morekeywords={as}]
class MyCustomError(Exception):
  pass

def FragileFunction():
  raise MyCustomError \end{lstlisting}
      \end{onlyenv}
    \end{column}
  \end{columns}

\end{frame}
