% -*- latex -*-

\part{Class, Files, and Directories}  % Class, Files, and Directories

%%%%%%%%%%%%%%%%%%%%%%%%%%%%%%%%%%%%%%%%%%%%%%%%%%%%%%%%

% How to create a class

\section{Classes}

\begin{frame}<beamer>{OOP}

  Object Oriented Programming (OOP) vocabulary

  \begin{itemize}
    \item<2-> Class: the type and structure
    \item<3-> Attribute: a variable that is a member of the class
    \item<4-> Method: a function/procedure that is a member of the class
  \end{itemize}

  \begin{itemize}
    \item<5-> Object: an instance of a class
    \item<6-> Constructor: the first method called when creating an object
    \item<7-> Destructor: the method called when deleting an object
  \end{itemize}

  \begin{itemize}
    \item<8-> Inheritance: structure of a class reused as a basis for a new one
    \item<9-> Private: when a member is accessible only by the object itself
    \item<10-> Public: when a member is accessible by any object
  \end{itemize}

\end{frame}


\begin{frame}<beamer>{Classes}

  Specificities of classes in Python:

  \begin{itemize}
    \item<2-> Only one constructor: \lstinline|__init__|
    \item<3-> No destructor (python manages the memory by references)
    \item<4-> Attributes are at least \textit{read-only}, and eventually \textit{writable}
    \item<5-> Writable attributes can be deleted from the object
    \item<6-> Member beginning by an underscore (\lstinline|_|) aren't strictly private, but should be considered internal to the class
    \item<7-> \TTBF{self} keyword is required as the first parameter of each method
  \end{itemize}

\end{frame}


\begin{frame}[fragile]{Classes}

  \begin{columns}[onlytextwidth]
    \begin{column}{\textwidth}

      \begin{onlyenv}<1>
        % Balises exception :  %* *)
        \begin{lstlisting}[style=python]











 \end{lstlisting}
      \end{onlyenv}

      \begin{onlyenv}<2>
        \begin{lstlisting}[style=python]
class Vehicle:
  """ General vehicles """









 \end{lstlisting}
      \end{onlyenv}

      \begin{onlyenv}<3>
        \begin{lstlisting}[style=python]
class Vehicle:
  """ General vehicles """
  __speed = 0
  Passengers = 0







 \end{lstlisting}
      \end{onlyenv}

      \begin{onlyenv}<4>
        \begin{lstlisting}[style=python]
class Vehicle:
  """ General vehicles """
  __speed = 0
  Passengers = 0
  # Constructor
  def __init__(self):
    self.Passengers = 1




 \end{lstlisting}
      \end{onlyenv}

      \begin{onlyenv}<5>
        \begin{lstlisting}[style=python]
class Vehicle:
  """ General vehicles """
  __speed = 0
  Passengers = 0
  # Constructor
  def __init__(self):
    self.Passengers = 1
  # Method
  def getSpeed(self):
    return (self.__speed)

 \end{lstlisting}
      \end{onlyenv}

      \begin{onlyenv}<6>
        \begin{lstlisting}[style=python]
class Vehicle:
  """ General vehicles """
  __speed = 0
  Passengers = 0
  # Constructor
  def __init__(self):
    self.Passengers = 1
  # Methods
  def getSpeed(self):
    return (self.__speed)
  def Accelerate(self, x):
    self.__speed += x \end{lstlisting}
      \end{onlyenv}


      \begin{onlyenv}<7->
        \begin{lstlisting}[style=python]
class Vehicle:
  """ General vehicles """
  __speed = 0
  Passengers = 0
  # Constructor
  def __init__(self):       # Constructor
    self.Passengers = 1
  # Methods
  def getSpeed(self):       # Accessor
    return (self.__speed)
  def Accelerate(self, x):  # Mutator
    self.__speed += x \end{lstlisting}
      \end{onlyenv}

    \end{column}
  \end{columns}

\end{frame}


\begin{frame}[fragile]{Classes}

  \begin{columns}[onlytextwidth]
    \begin{column}{\textwidth}

      \begin{onlyenv}<1>
        \begin{lstlisting}[style=python]









 \end{lstlisting}
      \end{onlyenv}

      \begin{onlyenv}<2>
        \begin{lstlisting}[style=python]
class Car(Vehicle):
  """ Cars inherit from Vehicle """
  __CV = 0






 \end{lstlisting}
      \end{onlyenv}

      \begin{onlyenv}<3>
        \begin{lstlisting}[style=python]
class Car(Vehicle):
  """ Cars inherit from Vehicle """
  __CV = 0
  # Constructor
  def __init__(self, CO2, P):
    self.Passengers = 1
    self.__CV = (CO2 / 45) + (P / 40)


 \end{lstlisting}
      \end{onlyenv}

      \begin{onlyenv}<4->
        \begin{lstlisting}[style=python]
class Car(Vehicle):
  """ Cars inherit from Vehicle """
  __CV = 0
  # Constructor
  def __init__(self, CO2, P):
    self.Passengers = 1
    self.__CV = (CO2 / 45) + (P / 40)
  # Method
  def getCV(self):
    return (self.__CV) \end{lstlisting}
      \end{onlyenv}

    \end{column}
  \end{columns}

\end{frame}



\begin{frame}[fragile]{Classes}

  \begin{columns}[onlytextwidth]
    \begin{column}{\textwidth}

      \begin{onlyenv}<1>
        \begin{lstlisting}[style=python]










 \end{lstlisting}
      \end{onlyenv}

      \begin{onlyenv}<2>
        \begin{lstlisting}[style=python]
Peugeot206Plus = Car(110, 44)  # 110g/km  44kW
AirbusA340 = Vehicle()








 \end{lstlisting}
      \end{onlyenv}

      \begin{onlyenv}<3>
        \begin{lstlisting}[style=python]
Peugeot206Plus = Car(110, 44)  # 110g/km  44kW
AirbusA340 = Vehicle()

Peugeot206Plus.Accelerate(80)
AirbusA340.Accelerate(260)





 \end{lstlisting}
      \end{onlyenv}

      \begin{onlyenv}<4>
        \begin{lstlisting}[style=python]
Peugeot206Plus = Car(110, 44)  # 110g/km  44kW
AirbusA340 = Vehicle()

Peugeot206Plus.Accelerate(80)
AirbusA340.Accelerate(260)

Peugeot206Plus.getSpeed()  # 80
AirbusA340.getSpeed()      # 260

Peugeot206Plus.getCV()
AirbusA340.getCV() \end{lstlisting}
      \end{onlyenv}


      \begin{onlyenv}<5->
        \begin{lstlisting}[style=python]
Peugeot206Plus = Car(110, 44)  # 110g/km  44kW
AirbusA340 = Vehicle()

Peugeot206Plus.Accelerate(80)
AirbusA340.Accelerate(260)

Peugeot206Plus.getSpeed()  # 80
AirbusA340.getSpeed()      # 260

Peugeot206Plus.getCV()
AirbusA340.getCV()  # Error \end{lstlisting}
      \end{onlyenv}

    \end{column}
  \end{columns}

\end{frame}


\begin{frame}<beamer>{Classes}

  Summary of vocabulary:

  \begin{itemize}
    \item<2-> \TTBF{Vehicle}: class
    \item<3-> \TTBF{Car}: class (inherits from \TTBF{Vehicle})
    \item<4-> \TTBF{Peugeot206Plus}: instance of \TTBF{Car}
    \item<5-> \TTBF{AirbusA340}: instance of \TTBF{Vehicle}
  \end{itemize}

  \begin{itemize}
  \item<6-> \TTBF{Peugeot206Plus} can calculate and give \TTBF{CV}
  \item<7-> \TTBF{AirbusA340}: doesn't have \TTBF{CV} method nor attribute
  \end{itemize}

  \begin{center}
    \onslide<8-> \url{https://docs.python.org/3/tutorial/classes.html}
  \end{center}

\end{frame}


% File management in python

\section{Files and Directories}

% How to open, read, write, close
\subsection{Open, Read, Write, Close}


\begin{frame}[fragile]{Files}

  Files are managed as usual:

  \begin{onlyenv}<2>
    \begin{itemize}
      \item \TTBF{open}
      \item \TTBF{read}
      \item \TTBF{write}
      \item \TTBF{close}
    \end{itemize}
  \end{onlyenv}

  \begin{onlyenv}<3>
    \begin{itemize}
      \item \TTBF{open(filename, flags}
      \item \TTBF{read}
      \item \TTBF{write}
      \item \TTBF{close}
    \end{itemize}
  \end{onlyenv}

  \begin{onlyenv}<4>
    \begin{itemize}
      \item \TTBF{open(filename, flags}
      \item \TTBF{read(NbCharacters)}
      \item \TTBF{write}
      \item \TTBF{close}
    \end{itemize}
  \end{onlyenv}

  \begin{onlyenv}<5>
    \begin{itemize}
      \item \TTBF{open(filename, flags}
      \item \TTBF{read(NbCharacters)}
      \item \TTBF{write(String)}
      \item \TTBF{close}
    \end{itemize}
  \end{onlyenv}

  \begin{onlyenv}<6->
    \begin{itemize}
      \item \TTBF{open(filename, flags}
      \item \TTBF{read(NbCharacters)}
      \item \TTBF{write(String)}
      \item \TTBF{close()}
    \end{itemize}
  \end{onlyenv}

  \bigskip

  \onslide<7-> Always close files in order to be sure to write changes on the physical support

  \bigskip

  \onslide<8-> Files are managed by an internal python's \textbf{class}

\end{frame}

\begin{frame}<beamer>{Files}

  Open flags:

  \begin{itemize}
    \item<2-> \TTBF{"r"} - open for reading
    \item<3-> \TTBF{"w"} - truncate the file for overwriting
    \item<4-> \TTBF{"a"} - append/begin from the end for writing
    \item<5-> \TTBF{"x"} - create a new file, fail if it already exists
    \item<6-> \TTBF{"r+"} - open for reading and writing
  \end{itemize}

  \bigskip

  \onslide<7-> Read variants:

  \begin{itemize}
    \item<8-> \TTBF{read()} - get the whole text
    \item<9-> \TTBF{read(Nb)} - get the next \textit{Nb} characters
    \item<10-> \TTBF{readline()} - get the next line
  \end{itemize}

\end{frame}


% Reading files

\begin{frame}[fragile]{Files}

  Open existing file for reading

  \begin{columns}[onlytextwidth]
    \begin{column}{\textwidth}

      \begin{onlyenv}<1>
        % Balises exception :  %* *)
        \begin{lstlisting}[style=python]





 \end{lstlisting}
      \end{onlyenv}

      \begin{onlyenv}<2>
        % Balises exception :  %* *)
        \begin{lstlisting}[style=python]
f = open("file.txt", "r")




 \end{lstlisting}
      \end{onlyenv}

      \begin{onlyenv}<3>
        % Balises exception :  %* *)
        \begin{lstlisting}[style=python]
f = open("file.txt", "r")
chars = f.read(10)
print(chars)


 \end{lstlisting}
      \end{onlyenv}

      \begin{onlyenv}<4>
        % Balises exception :  %* *)
        \begin{lstlisting}[style=python]
f = open("file.txt", "r")
chars = f.read(10)
print(chars)
line = f.readline()
print(line)
 \end{lstlisting}
      \end{onlyenv}

      \begin{onlyenv}<5->
        % Balises exception :  %* *)
        \begin{lstlisting}[style=python]
f = open("file.txt", "r")
chars = f.read(10)
print(chars)
line = f.readline()
print(line)
f.close() \end{lstlisting}
      \end{onlyenv}

    \end{column}
  \end{columns}

\end{frame}


\begin{frame}[fragile]{Files}

  Read line by line a file with python specificities

  \begin{columns}[onlytextwidth]
    \begin{column}{\textwidth}

      \begin{onlyenv}<1>
        % Balises exception :  %* *)
        \begin{lstlisting}[style=python]



 \end{lstlisting}
      \end{onlyenv}

      \begin{onlyenv}<2>
        % Balises exception :  %* *)
        \begin{lstlisting}[style=python]
f = open("file.txt", "r")


 \end{lstlisting}
      \end{onlyenv}

      \begin{onlyenv}<3>
        % Balises exception :  %* *)
        \begin{lstlisting}[style=python]
f = open("file.txt", "r")
for x in f:
  print(x)
 \end{lstlisting}
      \end{onlyenv}

      \begin{onlyenv}<4->
        % Balises exception :  %* *)
        \begin{lstlisting}[style=python]
f = open("file.txt", "r")
for x in f:
  print(x)
f.close() \end{lstlisting}
      \end{onlyenv}

    \end{column}
  \end{columns}

\end{frame}


% Writing files

\begin{frame}[fragile]{Files}

  Write at the end of a file (add content)

  \begin{columns}[onlytextwidth]
    \begin{column}{\textwidth}

      \begin{onlyenv}<1>
        % Balises exception :  %* *)
        \begin{lstlisting}[style=python]



 \end{lstlisting}
      \end{onlyenv}

      \begin{onlyenv}<2>
        % Balises exception :  %* *)
        \begin{lstlisting}[style=python]
f = open("file.txt", "a")


 \end{lstlisting}
      \end{onlyenv}

      \begin{onlyenv}<3>
        % Balises exception :  %* *)
        \begin{lstlisting}[style=python]
f = open("file.txt", "a")
nb = f.write("-add content-")

 \end{lstlisting}
      \end{onlyenv}

      \begin{onlyenv}<4>
        % Balises exception :  %* *)
        \begin{lstlisting}[style=python]
f = open("file.txt", "a")
nb = f.write("-add content-")
f.close()
 \end{lstlisting}
      \end{onlyenv}

      \begin{onlyenv}<5->
        % Balises exception :  %* *)
        \begin{lstlisting}[style=python]
f = open("file.txt", "a")
nb = f.write("-add content-")
f.close()
print(nb)  # 13 chars were added \end{lstlisting}
      \end{onlyenv}

    \end{column}
  \end{columns}

\end{frame}


\begin{frame}<beamer>{Files}

  \begin{center}
    \onslide<1-> Check documentation for more informations

    \bigskip

    \onslide<1-> \url{https://docs.python.org/3/tutorial/inputoutput.html}
  \end{center}

\end{frame}


% Test file existence and delete
\subsection{File existence and management}

\begin{frame}<beamer>{Files and Directories}

  Management of files is done through the \TTBF{os} module

  \begin{itemize}
    \item<2-> \TTBF{os.remove(filename)} - remove the file
    \item<3-> \TTBF{os.rmdir(dirname)} - remove the directory
    \item<4-> \TTBF{os.path.exist(path)} - test if \textit{path} is an existing file
    \item<5-> \TTBF{os.path.isfile(path)} - test if \textit{path} is a file
    \item<6-> \TTBF{os.path.isdir(path)} - test if \textit{path} is a directory
    \item<7-> \TTBF{os.path.abspath(path)} - get absolute pathname
    \item<8-> \TTBF{os.path.dirname(path)} - get dirname of pathname
    \item<9-> \TTBF{os.path.basename(path)} - get basename of pathname
  \end{itemize}

  \begin{center}
    \onslide<10-> \url{https://docs.python.org/3/library/os.path.html}
  \end{center}

\end{frame}
