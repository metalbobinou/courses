%% Exercice 4

%\ExoSpecs{\TTBF{CalculTVA.sh}}{\TTBF{\RenduDir/src/exo1/}}{750}{640}{\TTBF{write}}
\ExoSpecsCustom{\TTBF{exo4\_fun.php} [my\_TicTacToe(array)]}{\TTBF{\RenduDir/src/exo4/}}{}{}{Fonctions recommandées}{\TTBF{(Bases PHP)}, \TTBF{return}}

\vspace*{0.7cm}

\noindent \ExoObjectif{Le but de l'exercice est de déterminer quel est le gagnant d'une partie de morpion (ou \textit{Tic Tac Toe} en angais).}

\bigskip

\noindent Les règles sont relativement simples : chaque joueur place un \TTBF{X} ou un \TTBF{O} à chaque tour dans un tableau de 3 cases sur 3 cases, le premier joueur alignant trois fois son symbole a gagné (une ligne verticale, horizontale, ou en diagonale).
Il est relativement simple de faire un match nul (aucun joueur n'arrive à aligner trois fois son symbole).

\lstset{language=sh}
\begin{lstlisting}[frame=single,title={Exemple de jeu où le joueur X a gagné par la diagonale}]
X O X
O X O
X O O
\end{lstlisting}

\noindent Vous devez écrire une fonction nommée \TTBF{my\_TicTacToe} qui prendra en paramètre un tableau de tableaux.
La fonction renverra tout d'abord le tableau de jeu, puis elle renverra le gagnant.

\bigskip

\noindent Exemple d'entrée :

\lstset{language=php}
\begin{lstlisting}[frame=single,title={Exemple de tableau de jeu (exo4\_data.php)}]
$line1 = array("X", "O", "X");
$line2 = array("O", "X", "O");
$line3 = array("X", "O", "O");

$game[] = $line1;
$game[] = $line2;
$game[] = $line3;
\end{lstlisting}

\lstset{language=php}
\begin{lstlisting}[frame=single,title={Appel de la fonction}]
<textarea cols="80" rows="25" readonly="readonly">
<?php
  require_once("exo4_data.php");
  require_once("exo4_fun.php");
  $my_text = my_TicTacToe($game);
  echo($my_text);
?>
</textarea>
\end{lstlisting}

\bigskip

\noindent Si un joueur est gagnant, vous devez l'indiquer avec les chaînes de caractères :

\bigskip

\noindent \TTBF{Player\textvisiblespace X\textvisiblespace won}

\noindent \TTBF{Player\textvisiblespace O\textvisiblespace won}

\bigskip

\noindent Si aucun joueur n'est gagnant, vous devez l'indiquer ainsi :

\bigskip

\noindent \TTBF{It's\textvisiblespace a\textvisiblespace draw}

\bigskip

\lstset{language=php}
\begin{lstlisting}[frame=single,title={Cas général gagnant}]
<textarea cols="80" rows="25" readonly="readonly">
XOX
OXO
XOO
Player X won</textarea>
\end{lstlisting}

\lstset{language=php}
\begin{lstlisting}[frame=single,title={Cas général en match nul}]
<textarea cols="80" rows="25" readonly="readonly">
OXX
XXO
OOX
It's a draw</textarea>
\end{lstlisting}

\bigskip

\noindent Un cas d'erreur doit être géré avant tout affichage.
Si le tableau contient des caractères autres que \TTBF{X} ou \TTBF{O}, il faut indiquer le message suivant :

\bigskip

\noindent \TTBF{Incorrect\textvisiblespace game}

\bigskip

\lstset{language=php}
\begin{lstlisting}[frame=single,title={Exemple de tableau de jeu incorrect (exo4\_data.php)}]
$line1 = array("X", "O", "A");
$line2 = array("O", "X", "O");
$line3 = array("O", "O", "O");

$game[] = $line1;
$game[] = $line2;
$game[] = $line3;
\end{lstlisting}

\lstset{language=php}
\begin{lstlisting}[frame=single,title={Cas d'erreur}]
<textarea cols="80" rows="25" readonly="readonly">
Incorrect game</textarea>
\end{lstlisting}

\bigskip

\begin{RedBoxTitle}{ATTENTION}
    Les retours à la ligne ne doivent pas être faits avec la balise \TTBF{"<br />"}, mais avec \TTBF{"\textbackslash n"}.
    (se référer à la section \hyperref[sec:AideMemoire]{Aide Mémoire})
\end{RedBoxTitle}
