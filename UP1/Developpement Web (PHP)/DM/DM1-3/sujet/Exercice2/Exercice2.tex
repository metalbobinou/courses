%% Exercice 2

%\ExoSpecs{\TTBF{CalculTVA.sh}}{\TTBF{\RenduDir/src/exo3/}}{750}{640}{\TTBF{write}}
\ExoSpecsCustom{\TTBF{exo2\_fun.php} [my\_Sapin(int)]}{\TTBF{\RenduDir/src/exo2/}}{}{}{Fonctions recommandées}{\TTBF{(Bases PHP)}, \TTBF{return}}

\vspace*{0.7cm}

\noindent \ExoObjectif{Le but de l'exercice est d'afficher un sapin en ASCII art, dont la taille varie selon le paramètre donné.}

\bigskip

\noindent Vous devez écrire la fonction \TTBF{my\_Sapin} qui prendra un entier en paramètre, et affichera selon ce paramètre un sapin.

\bigskip

\lstset{language=php}
\begin{lstlisting}[frame=single,title={Cas général PHP}]
<textarea cols="80" rows="25" readonly="readonly">
<?php
  require_once("exo2_fun.php");
  $my_text = my_Sapin(2);
  echo($my_text);
?>
</textarea>
\end{lstlisting}

\lstset{language=php}
\begin{lstlisting}[frame=single,title={Cas général PHP exécuté}]
<textarea cols="80" rows="25" readonly="readonly">
  /\
 /  \
/    \
______
  ||
</textarea>
\end{lstlisting}

\bigskip

\noindent De façon précise, voici les spécifications pour les cas 1, 4, et 5 :

\bigskip

\lstset{language=sh}
\begin{lstlisting}[frame=single,title={Cas général 1}]
 /\     1 espace / 0 espace  \
/  \    0 espace / 2 espaces \
____    4 _
 ||     1 espace 2 |
\end{lstlisting}


\lstset{language=sh}
\begin{lstlisting}[frame=single,title={Cas général 4}]
    /\          4 espaces / 0 espace  \
   /  \         3 espaces / 2 espaces \
  /    \        2 espaces / 4 espaces \
 /      \       1 espace  / 6 espaces \
/        \      0 espace  / 8 espaces \
__________      10 _
    ||          4 espaces 2 |
\end{lstlisting}


\lstset{language=sh}
\begin{lstlisting}[frame=single,title={Cas général 5}]
     /\         5 espaces / 0 espace   \
    /  \        4 espaces / 2 espaces  \
   /    \       3 espaces / 4 espaces  \
  /      \      2 espaces / 6 espaces  \
 /        \     1 espace  / 8 espaces  \
/          \    0 espace  / 10 espaces \
____________    12 _
     ||         5 espaces 2 |
\end{lstlisting}


\bigskip

%\noindent Plusieurs cas spéciaux sont à prendre en compte.\\

\noindent Une exception doit être gérée, dans le cas où le paramètre donné est inférieur ou égal à 0, vous afficherez :

\bigskip

\lstset{language=php}
\begin{lstlisting}[frame=single,title={Cas 0}]
<textarea cols="80" rows="25" readonly="readonly">
<?php
  require_once("exo2_fun.php");
  $my_text = my_Sapin(0);
  echo($my_text);
?>
</textarea>
\end{lstlisting}

\lstset{language=php}
\begin{lstlisting}[frame=single,title={Cas 0 exécuté}]
<textarea cols="80" rows="25" readonly="readonly">
/\
||
</textarea>
\end{lstlisting}

\bigskip

%%%%%
%
%\noindent Si un paramètre texte est donné, il sera interprété comme 0 :
%
%\bigskip
%
%\lstset{language=php}
%\begin{lstlisting}[frame=single,title={Cas texte}]
%<textarea cols="80" rows="25" readonly="readonly">
%<?php
%  require_once("exo2_fun.php");
%  $my_text = my_Sapin("A");
%  echo($my_text);
%?>
%</textarea>
%\end{lstlisting}
%
%\lstset{language=php}
%\begin{lstlisting}[frame=single,title={Cas texte exécuté}]
%<textarea cols="80" rows="25" readonly="readonly">
%/\
%||
%</textarea>
%\end{lstlisting}

\bigskip

\begin{RedBoxTitle}{ATTENTION}
    Les retours à la ligne ne doivent pas être faits avec la balise \TTBF{"<br />"}, mais avec \TTBF{"\textbackslash n"}.
    (se référer à la section \hyperref[sec:AideMemoire]{Aide Mémoire})
\end{RedBoxTitle}