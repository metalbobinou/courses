%% Exercice 5

%\ExoSpecs{\TTBF{CalculTVA.sh}}{\TTBF{\RenduDir/src/exo1/}}{750}{640}{\TTBF{write}}
\ExoSpecsCustom{\TTBF{exo5\_fun.php} [my\_Loginisation(array, array)]}{\TTBF{\RenduDir/src/exo5/}}{}{}{Fonctions recommandées}{\TTBF{gettype}, \TTBF{in\_array}, \TTBF{strlen}, \TTBF{substr}}

\vspace*{0.7cm}

\noindent \ExoObjectif{Le but de l’exercice est de créer des logins à partir d'une liste de noms et prénoms.}

\bigskip

\noindent Il est courant en informatique de devoir créer des logins.
Dans cet exercice, il faudra créer des logins pour un groupe précis de personnes.

\bigskip

\noindent En entrée, deux paramètres sont donnés : un tableau avec une liste de référence, et un autre tableau avec une liste de personnes dont il faut créer les logins.
Afin de ne pas créer de login à n'importe qui, seules les personnes apparaissant dans la liste de référence disposeront d'un login.

\noindent La construction du login se fait comme suit : l'initiale du prénom en minuscule, le nom de famille en minuscules, les deux derniers chiffres de l'année de naissance.
Par exemple, pour \textit{\textbf{E}va \textbf{Bien}} née en \textit{19\textbf{98}}, le login attendu est \textbf{ebien98}.

\noindent Dans certains cas, certaines personnes ne disposent que de l'année de naissance.
Dans ce cas, on utilise toujours les deux dernières décimales pour le login, mais il faut s'assurer que la base de référence contienne la même information.
Si deux utilisateurs ont les mêmes nom et prénom, mais pas la même date de naissance, il s'agira de deux personnes distinctes.

\bigskip

\noindent Exemple d'entrées :

\lstset{language=sh}
\begin{lstlisting}[frame=single,title={Exemple de liste de référence}]
$ReferenceList[] = array("Marc", "Acin", "05-05-1998");
$ReferenceList[] = array("Charles", "Attan", "20-06-1997");
$ReferenceList[] = array("Eva", "Bien", "23-02-1998");
$ReferenceList[] = array("Jean", "Bombeur", "17-01-1999");
$ReferenceList[] = array("Jean", "Bonnot", "1997");
$ReferenceList[] = array("Claire", "Hon", "14-07-1997");
$ReferenceList[] = array("Lydie", "Le", "02-03-1997");
$ReferenceList[] = array("Julie", "Tali", "18-12-1998");
$ReferenceList[] = array("Oreste", "Torrent", "22-04-1998");
$ReferenceList[] = array("Elena", "Turelle", "08-10-1998");
\end{lstlisting}

\lstset{language=sh}
\begin{lstlisting}[frame=single,title={Exemple de liste de demandes}]
$RequestList[] = array("Eva", "Bien", "23-02-1998");
$RequestList[] = array("Jean", "Bombeur", "24-12-2000");
$RequestList[] = array("Jean", "Bonnot", "1997");
$RequestList[] = array("Julie", "Tali", "18-12-1998");
$RequestList[] = array("Clement", "Thine", "13-04-1998");
\end{lstlisting}

\bigskip

\noindent Les tableaux seront fournis à la fonction ainsi : \\

\lstset{language=html}
%\begin{lstlisting}[frame=single,caption={Useless code},label=useless]  % "useless" used to references
\begin{lstlisting}[frame=single,title={Appel de la fonction}]
<textarea cols="80" rows="25" readonly="readonly">
<?php
  require_once("exo5_listeref.php");
  require_once("exo5_data.php");
  require_once("exo5_fun.php");
  $my_text = my_Loginisation($ReferenceList, $RequestList);
  echo($my_text);
?>
</textarea>
\end{lstlisting}

%
%% OR
%% \lstinputlisting[language=[ANSI]C]{source.c}
%% \lstinputlisting[language=Python]{source_filename.py}
%% \lstinputlisting[language=Python, firstline=37, lastline=45]{source_filename.py}
%

\bigskip

\noindent La sortie sera de cette forme :\\

\lstset{language=html}
\begin{lstlisting}[frame=single,title={Sortie HTML attendue}]
<textarea cols="80" rows="25" readonly="readonly">
ebien98
jbonnot97
jtali98
</textarea>
\end{lstlisting}

\bigskip

\noindent Vous devez rendre un fichier nommé \TTBF{exo5\_fun.php} qui contiendra au moins la fonction \TTBF{my\_Loginisation} qui effectuera le traitement et renverra une chaîne de caractères.

\noindent La fonction prendra en paramètre deux tableaux : le premier contiendra la liste de référence des personnes autorisées, et le deuxième contiendra la liste des personnes demandeuses.

\bigskip

\noindent On considèrera que les listes en entrée sont correctement formattées, et jamais vides.

\bigskip

\begin{RedBoxTitle}{ATTENTION}
    Les retours à la ligne ne doivent pas être faits avec la balise \TTBF{"<br />"}, mais avec \TTBF{"\textbackslash n"}.
    (se référer à la section \hyperref[sec:AideMemoire]{Aide Mémoire})
\end{RedBoxTitle}

%\bigskip
%
%\begin{YellowBox}
%La fonction \TTBF{preg\_replace} travaille avec des expressions rationnelles (ou expressions régulières, ou regular expression, ou RegExp).
%L'usage classique des RegExp est de reconnaitre et extraire une sous-chaîne selon des critères précis.
%(\TTBF{preg\_replace} ne fonctionne pas exactement comme \TTBF{sed})
%\end{YellowBox}
