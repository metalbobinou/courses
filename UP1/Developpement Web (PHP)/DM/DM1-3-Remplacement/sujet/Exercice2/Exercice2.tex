%% Exercice 2

%\ExoSpecs{\TTBF{CalculTVA.sh}}{\TTBF{\RenduDir/src/exo3/}}{750}{640}{\TTBF{write}}
\ExoSpecsCustom{\TTBF{exo2\_fun.php} [my\_Square(int)]}{\TTBF{\RenduDir/src/exo2/}}{}{}{Fonctions recommandées}{\TTBF{(Bases PHP)}}

\vspace*{0.7cm}

\noindent \ExoObjectif{Le but de l'exercice est d'afficher un carré en ASCII art, dont la taille varie selon le paramètre donné.}

\bigskip

\noindent Vous devez écrire la fonction \TTBF{my\_Square} qui prendra un entier en paramètre, et affichera selon ce paramètre un carré fait de croisillons.

\bigskip

\lstset{language=html}
\begin{lstlisting}[frame=single,title={Cas général PHP}]
<textarea cols="80" rows="25" readonly="readonly">
<?php
  require_once("exo2_fun.php");
  $my_text = my_Square(3);
  echo($my_text);
?>
</textarea>
\end{lstlisting}

\lstset{language=html}
\begin{lstlisting}[frame=single,title={Cas général PHP exécuté}]
<textarea cols="80" rows="25" readonly="readonly">
###
# #
###
</textarea>
\end{lstlisting}

\bigskip

\noindent De façon détaillée, voici les spécifications pour les cas 1, 2, 3, et 4 :

\bigskip

\lstset{language=html}
\begin{lstlisting}[frame=single,title={Cas général 1}]
#    1 croisillon
\end{lstlisting}


\lstset{language=html}
\begin{lstlisting}[frame=single,title={Cas général 2}]
##    2 croisillons
##    2 croisillons
\end{lstlisting}


\lstset{language=html}
\begin{lstlisting}[frame=single,title={Cas général 3}]
###    3 croisillons
# #    1 croisillon  1 espace  1 croisillon
###    3 croisillons
\end{lstlisting}

\lstset{language=html}
\begin{lstlisting}[frame=single,title={Cas général 4}]
####    4 croisillons
#  #    1 croisillon  2 espaces  1 croisillon
#  #    1 croisillon  2 espaces  1 croisillon
####    4 croisillons
\end{lstlisting}

\bigskip

%\noindent Plusieurs cas spéciaux sont à prendre en compte.\\

\noindent Une exception doit être gérée, dans le cas où le paramètre donné est inférieur ou égal à 0, vous afficherez un point :

\bigskip

\lstset{language=html}
\begin{lstlisting}[frame=single,title={Cas 0}]
<textarea cols="80" rows="25" readonly="readonly">
<?php
  require_once("exo2_fun.php");
  $my_text = my_Square(0);
  echo($my_text);
?>
</textarea>
\end{lstlisting}

\lstset{language=html}
\begin{lstlisting}[frame=single,title={Cas 0 exécuté}]
<textarea cols="80" rows="25" readonly="readonly">
.
</textarea>
\end{lstlisting}

\bigskip

%%%%%
%
%\noindent Si un paramètre texte est donné, il sera interprété comme 0 :
%
%\bigskip
%
%\lstset{language=php}
%\begin{lstlisting}[frame=single,title={Cas texte}]
%<textarea cols="80" rows="25" readonly="readonly">
%<?php
%  require_once("exo2_fun.php");
%  $my_text = my_Sapin("A");
%  echo($my_text);
%?>
%</textarea>
%\end{lstlisting}
%
%\lstset{language=php}
%\begin{lstlisting}[frame=single,title={Cas texte exécuté}]
%<textarea cols="80" rows="25" readonly="readonly">
%/\
%||
%</textarea>
%\end{lstlisting}

\bigskip

\begin{RedBoxTitle}{ATTENTION}
    Les retours à la ligne ne doivent pas être faits avec la balise \TTBF{"<br />"}, mais avec \TTBF{"\textbackslash n"}.
    (se référer à la section \hyperref[sec:AideMemoire]{Aide Mémoire})
\end{RedBoxTitle}