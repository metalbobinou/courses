%% Exercice 5

%\ExoSpecs{\TTBF{CalculTVA.sh}}{\TTBF{\RenduDir/src/exo1/}}{750}{640}{\TTBF{write}}
\ExoSpecsCustom{\TTBF{exo5\_fun.php} [my\_NormalisationNom(array)]}{\TTBF{\RenduDir/src/exo5/}}{}{}{Fonctions recommandées}{\TTBF{strlen}, \TTBF{strpos}, \TTBF{substr}, \TTBF{preg\_replace}}

\vspace*{0.7cm}

\noindent \ExoObjectif{Le but de l’exercice est de remettre à la norme des noms de fichiers.}

\bigskip

\noindent Les appareils photos numériques et smartphones stockent souvent les images avec des noms de fichiers dans un format proche de : \texttt{IMG\_xxxx.jpg} (où xxxx correspond à un nombre).
Il arrive que ce nom soit parfois au format \texttt{IMG\_1.jpg} ou au format \texttt{IMG\_0001.jpg}.
Afin de standardiser ces noms sur un format de taille fixe, vous devrez créer un script qui transforme les noms dans le bon format.

\bigskip

\noindent En entrée, vous recevrez un tableau contenant des noms de fichiers.
Vous devrez afficher les noms de fichiers après les avoir mis au format : \TTBF{IMG\_xxxx.jpg} (4 chiffres)
Attention, l'extension \textbf{DOIT} être mise en minuscules (\textit{lower case}).
Les fichiers que nous testerons pourront avoir des préfixes différents que \TTBF{IMG\_}, tout comme l'extension peut être des \TTBF{png} ou autres.

\bigskip

\noindent Exemple d'entrée :

\lstset{language=sh}
\begin{lstlisting}[frame=single,title={Exemple de données en entrée}]
IMG_3.jpg
IMG_02.JPG
IMG_0005.jpg
IMG_8976.JPG
img_42.png
picture35.BMP
\end{lstlisting}

\bigskip

\noindent Sortie attendue pour cette entrée :

\lstset{language=sh}
\begin{lstlisting}[frame=single,title={Sortie attendue pour l'entrée précédente}]
IMG_0003.jpg
IMG_0002.jpg
IMG_0005.jpg
IMG_8976.jpg
img_0042.png
picture0035.bmp
\end{lstlisting}

\bigskip

\noindent Le tableau en entrée et l'appel se feront ainsi : \\

\lstset{language=php}
%\begin{lstlisting}[frame=single,caption={Useless code},label=useless]  % "useless" used to references
\begin{lstlisting}[frame=single,title={Exemple de tableau en entrée (exo5\_data.php)}]
$filenames[] = "IMG_3.jpg";
$filenames[] = "IMG_02.JPG";
$filenames[] = "IMG_0005.jpg";
$filenames[] = "IMG_8976.JPG";
$filenames[] = "img_42.png";
$filenames[] = "picture35.BMP";
\end{lstlisting}

\lstset{language=php}
%\begin{lstlisting}[frame=single,caption={Useless code},label=useless]  % "useless" used to references
\begin{lstlisting}[frame=single,title={Appel de la fonction}]
<textarea cols="80" rows="25" readonly="readonly">
<?php
  require_once("exo5_data.php");
  require_once("exo5_fun.php");
  $my_text = my_NormalisationNom($filenames);
  echo($my_text);
?>
</textarea>
\end{lstlisting}

%
%% OR
%% \lstinputlisting[language=[ANSI]C]{source.c}
%% \lstinputlisting[language=Python]{source_filename.py}
%% \lstinputlisting[language=Python, firstline=37, lastline=45]{source_filename.py}
%

\bigskip

\noindent La sortie sera de cette forme :\\

\lstset{language=php}
\begin{lstlisting}[frame=single,title={Sortie HTML attendue}]
<textarea cols="80" rows="25" readonly="readonly">
IMG_0003.jpg
IMG_0002.jpg
IMG_0005.jpg
IMG_8976.jpg
img_0042.png
picture0035.bmp
</textarea>
\end{lstlisting}

\bigskip

\noindent (Les noms de fichiers ne contiendront jamais de point \texttt{.} en dehors de l'extension, et une seule série de chiffre sera présente (celle à remplacer), elle contiendra de 1 à 4 chiffres)

\bigskip

\noindent Vous devez rendre un fichier nommé \TTBF{exo5\_fun.php} qui contiendra au moins la fonction \TTBF{my\_NormalisationNom} qui effectuera le traitement et renverra une chaîne de caractères.

\noindent La fonction prendra en paramètre un tableau qui contient des noms de fichiers.

\bigskip

\begin{RedBoxTitle}{ATTENTION}
    Les retours à la ligne ne doivent pas être faits avec la balise \TTBF{"<br />"}, mais avec \TTBF{"\textbackslash n"}.
    (se référer à la section \hyperref[sec:AideMemoire]{Aide Mémoire})
\end{RedBoxTitle}

\bigskip

\begin{YellowBox}
La fonction \TTBF{preg\_replace} travaille avec des expressions rationnelles (ou expressions régulières, ou regular expression, ou RegExp).
L'usage classique des RegExp est de reconnaitre et extraire une sous-chaîne selon des critères précis.
(\TTBF{preg\_replace} ne fonctionne pas exactement comme \TTBF{sed})
\end{YellowBox}
