%% Exercice 3

%\ExoSpecs{\TTBF{CalculTVA.sh}}{\TTBF{\RenduDir/src/exo1/}}{750}{640}{\TTBF{write}}
\ExoSpecsCustom{\TTBF{exo3\_fun.php} [my\_CalculTVA(array)]}{\TTBF{\RenduDir/src/exo3/}}{}{}{Fonctions recommandées}{\TTBF{(Bases PHP)}, \TTBF{return}, \TTBF{date\_format}, \TTBF{(int)}}

\vspace*{0.7cm}

\noindent \ExoObjectif{Le but de l'exercice est de recalculer le taux de TVA à partir de factures.
Pour simplifier l'exercice, tous les produits vendus seront taxés à 20\%.}

\bigskip

\noindent A partir d'un fichier de factures émises stockées dans des tableaux, vous allez afficher ces factures en ajoutant 3 colonnes supplémentaires (le montant hors taxes, le taux de TVA, et le montant des taxes).
Tous les montants sont stockés sous forme de centimes (300 signifie 300 centimes, soit 3 euros).

\bigskip

\noindent Pour rappel, voici comment se calculent les taxes :\\

\begin{tabular}{l c l}
Prix TTC depuis un prix H.T. : & $ Prix TTC = $ & $ Prix H.T. + Taxes $ \\
Taxes (20\%) depuis un prix H.T. : & $ Taxes = $ & $ Prix H.T. \times 20 \div 100 $ \\
 & & \\
Prix H.T. depuis un prix TTC (taxes à 20\%) : & $ Prix H.T. = $ & $ Prix TTC \div 1,2 $ \\
Taxes depuis des prix H.T. et TTC : & $ Taxes = $ & $ Prix TTC - Prix H.T. $ \\
\end{tabular}

\bigskip

\bigskip

\noindent Le tableau en entrée est de cette forme : \\

\hspace*{-\parindent} %% BIDOUILLE
\begin{minipage}{17cm} %% BIDOUILLE
\lstset{language=sh}
%\begin{lstlisting}[frame=single,caption={Useless code},label=useless]  % "useless" used to references
\begin{lstlisting}[frame=single,title={Tableau en entrée}]
[[ID_client (int), Date (DateTime), ID_Facture (int), Montant_TTC (int)] , [ID_client (int), Date (DateTime), ID_Facture (int), Montant_TTC (int)] , ...]
\end{lstlisting}
\end{minipage} %% FIN BIDOUILLE

%
%% OR
%% \lstinputlisting[language=[ANSI]C]{source.c}
%% \lstinputlisting[language=Python]{source_filename.py}
%% \lstinputlisting[language=Python, firstline=37, lastline=45]{source_filename.py}
%

\bigskip

\noindent Le fichier de sortie sera de cette forme : \\

\hspace*{-\parindent} %% BIDOUILLE
\begin{minipage}{17cm} %% BIDOUILLE
\lstset{language=sh,basicstyle=\ttfamily \footnotesize}
\begin{lstlisting}[frame=single,title={Sortie attendue}]
ID client ; Date ; Num Facture ; Montant TTC ; Montant HT ; Taux TVA ; Taxes
\end{lstlisting}
\end{minipage} %% FIN BIDOUILLE

\bigskip

\noindent Exemple de structure d'entrée :

\lstset{language=sh}
\begin{lstlisting}[frame=single,title={Exemple de données en entrée}]
42421337;2006/06/06;456789;200
36153615;2018/05/30;123456;123
\end{lstlisting}

\bigskip
%\newpage

\noindent Sortie attendue pour cette entrée :

\lstset{language=sh}
\begin{lstlisting}[frame=single,title={Sortie attendue pour l'entrée précédente}]
42421337;06/06/06;456789;200;166;20;34
36153615;18/05/30;123456;123;102;20;21
\end{lstlisting}

\bigskip

\noindent (Les fichiers que nous testerons ne seront pas de mauvaise qualité : les données seront toujours correctement formatées)

\bigskip

\noindent \textbf{Attention :} Certains nombres risquent de subir des virgules.
Pour éviter cela, utilisez \TTBF{((int) (x / y))} pour ne conserver que la partie entière (\texttt{111} TTC donnera donc par division \texttt{92} H.T. et, par soustraction, \texttt{19} de Taxes).
Attention, cette fonction tronque selon la virgule : si vous tronquez avant ou après certaines opérations, les résultats ne seront pas les mêmes !
Lors du calcul des valeurs, choisissez avec attention la ligne exacte où vous tronquerez les nombres, afin de ne pas ajouter ou retirer 1 centime.

\bigskip

\noindent Vous devez rendre un fichier nommé \TTBF{exo3\_fun.php} qui contiendra au moins la fonction \TTBF{my\_CalculTVA} qui effectuera le calcul et renverra une chaîne contenant le résultat.

\noindent La fonction prendra en paramètre un tableau qui contient lui-même des tableaux associatifs (chaque case du tableau général correspond à une facture).

\bigskip

\lstset{language=php}
\begin{lstlisting}[frame=single,title={Tableau en entrée (exo3\_data.php)}]
$tab1["ID_Client"] = 42421337;
$tab1["Date"] = new DateTime('2006-06-06');
$tab1["ID_Facture"] = 456789;
$tab1["Montant_TTC"] = 200;
$data[] = $tab1;

$tab1["ID_Client"] = 36153615;
$tab1["Date"] = new DateTime('2018-05-30');
$tab1["ID_Facture"] = 123456;
$tab1["Montant_TTC"] = 123;
$data[] = $tab1;
\end{lstlisting}

\lstset{language=php}
\begin{lstlisting}[frame=single,title={Appel de la fonction}]
<textarea cols="80" rows="25" readonly="readonly">
<?php
  require_once("exo3_data.php");
  require_once("exo3_fun.php");
  $my_text = my_CalculTVA($data);
  echo($my_text);
?>
</textarea>
\end{lstlisting}

\lstset{language=php}
\begin{lstlisting}[frame=single,title={Sortie HTML attendue}]
<textarea cols="80" rows="25" readonly="readonly">
42421337;06/06/06;456789;200;166;20;34
36153615;18/05/30;123456;123;102;20;21
</textarea>
\end{lstlisting}

\bigskip

\begin{RedBoxTitle}{ATTENTION}
    Les retours à la ligne ne doivent pas être faits avec la balise \TTBF{"<br />"}, mais avec \TTBF{"\textbackslash n"}.
    (se référer à la section \hyperref[sec:AideMemoire]{Aide Mémoire})
\end{RedBoxTitle}