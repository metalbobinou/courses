%% Exercice 4

%\ExoSpecs{\TTBF{CalculTVA.sh}}{\TTBF{\RenduDir/src/exo1/}}{750}{640}{\TTBF{write}}
\ExoSpecsCustom{\TTBF{exo4\_fun.php} [my\_Transposee(array)]}{\TTBF{\RenduDir/src/exo4/}}{}{}{Fonctions recommandées}{\TTBF{(Bases PHP)}, \TTBF{gettype}, \TTBF{sizeof}}

\vspace*{0.7cm}

\noindent \ExoObjectif{Le but de l'exercice est de transposer une matrice.}

\bigskip

\noindent Une matrice est donnée en paramètre, il faut simplement la transposer.

\noindent Vous devez écrire une fonction nommée \TTBF{my\_Transposee} qui prendra en paramètre un tableau de tableaux.
La fonction renverra la matrice transposée.

\bigskip

\noindent Exemple d'entrée :

\lstset{language=php}
\begin{lstlisting}[frame=single,title={Exemple de matrice (exo4\_data.php)}]
$line1 = array(0, 42, 1);
$line2 = array(8, 9, 13);

$matrix[] = $line1;
$matrix[] = $line2;
\end{lstlisting}

\lstset{language=html}
\begin{lstlisting}[frame=single,title={Appel de la fonction}]
<textarea cols="80" rows="25" readonly="readonly">
<?php
  require_once("exo4_data.php");
  require_once("exo4_fun.php");
  $my_text = my_Transposee($matrix);
  echo($my_text);
?>
</textarea>
\end{lstlisting}

\bigskip

\lstset{language=html}
\begin{lstlisting}[frame=single,title={Cas général}]
<textarea cols="80" rows="25" readonly="readonly">
0;8
42;9
1;13
</textarea>
\end{lstlisting}

\bigskip

\noindent Deux cas d'erreur doivent être gérés avant tout affichage : si la matrice est vide (ou n'existe pas), et si la matrice contient autre chose que des nombres.

\noindent Si la matrice est vide ou que la variable n'existe pas, il faut indiquer le message suivant :

\bigskip

\noindent \TTBF{Empty\textvisiblespace matrix}

\bigskip

\lstset{language=php}
\begin{lstlisting}[frame=single,title={Exemple de matrice vide (exo4\_data.php)}]
$line1 = array();
$line2 = array();

$game[] = $line1;
$game[] = $line2;
\end{lstlisting}

\lstset{language=html}
\begin{lstlisting}[frame=single,title={Cas d'erreur 1 : matrice vide}]
<textarea cols="80" rows="25" readonly="readonly">
Empty matrix
</textarea>
\end{lstlisting}

\noindent Si la matrice contient des caractères autres que des nombres, il faut indiquer le message suivant :

\bigskip

\noindent \TTBF{Incorrect\textvisiblespace matrix}

\bigskip

\lstset{language=php}
\begin{lstlisting}[frame=single,title={Exemple de matrice incorrecte (exo4\_data.php)}]
$line1 = array(0, "42", "A");
$line2 = array("-", "U", 13);

$game[] = $line1;
$game[] = $line2;
\end{lstlisting}

\lstset{language=html}
\begin{lstlisting}[frame=single,title={Cas d'erreur 2 : matrice incorrecte}]
<textarea cols="80" rows="25" readonly="readonly">
Incorrect matrix
</textarea>
\end{lstlisting}

\bigskip

\begin{RedBoxTitle}{ATTENTION}
    Les retours à la ligne ne doivent pas être faits avec la balise \TTBF{"<br />"}, mais avec \TTBF{"\textbackslash n"}.
    (se référer à la section \hyperref[sec:AideMemoire]{Aide Mémoire})
\end{RedBoxTitle}
