%% Exercice 3

%\ExoSpecs{\TTBF{CalculTVA.sh}}{\TTBF{\RenduDir/src/exo1/}}{750}{640}{\TTBF{write}}
\ExoSpecsCustom{\TTBF{exo3\_fun.php} [my\_Reduction(array, int)]}{\TTBF{\RenduDir/src/exo3/}}{}{}{Fonctions recommandées}{\TTBF{(Bases PHP)}, \TTBF{date\_format}, \TTBF{(int)}}

\vspace*{0.7cm}

\noindent \ExoObjectif{Le but de l'exercice est de calculer des réductions sur des listes de factures.}

\bigskip

\noindent A partir d'un fichier de factures émises stockées dans des tableaux, vous allez afficher ces factures en ajoutant 3 colonnes supplémentaires (le montant avec réduction appliquée, le taux de réduction, et le montant de la réduction).
Tous les montants sont stockés sous forme de centimes (300 signifie 300 centimes, soit 3 euros).

\bigskip

\noindent Pour rappel, voici comment se calculent les réductions :\\

\begin{tabular}{l c l}
Prix final après réduction : & $ Prix Final = $ & $ Prix Initial - Reduction $ \\
Réduction (20\%) depuis un prix : & $ Reduction = $ & $ Prix Initial \times 20 \div 100 $ \\
 & & \\
Prix initial si réduction (20\%) : & $ Prix Initial = $ & $ Prix Final \div 0,80 (Coef) $ \\
Coefficient de réduction de 20\% : & $ Coefficient (0,80) = $ & $ 1 - (20 \div 100) $ \\
\end{tabular}

\bigskip

\bigskip

\noindent Le tableau en entrée est de cette forme : \\

%\hspace*{-\parindent} %% BIDOUILLE
%\begin{minipage}{17cm} %% BIDOUILLE
\lstset{language=sh}
%\begin{lstlisting}[frame=single,caption={Useless code},label=useless]  % "useless" used to references
\begin{lstlisting}[frame=single,title={Tableau en entrée}]
[[ID_client (int), Date (DateTime), ID_Facture (int), Montant (int)] , [ID_client (int), Date (DateTime), ID_Facture (int), Montant (int)] , ...]
\end{lstlisting}
%\end{minipage} %% FIN BIDOUILLE

%
%% OR
%% \lstinputlisting[language=[ANSI]C]{source.c}
%% \lstinputlisting[language=Python]{source_filename.py}
%% \lstinputlisting[language=Python, firstline=37, lastline=45]{source_filename.py}
%

\bigskip

\noindent Le fichier de sortie sera de cette forme : \\

%\hspace*{-\parindent} %% BIDOUILLE
%\begin{minipage}{17cm} %% BIDOUILLE
\lstset{language=sh,basicstyle=\ttfamily \footnotesize}
\begin{lstlisting}[frame=single,title={Sortie attendue}]
ID client ; Date ; Num Facture ; Montant Final ; Montant Initial ; Taux Reduction ; Reduction
\end{lstlisting}
%\end{minipage} %% FIN BIDOUILLE

\bigskip

\noindent Exemple de structure d'entrée :

\lstset{language=sh}
\begin{lstlisting}[frame=single,title={Exemple de données en entrée}]
42421337;2006/06/06;456789;200
36153615;2018/05/30;123456;123
\end{lstlisting}

\bigskip
%\newpage

\noindent Sortie attendue pour cette entrée si on effectue une réduction de 33\% :

\lstset{language=sh}
\begin{lstlisting}[frame=single,title={Sortie attendue pour l'entrée précédente avec 33\% de réduction}]
42421337;06/06/06;456789;134;200;33;66
36153615;18/05/30;123456;83;123;33;40
\end{lstlisting}

\bigskip

\noindent (Les fichiers que nous testerons ne seront pas de mauvaise qualité : les données seront toujours correctement formatées)

\bigskip

\begin{YellowBox}
\textbf{Attention :} Certains nombres risquent de subir des virgules.
Pour éviter cela, utilisez \TTBF{((int) (x / y))} pour ne conserver que la partie entière (\texttt{123} en valeur initiale donnera donc par division \texttt{40} en réduction et \texttt{83} en valeur finale).
Attention, cette fonction tronque selon la virgule : si vous tronquez avant ou après certaines opérations, les résultats ne seront pas les mêmes !
Lors du calcul des valeurs, choisissez avec attention la ligne exacte où vous tronquerez les nombres, afin de ne pas ajouter ou retirer 1 centime.
\end{YellowBox}

\bigskip

\noindent Vous devez rendre un fichier nommé \TTBF{exo3\_fun.php} qui contiendra au moins la fonction \TTBF{my\_Reduction} qui effectuera le calcul et renverra une chaîne contenant le résultat.

\noindent La fonction prendra en paramètre un tableau qui contient lui-même des tableaux associatifs (chaque case du tableau général correspond à une facture).

\bigskip

\lstset{language=php}
\begin{lstlisting}[frame=single,title={Tableau en entrée (exo3\_data.php)}]
$tab1["ID_Client"] = 42421337;
$tab1["Date"] = new DateTime('2006-06-06');
$tab1["ID_Facture"] = 456789;
$tab1["Montant"] = 200;
$data[] = $tab1;

$tab1["ID_Client"] = 36153615;
$tab1["Date"] = new DateTime('2018-05-30');
$tab1["ID_Facture"] = 123456;
$tab1["Montant"] = 123;
$data[] = $tab1;
\end{lstlisting}

\bigskip

\lstset{language=html}
\begin{lstlisting}[frame=single,title={Appel de la fonction}]
<textarea cols="80" rows="25" readonly="readonly">
<?php
  require_once("exo3_data.php");
  require_once("exo3_fun.php");
  $my_text = my_Reduction($data, 33);
  echo($my_text);
?>
</textarea>
\end{lstlisting}

\lstset{language=html}
\begin{lstlisting}[frame=single,title={Sortie HTML attendue}]
<textarea cols="80" rows="25" readonly="readonly">
42421337;06/06/06;456789;134;200;33;66
36153615;18/05/30;123456;83;123;33;40
</textarea>
\end{lstlisting}

\bigskip

\begin{RedBoxTitle}{ATTENTION}
    Les retours à la ligne ne doivent pas être faits avec la balise \TTBF{"<br />"}, mais avec \TTBF{"\textbackslash n"}.
    (se référer à la section \hyperref[sec:AideMemoire]{Aide Mémoire})
\end{RedBoxTitle}