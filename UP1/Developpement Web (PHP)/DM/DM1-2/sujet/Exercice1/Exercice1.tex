%% Exercice 1

%\ExoSpecs{\TTBF{CalculTVA.sh}}{\TTBF{\RenduDir/src/exo1/}}{750}{640}{\TTBF{write}}
\ExoSpecsCustom{\TTBF{exo1\_fun.php} [my\_Calculette(int, int, string)]}{\TTBF{\RenduDir/src/exo1/}}{}{}{Fonctions recommandées}{\TTBF{(Bases PHP)}, \TTBF{(Maths PHP)}, \TTBF{return}}

\vspace*{0.7cm}

\noindent \ExoObjectif{Le but de l'exercice est de créer une mini calculatrice en PHP.}

\bigskip

\noindent Vous devez écrire une fonction nommée \TTBF{my\_Calculette} qui prendra trois paramètres (deux nombres, puis l'opérateur), et renverra le résultat de l'opération désignée ou un message d'erreur.

\noindent Vous devez implémenter les 5 opérations suivantes : l'addition (symbole \TTBF{+}), la soustraction (symbole \TTBF{-}), la multiplication (lettre \TTBF{*}), la division (symbole \TTBF{/}), et le reste de la division euclidienne (symbole  \TTBF{\%}).

\noindent \`A la fin du calcul, votre fonction doit renvoyer le résultat.

\bigskip

\lstset{language=html}
\begin{lstlisting}[frame=single,title={Cas général PHP}]
<textarea cols="80" rows="25" readonly="readonly">
<?php
  require_once("exo1_fun.php");
  $my_text = my_Calculette(42, 38, "+");
  echo($my_text);
?>
</textarea>
\end{lstlisting}

\lstset{language=html}
\begin{lstlisting}[frame=single,title={Cas général PHP exécuté}]
<textarea cols="80" rows="25" readonly="readonly">
80</textarea>
\end{lstlisting}

\bigskip

\noindent Les deux premiers paramètres doivent être des entiers, et le troisième doit être une chaîne de caractères. Si ça n'est pas le cas, vous devez renvoyer le texte suivant.

\bigskip

\noindent \TTBF{Incorrect\textvisiblespace parameters\textvisiblespace type}

\bigskip

\lstset{language=html}
\begin{lstlisting}[frame=single,title={Cas d'erreur 1 : mauvais paramètres}]
<textarea cols="80" rows="25" readonly="readonly">
<?php
  require_once("exo1_fun.php");
  $my_text = my_Calculette("+", 38, 42);
  echo($my_text);
?>
</textarea>
\end{lstlisting}

\lstset{language=html}
\begin{lstlisting}[frame=single,title={Cas d'erreur 1 exécuté}]
<textarea cols="80" rows="25" readonly="readonly">
Incorrect parameters type</textarea>
\end{lstlisting}

\bigskip

\noindent Si le troisième paramètre donné n'est ni un \TTBF{+}, ni un \TTBF{-}, ni un \TTBF{*}, ni un \TTBF{/}), ni un \TTBF{\%}, alors vous devez renvoyer le message d'erreur suivant.

\bigskip

\noindent \TTBF{Unknown\textvisiblespace operator}

\bigskip

\lstset{language=html}
\begin{lstlisting}[frame=single,title={Cas d'erreur 2 : opérateur inconnu}]
<textarea cols="80" rows="25" readonly="readonly">
<?php
  require_once("exo1_fun.php");
  $my_text = my_Calculette(42, 38, "A");
  echo($my_text);
?>
</textarea>
\end{lstlisting}

\lstset{language=html}
\begin{lstlisting}[frame=single,title={Cas d'erreur 2 exécuté}]
<textarea cols="80" rows="25" readonly="readonly">
Unknown operator</textarea>
\end{lstlisting}

\bigskip

\noindent Si le deuxième paramètre donné à la division est 0, vous devez renvoyer le message d'erreur suivant.

\bigskip

\noindent \TTBF{Division\textvisiblespace by\textvisiblespace 0\textvisiblespace is\textvisiblespace forbidden}

\bigskip

\lstset{language=html}
\begin{lstlisting}[frame=single,title={Cas d'erreur 3 : division par 0}]
<textarea cols="80" rows="25" readonly="readonly">
<?php
  require_once("exo1_fun.php");
  $my_text = my_Calculette(42, 0, "/");
  echo($my_text);
?>
</textarea>
\end{lstlisting}

\lstset{language=html}
\begin{lstlisting}[frame=single,title={Cas d'erreur 3 exécuté}]
<textarea cols="80" rows="25" readonly="readonly">
Division by 0 is forbidden</textarea>
\end{lstlisting}

\bigskip

%\noindent Si le deuxième paramètre donné au modulo est 0, vous devez renvoyer le message d'erreur suivant.
%
%\bigskip
%
%\noindent \TTBF{Modulo\textvisiblespace 0\textvisiblespace is\textvisiblespace forbidden}
%
%\bigskip
%
%\lstset{language=php}
%\begin{lstlisting}[frame=single,title={Cas d'erreur 3 : division par 0}]
%<textarea cols="80" rows="25" readonly="readonly">
%<?php
%  require_once("exo1_fun.php");
%  $my_text = my_Calculette(42, 0, "%");
%  echo($my_text);
%?>
%</textarea>
%\end{lstlisting}
%
%\lstset{language=php}
%\begin{lstlisting}[frame=single,title={Cas d'erreur 3 exécuté}]
%<textarea cols="80" rows="25" readonly="readonly">
%Modulo 0 is forbidden</textarea>
%\end{lstlisting}
%
%\bigskip

\noindent Si plusieurs des problèmes précédents sont rencontrés simultanément, vous devez les gérer dans cet ordre de priorité : le problème de type en priorité, l'opérateur inconnu en second, et la division par 0 en dernier.

%\noindent Si plusieurs des problèmes précédents sont rencontrés simultanément, vous devez les gérer dans cet ordre de priorité : le problème de type en priorité, l'opérateur inconnu en second, et enfin le modulo ou la division par 0 en dernier.

\bigskip

\lstset{language=html}
\begin{lstlisting}[frame=single,title={Cas d'erreurs}]
<textarea cols="80" rows="25" readonly="readonly">
<?php
  require_once("exo1_fun.php");
  $my_text = my_Calculette("A", 42, 0);
  echo($my_text);
?>
</textarea>
\end{lstlisting}

\lstset{language=html}
\begin{lstlisting}[frame=single,title={Cas d'erreurs exécuté}]
<textarea cols="80" rows="25" readonly="readonly">
Incorrect parameters type</textarea>
\end{lstlisting}

\lstset{language=html}
\begin{lstlisting}[frame=single,title={Cas d'erreurs}]
<textarea cols="80" rows="25" readonly="readonly">
<?php
  require_once("exo1_fun.php");
  $my_text = my_Calculette(42, 0, "A");
  echo($my_text);
?>
</textarea>
\end{lstlisting}

\lstset{language=html}
\begin{lstlisting}[frame=single,title={Cas d'erreurs exécuté}]
<textarea cols="80" rows="25" readonly="readonly">
Unknown operator</textarea>
\end{lstlisting}
