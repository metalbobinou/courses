\documentclass[11pt,a4paper]{article}
\usepackage[utf8]{inputenc}
\usepackage[french]{babel}
\usepackage[T1]{fontenc}

\usepackage{amsmath}
\usepackage{amsfonts}
\usepackage{amssymb}

\newcommand{\NomAuteur}{Fabrice BOISSIER - Ali JAFFAL}
\newcommand{\TitreMatiere}{Développement Web - PHP}
\newcommand{\NomUniv}{Paris 1 - Panthéon Sorbonne}
\newcommand{\NiveauUniv}{}
\newcommand{\NumGroupe}{Groupe 1}
\newcommand{\AnneeUniv}{}
\newcommand{\DateExam}{2 avril 2019}
\newcommand{\TypeExam}{Devoir sur Table}
\newcommand{\TitreExam}{\TitreMatiere}
\newcommand{\DureeExam}{1h00}
\newcommand{\MyWaterMark}{2018-2019} % Watermark de protection

% Ajout de mes classes & definitions
\usepackage{MetalExam} % Appelle un .sty

%%%%%%%%%%%%%%%%%%%%%%%
%Header
%%%%%%%%%%%%%%%%%%%%%%%
\lhead{\TypeExam}							%Gauche Haut
\chead{\NomUniv}							%Centre Haut
\rhead{\NumGroupe}							%Droite Haut
\lfoot{\DateExam}							%Gauche Bas
\cfoot{\thepage{} / \pageref*{LastPage}}	%Centre Bas
\rfoot{\texttt{\TitreMatiere}}				%Droite Bas


%%%%%

\usepackage{tabularx}

\begin{document}

% \MakeExamTitleDuree     % Pour afficher la duree
\MakeExamTitle                   % Ne pas afficher la duree

%% \MakeStudentName    %% A reutiliser sur chaque nouvelle page

% Questions cours Apache/HTTP
\section{Questions de Cours}

\subsection{(2 points) Expliquez le rôle de chacun des composants suivants : navigateur, base de données, serveur web. Décrivez les spécificités de LAMP/MAMP/WAMP par rapport à l'architecture générale théorique.}

\bigskip
\bigskip
\bigskip
\bigskip
\bigskip
\bigskip
\bigskip
\bigskip
\bigskip

\subsection{(2 points) Expliquez la différence de fonctionnement au niveau du serveur web entre les méthodes GET et POST.}

\bigskip
\bigskip
\bigskip
\bigskip
\bigskip
\bigskip
\bigskip
\bigskip
\bigskip

\subsection{(2 points) Qu'est-ce qu'un moteur de stockage dans une base de données ? Citer 3 exemples et leurs caractéristiques.}

\bigskip
\bigskip
\bigskip
\bigskip
\bigskip
\bigskip
\bigskip
\bigskip
\bigskip

\subsection{(2 points) Expliquez la différence entre les fonctions \textit{strip\_tags} et \textit{htmlspecialchars} .}

\bigskip
\bigskip
\bigskip
\bigskip
\bigskip
\bigskip
\bigskip
\bigskip
\bigskip

\newpage

\subsection{(3 points) Expliquez les différentes parties des URL suivantes, et ce qu'un serveur web standard comprendra :}

\bigskip

\TTBF{http://www.monsite.fr/files-storage/data.video/index.php}

\bigskip
\bigskip
\bigskip
\bigskip
\bigskip
\bigskip

\TTBF{ftp://server01.datastore.bigcorp.com/contents/fabrice/}

\bigskip
\bigskip
\bigskip
\bigskip
\bigskip
\bigskip

\TTBF{rtmp://live.streamcorp.xyz:1337/flux-video/film/4242.01}

\bigskip
\bigskip
\bigskip
\bigskip
\bigskip
\bigskip

\subsection{(3 points) Remplir le tableau avec la valeur booléenne de retour que chaque fonction renverrai pour chaque valeur de variable en entrée.}

\bigskip

% Allonge les cases en hauteur
\renewcommand\arraystretch{2.5}

\bigskip
\begin{center}
%  \begin{tabularx}{15.5cm}{| c | p{4cm} | p{4cm} | p{4cm} |}
  \begin{tabularx}{\linewidth}{| *{2}{>{\centering \arraybackslash}X |}}
  \hline
  Paramètre & \TTBF{empty()} \\ \hline
  \TTBF{null}  &  \\ \hline
  [\TTBF{unset(\$var)}] &  \\ \hline
  \TTBF{42} &  \\ \hline
  \TTBF{0} &  \\ \hline
  \TTBF{""} &  \\ \hline
  \TTBF{" "} [un espace] &  \\ \hline
  \end{tabularx}
\end{center}
\medskip

\renewcommand\arraystretch{1}

%\bigskip
\newpage

% On affiche le petit cartouche Nom/Prenom
\MakeStudentName

\subsection{(2,5 points) Remplir le tableau avec les codes HTTP ou leur description.}

\bigskip

% Allonge les cases en hauteur
\renewcommand\arraystretch{2.5}

\bigskip
\begin{center}
%  \begin{tabularx}{15.5cm}{| c | p{4cm} | p{4cm} | p{4cm} |}
%  \begin{tabularx}{\linewidth}{| *{2}{>{\centering \arraybackslash}X |}}
\begin{tabularx}{\linewidth}{| c | *{1}{>{\centering \arraybackslash}X |}}
  \hline
  Code HTTP & Description \\ \hline
  200 &  \\ \hline
  302 307 &  \\ \hline
  301 308 &  \\ \hline
  403 &  \\ \hline
  404 &  \\ \hline
  \end{tabularx}
\end{center}
\medskip

\renewcommand\arraystretch{1}

\bigskip

% Faire fonctionner code
\section{Développement}

\subsection{(3,5 points) Ce code devrait afficher le contenu du tableau, mais il ne fonctionne pas. Corrigez les erreurs de syntaxe pour que le tableau soit correctement rempli et affiché.}

\medskip

\lstset{language=php}
\begin{lstlisting}[frame=single,numbers=left]
<html>
<body>
<?php

my_tab = array();
my_tab[1] = Fabrice;
my_tab[2] = Ali;
my_tab[3] = Floriane;

foreach ( i  in  my_tab )
{
   echo('i \n');
}

?>
</body>
</html>
\end{lstlisting}

\end{document}
