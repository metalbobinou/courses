\documentclass[11pt,a4paper]{article}
\usepackage[utf8]{inputenc}
\usepackage[french]{babel}
\usepackage[T1]{fontenc}

\usepackage{amsmath}
\usepackage{amsfonts}
\usepackage{amssymb}

\newcommand{\NomAuteur}{Fabrice BOISSIER - Ali JAFFAL}
\newcommand{\TitreMatiere}{Développement Web - PHP}
\newcommand{\NomUniv}{Paris 1 - Panthéon Sorbonne}
\newcommand{\NiveauUniv}{}
\newcommand{\NumGroupe}{Groupe 1}
\newcommand{\AnneeUniv}{}
\newcommand{\DateExam}{2 avril 2019}
\newcommand{\TypeExam}{Devoir sur Table}
\newcommand{\TitreExam}{\TitreMatiere}
\newcommand{\DureeExam}{1h00}
\newcommand{\MyWaterMark}{2018-2019} % Watermark de protection

% Ajout de mes classes & definitions
\usepackage{MetalExam} % Appelle un .sty

%%%%%%%%%%%%%%%%%%%%%%%
%Header
%%%%%%%%%%%%%%%%%%%%%%%
\lhead{\TypeExam}							%Gauche Haut
\chead{\NomUniv}							%Centre Haut
\rhead{\NumGroupe}							%Droite Haut
\lfoot{\DateExam}							%Gauche Bas
\cfoot{\thepage{} / \pageref*{LastPage}}	%Centre Bas
\rfoot{\texttt{\TitreMatiere}}				%Droite Bas


%%%%%

\usepackage{tabularx}

\begin{document}

% \MakeExamTitleDuree     % Pour afficher la duree
\MakeExamTitle                   % Ne pas afficher la duree

%% \MakeStudentName    %% A reutiliser sur chaque nouvelle page

% Questions cours Apache/HTTP
\section{Questions de Cours}

\subsection{(2 points) Expliquez le rôle de chacun des composants suivants : navigateur, base de données, serveur web. Décrivez les spécificités de LAMP/MAMP/WAMP par rapport à l'architecture générale théorique.}

\bigskip

\begin{itemize}
\item Navigateur : logiciel permettant au client d'effectuer des requêtes à un serveur web... c'est à dire, il permet de naviguer sur le web.
\item Serveur Web : logiciel permettant de répondre aux requêtes d'un navigateur web en lui envoyant les ressources qu'il demande, ou éventuellement un message d'erreur
\item Base de Données : un conteneur de données organisées (n'est pas un SGBD)
\end{itemize}

LAMP/MAMP/WAMP : Les 3 suites logicielles permettent d'installer sur un seul système (Linux, Mac, Windows) un serveur web et un SGBD.

\bigskip

\subsection{(2 points) Expliquez la différence de fonctionnement au niveau du serveur web entre les méthodes GET et POST.}

\bigskip

Le navigateur, pour envoyer des données au serveur web, peut utiliser 2 méthodes : GET et POST.

GET : l'URL vers la ressource est étendue d'un \TTBF{?} qui sépare les ressources visées des données transmises.
Les variables sont séparées par des \TTBF{\&}. (format général : http://domain.com/ressource?var=42\& i=4).
On ne devrait pas envoyer plus de 256 octets par cette méthode.
Le serveur web reçoit donc une URL contenant les données supplémentaires.
Le serveur web enregistre dans les logs/journaux les URL demandées, il ne faut donc jamais envoyer de données sensibles par cette méthode. (les admins du serveur web peuvent lire les données)

POST : Le navigateur envoie les données dans le corps de la réponse, elles ne sont donc pas enregistrées dans les logs/journaux du serveur web.
Il n'y a pas de limite pour la taille des données envoyées en POST.

\bigskip

\subsection{(2 points) Qu'est-ce qu'un moteur de stockage dans une base de données ? Citer 3 exemples et leurs caractéristiques.}

\bigskip

Le moteur de stockage d'une base de données est le le logiciel/programme/code qui permet d'organiser les données dans la base.
Chaque moteur a des caractéristiques spécifiques (écriture ou lecture rapide, ...).

\begin{itemize}
\item MyISAM : lecture rapide, pas de clé étrangère
\item InnoDB : gère ACID et les clés étrangères
\item Memory : aucune écriture sur disque, uniquement en mémoire... ultra rapide mais ne sauvegarde rien en cas de crash
\end{itemize}

\bigskip

\subsection{(2 points) Expliquez la différence entre les fonctions \textit{strip\_tags} et \textit{htmlspecialchars} .}

\bigskip

\begin{itemize}
\item htmlspecialchars (chaine, flags) : Conversion des caractères spéciaux en entités HTML. Remplace par exemple \& (ET commercial) en \&amp;
\item strip\_tags (chaine, allowableTags) : Supprime les balises HTML et PHP d'une chaîne. Les commentaires HTML et PHP sont également supprimés
\end{itemize}

\bigskip

\lstset{language=php}
\begin{lstlisting}[frame=single]
$str = "This is some <b> bold </b> text.";

echo htmlspecialchars($str);
// affichera : This is some &lt;b&gt; bold &lt;/b&gt; text.




$text = '
<p>Paragraph.</p><!-- Comment --><a href="#fragment">Other text</a>
';

// Autorise <p> et <a>
echo strip_tags($text, '<p><a>');
// affichera :
// <p>Paragraph.</p> <a href="#fragment">Other text</a>
\end{lstlisting}

\bigskip

\subsection{(3 points) Expliquez les différentes parties des URL suivantes, et ce qu'un serveur web standard comprendra :}

\bigskip

\begin{enumerate}
\item \TTBF{http://www.monsite.fr/files-storage/data.video/index.php} : protocole : http (web), domaine : www.monsite.fr, ressource : files-storage/data.video/index.php (fichier : index.php)
\item \TTBF{ftp://server01.datastore.bigcorp.com/contents/fabrice/} : protocole : ftp, domaine : server01.datastore.bigcorp.com, ressource : /contents/fabrice/ (dossier)
\item \TTBF{rtmp://live.streamcorp.xyz:1337/flux-video/film/4242.01} : protocole : rtmp (streaming), live.streamcorp.xyz, ressource : /flux-video/film/4242.01
\end{enumerate}

\bigskip

\subsection{(3 points) Remplir le tableau avec la valeur booléenne de retour que chaque fonction renverrai pour chaque valeur de variable en entrée.}

\bigskip

% Allonge les cases en hauteur
\renewcommand\arraystretch{2.5}

\bigskip
\begin{center}
%  \begin{tabularx}{15.5cm}{| c | p{4cm} | p{4cm} | p{4cm} |}
  \begin{tabularx}{\linewidth}{| *{2}{>{\centering \arraybackslash}X |}}
  \hline
  Paramètre & \TTBF{empty()} \\ \hline
  \TTBF{null} & true \\ \hline
  [\TTBF{unset(\$var)}] & true \\ \hline
  \TTBF{42} & false \\ \hline
  \TTBF{0} & true \\ \hline
  \TTBF{""} & true \\ \hline
  \TTBF{" "} [un espace] & false \\ \hline
  \end{tabularx}
\end{center}
\medskip

\renewcommand\arraystretch{1}

\bigskip

\subsection{(2,5 points) Remplir le tableau avec les codes HTTP ou leur description.}

\bigskip

% Allonge les cases en hauteur
\renewcommand\arraystretch{2.5}

\bigskip
\begin{center}
%  \begin{tabularx}{15.5cm}{| c | p{4cm} | p{4cm} | p{4cm} |}
  \begin{tabularx}{\linewidth}{| *{2}{>{\centering \arraybackslash}X |}}
  \hline
  Code HTTP & Description \\ \hline
  200 & OK \\ \hline
  302 307 & Redirection Temporaire \\ \hline
  301 308 & Redirection Permanente \\ \hline
  403 & Le client n'a pas le droit d'accéder à la ressource \\ \hline
  404 & URL/Ressource demandée n'existe pas \\ \hline
  \end{tabularx}
\end{center}
\medskip

\renewcommand\arraystretch{1}

%\bigskip

\newpage

% Faire fonctionner code
\section{Développement}

\subsection{(3,5 points) Ce code devrait afficher le contenu du tableau, mais il ne fonctionne pas. Corrigez les erreurs de syntaxe pour que le tableau soit correctement rempli et affiché.}

\medskip

\lstset{language=php}
\begin{lstlisting}[frame=single,numbers=left]
<html>
<body>
<?php

my_tab = array();
my_tab[1] = Fabrice;
my_tab[2] = Ali;
my_tab[3] = Floriane;

foreach ( i  in  my_tab )
{
   echo('i \n');
}

?>
</body>
</html>
\end{lstlisting}


CORRECTION :

\medskip

\lstset{language=php}
\begin{lstlisting}[frame=single]
<html>
<body>
<?php

$my_tab = array();			// <== $   (0,5p)
$my_tab[1] = "Fabrice";		// <== $ " "    (0,5p)
$my_tab[2] = "Ali";			// <== $ " "    (0,5p)
$my_tab[3] = "Floriane";	// <== $ " "    (0,5p)

foreach ( $my_tab as $i )	// <== $my_tab as $i    (1p)
{
   echo("$i \n");			// <== " $ "    (0,5p)
}

?>
</body>
</html>
\end{lstlisting}

\bigskip

\end{document}
