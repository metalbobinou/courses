\documentclass[11pt,a4paper]{article}
\usepackage[utf8]{inputenc}
\usepackage[french]{babel}
\usepackage[T1]{fontenc}

\usepackage{amsmath}
\usepackage{amsfonts}
\usepackage{amssymb}

\newcommand{\NomAuteur}{Fabrice BOISSIER - Ali JAFFAL}
\newcommand{\TitreMatiere}{Développement Web - PHP}
\newcommand{\NomUniv}{Paris 1 - Panthéon Sorbonne}
\newcommand{\NiveauUniv}{}
\newcommand{\NumGroupe}{Groupe 2}
\newcommand{\AnneeUniv}{}
\newcommand{\DateExam}{4 avril 2019}
\newcommand{\TypeExam}{Devoir sur Table}
\newcommand{\TitreExam}{\TitreMatiere}
\newcommand{\DureeExam}{1h00}
\newcommand{\MyWaterMark}{2018-2019} % Watermark de protection

% Ajout de mes classes & definitions
\usepackage{MetalExam} % Appelle un .sty

%%%%%%%%%%%%%%%%%%%%%%%
%Header
%%%%%%%%%%%%%%%%%%%%%%%
\lhead{\TypeExam}							%Gauche Haut
\chead{\NomUniv}							%Centre Haut
\rhead{\NumGroupe}							%Droite Haut
\lfoot{\DateExam}							%Gauche Bas
\cfoot{\thepage{} / \pageref*{LastPage}}	%Centre Bas
\rfoot{\texttt{\TitreMatiere}}				%Droite Bas


%%%%%

\usepackage{tabularx}

\begin{document}

% \MakeExamTitleDuree     % Pour afficher la duree
\MakeExamTitle                   % Ne pas afficher la duree

%% \MakeStudentName    %% A reutiliser sur chaque nouvelle page

% Questions cours Apache/HTTP
\section{Questions de Cours}

\subsection{(2 points) PHP est-il un langage fortement ou faiblement typé ? Expliquez la différence.}

\bigskip

PHP est un langage faiblement typé.

Les langages fortement typés imposent que les types soient rigoureusement les mêmes que ceux utilisés dans les opérations/fonctions.
Une comparaison entre un nombre et une chaîne de carcatères ne peut pas fonctionner tel quel.

Les langages faiblements types ne vérifient pas les types avant d'effectuer des opérations, ils utilisent le plus simple/commun aux valeurs données en paramètres.
Par exemple, comparer une chaîne de caractères et un nombre sera effectué en comparant des chaînes de caractères uniquement.

\bigskip

\subsection{(2 points) Expliquez la différence d'utilisation en PHP des méthodes GET et POST. Indiquez les limites de chaque méthode, s'il y en a.}

\bigskip

En PHP, les variables envoyées avec GET et POST se trouvent dans des tableaux associatifs.
Un formulaire envoie les données selon la méthode indiquée, le serveur web les récupère et les insère dans des tableaux associatifs pour PHP.
Les variables GET et POST sont des superglobales.

\begin{verbatim}
$_GET['variable']
$_POST['variable']
\end{verbatim}

\bigskip

\subsection{(2 points) Dans quel cas vaut-il mieux utiliser des \textit{sessions} ? Dans quel cas vaut-il mieux utiliser des \textit{cookies} ?}

\bigskip

\begin{itemize}
\item Sessions : Très utile pour stocker les gros objets sans avoir à les renvoyer au serveur web à chaque requête.
\item Cookies : Utile pour identifier l'utilisateur. Déconseillé pour stocker les gros objets, car il faut les renvoyer à chaque requête.
\end{itemize}

\bigskip

\subsection{(2 points) Expliquez ce qu'est un SGBD par rapport à une BDD. Indiquez le nom du standard permettant d'interroger un SGBD.}

\bigskip

\begin{itemize}
\item BDD/DB : Base De Données/DataBase, l'ensemble des données organisées dans un fichier (ou en mémoire) selon un schéma
\item SGBD/DBMS : Système de Gestion de Base de Données/DataBase Management System, le logiciel/programme permettant d'interroger et modifier des bases de données
\end{itemize}

SQL (Standard Query Language) est le langage standard de requête de base de données.

\bigskip

\subsection{(3 points) Expliquez les différentes parties des URL suivantes, et ce qu'un serveur web standard comprendra :}

\bigskip

\begin{enumerate}
\item \TTBF{http://www2.website.jp:3864/homepages/portal/} : protocole : http (web), domaine : www2.website.jp, port : 3864, ressource : /homepages/portal/
\item \TTBF{https://univ-paris.fr} : protocole : http (web), domaine : univ-paris.fr, ressource : aucune, donc la racine /
\item \TTBF{http://www.dailynews.co.uk/article/2864} : protocole : http (web), domaine : www.dailynews.co.uk, ressource : /article/2864
\end{enumerate}

\bigskip

\subsection{(3 points) Remplir le tableau avec la valeur booléenne de retour que chaque fonction renverrai pour chaque valeur de variable en entrée.}

\bigskip

% Allonge les cases en hauteur
\renewcommand\arraystretch{2.5}

\bigskip
\begin{center}
%  \begin{tabularx}{15.5cm}{| c | p{4cm} | p{4cm} | p{4cm} |}
  \begin{tabularx}{\linewidth}{| *{4}{>{\centering \arraybackslash}X |}}
  \hline
  Paramètre & \TTBF{is\_null()} \\ \hline
  \TTBF{null} & true \\ \hline
  [\TTBF{unset(\$var)}] & error/true \\ \hline
  \TTBF{42} & false \\ \hline
  \TTBF{0} & false \\ \hline
  \TTBF{""} & false \\ \hline
  \TTBF{" "} [un espace] & false \\ \hline
  \end{tabularx}
\end{center}
\medskip

\renewcommand\arraystretch{1}

\bigskip

\subsection{(2,5 points) Remplir le tableau avec les familles de codes HTTP et leur description.}

\bigskip

% Allonge les cases en hauteur
\renewcommand\arraystretch{2.5}

\bigskip
\begin{center}
%  \begin{tabularx}{15.5cm}{| c | p{4cm} | p{4cm} | p{4cm} |}
  \begin{tabularx}{\linewidth}{| *{2}{>{\centering \arraybackslash}X |}}
  \hline
  Code HTTP & Description \\ \hline
  100 & Informations sur l'état de la connection \\ \hline
  200 & Réponses positives \\ \hline
  300 & Redirections \\ \hline
  400 & Erreurs côté client/dans la requête envoyée \\ \hline
  500 & Erreurs côté serveur \\ \hline
  \end{tabularx}
\end{center}
\medskip

\renewcommand\arraystretch{1}

\bigskip

% Faire fonctionner code
\section{Développement}

\subsection{(3,5 points) Ce code devrait compter le nombre de lignes contenant la valeur "Floriane", mais il ne fonctionne pas. Corrigez les erreurs pour que le compte soit bon.}

%\medskip

\lstset{language=php}
\begin{lstlisting}[frame=single,numbers=left]
<html>
<body>
<?php

// Fait une requete SQL et l'ecrit dans $result
require(functions.php)

$results = my_fun();

for ( $i = 1; $i <= length($results); $i-- )
{
	if ($results[$i] == "Floriane")
		$count++;
}

echo "$count";
?>
</body>
</html>
\end{lstlisting}


CORRECTION :

\medskip

\lstset{language=php}
\begin{lstlisting}[frame=single]
<html>
<body>
<?php

// Contient la fonction "my_fun" qui fait une requete SQL et
//  renvoie son resultat
require("functions.php");	// <== " " (0,5p)   ; (0,5p)   

$results = my_fun();
$count = 0;					// <== $count = 0;    (0,5p)

  // 0 (0,5p)    < (0,5p)    count/sizeof (0,5p)    $i++ (0,5p)
for ($i = 0; $i < count($results); $i++) // <== 0 < count/sizeof $i++
{
        if ($results[$i] == "Floriane")
                $count++;
}

echo "$count";
?>
</body>
</html>
\end{lstlisting}

\bigskip

\end{document}
