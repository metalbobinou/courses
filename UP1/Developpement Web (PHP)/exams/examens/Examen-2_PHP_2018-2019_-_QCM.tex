\documentclass[11pt,a4paper]{article}
\usepackage[utf8]{inputenc}
\usepackage[french]{babel}
\usepackage[T1]{fontenc}

\usepackage{amsmath}
\usepackage{amsfonts}
\usepackage{amssymb}

\newcommand{\NomAuteur}{Fabrice BOISSIER - Ali JAFFAL}
\newcommand{\TitreMatiere}{Développement Web - PHP}
\newcommand{\NomUniv}{Paris 1 - Panthéon Sorbonne}
\newcommand{\NiveauUniv}{}
\newcommand{\NumGroupe}{Groupe 2}
\newcommand{\AnneeUniv}{}
\newcommand{\DateExam}{4 avril 2019}
\newcommand{\TypeExam}{Devoir sur Table}
\newcommand{\TitreExam}{\TitreMatiere}
\newcommand{\DureeExam}{1h00}
\newcommand{\MyWaterMark}{2018-2019} % Watermark de protection

% Ajout de mes classes & definitions
\usepackage{MetalExam} % Appelle un .sty

%%%%%%%%%%%%%%%%%%%%%%%
%Header
%%%%%%%%%%%%%%%%%%%%%%%
\lhead{\TypeExam}							%Gauche Haut
\chead{\NomUniv}							%Centre Haut
\rhead{\NumGroupe}							%Droite Haut
\lfoot{\DateExam}							%Gauche Bas
\cfoot{\thepage{} / \pageref*{LastPage}}	%Centre Bas
\rfoot{\texttt{\TitreMatiere}}				%Droite Bas


%%%%%

\usepackage{tabularx}

\begin{document}

% \MakeExamTitleDuree     % Pour afficher la duree
\MakeExamTitle                   % Ne pas afficher la duree

%% \MakeStudentName    %% A reutiliser sur chaque nouvelle page

% Questions cours Apache/HTTP
\section{QCM (réponses multiples)}

\medskip

Réponse juste = 1 point ; Pas de réponse = 0 point ; Mauvaise réponse = -0,5 point

\medskip

\subsection{PHP est un langage...}

\begin{itemize}
\item[\CaseCoche] Fortement typé\\
\item[\CaseCoche] Faiblement typé\\
\item[\CaseCoche] Fonctionnel\\
\item[\CaseCoche] Orienté Objet\\
\end{itemize}


\subsection{Quels opérateurs/fonctions vérifient les types ?}

\begin{itemize}
\item[\CaseCoche] \TTBF{==} \\
\item[\CaseCoche] \TTBF{gettype()} \\
\item[\CaseCoche] \TTBF{get\_type()} \\
\item[\CaseCoche] \TTBF{is\_int()} \\
\end{itemize}


\subsection{Quelles est la bonne instruction pour annuler une transaction ?}

\begin{itemize}
\item[\CaseCoche] ROLLBACK \\
\item[\CaseCoche] STOP TRANSACTION \\
\item[\CaseCoche] CANCEL TRANSACTION \\
\item[\CaseCoche] COMMIT \\
\end{itemize}


\subsection{Comment récupérer les lignes du résultat sous forme de tableau ?}

\TTBF{\$result = mysqli\_query(select * from USER);}

\bigskip

\begin{itemize}
\item[\CaseCoche] \TTBF{mysqli\_fetch\_assoc(\$result) ;} \\
\item[\CaseCoche] \TTBF{mysqli\_list\_fields(\$result) ;} \\
\item[\CaseCoche] \TTBF{mysqli\_fetch\_array(\$result) ;} \\
\end{itemize}

\newpage

\subsection{Quels sont les séparateurs utilisés par GET pour transmettre des données par URL ?}

\begin{itemize}
\item[\CaseCoche] "?" Pour séparer l'adresse et "\%" pour les données \\
\item[\CaseCoche] "\%" Pour séparer l'adresse et "\&" pour les données \\
\item[\CaseCoche] "?" Pour séparer l'adresse et "\&" pour les données \\
\item[\CaseCoche] "\%" Pour séparer l'adresse et ";" pour les données \\
\end{itemize}


\subsection{Comment afficher la valeur du cookie \TTBF{ck1} ?}

\begin{itemize}
\item[\CaseCoche] \TTBF{print(COOKIE["ck1"]) ;} \\
\item[\CaseCoche] \TTBF{echo COOKIE["ck1"] ;} \\
\item[\CaseCoche] \TTBF{echo \$\_COOKIE["ck1"] ;} \\
\end{itemize}


\subsection{Comment écrit-on \TTBF{hello world} en PHP ?}

\begin{itemize}
\item[\CaseCoche] \TTBF{<p> Hello World </p>} \\
\item[\CaseCoche] \TTBF{Document.Write('Hello World');} \\
\item[\CaseCoche] \TTBF{echo 'Hello World' ;} \\
\end{itemize}


\subsection{Quelle est la façon correcte pour déclarer une fonction \TTBF{myFunction} en PHP ?}

\begin{itemize}
\item[\CaseCoche] \TTBF{function myFunction()} \\
\item[\CaseCoche] \TTBF{new function myFunction()} \\
\item[\CaseCoche] \TTBF{create myFunction()} \\
\end{itemize}


\subsection{En PHP, la seule façon d'afficher un texte est la commande \TTBF{echo} :}

\begin{itemize}
\item[\CaseCoche] Vrai \\
\item[\CaseCoche] Faux \\
\end{itemize}


\subsection{Cocher deux caractéristiques d'un cookie ?}

\begin{itemize}
\item[\CaseCoche] Il peut être refusé ou effacé par l'internaute \\
\item[\CaseCoche] Sa taille est fixe \\
\item[\CaseCoche] Sa durée de vie est variable \\
\end{itemize}


\newpage

% On affiche le petit cartouche Nom/Prenom
\MakeStudentName

\subsection{Les balises d'ouverture et de fermeture qui délimitent un script PHP sont :}

\begin{itemize}
\item[\CaseCoche] \TTBF{<php> ... </php>} \\
\item[\CaseCoche] \TTBF{<?php> ... </?>} \\
\item[\CaseCoche] \TTBF{<\& ... /\&>} \\
\item[\CaseCoche] \TTBF{<?php ... ?>} \\
\end{itemize}


\subsection{\TTBF{include("rep2/fich4.php")};}

\begin{itemize}
\item[\CaseCoche] Inclut \TTBF{fich4.php} situé dans le sous-répertoire \TTBF{rep2} de la racine du site \\
\item[\CaseCoche] Provoque une erreur PHP \\
\item[\CaseCoche] Crée une inclusion dans le fichier \TTBF{fich4.php} du répertoire \TTBF{rep4} \\
\item[\CaseCoche] Inclut le \TTBF{fich4.php}, situé dans le sous-répertoire \TTBF{rep2}, dans le script en cours \\
\end{itemize}


\subsection{Comment créer une session qui permet d'utiliser des variables de session ?}

\begin{itemize}
\item[\CaseCoche] \TTBF{start\_session();} \\
\item[\CaseCoche] \TTBF{session\_start();} \\
\item[\CaseCoche] \TTBF{session\_begin();} \\
\item[\CaseCoche] \TTBF{begin\_session();} \\
\end{itemize}


\subsection{Qu'affiche l'instruction suivante ?}

\TTBF{echo 'Voici un \textbackslash "test\textbackslash " php';}

\bigskip

\begin{itemize}
\item[\CaseCoche] \TTBF{Voici un "test" php} \\
\item[\CaseCoche] \TTBF{Voici un test php} \\
\item[\CaseCoche] \TTBF{Voici un \textbackslash "test\textbackslash " php} \\
\item[\CaseCoche] Rien car cela provoquerait une erreur \\
\end{itemize}


\newpage


\subsection{Comment supprimer l'ensemble des variables d'une session ?}

\begin{itemize}
\item[\CaseCoche] \TTBF{unset(\$\_SESSION[]);} \\
\item[\CaseCoche] \TTBF{session\_destroy();} \\
\item[\CaseCoche] \TTBF{session\_stop();} \\
\item[\CaseCoche] \TTBF{session\_unset();} \\
\end{itemize}


\subsection{Cocher la ou les affirmations vraies concernant cette URL :}

\TTBF{http://www2.example.com:3280/custompages/alex/cosplays.php}

\bigskip

\begin{itemize}
\item[\CaseCoche] Domaine : \TTBF{http://www2.example.com:3280} \\
\item[\CaseCoche] Domaine : \TTBF{example.com:3280} \\
\item[\CaseCoche] Protocole : Web \\
\item[\CaseCoche] Ressource : \TTBF{alex/cosplays.php} \\
\end{itemize}


\subsection{Cocher la ou les affirmations vraies concernant cette URL :}

\TTBF{http://www.journal.co.uk/paper:4364}

\bigskip

\begin{itemize}
\item[\CaseCoche] Ressource : \TTBF{/paper:4364} \\
\item[\CaseCoche] Domaine : \TTBF{www.journal.co.uk} \\
\item[\CaseCoche] Port : \TTBF{4364} \\
\item[\CaseCoche] Ressource : \TTBF{/paper} \\
\end{itemize}


\subsection{Cocher la ou les affirmations vraies concernant cette URL :}

\TTBF{https://machine01.site.org}

\bigskip

\begin{itemize}
\item[\CaseCoche] Ressource : \TTBF{site.org} \\
\item[\CaseCoche] Domaine : \TTBF{machine01.site.org} \\
\item[\CaseCoche] Port : \TTBF{4443} \\
\item[\CaseCoche] Protocole : http over secure shell \\
\end{itemize}

\newpage

% On affiche le petit cartouche Nom/Prenom
\MakeStudentName

\subsection{Quelle est la syntaxe correcte pour tester la valeur de la variable \TTBF{\$a} ?}

\begin{itemize}
\item[\CaseCoche] \TTBF{if (\$a = 5) {\$b = 4}; else {b = 2};} \\
\item[\CaseCoche] \TTBF{if {\$a = 5} then {\$b = 4}; else {\$b = 2};} \\
\item[\CaseCoche] \TTBF{if (\$a == 5) {\$b = 4}; else {\$b = 2};} \\
\end{itemize}


\subsection{Que provoque l'exécution du script suivant ?}

\TTBF{<?php \$a=7 ; do{\$a++ ;echo \$a;} while(0); ?>}

\bigskip

\begin{itemize}
\item[\CaseCoche] Une boucle infinie \\
\item[\CaseCoche] L'affichage du chiffre 7 \\
\item[\CaseCoche] L'affichage du chiffre 8 \\
\item[\CaseCoche] Rien car cela provoquerait une erreur \\
\end{itemize}


\end{document}
