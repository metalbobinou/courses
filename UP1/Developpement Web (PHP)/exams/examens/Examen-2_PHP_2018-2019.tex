\documentclass[11pt,a4paper]{article}
\usepackage[utf8]{inputenc}
\usepackage[french]{babel}
\usepackage[T1]{fontenc}

\usepackage{amsmath}
\usepackage{amsfonts}
\usepackage{amssymb}

\newcommand{\NomAuteur}{Fabrice BOISSIER - Ali JAFFAL}
\newcommand{\TitreMatiere}{Développement Web - PHP}
\newcommand{\NomUniv}{Paris 1 - Panthéon Sorbonne}
\newcommand{\NiveauUniv}{}
\newcommand{\NumGroupe}{Groupe 2}
\newcommand{\AnneeUniv}{}
\newcommand{\DateExam}{4 avril 2019}
\newcommand{\TypeExam}{Devoir sur Table}
\newcommand{\TitreExam}{\TitreMatiere}
\newcommand{\DureeExam}{1h00}
\newcommand{\MyWaterMark}{2018-2019} % Watermark de protection

% Ajout de mes classes & definitions
\usepackage{MetalExam} % Appelle un .sty

%%%%%%%%%%%%%%%%%%%%%%%
%Header
%%%%%%%%%%%%%%%%%%%%%%%
\lhead{\TypeExam}							%Gauche Haut
\chead{\NomUniv}							%Centre Haut
\rhead{\NumGroupe}							%Droite Haut
\lfoot{\DateExam}							%Gauche Bas
\cfoot{\thepage{} / \pageref*{LastPage}}	%Centre Bas
\rfoot{\texttt{\TitreMatiere}}				%Droite Bas


%%%%%

\usepackage{tabularx}

\begin{document}

% \MakeExamTitleDuree     % Pour afficher la duree
\MakeExamTitle                   % Ne pas afficher la duree

%% \MakeStudentName    %% A reutiliser sur chaque nouvelle page

% Questions cours Apache/HTTP
\section{Questions de Cours}

\subsection{(2 points) PHP est-il un langage fortement ou faiblement typé ? Expliquez la différence.}

\bigskip
\bigskip
\bigskip
\bigskip
\bigskip
\bigskip
\bigskip
\bigskip
\bigskip

\subsection{(2 points) Expliquez la différence d'utilisation en PHP des méthodes GET et POST. Indiquez les limites de chaque méthode, s'il y en a.}

\bigskip
\bigskip
\bigskip
\bigskip
\bigskip
\bigskip
\bigskip
\bigskip
\bigskip

\subsection{(2 points) Dans quel cas vaut-il mieux utiliser des \textit{sessions} ? Dans quel cas vaut-il mieux utiliser des \textit{cookies} ?}

\bigskip
\bigskip
\bigskip
\bigskip
\bigskip
\bigskip
\bigskip
\bigskip
\bigskip

\subsection{(2 points) Expliquez ce qu'est un SGBD par rapport à une BDD. Indiquez le nom du standard permettant d'interroger un SGBD.}

\bigskip
\bigskip
\bigskip
\bigskip
\bigskip
\bigskip
\bigskip
\bigskip
\bigskip

\newpage

\subsection{(3 points) Expliquez les différentes parties des URL suivantes, et ce qu'un serveur web standard comprendra :}

\bigskip

\TTBF{http://www2.website.jp:3864/homepages/portal/}

\bigskip
\bigskip
\bigskip
\bigskip
\bigskip
\bigskip

\TTBF{https://univ-paris.fr}

\bigskip
\bigskip
\bigskip
\bigskip
\bigskip
\bigskip

\TTBF{http://www.dailynews.co.uk/article/2864}

\bigskip
\bigskip
\bigskip
\bigskip
\bigskip
\bigskip

\subsection{(3 points) Remplir le tableau avec la valeur booléenne de retour que chaque fonction renverrai pour chaque valeur de variable en entrée.}

\bigskip

% Allonge les cases en hauteur
\renewcommand\arraystretch{2.5}

\bigskip
\begin{center}
%  \begin{tabularx}{15.5cm}{| c | p{4cm} | p{4cm} | p{4cm} |}
  \begin{tabularx}{\linewidth}{| *{4}{>{\centering \arraybackslash}X |}}
  \hline
  Paramètre & \TTBF{is\_null()} \\ \hline
  \TTBF{null} &  \\ \hline
  [\TTBF{unset(\$var)}] & \\ \hline
  \TTBF{42} & \\ \hline
  \TTBF{0} & \\ \hline
  \TTBF{""} & \\ \hline
  \TTBF{" "} [un espace] & \\ \hline
  \end{tabularx}
\end{center}
\medskip

\renewcommand\arraystretch{1}

\newpage

% On affiche le petit cartouche Nom/Prenom
\MakeStudentName

\subsection{(2,5 points) Remplir le tableau avec les familles de codes HTTP et leur description.}

\bigskip

% Allonge les cases en hauteur
\renewcommand\arraystretch{2.5}

\bigskip
\begin{center}
%  \begin{tabularx}{15.5cm}{| c | p{4cm} | p{4cm} | p{4cm} |}
  \begin{tabularx}{\linewidth}{| *{2}{>{\centering \arraybackslash}X |}}
  \hline
  Code HTTP & Description \\ \hline
  100 &  \\ \hline
  200 &  \\ \hline
  300 &  \\ \hline
  400 &  \\ \hline
  500 &  \\ \hline
  \end{tabularx}
\end{center}
\medskip

\renewcommand\arraystretch{1}

\bigskip

% Faire fonctionner code
\section{Développement}

\subsection{(3,5 points) Ce code devrait compter le nombre de lignes contenant la valeur "Floriane", mais il ne fonctionne pas. Corrigez les erreurs pour que le compte soit bon.}

\medskip

\lstset{language=php}
\begin{lstlisting}[frame=single,numbers=left]
<html>
<body>
<?php

// Fait une requete SQL et l'ecrit dans $result
require(functions.php)

$results = my_fun();

for ( $i = 1; $i <= length($results); $i-- )
{
	if ($results[$i] == "Floriane")
		$count++;
}

echo "$count";
?>
</body>
</html>
\end{lstlisting}

\end{document}
