\documentclass[11pt,a4paper]{article}
\usepackage[utf8]{inputenc}
\usepackage[french]{babel}
\usepackage[T1]{fontenc}

\usepackage{amsmath}
\usepackage{amsfonts}
\usepackage{amssymb}

\newcommand{\NomAuteur}{Fabrice BOISSIER - Ali JAFFAL}
\newcommand{\TitreMatiere}{Développement Web - PHP}
\newcommand{\NomUniv}{Paris 1 - Panthéon Sorbonne}
\newcommand{\NiveauUniv}{} % Dans titre
\newcommand{\NumGroupe}{Groupe 2}
\newcommand{\AnneeUniv}{} % Dans titre
\newcommand{\DateExam}{4 avril 2019}
\newcommand{\TypeExam}{Devoir sur Table}
\newcommand{\TitreExam}{\TitreMatiere}
\newcommand{\DureeExam}{1h30}
\newcommand{\MyWaterMark}{2018-2019} % Watermark de protection

% Ajout de mes classes & definitions
\usepackage{MetalExam} % Appelle un .sty

%%%%%%%%%%%%%%%%%%%%%%%
%Header
%%%%%%%%%%%%%%%%%%%%%%%
\lhead{\TypeExam}							%Gauche Haut
\chead{\NomUniv}							%Centre Haut
\rhead{\NumGroupe}							%Droite Haut
\lfoot{\DateExam}							%Gauche Bas
\cfoot{\thepage{} / \pageref*{LastPage}}	%Centre Bas
\rfoot{\texttt{\TitreMatiere}}				%Droite Bas


%%%%%

\usepackage{tabularx}

\begin{document}

% \MakeExamTitleDuree     % Pour afficher la duree
\MakeExamTitle                   % Ne pas afficher la duree

%% \MakeStudentName    %% A reutiliser sur chaque nouvelle page

% Questions cours Apache/HTTP
\section{Questions de Cours}

\bigskip

\subsection{Expliquez le rôle de chacun des composants suivants : navigateur, base de données, serveur web. Décrivez les spécificités de LAMP/MAMP/WAMP par rapport à l'architecture générale théorique.}

\bigskip

\begin{itemize}
\item Navigateur : logiciel permettant au client d'effectuer des requêtes à un serveur web... c'est à dire, il permet de naviguer sur le web.
\item Serveur Web : logiciel permettant de répondre aux requêtes d'un navigateur web en lui envoyant les ressources qu'il demande, ou éventuellement un message d'erreur
\item Base de Données : un conteneur de données organisées (n'est pas un SGBD)
\end{itemize}

LAMP/MAMP/WAMP : Les 3 suites logicielles permettent d'installer sur un seul système (Linux, Mac, Windows) un serveur web et un SGBD.

\bigskip

\subsection{Expliquez des points de vue utilisateur et développeur les différences entre un site web statique et dynamique. Citer quelques langages nécessaires pour faire un site web dynamique.}

\bigskip

Du point de vue utilisateur, un site web statique ne changera jamais lorsque l'on rafraichit la page/répète une même requête.
\`A l'inverse, un site dynamique peut éventuellement changer pour une même requête (exemple : stocks d'objets évoluent).

Du point de vue développeur, le site statique est écrit avec un langage de présentation : on écrit exactement ce qui doit être affiché sans aucune logique métier.
Dans un site dynamique, on écrit une logique métier/des règles qui permettent de faire évoluer différentes valeurs affichées.

Site dynamique : PHP, Javascript, Java

\bigskip

\subsection{Citez les 3 versions majeures du protocole HTTP. Expliquez très succinctement la différence entre HTTP et HTTPS.}

\bigskip

HTTP 1.0, HTTP 1.1, HTTP/2

HTTP transmet toutes les informations en clair, tandis que HTTPS chiffre les communications entre le client web et le serveur web.
On peut vérifier l'authenticiter du serveur web avec les certificats fournis avec HTTPS.
S pour Secure.

\bigskip

\subsection{PHP est-il un langage fortement ou faiblement typé ? Expliquez la différence.}

\bigskip

PHP est un langage faiblement typé.

Les langages fortement typés imposent que les types soient rigoureusement les mêmes que ceux utilisés dans les opérations/fonctions.
Une comparaison entre un nombre et une chaîne de carcatères ne peut pas fonctionner tel quel.

Les langages faiblements types ne vérifient pas les types avant d'effectuer des opérations, ils utilisent le plus simple/commun aux valeurs données en paramètres.
Par exemple, comparer une chaîne de caractères et un nombre sera effectué en comparant des chaînes de caractères uniquement.

\bigskip

\subsection{Expliquez comment PHP est utilisé dans le cadre d'un serveur web.}

\bigskip

PHP est un pré-processeur, c'est à dire qu'il va lire un fichier et modifier certaines valeurs, pour générer un fichier de sortie qui sera lu/utilisé par un autre logiciel ou service.

Le serveur web reçoit des requêtes et y répond en renvoyant les ressources demandées.
PHP est pafois appelé par le serveur web pour traiter les ressources avant de les envoyer aux clients.
Le serveur web fournit le nom de la ressource et des variables à PHP, PHP fait les traitements, puis PHP renvoie le résultat au serveur web (qui transmet au client la réponse).

\bigskip

\subsection{Expliquez la différence de fonctionnement au niveau du serveur web entre les méthodes GET et POST.}

\bigskip

Le navigateur, pour envoyer des données au serveur web, peut utiliser 2 méthodes : GET et POST.

GET : l'URL vers la ressource est étendue d'un \TTBF{?} qui sépare les ressources visées des données transmises.
Les variables sont séparées par des \TTBF{\&}. (format général : http://domain.com/ressource?var=42\& i=4).
(On ne devrait pas envoyer plus de 256 octets par cette méthode.)
Le serveur web reçoit donc une URL contenant les données supplémentaires.
Le serveur web enregistre dans les logs/journaux les URL demandées, il ne faut donc jamais envoyer de données sensibles par cette méthode. (les admins du serveur web peuvent lire les données)

POST : Le navigateur envoie les données dans le corps de la réponse, elles ne sont donc pas enregistrées dans les logs/journaux du serveur web.
(Il n'y a pas de limite pour la taille des données envoyées en POST.)

\bigskip

\subsection{Expliquez la différence d'utilisation en PHP des méthodes GET et POST. Indiquez les limites de chaque méthode, s'il y en a.}

\bigskip

En PHP, les variables envoyées avec GET et POST se trouvent dans des tableaux associatifs.
Un formulaire envoie les données selon la méthode indiquée, le serveur web les récupère et les insère dans des tableaux associatifs pour PHP.
Les variables GET et POST sont des superglobales.

\begin{verbatim}
$_GET['variable']
$_POST['variable']
\end{verbatim}

\bigskip

\subsection{Expliquez la différence entre les fonctions \textit{strip\_tags} et \textit{htmlspecialchars}.}

\bigskip

\begin{itemize}
\item htmlspecialchars (chaine, flags) : Conversion des caractères spéciaux en entités HTML. Remplace par exemple \& (ET commercial) en \&amp;
\item strip\_tags (chaine, allowableTags) : Supprime les balises HTML et PHP d'une chaîne. Les commentaires HTML et PHP sont également supprimés
\end{itemize}

\begin{verbatim}
$str = "This is some <b> bold </b> text.";

echo htmlspecialchars($str);
// affichera : This is some &lt;b&gt; bold &lt;/b&gt; text.




$text = ‘
<p>Paragraph.</p><!-- Comment --><a href="#fragment">Other text</a>
';

// Autorise <p> et <a>
echo strip_tags($text, '<p><a>');
// affichera :
// <p>Paragraph.</p> <a href="#fragment">Other text</a>
\end{verbatim}

\bigskip

\subsection{Expliquez les différentes parties des URL suivantes, et ce qu'un serveur web standard comprendra :}

\bigskip

\begin{enumerate}
\item \TTBF{https://univ-paris.fr} : protocole : http (web), domaine : univ-paris.fr, ressource : aucune, donc la racine /
\item \TTBF{http://www2.website.jp:3864/homepages/portal/} : protocole : http (web), domaine : www2.website.jp, port : 3864, ressource : /homepages/portal/
\item \TTBF{http://www.dailynews.co.uk/article/2864} : protocole : http (web), domaine : www.dailynews.co.uk, ressource : /article/2864
\item \TTBF{http://www.monsite.fr/files-storage/data.video/index.php} : protocole : http (web), domaine : www.monsite.fr, ressource : files-storage/data.video/index.php (fichier : index.php)
\item \TTBF{ftp://server01.datastore.bigcorp.com/contents/fabrice/} : protocole : ftp, domaine : server01.datastore.bigcorp.com, ressource : /contents/fabrice/ (dossier)
\item \TTBF{rtmp://live.streamcorp.xyz:1337/flux-video/film/4242.01} : protocole : rtmp (streaming), live.streamcorp.xyz, ressource : /flux-video/film/4242.01
\end{enumerate}

\bigskip

\subsection{Remplir le tableau avec la valeur booléenne de retour que chaque fonction renverrai pour chaque valeur de variable en entrée.}

\bigskip

% Allonge les cases en hauteur
\renewcommand\arraystretch{2.5}

\bigskip
\begin{center}
%  \begin{tabularx}{15.5cm}{| c | p{4cm} | p{4cm} | p{4cm} |}
  \begin{tabularx}{\linewidth}{| *{4}{>{\centering \arraybackslash}X |}}
  \hline
  Paramètre & \TTBF{isset()} & \TTBF{empty()} & \TTBF{is\_null()} \\ \hline
  \TTBF{null} & false & true & true \\ \hline
  [\TTBF{unset(\$var)}] & false & true & error/true \\ \hline
  \TTBF{42} & true & false & false \\ \hline
  \TTBF{0} & true & true & false \\ \hline
  \TTBF{""} & true & true & false \\ \hline
  \TTBF{" "} [un espace] & true & false & false \\ \hline
  \end{tabularx}
\end{center}
\medskip

\renewcommand\arraystretch{1}

\bigskip

\subsection{Remplir le tableau avec les codes HTTP ou leur description.}

\bigskip

% Allonge les cases en hauteur
\renewcommand\arraystretch{2.5}

\bigskip
\begin{center}
%  \begin{tabularx}{15.5cm}{| c | p{4cm} | p{4cm} | p{4cm} |}
  \begin{tabularx}{\linewidth}{| *{2}{>{\centering \arraybackslash}X |}}
  \hline
  \textbf{Code HTTP} & \textbf{Description} \\ \hline
  200 & OK \\ \hline
  302 307 & Temporary Redirect \\ \hline
  301 308 & Permanent Redirect \\ \hline
  403 & Le client n'a pas le droit d'accéder à la ressource \\ \hline
  404 & URL/Ressource demandée n'existe pas \\ \hline
  500 & Internal Server Error \\ \hline
  502 & Bad Gateway \\ \hline
  \end{tabularx}
\end{center}
\medskip

\renewcommand\arraystretch{1}

\bigskip

\subsection{Remplir le tableau avec les familles de codes HTTP et leur description.}

\bigskip

% Allonge les cases en hauteur
\renewcommand\arraystretch{2.5}

\bigskip
\begin{center}
%  \begin{tabularx}{15.5cm}{| c | p{4cm} | p{4cm} | p{4cm} |}
  \begin{tabularx}{\linewidth}{| *{2}{>{\centering \arraybackslash}X |}}
  \hline
  \textbf{Code HTTP} & \textbf{Description} \\ \hline
  100 & Informations sur l'état de la connection \\ \hline
  200 & Réponses positives \\ \hline
  300 & Redirections \\ \hline
  400 & Erreurs côté client/dans la requête envoyée \\ \hline
  500 & Erreurs côté serveur \\ \hline
  \end{tabularx}
\end{center}
\medskip

\renewcommand\arraystretch{1}

\bigskip

\subsection{Expliquez les différents niveaux de visibilité des variables. Donnez un exemple succinct pour chaque niveau.}

\bigskip

\begin{itemize}
\item Variables locales : n'existent que dans le scope local à la fonction
\item Variables globales : elles existent dans l'ensemble des fonctions appelées par le script principal
\item Variables superglobales : elles existent dans l'ensemble de PHP
\end{itemize}

\bigskip

\subsection{Expliquer les différences entre les \textit{sessions} et les \textit{cookies}.}

\bigskip

\begin{itemize}
\item Sessions : variables stockées côté serveur. Le numéro de session est envoyé à l'utilisateur qui le rappelle à chaque requête vers le serveur web. (Très utile pour stocker les gros objets sans avoir à les renvoyer au serveur web à chaque requête.)
\item Cookies : variables stockées côté client. Les variables et valeurs sont renvoyées constament au serveur web. (Utile pour identifier l'utilisateur. Déconseillé pour stocker les gros objets, car il faut les renvoyer à chaque requête)
\end{itemize}

\bigskip

\subsection{Dans quel cas vaut-il mieux utiliser des \textit{sessions} ? Dans quel cas vaut-il mieux utiliser des \textit{cookies} ?}

\bigskip

\begin{itemize}
\item Sessions : Très utile pour stocker les gros objets sans avoir à les renvoyer au serveur web à chaque requête.
\item Cookies : Utile pour identifier l'utilisateur. Déconseillé pour stocker les gros objets, car il faut les renvoyer à chaque requête.
\end{itemize}

\bigskip

\subsection{Expliquez ce qu'est un SGBD par rapport à une BDD. Indiquez le nom du standard permettant d'interroger un SGBD.}

\bigskip

\begin{itemize}
\item BDD/DB : Base De Données/DataBase, l'ensemble des données organisées dans un fichier (ou en mémoire) selon un schéma
\item SGBD/DBMS : Système de Gestion de Base de Données/DataBase Management System, le logiciel/programme permettant d'interroger et modifier des bases de données
\end{itemize}

SQL (Standard Query Language) est le langage standard de requête de base de données.

\bigskip

\subsection{Que signifie l'acronyme ACID ? Quelles sont les 2 étapes qui peuvent alternativement se produire à la fin d'une transaction ? Donnez un exemple de transaction (du point de vue métier, puis du point de vue technique).}

\bigskip

ACID : Atomicité, Cohérence, Isolation, Durabilité

\`A la fin d'une transaction, il se produit soit un COMMIT, soit un ROLLBACK.

Exemple de transaction : un achat en ligne (métier).
L'achat se découpe en plusieurs étapes techniques : sélection des objets selon les stocks disponibles, création d'une commande, retrait des objets des stocks.

\bigskip

\subsection{Qu'est-ce que :}
\begin{enumerate}
\item \TTBF{L'Atomicité} : transaction se déroule intégralement ou pas du tout
\item \TTBF{La Cohérence} : la base de données doit passer d'un état cohérent à un autre état cohérent
\item \TTBF{L'Isolation} : une transaction ne peut pas dépendre d'une autre transaction
\item \TTBF{La Durabilité} : les modifications de la base de données à la fin de chaque transaction sont sauvegardées et doivent résister aux crashs/une reprise dans le dernier état cohérent en cas de crash doit être possible
\end{enumerate}

\bigskip

\subsection{Qu'est-ce qu'un moteur de stockage dans une base de données ? Citer 3 exemples et leurs caractéristiques.}

\bigskip

Le moteur de stockage d'une base de données est le le logiciel/programme/code qui permet d'organiser les données dans la base.
Chaque moteur a des caractéristiques spécifiques (écriture ou lecture rapide, ...).

\begin{itemize}
\item MyISAM : lecture rapide, pas de clé étrangère
\item InnoDB : gère ACID et les clés étrangères
\item Memory : aucune écriture sur disque, uniquement en mémoire... ultra rapide mais ne sauvegarde rien en cas de crash
\end{itemize}

\bigskip

% Faire fonctionner code
\section{Développement}

\subsection{Ce code devrait afficher le contenu du tableau, mais il ne fonctionne pas. Corrigez les erreurs de syntaxe pour que le tableau soit correctement rempli et affiché.}

\medskip

\lstset{language=php}
\begin{lstlisting}[frame=single,numbers=left]
<html>
<body>
<?php

my_tab = array();
my_tab[1] = Fabrice;
my_tab[2] = Ali;
my_tab[3] = Floriane;

foreach ( i  in  my_tab )
{
   echo('i \n');
}

?>
</body>
</html>
\end{lstlisting}


CORRECTION :

\medskip

\lstset{language=php}
\begin{lstlisting}[frame=single]
<html>
<body>
<?php

$my_tab = array();			// <== $
$my_tab[1] = "Fabrice";		// <== $ " "
$my_tab[2] = "Ali";			// <== $ " "
$my_tab[3] = "Floriane";	// <== $ " "

foreach ( $my_tab as $i )	// <== $my_tab as $i
{
   echo("$i \n");			// <== " $ "
}

?>
</body>
</html>
\end{lstlisting}

\bigskip

\subsection{Ce code devrait compter le nombre de lignes contenant la valeur "Floriane", mais il ne fonctionne pas. Corrigez les erreurs pour que le compte soit bon.}

\medskip

\lstset{language=php}
\begin{lstlisting}[frame=single,numbers=left]
<html>
<body>
<?php

// Fait une requete SQL et l'ecrit dans $result
require(functions.php)

$results = my_fun();

for ( $i = 1; $i <= length($results); $i-- )
{
	if ($results[$i] == "Floriane")
		$count++;
}

echo "$count";
?>
</body>
</html>
\end{lstlisting}


CORRECTION :

\medskip

\lstset{language=php}
\begin{lstlisting}[frame=single]
<html>
<body>
<?php

// Contient la fonction "my_fun" qui fait une requete SQL et
//  renvoie son resultat
require("functions.php");	// <== " " ;

$results = my_fun();
$count = 0;					// <== $count = 0;

for ($i = 0; $i < count($results); $i++) // <== 0 < count/sizeof $i++
{
        if ($results[$i] == "Floriane")
                $count++;
}

echo "$count";
?>
</body>
</html>
\end{lstlisting}

\bigskip


% Ecrire un bout de code de 0
\subsection{Cet ancien code écrit avec l'extension obsolète \textit{MySQL} doit être mis à jour. \'Ecrivez la nouvelle version en utilisant PDO ou MySQLi.}

\medskip

\lstset{language=php}
\begin{lstlisting}[frame=single,numbers=left]
<?php
function MyConnection($host, $db)
{
  $dbms = mysql_connect($host, "login", "pass");
  if ($dbms)
  {
      mysql_select_db($db $dbms);
  }
  return ($dbms);
}

function CloseMyConnection($sgbd)
{
  mysql_close($sgbd);
}

function MakeQuery($sql, $db)
{
  $result = FALSE;
  if ($db)
  {
    $result = mysql_query($sql, $db);
  }
  return ($result);
}

function IterResSql($results)
{
  $tab = array();
  $i = 0;
  if ($results)
  {
    while ($row = mysql_fetch_assoc($results))
    {
      $tab[$i] = array();
      $tab[$i]['prenom'] = $row['prenom'];
      $tab[$i]['nom'] = $row['nom'];
      $tab[$i]['adresse'] = $row['addresse'];
      $tab[$i]['age'] = $row['age'];
      $i = $i + 1;
    }
  }
  return ($tab);
}

$co = MyConnection("localhost", "MaBDD");
$requete = "SELECT * FROM clients";
$res = MakeQuery($requete, $co);
$tab = IterResSql($res);
CloseMyConnection($co);
?>
\end{lstlisting}

\bigskip

\end{document}
