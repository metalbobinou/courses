\documentclass[12pt,a4paper]{article}

\usepackage[french]{babel}
\usepackage[utf8]{inputenc}
\usepackage[T1]{fontenc}

\usepackage{amsmath}
\usepackage{amsfonts}
\usepackage{amssymb}

% Preparation des infos generales
\newcommand{\TitreMatiere}{Développement Web - PHP}
\newcommand{\DateExo}{03 juillet 2020}
\newcommand{\TypeExo}{DM}
\newcommand{\TitreExercice}{PHP - Rattrapages}
\newcommand{\NomAuteurA}{Fabrice BOISSIER}
\newcommand{\MailAuteurA}{fabrice.boissier@univ-paris1.fr}
%\newcommand{\NomAuteurB}{}
%\newcommand{\MailAuteurB}{}
\newcommand{\VersionExo}{Version 1}
\newcommand{\LoginEtudiant}{2019-2020} % Watermark de protection

\newcommand{\RenduTarball}{nom-Rattrapages.zip}
\newcommand{\RenduDir}{nom-Rattrapages}


% Ajout de mes classes & definitions
\usepackage{MetalExo} % Appelle un .sty


% Redefinition headers
\lhead{\TitreExercice}		%Gauche Haut
\chead{\TypeExo}			%Centre Haut
\rhead{\thepage}			%Droite Haut
\lfoot{}					%Gauche Bas
\cfoot{\TitreMatiere}		%Centre Bas
\rfoot{}					%Droite Bas


\definecolor{mygreen}{rgb}{0,0.6,0}		% RGB model
\definecolor{mygray}{rgb}{0.5,0.5,0.5}
\definecolor{mymauve}{rgb}{0.58,0,0.82}


\begin{document}

%% Titre
\maketitle

%\newpage

%% Copyright
\pagenumbering{Roman}

%% Copyright

{\Large \textbf{Copyright}}

\vspace{30px}

Ce document est destiné à l'usage interne de Paris 1 - Panthéon Sorbonne.\\

Copyright \space \copyright \space Fabrice BOISSIER - 2019\\

\bigskip

\begin{center}
	\fcolorbox{black}{white}{\makebox[14cm]{
	\begin{minipage}[l]{12cm}
		\vspace*{10px}
		\textbf{Ce document est soumis à conditions :}
		\medskip

		Il est interdit de partager ce document avec d'autres personnes.
		\smallskip

		Vérifiez que vous disposez de la dernière révision de ce document.
	\vspace*{10px}
	\end{minipage}
	}}
\end{center}


\newpage

%% Table des matieres
\tableofcontents

\newpage

%% Consignes generales
\section{Consignes Générales}

\bigskip

%% Consignes generales
\noindent \textit{Les informations suivantes sont très importantes :}

\bigskip

\noindent \textit{Le non-respect d'une des consignes suivantes entraînera des sanctions pouvant aller jusqu'à la multiplication de la note finale par 0.}

\bigskip

\noindent \textit{Ces consignes sont claires, non-ambiguës, et ont un objectif précis. En outre, elles ne sont pas négociables.}

\bigskip

\noindent N'hésitez pas à demander si vous ne comprenez pas une des règles.

\bigskip

\newcounter{My_Counter}

\ConsigneGenerale{My_Counter}{Vous devez lire le sujet.}
\ConsigneGenerale{My_Counter}{Vous devez respecter les consignes.}
\ConsigneGenerale{My_Counter}{Vous devez rendre le travail dans les délais prévus.}

\medskip

\ConsigneGenerale{My_Counter}{Le travail doit être rendu dans le format décrit à la section \hyperref[sec:FormatDeRendu]{Format de Rendu}.}

\ConsigneGenerale{My_Counter}{Le travail rendu ne doit pas contenir de fichiers binaires, temporaires, ou d'erreurs (\texttt{\textbf{*\textasciitilde}}, \texttt{\textbf{*.o}}, \texttt{\textbf{*.a}}, \texttt{\textbf{*.so}}, \texttt{\textbf{*\#*}}, \texttt{\textbf{*core}}, \texttt{\textbf{*.log}}, \texttt{\textbf{*.exe}}, binaires, ...).}

\ConsigneGenerale{My_Counter}{Dans l'ensemble de ce document, la casse (caractères majuscules et minuscules) est très importante. Vous devez strictement respecter les majuscules et minuscules imposées dans les messages et noms de fichiers du sujet.}

%\ConsigneGenerale{My_Counter}{Dans l'ensemble de ce document, \LoginX \space correspond à votre login.}
\ConsigneGenerale{My_Counter}{Dans l'ensemble de ce document, \TTBF{nom1-nom2} \space correspond à la combinaison des deux noms de votre binôme (par exemple pour Fabrice BOISSIER et Mark ANGOUSTURES, cela donnera \TTBF{boissier-angoustures}).}

\ConsigneGenerale{My_Counter}{Dans l'ensemble de ce document, le caractère \TTBF{\textvisiblespace } correspond à une espace (s'il vous est demandé d'afficher \TTBF{\textvisiblespace \textvisiblespace \textvisiblespace }, vous devez afficher trois espaces consécutives).}

\ConsigneGenerale{My_Counter}{Tout retard, même d'une seconde, entraîne la note non négociable de 0.}

\ConsigneGenerale{My_Counter}{La triche (échange de code, copie de code ou de texte, ...) entraîne \textbf{au mieux} la note non négociable de 0.}

\ConsigneGenerale{My_Counter}{En cas de problème avec le projet, vous devez contacter le plus tôt possible les responsables du sujet aux adresses mail indiquées.}

\bigskip

\noindent \textbf{Conseil :} N'attendez pas la dernière minute pour commencer à travailler sur le sujet.


\newpage

%% Format de Rendu
\section{Format de Rendu}
\label{sec:FormatDeRendu}

\vspace*{1cm}

%% Format de Rendu

%\ResponsablesProjet{Metal Man/metalman@example.org, Damdoshi/damdoshi@example.org, Tayst/tayst@tayst.org}
%\begin{tabular}{p{7cm} p{10cm}}
\begin{tabular}{p{7cm} p{8.5cm}}
	%\ResponsablesProjetRow{Fabrice BOISSIER/fabrice.boissier@univ-paris1.fr, Ali JAFFAL/ali.jaffal@univ-paris1.fr}
	\ResponsablesProjetRow{Fabrice BOISSIER/fabrice.boissier@epita.fr}
	& \\
%	\RenduSpecsGenerales{[PHP][DM]}{2}{Envoi par mail}{\RenduDir}{\RenduTarball}{10/02/2020 23h42}{2 semaines}
%	\RenduSpecsGenerales{[CAV][TP1]}{1}{Pas de rendu}{\RenduDir}{\RenduTarball}{Pas de rendu}{Pas de rendu}
	\RenduSpecsGenerales{[C][STR]}{1}{Devoir/Assignment sur Teams}{\RenduDir}{\RenduTarball}{18/12/2022 23h42}{2 semaines}
	& \\
%	\RenduSpecsTechniques{WAMP ou MAMP}{PHP}{Apache/PHP}{ }
%	\RenduSpecsTechniques{Linux - Ubuntu (x86\_64)}{C}{/usr/bin/gcc}{-W -Wall -Werror -std=c99 -pedantic}
	\RenduSpecsTechniques{Linux}{C}{gcc}{-W -Wall -Werror -std=c99 -pedantic}
%	& \\
%	Fonctions autorisées : & malloc(3), free(3), printf(3)
\end{tabular}


\vspace*{1cm}


\noindent Les fichiers suivants sont requis :

\medskip

\begin{tabular}{l p{12cm}}
\texttt{AUTHORS} & contient le(s) nom(s) et prénom(s) de(s) auteur(s).\\
%\texttt{Makefile} & le Makefile principal.\\
\texttt{README} & contient la description du projet et des exercices, ainsi que la fa\c con d'utiliser le projet.\\
%\texttt{configure} & le script shell de configuration pour l'environnement de compilation.\\
\end{tabular}


\vspace*{1cm}


%\noindent Un fichier \TTBF{Makefile} doit être présent à la racine du dossier, et doit obligatoirement proposer ces règles :

\medskip

%\begin{tabular}{l p{13cm}}
%\texttt{all} & \textit{[Première règle]} lance la règle \texttt{libmyqueue}.\\
%\texttt{clean} & supprime tous les fichiers temporaires et ceux créés par le compilateur.\\
%\texttt{dist} & crée une archive propre, valide, et répondant aux exigences de rendu.\\
%\texttt{distclean} & lance la règle \texttt{clean}, puis supprime les binaires et bibliothèques.\\
%\texttt{check} & lance le(s) script(s) de test.\\
%\texttt{libmyqueue} & lance les règles \texttt{shared} et \texttt{static} \\
%\texttt{shared} & compile l'ensemble du projet avec les options de compilations exigées et génère une bibliothèque dynamique.\\
%\texttt{static} & compile l'ensemble du projet avec les options de compilations exigées et génère une bibliothèque statique.\\
%\end{tabular}


%\vspace*{1cm}
%\newpage

\noindent Votre code sera testé automatiquement, vous devez donc scrupuleusement respecter les spécifications pour pouvoir obtenir des points en validant les exercices.
%
Votre code sera testé en l'intégrant à une série de tests automatisés qui seront fournis un peu plus tard.
N'attendez SURTOUT PAS que ces tests soient envoyés pour commencer à produire vos fonctions et vos propres tests.
%
%Votre code sera testé en générant un exécutable ou des bibliothèques avec les commandes suivantes :

%\medskip
%
%\begin{tabular}{l}
%\texttt{./configure}\\
%\texttt{make}\\
%\end{tabular}
%
%\bigskip
%
%\noindent Suite à cette étape de génération, les exécutables ou bibliothèques doivent être placés à ces endroits :
%
%\medskip
%
%\begin{tabular}{l}
%\TTBF{\RenduDir/libmyqueue.a}\\
%\TTBF{\RenduDir/libmyqueue.so}\\
%\end{tabular}

\bigskip

\noindent L'arborescence attendue pour le projet est la suivante :

\medskip

\begin{tabular}{l}
\TTBF{\RenduDir/}\\
\TTBF{\RenduDir/AUTHORS}\\
\TTBF{\RenduDir/README}\\
%\TTBF{\RenduDir/Makefile}\\
%\TTBF{\RenduDir/configure}\\
%\TTBF{\RenduDir/check/}\\
%\TTBF{\RenduDir/check/check.sh}\\
\TTBF{\RenduDir/src/}\\
\TTBF{\RenduDir/src/StrBasics.c}\\
\TTBF{\RenduDir/src/StrBasics.h}\\
\TTBF{\RenduDir/src/StrPart1.c}\\
\TTBF{\RenduDir/src/StrPart1.h}\\
\TTBF{\RenduDir/src/StrPart2.c}\\
\TTBF{\RenduDir/src/StrPart2.h}\\
%\TTBF{\RenduDir/src/StrBonus.c}\\
%\TTBF{\RenduDir/src/StrBonus.h}\\
\end{tabular}


\vspace*{1cm}


%\noindent \textit{Vous ne serez jamais pénalisés pour la présence de makefiles ou de fichiers sources (code et/ou headers) dans les différents dossiers du projet tant que leur existence peut être justifiée (des makefiles vides ou jamais utilisés sont pénalisés).}

%\noindent \textit{Vous ne serez jamais pénalisés pour la présence de fichiers de différentes natures dans le dossier \texttt{check} tant que leur existence peut être justifiée (des fichiers de test jamais utilisés sont pénalisés).}


%%%%%%%%%%%%%%%%%%%%%%%%%%%%
%% AIDE MEMOIRE AU CAS OU %%
%%%%%%%%%%%%%%%%%%%%%%%%%%%%
\newpage

\section{Aide Mémoire}
\label{sec:AideMemoire}

\vspace*{1cm}

%%%%%%%%%%%%%%%%%%%%%%%%%%%%
%% AIDE MEMOIRE AU CAS OU %%
%%%%%%%%%%%%%%%%%%%%%%%%%%%%
%\newpage

%{\Large \textbf{Aide Mémoire}}

%\vspace{30px}

%\noindent Le travail doit être rendu au format \textbf{\textit{.zip}}, c'est-à-dire une archive \textbf{zip} compressée avec un outil adapté (les logiciels \textit{7zip} ou \textit{Keka} sont gratuits et adaptés).
\noindent Le travail doit être rendu au format \textbf{\textit{.tar.bz2}}, c'est-à-dire une archive \textbf{bz2} compressée avec un outil adapté (voir \TTBF{man 1 tar} et \TTBF{man 1 bz2}).

%\noindent Tout autre format d'archive (rar, 7zip, gz, gzip, bzip, ...) ne sera pas pris en compte, et votre travail ne sera pas corrigé (entraînant la note de 0).
\noindent Tout autre format d'archive (zip, rar, 7zip, gz, gzip, ...) ne sera pas pris en compte, et votre travail ne sera pas corrigé (entraînant la note de 0).

\bigskip

\noindent Pour générer une archive \textit{tar} en y mettant les dossiers \textit{folder1} et \textit{folder2}, vous devez taper :

\TTBF{tar cvf MyTarball.tar folder1 folder2}


\bigskip


\noindent Pour générer une archive \textit{tar} et la compresser avec GZip, vous devez taper :

\TTBF{tar cvzf MyTarball.tar.gz folder1 folder2}


\bigskip


\noindent Pour générer une archive \textit{tar} et la compresser avec BZip2, vous devez taper :

\TTBF{tar cvjf MyTarball.tar.bz2 folder1 folder2}


\bigskip


\noindent Pour lister le contenu d'une archive \textit{tar}, vous devez taper :

\TTBF{tar tf MyTarball.tar.bz2}


\bigskip


\noindent Pour extraire le contenu d'une archive \textit{tar}, vous devez taper :

\TTBF{tar xvf MyTarball.tar.bz2}


\vspace*{1cm}

%\noindent Dans ce sujet précis, vous ferez du code en C et des appels à des scripts shell qui afficheront les résultats dans le terminal (donc des flux de sortie qui pourront être redirigés vers un fichier texte).

%\noindent Dans ce sujet précis, vous ferez du code en script shell, qui affichera les résultats dans le terminal (donc des flux de sortie qui pourront être redirigés vers un fichier texte).

%\noindent Dans ce sujet précis, vous ferez du code en PHP, qui affichera les résultats dans une page HTML. Les valeurs seront affichées dans une \textit{textarea} dont le texte est généré par des outils multiplateformes supportant les retours à la ligne UNIX (\textbf{\textbackslash n}). Il ne faut donc pas inclure de balise \TTBF{"<br />"} pour retourner à la ligne, mais un \TTBF{"\textbackslash n"}.

\noindent Dans ce sujet précis, vous ferez du code en Python, qui affichera les résultats dans le terminal (donc des flux de sortie qui pourront être redirigés vers un fichier texte).

%\vspace*{1cm}

%\noindent Pour réaliser le travail demandé, nous vous fournirons pour chaque exercice au moins 2 fichiers : \TTBF{exoN\_res.php} (le fichier qui sera appelé pour voir le résultat de votre travail), et \TTBF{exoN\_fun.php} (le fichier contenant la fonction que vous devez coder dans chaque exercice).
%Optionnellement, un fichier \TTBF{exoN\_data.php} peut être fourni pour indiquer le format de données en entrée.
%Le \TTBF{N} correspond au numéro de l'exercice.

%\medskip

%\noindent Dans tous les cas, vous ne devez rendre que le fichier \TTBF{exoN\_fun.php} avec au moins la fonction demandée remplie (qui peut faire appel à d'autres fonctions que vous définirez dans le \textbf{même} fichier). Les autres fichiers seront générés par nos soins pour tester vos fonctions.

%\vspace*{1cm}

%\noindent Vous ne devez \textbf{PAS} utiliser la fonction \TTBF{echo} pour écrire !
%Il faut retourner une chaîne de caractères correctement formattée.



%%%%%%%%%%%%%%%
%% EXERCICES %%
%%%%%%%%%%%%%%%
\newpage

\pagenumbering{arabic}

%% Introduction
\section{Mini Projet}

\vspace*{0.7cm}

%% Introduction / Mini Projet

%\ExoSpecs{\TTBF{CalculTVA.sh}}{\TTBF{\RenduDir/src/exo1/}}{750}{640}{\TTBF{write}}
%\ExoSpecsCustom{\TTBF{BDD.sql}}{\TTBF{\RenduDir/BDD.sql}}{}{}{}

\vspace*{0.7cm}

\noindent \ExoObjectif{Le but de l'exercice est de faire une maquette minimaliste d'un site de vente en ligne.}

\bigskip

\noindent Ce mini projet, à réaliser \textbf{seul(e)}, vous amènera à développer une interface minimaliste pour gérer des produits d'un magasin en ligne.
Les notions abordées dedans impliquent de coder en PHP, de gérer des formulaires (avec GET et POST), et de créer des tables SQL sous PhpMyAdmin avec des clés étrangères pour faire une une jointure dans des requêtes SQL.

\bigskip

\noindent La qualité graphique du site n'est absolument pas évaluée !
Seule la \textit{logique métier} est testée (néanmoins, un site inutilisable à cause de l'interface graphique inexistante ou inutilisable vaut 0).
Vous devrez donc uniquement utiliser quelques balises HTML telles que :

\medskip

\begin{itemize}
\item <html>, <header>, <title>, <body>
\item <center>, <h1>, <h2>, <h3>, <b>, <i>, <p>, <br>
\item <a href=...>
\item <table>, <tr>, <th>
\item <form action=... method=...>, <input type=... name=...>
\end{itemize}

\bigskip

\noindent Ce projet n'est absolument pas complexe si vous le réalisez pas à pas, mais il nécessite plusieurs jours de travail.

\medskip

\noindent Vous devez suivre l'ordre des paliers pour obtenir les points : tant que le palier 1 n'est pas fonctionnel, il est inutile de tenter le palier 2, car il s'appuiera inévitablement sur le palier précédent (de plus, chaque palier peut vous aider à débugger les suivants, et beaucoup de code peut être dupliqué de page en page).
Se concentrer exclusivement sur les paliers 5 et/ou 6 ne vous octroiera aucun point : vous \textbf{devez} réaliser les paliers précédents.

\medskip

\noindent N'oubliez pas de tester les valeurs de retour du connecteur SQL et surtout d'afficher les messages d'erreur lorsque vous développez !
Ceux-ci indiquent explicitement ce qui ne fonctionne pas.


\newpage

%% Pré-Requis
\section{Palier 0 - Base de Données (Pré-Requis)}

\vspace*{0.7cm}

%% Exercice 1

\ExoSpecs{\TTBF{BDD.sql.gz}}{\TTBF{\RenduDir/}}{}{}{}
%\ExoSpecsCustom{\TTBF{BDD.sql}}{\TTBF{\RenduDir/BDD.sql}}{}{}{}

\vspace*{0.7cm}

\noindent \ExoObjectif{Le but du palier est de faire une base de données sur PhpMyAdmin pour la suite du mini projet.}

\bigskip

\noindent Vous devez créer une base de données simple qui contiendra des produits, des numéros de commande, et des liens entre les commandes et les produits.
Vous devrez donc créer 3 tables dans MySQL en utilisant PhpMyAdmin, et vous devrez fournir l'extraction de la table et des données de test que vous insèrerez.
Le fichier contenant l'export doit être nommé : \TTBF{BDD.sql.gz}.

\noindent \textit{\'Etant donné que les paliers suivants impliquent d'ajouter d'autres tables, vous enverrez l'export de la base de données avec l'ensemble des tables requises pour le fonctionnement du site (et pas uniquement les 3 premières tables décrites ici, sauf si vous n'avez pas atteint les paliers en question)}

\noindent Les 3 tables à construire pour ce palier sont : \textit{articles}, \textit{commandes}, \textit{paniers}.

\noindent La table \textit{articles} contiendra les produits disponibles dans le magasin.
Chaque article dispose d'un ID unique, d'un nom, d'une quantité en stock, et d'un prix unitaire de vente.

\medskip

\begin{figure}[h!]
\begin{center}
\begin{tabular}{| c | c | c | c |}
\hline
\textbf{IDArticle} & \textbf{NomArticle} & \textbf{QuantiteStock} & \textbf{PrixUnitaire} \\
\hline
1 & Pomme & 42 & 1 \\
\hline
2 & Carotte & 36 & 1 \\
\hline
3 & Chocolat & 21 & 5 \\
\hline
... & ... & ... & ... \\
\hline
\end{tabular}
\end{center}
\caption{Table : articles}
\end{figure}

\bigskip

\noindent La table \textit{commandes} contiendra les commandes passées.
Chaque commande dispose d'un ID unique, ainsi que du prix total de vente.

\medskip

\begin{figure}[h!]
\begin{center}
\begin{tabular}{| c | c |}
\hline
\textbf{IDCommande} & \textbf{PrixTotal} \\
\hline
1 & 7 \\
\hline
2 & 1 \\
\hline
3 & 38 \\
\hline
... & ... \\
\hline
\end{tabular}
\end{center}
\caption{Table : commandes}
\end{figure}

\bigskip
%\newpage

\noindent La table \textit{paniers} contiendra l'ensemble des produits achetés lors des commandes.
Cette table fait la jointure entre la table contenant les articles et les commandes : c'est ici que chaque article du panier d'un utilisateur est lié à une commande.
Chaque ligne dans cette table correspond donc à un article dans un panier, et elle contient logiquement : l'ID d'une commande, l'ID d'un article, la quantité de cet article dans le panier.

\medskip

\begin{figure}[h!]
\begin{center}
\begin{tabular}{| c | c | c |}
\hline
\textbf{IDCommande} & \textbf{IDArticle} & \textbf{QuantiteCommande} \\
\hline
1 & 1 \textit{(pomme)} & 2 \\
\hline
1 & 3 \textit{(chocolat)} & 1 \\
\hline
2 & 2 \textit{(carotte)} & 1 \\
\hline
... & ... & ... \\
\hline
\end{tabular}
\end{center}
\caption{Table : paniers}
\end{figure}


\newpage

%% Exercice 1
\section{Palier 1 - Présentation des Produits (2 points)}

\vspace*{0.7cm}

%% Exercice 1

\ExoSpecs{\TTBF{index.php}}{\TTBF{\RenduDir/MiniProjet/}}{}{}{}
%\ExoSpecsCustom{\TTBF{exo1\_fun.php} [my\_Calculette(int, int, string)]}{\TTBF{\RenduDir/src/exo1/}}{}{}{Fonctions recommandées}{\TTBF{(Bases PHP)}, \TTBF{(Maths PHP)}, \TTBF{return}}

\vspace*{0.7cm}

\noindent \ExoObjectif{Le but du palier est d'afficher les produits en vente.}

\bigskip

\noindent Vous devez écrire un fichier \TTBF{index.php} permettant de lire le contenu de la table \textbf{articles} et d'en afficher le contenu.
Chaque produit contenu dans la base de données doit être indiqué avec son nom, son prix, et les quantités restantes.

\medskip

\noindent Vous aurez besoin de faire une requête SQL contenant un \TTBF{SELECT}.


\newpage

%% Exercice 2
\section{Palier 2 - Modification des Produits (3 points)}

\vspace*{0.7cm}

%% Exercice 2

\ExoSpecs{\TTBF{admin\_simple.php}}{\TTBF{\RenduDir/MiniProjet/}}{}{}{}
%\ExoSpecsCustom{\TTBF{exo1\_fun.php} [my\_Calculette(int, int, string)]}{\TTBF{\RenduDir/src/exo1/}}{}{}{Fonctions recommandées}{\TTBF{(Bases PHP)}, \TTBF{(Maths PHP)}, \TTBF{return}}

\vspace*{0.7cm}

\noindent \ExoObjectif{Le but du palier est de créer une micro interface de gestion.}

\bigskip

%\noindent Vous devez créer une page \TTBF{admin\_simple.php}, protégée par un mot de passe (vous mettrez comme mot de passe : \TTBF{password1234}), dans laquelle un utilisateur pourra modifier les propriétés de chaque produit existant, et éventuellement le supprimer.

\noindent Vous devez créer une page \TTBF{admin\_simple.php} dans laquelle un utilisateur pourra modifier les propriétés de chaque produit existant, et éventuellement le supprimer.

\medskip

\noindent L'utilisateur disposera également d'un formulaire lui permettant d'ajouter un produit.

%\medskip
%
%\noindent La "sécurité" de la page n'a pas besoin d'être forte.
%Une simple valeur en URL récupérée avec GET par un formulaire est suffisante pour l'exercice (à ne surtout pas reproduire dans la vie réelle).

\medskip

\noindent Vous aurez besoin de faire des requêtes SQL contenant un \TTBF{SELECT}, un \TTBF{UPDATE}, un \TTBF{DELETE}, ou un \TTBF{INSERT INTO}.


\newpage

%% Exercice 3
\section{Palier 3 - Commande (3 points)}

\vspace*{0.7cm}

%% Exercice 3

\ExoSpecs{\TTBF{commande.php}}{\TTBF{\RenduDir/MiniProjet/}}{}{}{}
%\ExoSpecsCustom{\TTBF{exo1\_fun.php} [my\_Calculette(int, int, string)]}{\TTBF{\RenduDir/src/exo1/}}{}{}{Fonctions recommandées}{\TTBF{(Bases PHP)}, \TTBF{(Maths PHP)}, \TTBF{return}}

\vspace*{0.7cm}

\noindent \ExoObjectif{Le but du palier est d'afficher les commandes précédentes.}

\bigskip

\noindent Vous devez créer une page \TTBF{commande.php} qui affichera les commandes précédentes et leur contenu.
Il est fortement conseillé d'afficher un tableau HTML par commande, et de les espacer de quelques lignes.

\medskip

\noindent Pour simplifier le développement (et pour débugger), n'hésitez pas à d'abord remplir votre base de données de quelques lignes dans la table \textbf{commandes}, et quelques lignes dans la table \textbf{paniers}.
\textit{(Dans l'exemple donné dans le palier 0, la commande 1 contient 2 pommes et 1 chocolat, tandis que la commande 2 contient 1 carotte)}

\medskip

\noindent Vous aurez besoin de faire une requête SQL contenant un \TTBF{SELECT} et un \TTBF{JOIN} (ou équivalent).


\newpage

%% Exercice 4
\section{Palier 4 - Panier (5 points)}

\vspace*{0.7cm}

%% Exercice 4

\ExoSpecs{\TTBF{panier.php}}{\TTBF{\RenduDir/MiniProjet/}}{}{}{}
%\ExoSpecsCustom{\TTBF{exo1\_fun.php} [my\_Calculette(int, int, string)]}{\TTBF{\RenduDir/src/exo1/}}{}{}{Fonctions recommandées}{\TTBF{(Bases PHP)}, \TTBF{(Maths PHP)}, \TTBF{return}}

\vspace*{0.7cm}

\noindent \ExoObjectif{Le but du palier est de créer un panier, d'y ajouter des produits, et enregistrer les commandes.}

\bigskip

\noindent Vous devez tout d'abord créer une page \TTBF{panier.php} dans laquelle le contenu de la commande courante est affiché (en utilisant l'ID de la commande en cours).
Chaque article ajouté dans le panier est enregistré dans la table \textit{paniers}.

\medskip

\noindent L'ID de la commande en cours s'obtiendra de 2 façons différentes selon le cas :
\begin{enumerate}
\item Si la table \textbf{commandes} est vide, alors la page \textbf{panier.php} doit créer une ligne dedans avec l'ID \TTBF{1}
\item Si la table n'est pas vide, alors on utilise le dernier ID existant (l'ID le plus grand) de la table \textbf{commandes} pour créer une nouvelle ligne
\end{enumerate}

\medskip

\noindent Un bouton permettra à l'utilisateur de supprimer un article de son panier.

\medskip

\noindent Cette page contiendra également un bouton \textbf{commander} qui permettra de passer une commande.
C'est-à-dire, dans le cadre de ce mini-projet et de ce palier, de fixer le montant total de la commande en cours, puis de créer une nouvelle ligne vide dans la table \textbf{commandes} avec un nouvel ID.

\medskip

\noindent N'oubliez pas d'ajouter des boutons sur la page \textbf{index.php} pour pouvoir ajouter des articles au panier.

\medskip

\noindent Vous aurez besoin de faire des requêtes SQL contenant un \TTBF{SELECT}, un \TTBF{UPDATE}, et un \TTBF{INSERT INTO}.


\newpage

%% Exercice 5
\section{Palier 5 - Utilisateurs (4 points)}

\vspace*{0.7cm}

%% Exercice 5

\ExoSpecs{\TTBF{login.php profil.php}}{\TTBF{\RenduDir/MiniProjet/}}{}{}{}
%\ExoSpecsCustom{\TTBF{exo1\_fun.php} [my\_Calculette(int, int, string)]}{\TTBF{\RenduDir/src/exo1/}}{}{}{Fonctions recommandées}{\TTBF{(Bases PHP)}, \TTBF{(Maths PHP)}, \TTBF{return}}

\vspace*{0.7cm}

\noindent \ExoObjectif{Le but du palier est de créer une gestion des sessions utilisateurs.}

\bigskip

\noindent Vous devez tout d'abord créer une page \TTBF{login.php} dans laquelle un utilisateur peut créer un compte, s'identifier, ou se délogger.
Sur cette page, plusieurs boutons permettront de gérer chaque cas (authentification, fin de session, ou création de compte).
Vous devez créer une table \TTBF{utilisateurs} dans la base de données qui contient au moins un nom d'utilisateur et un mot de passe, mais vous êtes libres d'ajouter d'éventuelles colonnes supplémentaires.

\medskip

\noindent Ensuite, vous devrez créer une page \TTBF{profil.php} dans laquelle l'utilisateur peut modifier son mot de passe et éventuellement les autres informations.

\medskip

\noindent La gestion des utilisateurs authentifiés implique que les commandes soient gérées séparément pour chacun.
À vous d'ajouter les colonnes nécessaires dans les autres tables de la base de données, ainsi que le code nécessaire dans les autres pages PHP (\textit{index.php}, \textit{commande.php}, \textit{panier.php} principalement) pour pouvoir gérer les sessions utilisateur.


\newpage

%% Exercice 5
\section{Palier 6 - Administrateurs (3 points)}

\vspace*{0.7cm}

%% Exercice 2

\ExoSpecs{\TTBF{admin.php}}{\TTBF{\RenduDir/MiniProjet/}}{}{}{}
%\ExoSpecsCustom{\TTBF{exo1\_fun.php} [my\_Calculette(int, int, string)]}{\TTBF{\RenduDir/src/exo1/}}{}{}{Fonctions recommandées}{\TTBF{(Bases PHP)}, \TTBF{(Maths PHP)}, \TTBF{return}}

\vspace*{0.7cm}

\noindent \ExoObjectif{Le but du palier est de créer une interface complète de gestion d'un site de vente.}

\bigskip

\noindent Vous devez créer une page \TTBF{admin.php} qui permet de suivre les commandes et administrer l'ensemble du site.
Seuls les utilisateurs considérés comme des \textit{administrateurs} pourront y accéder (à vous d'ajouter une colonne dans la base de données pour le faire).

\medskip

\noindent Les commandes disposeront maintenant d'un état : elles seront soit en mode \textit{panier} (non validée par l'utilisateur), soit en mode \textit{validée} (l'utilisateur a validé la commande), soit en mode \textit{expédiée} (un administrateur a confirmé sur la page \textit{admin.php} qu'une commande \textit{validée} était expédiée).
À vous d'ajouter une colonne dans la base de données pour le faire.

\medskip

\noindent Les fonctionnalités devront permettre de :

\begin{enumerate}
\item lister les utilisateurs inscrits sur le site
\item ajouter/retirer le droit administrateur aux utilisateurs
\item lister les articles et modifier leurs informations (nom, quantité, prix)
\item lister les commandes validées
\item faire passer une commande de l'état \textit{validée} a l'état \textit{expédiée} avec un bouton
\item lister les commandes expédiées
\end{enumerate}


\end{document}
