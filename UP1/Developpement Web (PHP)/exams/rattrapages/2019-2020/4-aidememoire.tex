%%%%%%%%%%%%%%%%%%%%%%%%%%%%
%% AIDE MEMOIRE AU CAS OU %%
%%%%%%%%%%%%%%%%%%%%%%%%%%%%
%\newpage

%{\Large \textbf{Aide Mémoire}}

%\vspace{30px}

\noindent Le travail doit être rendu au format \textbf{\textit{.zip}}, c'est-à-dire une archive \textbf{zip} compressée avec un outil adapté (les logiciels \textit{7zip} ou \textit{Keka} sont gratuits et adaptés).

\noindent Tout autre format d'archive (rar, 7zip, gz, gzip, bzip, ...) ne sera pas pris en compte, et votre travail ne sera pas corrigé (entraînant la note de 0).

\vspace*{1cm}

\noindent Dans ce sujet précis, vous aurez à construire des tables sur MySQL avec PhpMyAdmin.
Vous devrez donc exporter vos tables depuis PhpMyAdmin et joindre le fichier.
Pour exporter une base de données depuis PhpMyAdmin : cliquez sur le nom de la base de données dans le menu de gauche, puis dans le menu en haut à droite, cliquez sur \textbf{Export}.
Sélectionnez \textbf{Custom} pour personnaliser l’export.
Choisissez le format \textbf{SQL}.
%Dans la partie \textbf{Output}, cliquez sur \textit{Rename exported databases/tables/columns}, et indiquez le nom de la nouvelle base de données donnée dans le mail.
Puis cliquez sur \textbf{Save output to a file}, et sélectionnez dans \textbf{Compression} : \textit{gzipped}.
Puis cliquez tout en bas sur \textbf{Go}.
