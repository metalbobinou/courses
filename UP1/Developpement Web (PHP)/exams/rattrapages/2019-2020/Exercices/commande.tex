%% Exercice 3

\ExoSpecs{\TTBF{commande.php}}{\TTBF{\RenduDir/MiniProjet/}}{}{}{}
%\ExoSpecsCustom{\TTBF{exo1\_fun.php} [my\_Calculette(int, int, string)]}{\TTBF{\RenduDir/src/exo1/}}{}{}{Fonctions recommandées}{\TTBF{(Bases PHP)}, \TTBF{(Maths PHP)}, \TTBF{return}}

\vspace*{0.7cm}

\noindent \ExoObjectif{Le but du palier est d'afficher les commandes précédentes.}

\bigskip

\noindent Vous devez créer une page \TTBF{commande.php} qui affichera les commandes précédentes et leur contenu.
Il est fortement conseillé d'afficher un tableau HTML par commande, et de les espacer de quelques lignes.

\medskip

\noindent Pour simplifier le développement (et pour débugger), n'hésitez pas à d'abord remplir votre base de données de quelques lignes dans la table \textbf{commandes}, et quelques lignes dans la table \textbf{paniers}.
\textit{(Dans l'exemple donné dans le palier 0, la commande 1 contient 2 pommes et 1 chocolat, tandis que la commande 2 contient 1 carotte)}

\medskip

\noindent Vous aurez besoin de faire une requête SQL contenant un \TTBF{SELECT} et un \TTBF{JOIN} (ou équivalent).
