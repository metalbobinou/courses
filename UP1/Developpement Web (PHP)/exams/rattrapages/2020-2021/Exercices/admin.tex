%% Exercice 2

\ExoSpecs{\TTBF{admin.php}}{\TTBF{\RenduDir/MiniProjet/}}{}{}{}
%\ExoSpecsCustom{\TTBF{exo1\_fun.php} [my\_Calculette(int, int, string)]}{\TTBF{\RenduDir/src/exo1/}}{}{}{Fonctions recommandées}{\TTBF{(Bases PHP)}, \TTBF{(Maths PHP)}, \TTBF{return}}

\vspace*{0.7cm}

\noindent \ExoObjectif{Le but du palier est de créer une interface complète de gestion d'un site de vente.}

\bigskip

\noindent Vous devez créer une page \TTBF{admin.php} qui permet de suivre les commandes et administrer l'ensemble du site.
Seuls les utilisateurs considérés comme des \textit{administrateurs} pourront y accéder (à vous d'ajouter une colonne dans la base de données pour le faire).

\medskip

\noindent Les commandes disposeront maintenant d'un état : elles seront soit en mode \textit{panier} (non validée par l'utilisateur), soit en mode \textit{validée} (l'utilisateur a validé la commande), soit en mode \textit{expédiée} (un administrateur a confirmé sur la page \textit{admin.php} qu'une commande \textit{validée} était expédiée).
À vous d'ajouter une colonne dans la base de données pour le faire.

\medskip

\noindent Les fonctionnalités devront permettre de :

\begin{enumerate}
\item lister les utilisateurs inscrits sur le site
\item ajouter/retirer le droit administrateur aux utilisateurs
\item lister les articles et modifier leurs informations (nom, quantité, prix)
\item lister les commandes validées
\item faire passer une commande de l'état \textit{validée} a l'état \textit{expédiée} avec un bouton
\item lister les commandes expédiées
\end{enumerate}
