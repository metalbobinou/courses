%% Exercice 5

\ExoSpecs{\TTBF{login.php profil.php}}{\TTBF{\RenduDir/MiniProjet/}}{}{}{}
%\ExoSpecsCustom{\TTBF{exo1\_fun.php} [my\_Calculette(int, int, string)]}{\TTBF{\RenduDir/src/exo1/}}{}{}{Fonctions recommandées}{\TTBF{(Bases PHP)}, \TTBF{(Maths PHP)}, \TTBF{return}}

\vspace*{0.7cm}

\noindent \ExoObjectif{Le but du palier est de créer une gestion des sessions utilisateurs.}

\bigskip

\noindent Vous devez tout d'abord créer une page \TTBF{login.php} dans laquelle un utilisateur peut créer un compte, s'identifier, ou se délogger.
Sur cette page, plusieurs boutons permettront de gérer chaque cas (authentification, fin de session, ou création de compte).
Vous devez créer une table \TTBF{utilisateurs} dans la base de données qui contient au moins un nom d'utilisateur et un mot de passe, mais vous êtes libres d'ajouter d'éventuelles colonnes supplémentaires.

\medskip

\noindent Ensuite, vous devrez créer une page \TTBF{profil.php} dans laquelle l'utilisateur peut modifier son mot de passe et éventuellement les autres informations.

\medskip

\noindent La gestion des utilisateurs authentifiés implique que les commandes soient gérées séparément pour chacun.
À vous d'ajouter les colonnes nécessaires dans les autres tables de la base de données, ainsi que le code nécessaire dans les autres pages PHP (\textit{index.php}, \textit{commande.php}, \textit{panier.php} principalement) pour pouvoir gérer les sessions utilisateur.
