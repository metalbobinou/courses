%% Exercice 1

%\ExoSpecs{\TTBF{CalculTVA.sh}}{\TTBF{\RenduDir/src/exo1/}}{750}{640}{\TTBF{write}}
\ExoSpecsCustom{\TTBF{Base-de-Donnees.xlsx} \\ & \TTBF{Base-de-Donnees.pdf}}{\TTBF{\RenduDir/}}{}{}{Fonctions Recommandées}{ }
%\ExoSpecs{\TTBF{Cahier-des-Charges.docx} \\ & \TTBF{Cahier-des-Charges.pdf} }{\TTBF{\RenduDir/}}{ }{ }{ }

\vspace*{0.7cm}

\noindent \ExoObjectif{Le but de l'exercice est de fournir le schéma de la base de données que vous visez, après avoir réfléchi aux requêtes à exécuter pour votre site.}

\bigskip

\noindent Vous devez représenter les tables de votre base de données sous forme de tableaux (si possible avec un ou deux exemple(s) illustratifs).
Afin de préparer la suite du travail, indiquez pour chaque table dans quel contexte elle sera utilisée.
(exemple : \textit{Affichage de la liste des produits}, \textit{Modification des informations utilisateurs}, ...)

\bigskip

\noindent L'objectif de ce travail est de préparer le format de vos base de données et les requêtes, pour faciliter la suite.
Si vous souhaitez de l'aide sur certaines requêtes, ou souhaitez des explications sur certaines, n'hésitez pas à l'indiquer dans le document (avec les requêtes en question, ou leurs ébauches).
