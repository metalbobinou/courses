%% Exercice 1

%\ExoSpecs{\TTBF{CalculTVA.sh}}{\TTBF{\RenduDir/src/exo1/}}{750}{640}{\TTBF{write}}
\ExoSpecsCustom{\TTBF{Cahier-des-Charges.docx} \\ & \TTBF{Cahier-des-Charges.pdf}}{\TTBF{\RenduDir/}}{}{}{Fonctions Recommandées}{ }
%\ExoSpecs{\TTBF{Cahier-des-Charges.docx} \\ & \TTBF{Cahier-des-Charges.pdf} }{\TTBF{\RenduDir/}}{ }{ }{ }

\vspace*{0.7cm}

\noindent \ExoObjectif{Le but de l'exercice est de fournir un cahier des charges pour le site que vous allez construire ou améliorer.
Vous rédigerez donc un cahier des charges (aussi appelé dans ce cas \textit{cahier de spécifications fonctionnelles}).}

\bigskip

\noindent Vous devez rédiger l'ensemble des fonctionnalités que vous visez pour votre site dynamique en PHP.
Vous pouvez repartir depuis votre site web réalisé au premier semestre (dans le cadre du cours de Développement Web/HTML), ou refaire un site de zéro (vous devrez l'indiquer dans le document avant la liste des fonctionnalités).
La qualité graphique ne sera pas notée négativement tant qu'elle est un minimum logique (par exemple : mettre des numéros de pages au milieu des articles sans raison valable sera considéré comme une erreur d'affichage), mais elle rendra évidemment la présentation plus attrayante.
Vous indiquerez dans un court paragraphe d'introduction le thème de votre site, et son objectif.

Les fonctionnalités attendues concernent la partie \textit{dynamique}, c'est-à-dire la gestion des utilisateurs du site web (visiteur, utilisateur identifié, vendeur, administrateur, ...), la présentation des produits, la capacité à gérer un panier et des commandes, etc...
Vous devez donc présenter les fonctionnalités que vous ajouterez pour réaliser un site web opérationnel.
Un certain niveau de détail est attendu dans le cahier des charges : \textit{gestion des utilisateurs} serait une partie, et elle contiendrait un petit descriptif précis sur \textit{identification des visiteurs, utilisateurs, vendeurs, administrateurs} (qu'est-ce qu'un visiteur ? qu'est-ce qu'un utilisateur identifié ? etc...).

\bigskip

\noindent Il n'y a pas de minimum ou de maximum de pages.
Le document doit cependant avoir un aspect un minimum professionnel (faites attention à l'orthographe et la grammaire, par exemple).
Il doit également présenter clairement les informations pertinentes pour l'évaluation (évitez les paragraphes inutiles).

\bigskip

\noindent L'objectif de ce travail est de formaliser en amont ce que vous comptez réaliser, pour que l'enseignant puisse vous proposer dès maintenant des idées d'améliorations (afin que le projet soit un minimum complet) ou vous indique les difficultés que vous rencontrerez sur vos propositions (et vous proposer, si cela vous convient, quelque chose de plus réalisable à la place).
En cas de doute sur ce que vous souhaitez implémenter, n'hésitez pas à l'indiquer afin que l'enseignant vous explique au plus tôt comment le réaliser et la difficulté réelle.
%L'objectif de ce travail est évidemment de partager vos idées et expériences entre les deux groupes, pour trouver des fonctionnalités évidentes manquantes.
Il n'y a pas de \textit{triche} réellement possible dans cet exercice, excepté le copier/coller de paragraphes ou de phrases entre plusieurs groupes.
