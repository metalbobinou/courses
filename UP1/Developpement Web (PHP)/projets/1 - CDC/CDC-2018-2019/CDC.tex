%% Exercice 1

%\ExoSpecs{\TTBF{CalculTVA.sh}}{\TTBF{\RenduDir/src/exo1/}}{750}{640}{\TTBF{write}}
\ExoSpecsCustom{\TTBF{Cahier-des-Charges.docx} \\ & \TTBF{Cahier-des-Charges.pdf}}{\TTBF{\RenduDir/}}{}{}{Fonctions Recommandées}{ }
%\ExoSpecs{\TTBF{Cahier-des-Charges.docx} \\ & \TTBF{Cahier-des-Charges.pdf} }{\TTBF{\RenduDir/}}{ }{ }{ }

\vspace*{0.7cm}

\noindent \ExoObjectif{Le but de l'exercice est de fournir un cahier des charges au groupe auquel vous êtes associé. Vous jouerez donc le rôle d'une MOA/MOE en indiquant les fonctionnalités que vous souhaitez voir apparaitre sur le site fait par l'autre groupe.}

\bigskip

\noindent Vous devez analyser le site web (réalisé au premier semestre dans le cadre du cours de Développement Web/HTML) du groupe avec lequel vous avez été assigné, et vous devez indiquer dans un cahier des charges l'ensemble des fonctionnalités manquantes.
Les fonctionnalités concernent la gestion des utilisateurs du site web (visiteur, utilisateur identifié, vendeur, administrateur, ...), la présentation des produits, la capacité de gérer un panier et des commandes, etc...
Vous devez donc présenter les fonctionnalités que l'autre groupe devrait ajouter pour réaliser un site web opérationnel.
Un certain niveau de détail est attendu : \textit{gestion des utilisateurs} serait une partie, et elle contiendrait un petit descriptif précis sur  \textit{identification des visiteurs, utilisateurs, vendeurs, administrateurs} (qu'est-ce qu'un visiteur ? qu'est-ce qu'un utilisateur identifié ? etc...).

\bigskip

\noindent Il n'y a pas de minimum ou de maximum de pages.
Le document doit cependant avoir un aspect un minimum professionnel (faites attention à l'orthographe et la grammaire, par exemple).
Il doit également présenter clairement les informations pertinentes pour l'autre groupe (évitez les paragraphes inutiles).

\bigskip

\noindent L'objectif de ce travail est évidemment de partager vos idées et expériences entre les deux groupes, pour trouver des fonctionnalités évidentes manquantes.
Il n'y a pas de \textit{triche} réellement possible dans cet exercice, excepté le copier/coller de paragraphes ou de phrases entre plusieurs groupes.
