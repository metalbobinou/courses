%%%%%%%%%%%%%%%%%%%%%%%%%%%%
%% AIDE MEMOIRE AU CAS OU %%
%%%%%%%%%%%%%%%%%%%%%%%%%%%%
%\newpage

%{\Large \textbf{Aide Mémoire}}

%\vspace{30px}

\noindent Le travail doit être rendu au format \textbf{\textit{.zip}}, c'est-à-dire une archive \textbf{zip} compressée avec un outil adapté (les logiciels \textit{7zip} ou \textit{Keka} sont gratuits et adaptés).

\noindent Tout autre format d'archive (rar, 7zip, gz, gzip, bzip, ...) ne sera pas pris en compte, et votre travail ne sera pas corrigé (entraînant la note de 0).

\vspace*{1cm}

\noindent Vous devez rendre un document au format \textbf{DOCX} et un autre exemplaire du même document au format \textbf{PDF}. Plusieurs logiciels peuvent vous permettre de générer des documents dans ces deux formats (LibreOffice, Apache OpenOffice, Oracle StarOffice, Microsoft Word, ...).
