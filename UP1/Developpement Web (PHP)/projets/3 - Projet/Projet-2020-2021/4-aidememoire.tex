%%%%%%%%%%%%%%%%%%%%%%%%%%%%
%% AIDE MEMOIRE AU CAS OU %%
%%%%%%%%%%%%%%%%%%%%%%%%%%%%
%\newpage

%{\Large \textbf{Aide Mémoire}}

%\vspace{30px}

\noindent Vous allez recevoir sur vos boîtes mails un login et un mot de passe pour vous connecter et envoyer vos fichiers.
La première consigne évidente : ne tentez pas de casser la machine ou d'en prendre le contrôle, cela est pénalement répréhensible (sans parler du 0 qui viendrait évidemment avec...).
Même chose sur les projets de vos camarades de classe : vous n'êtes autorisés à accéder et modifier \textit{que} votre espace personnel.

\bigskip

\subsection{Accès aux fichiers}

\noindent Pour vous connecter à la machine et envoyer votre site, vous devrez installer un client SCP/SFTP.
Sur Windows, vous pouvez utiliser \href{https://winscp.net}{WinSCP}.
Sur macOS vous pouvez utiliser \href{https://cyberduck.io}{CyberDuck}.

\bigskip

\noindent Une fois installé, vous pourrez lancer l'application et celle-ci vous proposera de vous connecter à un serveur.
Sur CyberDuck, cliquez sur \textit{Open Connection}.
Sur WinSCP, la fenêtre de connexion s'ouvre automatiquement.
Dans le menu suivant, vous devrez indiquer ces informations :

\bigskip

\begin{itemize}
\item Protocole : SFTP (ou SCP) - \textit{(n'utilisez pas FTP)}
\item Serveur : p1web2021.metalman.eu
\item Port : 22
\item Login : (celui indiqué dans le mail)
\item Password : (celui indiqué dans le mail)
\end{itemize}

\bigskip

\noindent Une fois les informations validées, à la première connexion vous aurez une alerte vous demandant si vous souhaitez enregistrer la clé, répondez \textit{oui} (la clé SSH ne devrait pas changer d'ici la fin de l'année universitaire).

\bigskip

\noindent Vous arriverez sur votre dossier personnel, entrez dans le dossier \textit{public\_html} dans lequel vous pourrez déposer votre site.
Ne supprimez jamais ce dossier.
Tout site web démarre par un fichier \textit{index.php} ou \textit{index.htm} : assurez-vous d'en mettre un.
Vous pourrer accéder à votre site en vous connectant à l'url suivante (remplacez la lettre A par celle(s) de votre groupe) : \url{http://p1web2021.metalman.eu/~groupeA/}

\bigskip

\subsection{Accès à la base de données}

\noindent Pour accéder à PhpMyAdmin, vous n'aurez qu'à vous rendre sur : \url{http://p1web2021.metalman.eu/phpmyadmin/}.
Pour vous y connecter, utilisez le login et le mot de passe indiqués dans le mail.
Il n'y a pas de connexion sécurisée vers PhpMyAdmin pour cette machine.

\bigskip

\noindent Afin de transférer sans difficulté votre base de données depuis votre machine, vous devrez faire un \textit{export} depuis votre PhpMyAdmin local, puis un \textit{import} sur le PhpMyAdmin distant.
N'oubliez pas de modifier dans le code de votre site les accès au système de gestion de base de données !
Cela sera toujours sur localhost, avec le port par défaut, mais la base de données, les login, et le mot de passe sont ceux fournis dans le mail.

\bigskip

\noindent Pour exporter une base de données depuis PhpMyAdmin : cliquez sur le nom de la base de données dans le menu de gauche, puis dans le menu en haut à droite, cliquez sur \textit{Export}.
Sélectionnez \textit{Custom} pour personnaliser l'export.
Choisissez le format \textit{SQL}.
Dans la partie \textit{Output}, cliquez sur \textit{Rename exported databases/tables/columns}, et indiquez le nom de la nouvelle base de données donnée dans le mail.
Puis cliquez sur \textit{Save output to a file}, et sélectionnez dans \textit{Compression} : \textit{gzipped}.
Puis cliquez tout en bas sur \textit{Go}.

\bigskip

\noindent Une fois l'archive récupérée, pour l'importer sur une autre machine, cliquez sur \textit{Import}, et envoyez simplement l'archive que vous venez de créer.
