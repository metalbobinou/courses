%% Exercice 1

%\ExoSpecs{\TTBF{CalculTVA.sh}}{\TTBF{\RenduDir/src/exo1/}}{750}{640}{\TTBF{write}}
\ExoSpecsCustom{\TTBF{ } \\ & \TTBF{ }}{\TTBF{ }}{ }{ }{Exigence}{Site Web (http)}
%\ExoSpecs{\TTBF{Cahier-des-Charges.docx} \\ & \TTBF{Cahier-des-Charges.pdf} }{\TTBF{\RenduDir/}}{ }{ }{ }

\vspace*{0.7cm}

\noindent \ExoObjectif{Le but du projet est de fournir un site web dynamique en PHP qui fonctionne.}

\bigskip

\noindent Vous devez maintenant développer le site web prévu par le cahier des charges.
La base de données est normalement adaptée aux exigences inscrites dans le cahier des charges.
Il suffit donc de développer en PHP le site web.

\bigskip

\noindent Vous devrez donc envoyer votre site web (les fichiers PHP) sur la machine distante, recréer les tables (ou exporter votre schéma et vos données) dessus, et adapter le code si nécessaire (principalement les accès à la base de données).
Cette phase de déploiement est souvent appelée \textbf{mise en production}, et est une étape critique et souvent stressante en situation réelle.
\'Etant donné que la machine distante n'aura à peu près aucune contrainte technique (un LAMP y est déjà installé), il n'y a pas d'inquiétude à avoir : il s'agit du même environnement que le vôtre, excepté qu'il sera sous Linux.

\bigskip

\noindent Les informations pour l'envoi sont indiquées dans la partie \textit{Aide Mémoire}.

\bigskip

\noindent En cas de problème avec la machine, contactez Fabrice BOISSIER par mail \linebreak (\textit{fabrice.boissier@univ-paris1.fr}), en indiquant explicitement dans le mail votre groupe.
