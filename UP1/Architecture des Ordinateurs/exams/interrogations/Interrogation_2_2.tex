\documentclass[11pt,a4paper]{article}
\usepackage[utf8]{inputenc}
\usepackage[french]{babel}
\usepackage[T1]{fontenc}

\usepackage{amsmath}
\usepackage{amsfonts}
\usepackage{amssymb}

\usepackage[left=2cm,right=2cm,top=2cm,bottom=2cm]{geometry}

\usepackage{courier}

\usepackage{fancyhdr}

\usepackage{tabularx}

\pagestyle{fancy}														% definition du style : FancyHDR
\renewcommand{\footrulewidth}{0.4pt}									% Ligne au dessus du footer
\fancyhf{}																% Remise a zero des entetes
\newenvironment{vcenterpage}
{\newpage\vspace*{\fill}}
{\vspace*{\fill}\par\pagebreak}

%%%%%%%%%%%%%%%%%%%%%%%
%Header
%%%%%%%%%%%%%%%%%%%%%%%
\rhead{Groupe 2}									%Droite Haut
\chead{18 décembre 2017}							%Centre Haut
\lhead{Interrogation 2}								%Gauche Haut
\rfoot{}											%Droite Bas
\cfoot{\texttt{Architecture des Ordinateurs et Systèmes d'Exploitation}}	%Centre Bas
\lfoot{}											%Gauche Bas

% \texttt{Paris 1 - Panthéon Sorbonne}

\author{Fabrice BOISSIER}
\begin{document}

\setlength{\fboxrule}{2pt}

\noindent \framebox[17cm]{
\begin{minipage}{1\textwidth}
     \begin{center}
          \LARGE\textbf{INTERROGATION 2} \\
          \normalsize\textbf{Système d'Exploitation et Shell}
     \end{center}
\end{minipage}
}

\bigskip

NOM : \hspace{6.5cm} PR\'ENOM :

\smallskip

\section{Citer les 6 appels systèmes UNIX permettant des manipulations sur le système de fichier : }

\bigskip
\bigskip
\bigskip
\bigskip
\bigskip

\section{Indiquer les commandes pour donner les bons droits aux fichiers suivants :}

\medskip

\noindent fichier1 (lecture et exécution pour le propriétaire, tous les droits pour le groupe, et aucun droit pour les autres)

\medskip

\noindent fichier2 (écriture pour le propriétaire, lecture pour le groupe, exécution pour les autres)

\medskip

\noindent \#monfichier\# (tous les droits pour le propriétaire, lecture et écriture pour tous les autres)

\medskip

\bigskip
\bigskip
\bigskip
\bigskip
\bigskip
\bigskip
\bigskip
\bigskip
\bigskip
\bigskip
\bigskip

\section{\'Ecrire la ligne de commande permettant de créer une archive \newline \texttt{archive.tar.bz2} contenant le répertoire \texttt{UnDossier} : }

\bigskip
\bigskip
\bigskip
\bigskip
\bigskip

\section{Remplir les phrases suivantes avec les bons mots : }

\bigskip

\noindent Un système d'exploitation ................................... permet de stopper des processus/tâches en cours d'exécution pour pouvoir donner du temps processeur à d'autres. On dit aussi que le système est à ................................... ................................... .\\

\noindent La méthode d'accès ................................... permet d'accéder directement à la donnée souhaitée, tandis que la méthode d'accès ................................... nécessite de lire toutes les données précédentes depuis le début du support.\\

\bigskip

\section{Remettre dans l'ordre les phases de compilation, indiquer la commande du principal compilateur/linkeur et ses options si nécessaire}

\renewcommand\arraystretch{2.5}

\bigskip
\begin{center}
  \begin{tabularx}{15.5cm}{| c | c | c | X |}
  \hline
  \No \'Etape & Nom de l'\'Etape & Programme/Commande & Paramètre(s) \\ \hline
  & \textbf{Assemblage} & & \\
  \hline
  & \textbf{Compilation} & & \\
  \hline
  & \textbf{Link Edit/\'Edition de Liens} & & \\
  \hline
  & \textbf{Pré-Compilation} & & \\ 
  \hline
  \end{tabularx}
\end{center}
\medskip

\renewcommand\arraystretch{1}

\section{Remplir le tableau avec les commandes ou leur description}

\renewcommand\arraystretch{2.5}

\bigskip
\begin{center}
  \begin{tabular}{| c | c |}
  \hline
  \textbf{Commande/Programme} & \textbf{Description} \\ \hline
  kill & \\ \hline
  ps & \\ \hline
  & Renommer un fichier \\ \hline
  & Supprimer un fichier \\ \hline
  & Supprimer un dossier \\ \hline
  ln & \\ \hline
  bg & \\ \hline
  touch & \\ \hline
  & Programme de traitement de texte par colonne \\ \hline
  cp & \\ \hline
  & Déplacer un dossier \\
  \hline
  \end{tabular}
\end{center}
\bigskip

\renewcommand\arraystretch{1}

\newpage

NOM : \hspace{6.5cm} PR\'ENOM :

\section{Convertir ces nombres décimaux en binaires sur 8 bits : \texttt{222}, \texttt{-14}}

\bigskip
\bigskip
\bigskip

\section{Convertir ces nombres binaires (8 bits signés et non signés) en décimaux : \texttt{\%1000 1101}, \texttt{\%1011 1101}}

\bigskip
\bigskip
\bigskip
\bigskip

\section{\'Ecrire un script sh qui affichera \texttt{"Coucou !"} si on lui donne en unique paramètre \texttt{"Bonjour ?"}, mais qui affichera \texttt{"Non."} si on lui donne d'autres paramètres (multiples ou non), et n'affichera rien si aucun paramètre n'est donné.}

\bigskip
\bigskip
\bigskip
\bigskip
\bigskip
\bigskip
\bigskip
\bigskip
\bigskip
\bigskip



\end{document}
