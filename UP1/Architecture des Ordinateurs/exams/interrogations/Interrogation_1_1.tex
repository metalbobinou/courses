\documentclass[11pt,a4paper]{article}
\usepackage[utf8]{inputenc}
\usepackage[francais]{babel}
\usepackage[T1]{fontenc}

\usepackage{amsmath}
\usepackage{amsfonts}
\usepackage{amssymb}

\usepackage[left=2cm,right=2cm,top=2cm,bottom=2cm]{geometry}

\usepackage{courier}

\usepackage{fancyhdr}

\pagestyle{fancy}														% definition du style : FancyHDR
\renewcommand{\footrulewidth}{0.4pt}									% Ligne au dessus du footer
\fancyhf{}																% Remise a zero des entetes
\newenvironment{vcenterpage}
{\newpage\vspace*{\fill}}
{\vspace*{\fill}\par\pagebreak}

%%%%%%%%%%%%%%%%%%%%%%%
%Header
%%%%%%%%%%%%%%%%%%%%%%%
\rhead{Groupe 2}									%Droite Haut
\chead{13 octobre 2016}								%Centre Haut
\lhead{Interrogation 1}								%Gauche Haut
\rfoot{}											%Droite Bas
\cfoot{\texttt{Architecture des Ordinateurs et Systèmes d'Exploitation}}	%Centre Bas
\lfoot{}											%Gauche Bas

% \texttt{Paris 1 - Panthéon Sorbonne}

\author{Fabrice BOISSIER}
\begin{document}

\setlength{\fboxrule}{2pt}

\noindent \framebox[17cm]{
\begin{minipage}{1\textwidth}
     \begin{center}
          \LARGE\textbf{INTERROGATION 1} \\
          \normalsize\textbf{Architecture des Ordinateurs ; Déplacement dans le Shell}
     \end{center}
\end{minipage}
}

\bigskip

NOM : \hspace{6.5cm} PR\'ENOM :

\smallskip

\section{Citer 3 classes d'ordinateurs et leurs caractéristiques}

\bigskip
\bigskip
\bigskip
\bigskip
\bigskip
\bigskip
\bigskip
\bigskip

\section{Quels sont les 3 principaux bus d'un processeur ? Expliquer leurs fonctions}

\bigskip
\bigskip
\bigskip
\bigskip
\bigskip
\bigskip
\bigskip
\bigskip

\section{Un périphérique peut-il écrire en mémoire ? Quel mécanisme lui permet de communiquer avec le processeur ?}

\bigskip
\bigskip
\bigskip

\section{Comment est chargé un programme en mémoire pour être exécuté ?}

\bigskip
\bigskip
\bigskip
\bigskip
\bigskip
\bigskip
\bigskip
\bigskip

\section{Dans le contexte d'un processeur, par quels moyens des paramètres peuvent être donnés à une fonction ?}

%\bigskip
%\bigskip
%\bigskip
%\bigskip

%\section{Expliquer les relations entre les différents langages : "langage machine", "assembleur", "C"}



\end{document}
