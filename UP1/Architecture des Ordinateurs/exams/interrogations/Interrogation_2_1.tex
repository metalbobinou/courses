\documentclass[11pt,a4paper]{article}
\usepackage[utf8]{inputenc}
\usepackage[francais]{babel}
\usepackage[T1]{fontenc}

\usepackage{amsmath}
\usepackage{amsfonts}
\usepackage{amssymb}

\usepackage[left=2cm,right=2cm,top=2cm,bottom=2cm]{geometry}

\usepackage{courier}

\usepackage{fancyhdr}

\usepackage{tabularx}

\pagestyle{fancy}														% definition du style : FancyHDR
\renewcommand{\footrulewidth}{0.4pt}									% Ligne au dessus du footer
\fancyhf{}																% Remise a zero des entetes
\newenvironment{vcenterpage}
{\newpage\vspace*{\fill}}
{\vspace*{\fill}\par\pagebreak}

%%%%%%%%%%%%%%%%%%%%%%%
%Header
%%%%%%%%%%%%%%%%%%%%%%%
\rhead{Groupe 2}									%Droite Haut
\chead{01 décembre 2016}							%Centre Haut
\lhead{Interrogation 2}								%Gauche Haut
\rfoot{}											%Droite Bas
\cfoot{\texttt{Architecture des Ordinateurs et Systèmes d'Exploitation}}	%Centre Bas
\lfoot{}											%Gauche Bas

% \texttt{Paris 1 - Panthéon Sorbonne}

\author{Fabrice BOISSIER}
\begin{document}

\setlength{\fboxrule}{2pt}

\noindent \framebox[17cm]{
\begin{minipage}{1\textwidth}
     \begin{center}
          \LARGE\textbf{INTERROGATION 2} \\
          \normalsize\textbf{Système d'Exploitation et Shell}
     \end{center}
\end{minipage}
}

\bigskip

NOM : \hspace{6.5cm} PR\'ENOM :

\smallskip

\section{Citer 4 appels systèmes}

\bigskip
\bigskip

\section{Quels sont les 2 modes d'accès aux données sur les supports physiques ? Donner un exemple de support pour chacun}

\bigskip
\bigskip
\bigskip
\bigskip
\bigskip

\section{Remplir le tableau avec les commandes ou leur description}

\renewcommand\arraystretch{2.5}

\bigskip
\begin{center}
  \begin{tabular}{| c | c |}
  \hline
  \textbf{Commande/Programme} & \textbf{Description} \\ \hline
  & Lister les fichiers \\ \hline
  pwd &  \\ \hline
  & Renommer un fichier \\ \hline
  & Supprimer un fichier \\ \hline
  & Supprimer un dossier \\ \hline
  & Créer un dossier \\ \hline
  cd & \\ \hline
  & Modifier la date de modification d'un fichier \\ \hline
  grep & \\ \hline
  ps & \\ \hline
  & Déplacer un dossier \\
  \hline
  \end{tabular}
\end{center}
\bigskip

\renewcommand\arraystretch{1}

\section{Remettre dans l'ordre les phases de compilation, indiquer la commande du principal compilateur/linkeur et ses options si nécessaire}

\renewcommand\arraystretch{2.5}

\bigskip
\begin{center}
  \begin{tabularx}{15.5cm}{| c | c | c | X |}
  \hline
  \No \'Etape & Nom de l'\'Etape & Programme/Commande & Paramètre(s) \\ \hline
  & \textbf{Assemblage} & & \\
  \hline
  & \textbf{Compilation} & & \\
  \hline
  & \textbf{Link Edit/\'Edition de Liens} & & \\
  \hline
  & \textbf{Pré-Compilation} & & \\ 
  \hline
  \end{tabularx}
\end{center}
\medskip

\renewcommand\arraystretch{1}

\section{Indiquer les commandes pour donner les bons droits aux fichiers suivants :}

\medskip

\noindent fichier1 (tous les droits pour le propriétaire, lecture et exécution pour le groupe et les autres)

\bigskip
\bigskip

\noindent fichier2 (lecture et exécution pour le propriétaire, exécution pour le groupe, aucun droit pour les autres)

\bigskip
\bigskip

\noindent \#monfichier\# (lecture et écriture pour le propriétaire, aucun droit pour tous les autres)

\bigskip
\bigskip

\section{Citer les sections de l'espace d'adressage où peuvent se situer les variables d'un programme}

\bigskip
\bigskip
\bigskip

\section{\textit{\texttt{SUPPL\'EMENT :}} Que font >, >{}>, |, et < dans le shell ?}

\bigskip
\bigskip
\bigskip
\bigskip
\bigskip
\bigskip
\bigskip

\section{\textit{\texttt{SUPPL\'EMENT :}} Que font les registres "PC" (Program Counter) et "SP" (Stack Pointer) dans un processeur ?}

\bigskip

\end{document}
