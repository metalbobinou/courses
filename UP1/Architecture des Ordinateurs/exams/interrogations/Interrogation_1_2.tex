\documentclass[11pt,a4paper]{article}
\usepackage[utf8]{inputenc}
\usepackage[french]{babel}
\usepackage[T1]{fontenc}

\usepackage{amsmath}
\usepackage{amsfonts}
\usepackage{amssymb}

\usepackage[left=2cm,right=2cm,top=2cm,bottom=2cm]{geometry}

\usepackage{courier}

\usepackage{fancyhdr}

\pagestyle{fancy}														% definition du style : FancyHDR
\renewcommand{\footrulewidth}{0.4pt}									% Ligne au dessus du footer
\fancyhf{}																% Remise a zero des entetes
\newenvironment{vcenterpage}
{\newpage\vspace*{\fill}}
{\vspace*{\fill}\par\pagebreak}

%%%%%%%%%%%%%%%%%%%%%%%
%Header
%%%%%%%%%%%%%%%%%%%%%%%
\rhead{Groupe 2}									%Droite Haut
\chead{06 novembre 2017}							%Centre Haut
\lhead{Interrogation 1}								%Gauche Haut
\rfoot{}											%Droite Bas
\cfoot{\texttt{Architecture des Ordinateurs et Systèmes d'Exploitation}}	%Centre Bas
\lfoot{}											%Gauche Bas

% \texttt{Paris 1 - Panthéon Sorbonne}

\author{Fabrice BOISSIER}
\begin{document}

\setlength{\fboxrule}{2pt}

\noindent \framebox[17cm]{
\begin{minipage}{1\textwidth}
     \begin{center}
          \LARGE\textbf{INTERROGATION 1} \\
          \normalsize\textbf{Architecture des Ordinateurs ; Déplacement dans le Shell}
     \end{center}
\end{minipage}
}

\bigskip

NOM : \hspace{6.5cm} PR\'ENOM :

\smallskip

\section{Convertir ces nombres décimaux en binaires sur 8 bits : \texttt{144}, \texttt{-120}}

\bigskip
\bigskip
\bigskip

\section{Convertir ces nombres binaires (8 bits signés et non signés) en décimaux : \texttt{\%1110 0110}, \texttt{\%1001 0111}}

\bigskip
\bigskip
\bigskip

\section{Nommer le langage que comprennent les processeurs, puis son équivalent lisible par un.e humain.e}

\bigskip
\bigskip
\bigskip

\section{Citer les 3 bus d'un processeur, puis les 3 classes générales d'ordinateurs}

\bigskip
\bigskip
\bigskip
\bigskip
\bigskip

\section{Expliquer le fonctionnement de la pile, puis faites un petit schéma représentant l'appel de la fonction suivante : \texttt{toupper("Aie", str)} où \textit{str} aura pour adresse \texttt{@1440}}

\end{document}
