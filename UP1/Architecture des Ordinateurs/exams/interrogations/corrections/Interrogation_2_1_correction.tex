\documentclass[11pt,a4paper]{article}
\usepackage[utf8]{inputenc}
\usepackage[francais]{babel}
\usepackage[T1]{fontenc}

\usepackage{amsmath}
\usepackage{amsfonts}
\usepackage{amssymb}

\usepackage[left=2cm,right=2cm,top=2cm,bottom=2cm]{geometry}

\usepackage{courier}

\usepackage{fancyhdr}

\usepackage{tabularx}

\pagestyle{fancy}														% definition du style : FancyHDR
\renewcommand{\footrulewidth}{0.4pt}									% Ligne au dessus du footer
\fancyhf{}																% Remise a zero des entetes
\newenvironment{vcenterpage}
{\newpage\vspace*{\fill}}
{\vspace*{\fill}\par\pagebreak}

%%%%%%%%%%%%%%%%%%%%%%%
%Header
%%%%%%%%%%%%%%%%%%%%%%%
\rhead{Groupe 2}									%Droite Haut
\chead{01 décembre 2016}							%Centre Haut
\lhead{Interrogation 2}								%Gauche Haut
\rfoot{}											%Droite Bas
\cfoot{\texttt{Architecture des Ordinateurs et Systèmes d'Exploitation}}	%Centre Bas
\lfoot{}											%Gauche Bas

% \texttt{Paris 1 - Panthéon Sorbonne}

\author{Fabrice BOISSIER}
\begin{document}

\setlength{\fboxrule}{2pt}

\noindent \framebox[17cm]{
\begin{minipage}{1\textwidth}
     \begin{center}
          \LARGE\textbf{INTERROGATION 2} \\
          \normalsize\textbf{Système d'Exploitation et Shell}
     \end{center}
\end{minipage}
}

\bigskip

NOM : \hspace{6.5cm} PR\'ENOM :

\smallskip

\section{Citer 4 appels systèmes}

\bigskip
\begin{itemize}
\item Open : ouvre un fichier selon des règles (création, ajout, remise à zéro) et crée un file descriptor
\item Read : lit depuis un file descriptor des données, et les copie au bout d'un pointeur
\item Write : écrit dans un file descriptor des données issues d'un pointeur
\item Close : ferme un file descriptor, et libère le fichier
\item Fork : duplique le processus en cours (création d'un processus "fils")
\item Pipe : créer un pipe entre le processus et lui-même
\item Dup : duplique un file descriptor
\item Wait/WaitPID : attend la fin d'un processus fils
\item GetPID/GetPPID : récupère son PID ou celui de son "père"
\end{itemize}
\bigskip

\section{Quels sont les 2 modes d'accès aux données sur les supports physiques ? Donner un exemple de support pour chacun}

\bigskip
\begin{itemize}
\item Accès Séquentiel : le support physique contient des données, dont l'organisation oblige à lire (ou au moins faire passer) tout ce qui précède la donnée recherchée. On ne peut accéder à une case que si on a lu/est passé sur ses prédécesseurs. Par exemple : les bandes magnétiques doivent être déroulées pour accéder à la donnée qui se trouve au milieu de la bande.\\
\item Accès Aléatoire : le support physique permet un accès à n'importe quelle donnée de façon aisée/peu contraignante/rapide. On peut accéder à n'importe quelle case à la demande. Par exemple : les disques permettent d'accéder à n'importe quelle case en faisant simplement tourner le disque et en déplaçant la tête de lecture (accès plutôt rapide). Les mémoires flashs activent les cases à lire/écrire grâce au câblage électronique gravé dans le silicium (accès très rapide).\\
\end{itemize}
\bigskip

\section{Remplir le tableau avec les commandes ou leur description}

%\renewcommand\arraystretch{2.5}

\bigskip
\begin{center}
  \begin{tabular}{| c | c |}
  \hline
  \textbf{Commande/Programme} & \textbf{Description} \\ \hline
  \texttt{ls} & Lister les fichiers \\ \hline
  \texttt{pwd} & Afficher le répertoire courant \\ \hline
  \texttt{mv} / \texttt{rename} & Renommer un fichier \\ \hline
  \texttt{rm} & Supprimer un fichier \\ \hline
  \texttt{rmdir} / \texttt{rm -r} & Supprimer un dossier \\ \hline
  \texttt{mkdir} & Créer un dossier \\ \hline
  \texttt{cd} & Changer de répertoire courant \\ \hline
  \texttt{touch} & Modifier la date de modification d'un fichier \\ \hline
  \texttt{grep} & Afficher les chaînes de caractères correspondantes à un motif \\ \hline
  \texttt{ps} & Afficher la liste des processus \\ \hline
  \texttt{mv} & Déplacer un dossier \\
  \hline
  \end{tabular}
\end{center}
\bigskip

%\renewcommand\arraystretch{1}

\section{Remettre dans l'ordre les phases de compilation, indiquer la commande du principal compilateur/linkeur et ses options si nécessaire}

%\renewcommand\arraystretch{2.5}

\bigskip
\begin{center}
  \begin{tabularx}{15.5cm}{| c | c | c | X |}
  \hline
  \No \'Etape & Nom de l'\'Etape & Programme/Commande & Paramètre(s) \\ \hline
  1 & \textbf{Pré-Compilation} & gcc (OU cpp) & -E\\ 
  \hline
  2 & \textbf{Compilation} & gcc (OU cc1) & -S \\
  \hline
  3 & \textbf{Assemblage} & gcc (OU as) & -c \\
  \hline
  4 & \textbf{Link Edit/\'Edition de Liens} & gcc (OU ld) & \\
  \hline
  \end{tabularx}
\end{center}
\medskip

%\renewcommand\arraystretch{1}

\section{Indiquer les commandes pour donner les bons droits aux fichiers suivants :}

\medskip

\noindent fichier1 (tous les droits pour le propriétaire, lecture et exécution pour le groupe et les autres)

\noindent \textbf{chmod 755 fichier1} \hfill OU \hfill \textbf{chmod u=rwx,go=rx fichier1}

\bigskip

\noindent fichier2 (lecture et exécution pour le propriétaire, exécution pour le groupe, aucun droit pour les autres)

\noindent \textbf{chmod 510 fichier2} \hfill OU \hfill \textbf{chmod u=rx,g=x,o= fichier2}

\bigskip

\noindent \#monfichier\# (lecture et écriture pour le propriétaire, aucun droit pour tous les autres)

\noindent \textbf{chmod 600 \textbackslash\#monfichier\textbackslash\#} \hfill OU \hfill \textbf{chmod u=rw,go= \textbackslash\#monfichier\textbackslash\#}

\bigskip

\section{Citer les sections de l'espace d'adressage où peuvent se situer les variables d'un programme}

\bigskip
\begin{itemize}
\item Data : données initialisées (exemple : \textit{\texttt{int a = 42;}})
\item BSS : variables initialisées à 0, ou non initialisées (exemple : \textit{\texttt{int a, b = 0;}})
\item (acceptée) ROData : données initialisées et chaînes de caractères pré-déclarées \\ (exemple : \textit{\texttt{char *str = "LOL";}})
\item (acceptée si les précédentes sont citées) Pile/Stack : certaines variables peuvent se situer dans la pile lors de l'exécution du programme
\end{itemize}
\bigskip

\section{\textit{\texttt{SUPPL\'EMENT :}} Que font >, >{}>, |, et < dans le shell ?}

\bigskip
\begin{itemize}
\item \textbf{>} : la sortie standard de la commande à gauche est envoyée dans un fichier (le fichier est écrasé)
\item \textbf{>{}>} : la sortie standard de la commande à gauche est envoyée dans un fichier (ajout en fin de fichier)
\item \textbf{<} : l'entrée standard de la commande à gauche est remplie avec le contenu d'un fichier
\item \textbf{|} : la sortie standard de la commande à gauche est connectée à l'entrée standard de la commande à droite

\end{itemize}
\bigskip

\section{\textit{\texttt{SUPPL\'EMENT :}} Que font les registres "PC" (Program Counter) et "SP" (Stack Pointer) dans un processeur ?}

\bigskip
\begin{itemize}
\item PC (Program Counter) : donne l'adresse de l'instruction en cours d'exécution
\item SP (Stack Pointer) : donne l'adresse de la pile, et plus particulièrement du sommet de la pile
\end{itemize}

\end{document}
