\documentclass[11pt,a4paper]{article}
\usepackage[utf8]{inputenc}
\usepackage[francais]{babel}
\usepackage[T1]{fontenc}

\usepackage{amsmath}
\usepackage{amsfonts}
\usepackage{amssymb}

\usepackage[left=2cm,right=2cm,top=2cm,bottom=2cm]{geometry}

\usepackage{courier}

\usepackage{fancyhdr}

\pagestyle{fancy}														% definition du style : FancyHDR
\renewcommand{\footrulewidth}{0.4pt}									% Ligne au dessus du footer
\fancyhf{}																% Remise a zero des entetes
\newenvironment{vcenterpage}
{\newpage\vspace*{\fill}}
{\vspace*{\fill}\par\pagebreak}

%%%%%%%%%%%%%%%%%%%%%%%
%Header
%%%%%%%%%%%%%%%%%%%%%%%
\rhead{Groupe 2}									%Droite Haut
\chead{13 octobre 2016}								%Centre Haut
\lhead{Interrogation 1}								%Gauche Haut
\rfoot{}											%Droite Bas
\cfoot{\texttt{Architecture des Ordinateurs et Systèmes d'Exploitation}}	%Centre Bas
\lfoot{}											%Gauche Bas

% \texttt{Paris 1 - Panthéon Sorbonne}

\author{Fabrice BOISSIER}
\begin{document}

\setlength{\fboxrule}{2pt}

\noindent \framebox[17cm]{
\begin{minipage}{1\textwidth}
     \begin{center}
          \LARGE\textbf{INTERROGATION 1} \\
          \normalsize\textbf{Architecture des Ordinateurs}
     \end{center}
\end{minipage}
}

\bigskip

NOM : \hspace{6.5cm} PR\'ENOM :

\smallskip

\section{Citer 3 classes d'ordinateurs et leurs caractéristiques}

\bigskip
\noindent 4 classes d'ordinateurs :
\begin{itemize}
\item \textbf{Micro-Ordinateurs} (Micros) : Nos ordinateurs classiques à la maison, généralement des PC (Personal Computer), dont l'usage est plutôt orienté graphisme et expérience utilisateur (GUI, IHM/HCI, ...) pour nos besoins quotidiens.
\item \textbf{Mini-Ordinateurs} (Minis) : Machines souvent mises en réseau ensemble, légèrement plus puissantes que des Micros, et orientées dans le traitement des requêtes émises par des Micros.
\item \textbf{Mainframes} : Machines les plus puissantes en terme de traitement IO, dédiées pour traiter des quantités énormes de requêtes (typiquement : les transactions bancaires, les réservations de billets, etc.)
\item \textbf{Super-Ordinateurs} / \textbf{Super-Calculateurs} : Ensemble de Minis reliés pour du calcul scientifique intensif.
\item [ANNEXE] \textit{"Nano-Ordinateurs" : SmartPhones (Micros dans l'usage), sondes, webcams autonomes, FPGA, RFID, etc. Tout pour l'Internet of Things (IOT).}\\
\end{itemize}

\section{Quels sont les 3 principaux bus d'un processeur ? Expliquer leurs fonctions}

\bigskip
\begin{itemize}
\item \textbf{Bus d'Adresse} : sélectionne une adresse en mémoire, pour y écrire ou lire des données.
\item \textbf{Bus de Données} : envoie ou récupère des données en dehors du processeur.
\item \textbf{Bus de Contrôle} : contrôle le processeur lui-même, la synchronisation électronique avec la mémoire, et les périphériques externes avec les \textbf{interruptions}.
\end{itemize}

\section{Un périphérique peut-il écrire en mémoire ? Quel mécanisme lui permet de communiquer avec le processeur ?}

\bigskip
\noindent Oui un périphérique peut écrire directement en mémoire sans passer par le processeur (avec des DMA/Direct Memory Access).
Les interruptions permettent de communiquer avec le processeur.

\section{Comment est chargé un programme en mémoire pour être exécuté ?}

\bigskip
\noindent Le programme est stocké sur le disque dur sous forme de fichier.
Grâce au nom logique du programme (par exemple : \textit{/bin/ls}), le système d'exploitation va rechercher la position physique du fichier sur le disque dur.
Le système d'exploitation va copier ce programme depuis le disque dur vers la mémoire (le processeur va exécuter du code dédié à l'OS qui fera de multiples interruptions entre le disque et le processeur).
Une fois que le programme est copié en mémoire, le système d'exploitation va sauter à l'adresse de la première instruction du programme (JUMP, CALL, etc.).
Lorsque le code du système d'exploitation est exécuté, le processeur est en mode superviseur.
Lorsque le code du programme est exécuté, le processeur est en mode normal/utilisateur.

\section{Dans le contexte d'un processeur, par quels moyens des paramètres peuvent être donnés à une fonction ?}

\bigskip
\noindent Dans la plupart des processeurs, la pile/stack permet de passer des paramètres à une fonction.
Chaque argument est empilé l'un après l'autre, la fonction est appelée (CALL), puis le code de la fonction dépile chaque argument nécessaire.

\end{document}
