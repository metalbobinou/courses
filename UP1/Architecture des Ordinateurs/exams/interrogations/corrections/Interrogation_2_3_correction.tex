\documentclass[11pt,a4paper]{article}
\usepackage[utf8]{inputenc}
\usepackage[french]{babel}
\usepackage[T1]{fontenc}

\usepackage{amsmath}
\usepackage{amsfonts}
\usepackage{amssymb}

\usepackage[left=2cm,right=2cm,top=2cm,bottom=2cm]{geometry}

\usepackage{courier}

\usepackage{fancyhdr}

\usepackage{tabularx}

% Definition de \CaseCoche{*} qui produira une ligne avec une case et un texte
%\newcommand{\CaseCoche}[1]{{\fontsize{20}{20}\selectfont $ \square $} {\large #1}}

% Definition de \CaseCoche qui produira une case à cocher
\newcommand{\CaseCoche}{\fontsize{20}{20}\selectfont $ \square $}

\newcommand{\TTBF}[1]{\texttt{\textbf{#1}}}

\pagestyle{fancy}														% definition du style : FancyHDR
\renewcommand{\footrulewidth}{0.4pt}									% Ligne au dessus du footer
\fancyhf{}																% Remise a zero des entetes
\newenvironment{vcenterpage}
{\newpage\vspace*{\fill}}
{\vspace*{\fill}\par\pagebreak}

%%%%%%%%%%%%%%%%%%%%%%%
%Header
%%%%%%%%%%%%%%%%%%%%%%%
\rhead{Groupe 1}									%Droite Haut
\chead{14 décembre 2017}							%Centre Haut
\lhead{Interrogation 2}								%Gauche Haut
\rfoot{}											%Droite Bas
\cfoot{\texttt{Architecture des Ordinateurs et Systèmes d'Exploitation}}	%Centre Bas
\lfoot{}											%Gauche Bas

% \texttt{Paris 1 - Panthéon Sorbonne}

\author{Fabrice BOISSIER}
\begin{document}

\setlength{\fboxrule}{2pt}

\noindent \framebox[17cm]{
\begin{minipage}{1\textwidth}
     \begin{center}
          \LARGE\textbf{INTERROGATION 2} \\
          \normalsize\textbf{Système d'Exploitation et Shell}
     \end{center}
\end{minipage}
}

\bigskip

NOM : \hspace{6.5cm} PR\'ENOM :

\smallskip

\section{Choisir la notion ou l'élément le mieux illustré par chaque exemple : }

\medskip
\noindent Il/Elle permet de partager équitablement le temps d'exécution entre plusieurs processus.\\

\begin{itemize}
\item[\CaseCoche] L'Ordonnanceur\\
\end{itemize}

\bigskip
\noindent Plusieurs utilisateurs se connectent sur la même machine et peuvent travailler sur plusieurs tâches grâce à ce système.\\

\begin{itemize}
\item[\CaseCoche] Système d'Exploitation Multi-Tâches à Temps Partagé\\
\end{itemize}

\bigskip
\noindent Il/Elle identifie et stocke les méta-données d'un fichier sur disque.\\

\begin{itemize}
\item[\CaseCoche] L'I-Node\\
\end{itemize}

\bigskip
\noindent Un utilisateur avec bash peut personnaliser son shell à chaque connection avec cela.\\

\begin{itemize}
\item[\CaseCoche] .bashrc\\
\end{itemize}

\bigskip

\section{Définir ce que sont les appels systèmes (au delà des 6 que vous connaissez).}

\bigskip
C'est l'ensemble des services offerts par le système d'exploitation.
\bigskip

\section{Indiquer les commandes pour donner les bons droits aux fichiers suivants :}

\medskip

\noindent fichier1 (tous les droits pour le propriétaire, lecture et exécution pour le groupe et les autres)

\bigskip

chmod 755 fichier1 \qquad chmod u=rwx,go=rx fichier1

\bigskip

\noindent fichier2 (lecture pour le propriétaire, exécution pour le groupe, écriture pour les autres)

\bigskip

chmod 412 fichier2 \qquad chmod u=r,g=x,o=w fichier2

\bigskip

\noindent \#monfichier\# (lecture et écriture pour le propriétaire, aucun droit pour tous les autres)

\bigskip

chmod 600 \textbackslash \#monfichier\textbackslash \# \qquad chmod u=rw,go= \textbackslash \#monfichier\textbackslash \#

\bigskip

\section{\'Ecrire les lignes de commande permettant de créer cette arborescence : }

\medskip

\noindent \TTBF{Dossier/}\\
\noindent \TTBF{Dossier/Fichier1}\\
\noindent \TTBF{Dossier/Fichier2}\\
\noindent \TTBF{Dossier/Contenu}\\
\noindent \TTBF{Dossier/Contenu/Fichier}\\

\bigskip
\noindent \TTBF{mkdir Dossier}\\
\noindent \TTBF{touch Dossier/Fichier1}\\
\noindent \TTBF{touch Dossier/Fichier2}\\
\noindent \TTBF{mkdir Dossier/Contenu}\\
\noindent \TTBF{touch Dossier/Contenu/Fichier}\\
\bigskip

\newpage

NOM : \hspace{6.5cm} PR\'ENOM :

\section{Remettre dans l'ordre les phases de compilation, indiquer la commande du principal compilateur/linkeur et ses options si nécessaire}

\renewcommand\arraystretch{2.5}

\bigskip
\begin{center}
  \begin{tabularx}{15.5cm}{| c | c | c | X |}
  \hline
  \No \'Etape & Nom de l'\'Etape & Programme/Commande & Paramètre(s) \\ \hline
  1 & \textbf{Pré-Compilation} & gcc (OU cpp) & -E\\ 
  \hline
  2 & \textbf{Compilation} & gcc (OU cc1) & -S \\
  \hline
  3 & \textbf{Assemblage} & gcc (OU as) & -c \\
  \hline
  4 & \textbf{Link Edit/\'Edition de Liens} & gcc (OU ld) & \\
  \hline
  \end{tabularx}
\end{center}
\medskip

\renewcommand\arraystretch{1}


\section{Remplir le tableau avec les commandes ou leur description}

\renewcommand\arraystretch{2.5}

\bigskip
\begin{center}
  \begin{tabular}{| c | c |}
  \hline
  \textbf{Commande/Programme} & \textbf{Description} \\ \hline
  ls & Lister les fichiers \\ \hline
  pwd & Afficher le chemin du répertoire courant \\ \hline
  mv & Renommer un fichier \\ \hline
  rm & Supprimer un fichier \\ \hline
  rmdir & Supprimer un dossier \\ \hline
  kill & Envoyer un signal à un processus \\ \hline
  fg & Faire revenir au premier plan une tâche \\ \hline
  touch & Modifier la date de modification d'un fichier \\ \hline
  grep & Afficher les chaînes de caractères correspondantes à un motif \\ \hline
  ps & Afficher la liste des processus \\ \hline
  mv & Déplacer un dossier \\
  \hline
  \end{tabular}
\end{center}
\bigskip

\renewcommand\arraystretch{1}

\section{Convertir ces nombres décimaux en binaires sur 8 bits : \texttt{214}, \texttt{-78}}

\bigskip
214 : \% 1101 0110 (\$ D6)	\qquad	-78 : \% 1011 0010 (\$ B2)
\bigskip

\section{Convertir ces nombres binaires (8 bits signés et non signés) en décimaux : \texttt{\%1010 1101}, \texttt{\%1110 1011}}

\bigskip
\% 1010 1101 : -83 ou 173	\qquad	\% 1110 1011 : 235 ou -21
\bigskip

\section{\'Ecrivez un script shell qui lancera le script \TTBF{autre\_script.sh} (et redirigera sa sortie d'erreur vers le fichier \TTBF{error.log}) si aucun paramètre n'est donné, sinon, le nom donné en premier paramètre sera considéré comme un script et sera exécuté (et sa sortie standard sera redirigée vers le fichier \TTBF{std.log}) }

\bigskip

\begin{verbatim}
#! /bin/sh

if [ $# = 0 ]; then
  sh autre_script.sh 2> error.log
else
  sh $1 1> std.log
fi

\end{verbatim}


\end{document}
