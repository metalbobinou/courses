\documentclass[11pt,a4paper]{article}
\usepackage[utf8]{inputenc}
\usepackage[french]{babel}
\usepackage[T1]{fontenc}

\usepackage{amsmath}
\usepackage{amsfonts}
\usepackage{amssymb}

\usepackage[left=2cm,right=2cm,top=2cm,bottom=2cm]{geometry}

\usepackage{courier}

\usepackage{fancyhdr}

\pagestyle{fancy}														% definition du style : FancyHDR
\renewcommand{\footrulewidth}{0.4pt}									% Ligne au dessus du footer
\fancyhf{}																% Remise a zero des entetes
\newenvironment{vcenterpage}
{\newpage\vspace*{\fill}}
{\vspace*{\fill}\par\pagebreak}

%%%%%%%%%%%%%%%%%%%%%%%
%Header
%%%%%%%%%%%%%%%%%%%%%%%
\rhead{Groupe 2}									%Droite Haut
\chead{06 novembre 2017}							%Centre Haut
\lhead{Interrogation 1}								%Gauche Haut
\rfoot{}											%Droite Bas
\cfoot{\texttt{Architecture des Ordinateurs et Systèmes d'Exploitation}}	%Centre Bas
\lfoot{}											%Gauche Bas

% \texttt{Paris 1 - Panthéon Sorbonne}

\author{Fabrice BOISSIER}
\begin{document}

\setlength{\fboxrule}{2pt}

\noindent \framebox[17cm]{
\begin{minipage}{1\textwidth}
     \begin{center}
          \LARGE\textbf{INTERROGATION 1} \\
          \normalsize\textbf{Architecture des Ordinateurs ; Déplacement dans le Shell}
     \end{center}
\end{minipage}
}

\bigskip

NOM : \hspace{6.5cm} PR\'ENOM :

\smallskip

\section{Convertir ces nombres décimaux en binaires sur 8 bits : \texttt{144}, \texttt{-120}}

\bigskip
144 : \% 1001 0000 (\$ 90)	\qquad	-120 : \% 1000 1000 (\$ 88)
\bigskip

\section{Convertir ces nombres binaires (8 bits signés et non signés) en décimaux : \texttt{\%1110 0110}, \texttt{\%1001 0111}}

\bigskip
\% 1110 0110 : 230 ou -26	\qquad	\% 1001 0111 : 151 ou -105
\bigskip

\section{Nommer le langage que comprennent les processeurs, puis son équivalent lisible par un.e humain.e}

\bigskip
Le langage machine (suite de 0 et 1/binaire) est compris par les processeurs, l'assembleur est son équivalent lisible par un humain.
\bigskip

\section{Citer les 3 bus d'un processeur, puis les 3 classes générales d'ordinateurs}

\bigskip
Bus de Données, Bus d'Adresses, Bus de Contrôle \qquad Mainframes, Minis, Micros
\bigskip

\section{Expliquer le fonctionnement de la pile, puis faites un petit schéma représentant l'appel de la fonction suivante : \texttt{toupper("Aie", str)} où \textit{str} aura pour adresse \texttt{@1440}}

\bigskip

\noindent Une pile ou LIFO (Last In First Out) est une forme de liste d'éléments, où les éléments sont accédés par ordre inverse d'ajout.
En d'autres termes, le dernier élément ajouté à la pile est le premier que l'on récupèrera.
La pile s'oppose à la file.

\medskip

\noindent Dans la plupart des processeurs, il existe une pile permettant le passage d'arguments aux fonctions.

\bigskip

\noindent Pour appeler la fonction \textbf{toupper} avec en premier paramètre "Aie" (chaîne de caractères) et str (adresse 1440), deux possibilités existent :
\begin{itemize}
\item cas d'un processeur 32 bits : on place la chaîne "aie" dans un mot de 32 bits, et 1440 dans un autre mot, et on les empile \\

\begin{tabular}{| c |}
\hline
 \\ \hline
"a", "i", "e", \textbackslash 0 \\ \hline
1440 \\ \hline
... \\ \hline
... \\ \hline
\end{tabular}

\item cas général : on alloue "aie" en mémoire, puis on empile son adresse, puis celle de str \\

\begin{tabular}{| c |}
\hline
 \\ \hline
adresse du pointeur vers la chaîne "aie" \\ \hline
1440 \\ \hline
... \\ \hline
... \\ \hline
\end{tabular}

\end{itemize}



\end{document}
