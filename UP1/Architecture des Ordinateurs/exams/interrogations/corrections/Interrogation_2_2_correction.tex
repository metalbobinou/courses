\documentclass[11pt,a4paper]{article}
\usepackage[utf8]{inputenc}
\usepackage[french]{babel}
\usepackage[T1]{fontenc}

\usepackage{amsmath}
\usepackage{amsfonts}
\usepackage{amssymb}

\usepackage[left=2cm,right=2cm,top=2cm,bottom=2cm]{geometry}

\usepackage{courier}

\usepackage{fancyhdr}

\usepackage{tabularx}

\pagestyle{fancy}														% definition du style : FancyHDR
\renewcommand{\footrulewidth}{0.4pt}									% Ligne au dessus du footer
\fancyhf{}																% Remise a zero des entetes
\newenvironment{vcenterpage}
{\newpage\vspace*{\fill}}
{\vspace*{\fill}\par\pagebreak}

%%%%%%%%%%%%%%%%%%%%%%%
%Header
%%%%%%%%%%%%%%%%%%%%%%%
\rhead{Groupe 2}									%Droite Haut
\chead{18 décembre 2017}							%Centre Haut
\lhead{Interrogation 2}								%Gauche Haut
\rfoot{}											%Droite Bas
\cfoot{\texttt{Architecture des Ordinateurs et Systèmes d'Exploitation}}	%Centre Bas
\lfoot{}											%Gauche Bas

% \texttt{Paris 1 - Panthéon Sorbonne}

\author{Fabrice BOISSIER}
\begin{document}

\setlength{\fboxrule}{2pt}

\noindent \framebox[17cm]{
\begin{minipage}{1\textwidth}
     \begin{center}
          \LARGE\textbf{INTERROGATION 2} \\
          \normalsize\textbf{Système d'Exploitation et Shell}
     \end{center}
\end{minipage}
}

\bigskip

NOM : \hspace{6.5cm} PR\'ENOM :

\smallskip

\section{Citer les 6 appels systèmes UNIX permettant des manipulations sur le système de fichier : }

\bigskip
open, read, write, close, lseek, stat
\bigskip

\section{Indiquer les commandes pour donner les bons droits aux fichiers suivants :}

\medskip

\noindent fichier1 (lecture et exécution pour le propriétaire, tous les droits pour le groupe, et aucun droit pour les autres)

\bigskip

chmod 570 fichier1 \qquad chmod u=rx,g=rwx,o= fichier1

\bigskip

\noindent fichier2 (écriture pour le propriétaire, lecture pour le groupe, exécution pour les autres)

\bigskip

chmod 241 fichier2 \qquad chmod u=w,g=r,o=x fichier2

\bigskip

\noindent \#monfichier\# (tous les droits pour le propriétaire, lecture et écriture pour tous les autres)

\bigskip

chmod 766 \textbackslash \#monfichier\textbackslash \# \qquad chmod u=rwx,go=rw \textbackslash \#monfichier\textbackslash \#

\bigskip

\section{\'Ecrire la ligne de commande permettant de créer une archive \newline \texttt{archive.tar.bz2} contenant le répertoire \texttt{UnDossier} : }

\bigskip
\texttt{tar cvjf archive.tar.bz2 UnDossier}
\bigskip

\section{Remplir les phrases suivantes avec les bons mots : }

\bigskip
\noindent Un système d'exploitation \textbf{préemptif} permet de stopper des processus/tâches en cours d'exécution pour pouvoir donner du temps processeur à d'autres. On dit aussi que le système est à \textbf{temps partagé}.\\


\noindent La méthode d'accès \textbf{aléatoire} permet d'accéder directement à la donnée souhaitée, tandis que la méthode d'accès \textbf{séquentielle} nécessite de lire toutes les données précédentes depuis le début du support.

\bigskip

\section{Remettre dans l'ordre les phases de compilation, indiquer la commande du principal compilateur/linkeur et ses options si nécessaire}

\renewcommand\arraystretch{2.5}

\bigskip
\begin{center}
  \begin{tabularx}{15.5cm}{| c | c | c | X |}
  \hline
  \No \'Etape & Nom de l'\'Etape & Programme/Commande & Paramètre(s) \\ \hline
  1 & \textbf{Pré-Compilation} & gcc (OU cpp) & -E\\ 
  \hline
  2 & \textbf{Compilation} & gcc (OU cc1) & -S \\
  \hline
  3 & \textbf{Assemblage} & gcc (OU as) & -c \\
  \hline
  4 & \textbf{Link Edit/\'Edition de Liens} & gcc (OU ld) & \\
  \hline
  \end{tabularx}
\end{center}
\medskip

\renewcommand\arraystretch{1}


\section{Remplir le tableau avec les commandes ou leur description}

\renewcommand\arraystretch{2.5}

\bigskip
\begin{center}
  \begin{tabular}{| c | c |}
  \hline
  \textbf{Commande/Programme} & \textbf{Description} \\ \hline
  kill & Envoyer un signal à un processus \\ \hline
  ps & Afficher la liste des processus \\ \hline
  mv & Renommer un fichier \\ \hline
  rm & Supprimer un fichier \\ \hline
  rmdir & Supprimer un dossier \\ \hline
  ln & Créer un lien symbolique ou phyique vers un fichier \\ \hline
  bg & Relancer en arrière plan une tâche \\ \hline
  touch & Modifier la date de modification d'un fichier \\ \hline
  awk & Programme de traitement de texte par colonne \\ \hline
  cp & Copier un fichier ou dossier \\ \hline
  mv & Déplacer un dossier \\
  \hline
  \end{tabular}
\end{center}
\bigskip

\renewcommand\arraystretch{1}

\section{Convertir ces nombres décimaux en binaires sur 8 bits : \texttt{222}, \texttt{-14}}

\bigskip
222 : \% 1101 1110 (\$ DE)	\qquad	-14 : \% 1111 0010 (\$ F2)
\bigskip

\section{Convertir ces nombres binaires (8 bits signés et non signés) en décimaux : \texttt{\%1000 1101}, \texttt{\%1011 1101}}

\bigskip
\% 1000 1101 : -115 ou 141	\qquad	\% 1011 1101 : 189 ou -67
\bigskip

\section{\'Ecrire un script sh qui affichera \texttt{"Coucou !"} si on lui donne en unique paramètre \texttt{"Bonjour ?"}, mais qui affichera \texttt{"Non."} si on lui donne d'autres paramètres (multiples ou non), et n'affichera rien si aucun paramètre n'est donné.}

\begin{verbatim}
if [ $# -ne 0 ]; then
  if [ "$1" = "Bonjour ?" ]; then
    echo "Coucou !"
  else
    echo "Non."
  fi
fi
\end{verbatim}

\end{document}
