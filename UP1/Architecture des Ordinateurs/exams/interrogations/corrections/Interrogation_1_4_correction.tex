\documentclass[11pt,a4paper]{article}
\usepackage[utf8]{inputenc}
\usepackage[french]{babel}
\usepackage[T1]{fontenc}

\usepackage{amsmath}
\usepackage{amsfonts}
\usepackage{amssymb}

\usepackage[left=2cm,right=2cm,top=2cm,bottom=2cm]{geometry}

\usepackage{courier}

\usepackage{fancyhdr}

\pagestyle{fancy}														% definition du style : FancyHDR
\renewcommand{\footrulewidth}{0.4pt}									% Ligne au dessus du footer
\fancyhf{}																% Remise a zero des entetes
\newenvironment{vcenterpage}
{\newpage\vspace*{\fill}}
{\vspace*{\fill}\par\pagebreak}

%%%%%%%%%%%%%%%%%%%%%%%
%Header
%%%%%%%%%%%%%%%%%%%%%%%
\rhead{Groupe X}									%Droite Haut
\chead{10 novembre 2017}							%Centre Haut
\lhead{Interrogation 1}								%Gauche Haut
\rfoot{}											%Droite Bas
\cfoot{\texttt{Architecture des Ordinateurs et Systèmes d'Exploitation}}	%Centre Bas
\lfoot{}											%Gauche Bas

% \texttt{Paris 1 - Panthéon Sorbonne}

\author{Fabrice BOISSIER}
\begin{document}

\setlength{\fboxrule}{2pt}

\noindent \framebox[17cm]{
\begin{minipage}{1\textwidth}
     \begin{center}
          \LARGE\textbf{INTERROGATION 1} \\
          \normalsize\textbf{Architecture des Ordinateurs ; Déplacement dans le Shell}
     \end{center}
\end{minipage}
}

\bigskip

NOM : \hspace{6.5cm} PR\'ENOM :

\smallskip

\section{Convertir ces nombres décimaux en binaires sur 8 bits : \texttt{124}, \texttt{-66}}

\bigskip
124 : \% 0111 1100 (\$ 7C)	\qquad	-66 : \% 1011 1110 (\$ BE)
\bigskip

\section{Convertir ces nombres binaires (8 bits signés et non signés) en décimaux : \texttt{\%1001 1111}, \texttt{\%1010 0010}}

\bigskip
\% 1001 1111 : -97 ou 159	\qquad	\% 1010 0010 : 162 ou -94
\bigskip

\section{Citer les 3 types de périphériques, et les 2 types d'interruptions d'un processeur}

\bigskip
Périphérique d'entrée, de sortie, d'entrée/sortie \qquad Interruptions matérielles et logicielles
\bigskip

\section{Citer 5 flags dont disposent généralement tous les processeurs}

\bigskip
Zero, Negative, Carry, Overflow, Supervisor
\bigskip

\section{Expliquer comment un système d'exploitation charge un programme pour l'exécuter}

\bigskip

\noindent L'OS va tout d'abord chercher le fichier contenant le programme sur le disque dur en résolvant le chemin où il se trouve.
Puis, un processus va être créé, et l'OS va lire le contenu du fichier pour le transférer en mémoire dans l'espace d'adressage alloué au processus.
Enfin, le processeur sautera (jump ou call) à l'adresse du début du programme, et exécutera les instructions.

\end{document}
