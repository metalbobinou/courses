\documentclass[11pt,a4paper]{article}
\usepackage[utf8]{inputenc}
\usepackage[french]{babel}
\usepackage[T1]{fontenc}

\usepackage{amsmath}
\usepackage{amsfonts}
\usepackage{amssymb}

\usepackage[left=2cm,right=2cm,top=2cm,bottom=2cm]{geometry}

\usepackage{courier}

\usepackage{fancyhdr}

\pagestyle{fancy}														% definition du style : FancyHDR
\renewcommand{\footrulewidth}{0.4pt}									% Ligne au dessus du footer
\fancyhf{}																% Remise a zero des entetes
\newenvironment{vcenterpage}
{\newpage\vspace*{\fill}}
{\vspace*{\fill}\par\pagebreak}

%%%%%%%%%%%%%%%%%%%%%%%
%Header
%%%%%%%%%%%%%%%%%%%%%%%
\rhead{Groupe 1}									%Droite Haut
\chead{09 novembre 2017}							%Centre Haut
\lhead{Interrogation 1}								%Gauche Haut
\rfoot{}											%Droite Bas
\cfoot{\texttt{Architecture des Ordinateurs et Systèmes d'Exploitation}}	%Centre Bas
\lfoot{}											%Gauche Bas

% \texttt{Paris 1 - Panthéon Sorbonne}

\author{Fabrice BOISSIER}
\begin{document}

\setlength{\fboxrule}{2pt}

\noindent \framebox[17cm]{
\begin{minipage}{1\textwidth}
     \begin{center}
          \LARGE\textbf{INTERROGATION 1} \\
          \normalsize\textbf{Architecture des Ordinateurs ; Déplacement dans le Shell}
     \end{center}
\end{minipage}
}

\bigskip

NOM : \hspace{6.5cm} PR\'ENOM :

\smallskip

\section{Convertir ces nombres décimaux en binaires sur 8 bits : \texttt{166}, \texttt{-47}}

\bigskip
166 : \% 1010 0110 (\$ A6)	\qquad	-47 : \% 1101 0001 (\$ D1)
\bigskip

\section{Convertir ces nombres binaires (8 bits signés et non signés) en décimaux : \texttt{\%1011 1001}, \texttt{\%1101 1101}}

\bigskip
\% 1011 1001 : 185 ou -71	\qquad	\% 1101 1101 : 221 ou -35
\bigskip

\section{Citer les 3 bus d'un processeur, puis les 3 classes générales d'ordinateurs}

\bigskip
Bus de Données, Bus d'Adresses, Bus de Contrôle \qquad Mainframes, Minis, Micros
\bigskip

\section{Citer les 5 étages classiques du pipeline d'un processeur}

\bigskip
Fetch, Decode, Execute, Memory, Write Back
\bigskip

\section{Expliquer les relations entre les différents langages : "langage machine", "assembleur", "C"}

\bigskip

\noindent Le langage machine est le langage compris par les processeurs.
Il est souvent représenté par du binaire (0 et 1), et correspond aux instructions exécutées par le processeur.
Chaque famille de processeurs comprend un langage machine précis.
Le code machine n'est donc pas portable d'une architecture à une autre.

\bigskip

\noindent L'assembleur est la version "lisible (par un humain)" du langage machine.
Chaque instruction du langage machine correspond à une instruction assembleur (et inversement).
Il n'y a aucune différence de "logique" dans le code des instructions assembleur et machine.
Chaque famille de processeurs comprend un assembleur précis.
Le code assembleur n'est donc pas portable d'une architecture à une autre.

\bigskip

\noindent Le langage C est un langage haut niveau plutôt indépendant du processeur sous-jacent.
On peut écrire du code dans une logique plus humaine : on peut créer des boucles et conditions complexes en une seule ligne, là où l'assembleur nécessiterait plusieurs instructions.
Le C permet de s'abtraire légèrement du fonctionnement exact d'un processeur.
Pour un même processeur, un seul code C peut générer plusieurs versions différentes de code assembleur.
Un code C est portable sur plusieurs architectures différentes de processeurs sans changement majeur.

\end{document}
