\documentclass[11pt,a4paper]{article}
\usepackage[utf8]{inputenc}
\usepackage[french]{babel}
\usepackage[T1]{fontenc}

\usepackage{amsmath}
\usepackage{amsfonts}
\usepackage{amssymb}

\usepackage[left=2cm,right=2cm,top=2cm,bottom=2cm]{geometry}

\usepackage{courier}

\newcommand{\TTBF}[1]{\texttt{\textbf{#1}}}

\usepackage{fancyhdr}

\pagestyle{fancy}														% definition du style : FancyHDR
\renewcommand{\footrulewidth}{0.4pt}									% Ligne au dessus du footer
\fancyhf{}																% Remise a zero des entetes
\newenvironment{vcenterpage}
{\newpage\vspace*{\fill}}
{\vspace*{\fill}\par\pagebreak}

%%%%%%%%%%%%%%%%%%%%%%%%%%%%%%%%%%%%%%%%%%%%%%%%%%%%%%
% MODIF TITRES ("Exercice 1", "Question 1.1", ...)

\usepackage{titlesec} % Personaliser les titres, sections, ...

\titleformat{\chapter}[display]
  {\normalfont\huge\bfseries}{Partie \thechapter : }  % {\chaptertitlename\ \thechapter -}
  {20pt}{\Huge}

\titleformat{\section}
  {\normalfont\Large\bfseries}{Exercice \thesection : }{1em}{}

\titleformat{\subsection}
  {\normalfont\large\bfseries}{Question \thesubsection : }{1em}{}

% FIN MODIF TITRES
%%%%%%%%%%%%%%%%%%%%%%%%%%%%%%%%%%%%%%%%%%%%%%%%%%%%%%


%%%%%%%%%%%%%%%%%%%%%%%
%Header
%%%%%%%%%%%%%%%%%%%%%%%
\rhead{11 janvier 2018}								%Droite Haut
\chead{Paris 1 - Panthéon Sorbonne}					%Centre Haut
\lhead{Partiel}										%Gauche Haut
\rfoot{}											%Droite Bas
\cfoot{\texttt{Architecture des Ordinateurs et Systèmes d'Exploitation}}	%Centre Bas
\lfoot{}											%Gauche Bas

\usepackage{xcolor}

\definecolor{light-gray}{gray}{0.95}	% Gray model : 1 (white), 0 (dark)
%\definecolor{bg-gray}{gray}{0.97}	% Gray model : 1 (white), 0 (dark)
\definecolor{mauve}{rgb}{0.58,0,0.82}	% RGB model : R (0 - 1), G (0 - 1), B (0 - 1)

\usepackage{listings} % Affichage du code dans le document

\lstset{ %
  backgroundcolor=\color{light-gray},    % choose the background color; should come as last argument
  basicstyle=\ttfamily,            % the size of the fonts that are used for the code (e.g \footnotesize\ttfamily)
  breakatwhitespace=false,         % sets if automatic breaks should only happen at whitespace
  breaklines=true,                 % sets automatic line breaking
  captionpos=b,                    % sets the caption-position to bottom
  commentstyle=\color{red},      % comment style
  deletekeywords={...},            % if you want to delete keywords from the given language
  escapeinside={\%*}{*)},          % if you want to add LaTeX within your code
  extendedchars=true,              % lets you use non-ASCII characters; for 8-bits encodings only, does not work with UTF-8
  frame=single,	                   % adds a frame around the code
  framesep=10pt,                   % separate the code from the border
  keepspaces=true,                 % keeps spaces in text, useful for keeping indentation of code (possibly needs columns=flexible)
  keywordstyle=\color{blue},       % keyword style
  language=Octave,                 % the language of the code
  morekeywords={*,...},            % if you want to add more keywords to the set
  numbers=none,                    % where to put the line-numbers; possible values are (none, left, right)
  numbersep=5pt,                   % how far the line-numbers are from the code
  numberstyle=\tiny\color{gray},   % the style that is used for the line-numbers
  rulecolor=\color{black},         % if not set, the frame-color may be changed on line-breaks within not-black text (e.g. comments (green here))
  showspaces=false,                % show spaces everywhere adding particular underscores; it overrides 'showstringspaces'
  showstringspaces=false,          % underline spaces within strings only
  showtabs=false,                  % show tabs within strings adding particular underscores
  stepnumber=1,                    % the step between two line-numbers. If it's 1, each line will be numbered
  stringstyle=\color{mauve},       % string literal style
  tabsize=2,	                   % sets default tabsize to 2 spaces
  title=\lstname                   % show the filename of files included with \lstinputlisting; also try caption instead of title
}

%%%%%%%%%%%%%%%%%%%%%%%%%%%%%%%%%%%%%%%%%%%%%%%%%%%%%%

\author{Fabrice BOISSIER}
\begin{document}

\setlength{\fboxrule}{2pt}

\noindent \framebox[17cm]{
\begin{minipage}{1\textwidth}
     \begin{center}
          \LARGE\textbf{PARTIEL 2017-2018 - L2 (2h00)} \\
          \normalsize\textbf{Architecture des Ordinateurs et Systèmes d'Exploitation}
     \end{center}
\end{minipage}
}

%\bigskip

%%%%%%%%%%%%%%%%%%%%%%%%%%%%%%%%%%%%%%%%%%%%%%%%

%NOM : \hspace{6.5cm} PR\'ENOM :
%
%\smallskip

%%%%%%%%%%%%%%%%%%%%%%%%%%%%%%%%%%%%%%%%%%%%%%%%

\section{Conversions (2 points)} % 15 mins

\subsection{(1 point) Convertir ces nombres décimaux en binaires sur 8 bits : \texttt{164}, \texttt{-34}}

\bigskip

\subsection{(1 point) Convertir ces nombres binaires (8 bits signés et non signés) en décimaux : \texttt{\%1111 0101}, \texttt{\%1101 1011}}

\bigskip
\bigskip
\bigskip

\section{Compilation (2 points)} % 10 mins

\subsection{(2 points) Expliquer succinctement les 4 étapes du pipeline de compilation. (une ou deux phrase(s) par étape)}

\bigskip
\bigskip
\bigskip

\section{Architecture (4 points)} % 20 mins

\subsection{(2 points) Expliquer succinctement les 5 principales étapes qu'un processeur effectue lorsqu'il lit et exécute une instruction. (une phrase par étape)}

\bigskip

\subsection{(1 point) Expliquer succinctement quelles sont les fonctions/l'utilité du processeur, de la mémoire, et des périphériques dans un ordinateur.}

\bigskip

\subsection{(1 point) Pour un même programme dont les sources sont compilées pour un processeur CISC puis pour un processeur RISC, expliquer quelles seront les différences dans les deux binaires exécutables finaux. En admettant que les deux processeurs fonctionnent exactement à la même vitesse (chaque instruction prend un temps fixe à s'exécuter sur les deux processeurs), quelles seraient les conséquences lors de l'exécution ?}

\bigskip
\bigskip
\bigskip

%\pagebreak

\section{Système d'Exploitation (12 points)} % 20 mins

\subsection{(2 points) Expliquer la notion de "temps partagé". Citer le composant du système d'exploitation qui gère cette notion, le mécanisme du processeur sur lequel il s'appuie, et le composant physique permettant cela. Puis illustrer avec un exemple où le temps est partagé puis un autre exemple où le temps n'est pas partagé.}

\bigskip

\subsection{(1,5 points) Expliquer la différence entre un programme et un processus, et donner les propriétés classiques d'un processus.}

\bigskip

\pagebreak

\subsection{(1 point) Expliquer la différence entre processus et threads.}

\bigskip

\subsection{(2 points) Expliquer la relation entre i-node et blocs, puis expliquer la différence entre fichier et dossier et particulièrement où sont stockés les noms de fichiers.}

\bigskip

\subsection{(1,5 points) Donnez 3 raisons différentes qui peuvent empêcher cette ligne de commande de fonctionner : \TTBF{./mon\_script.sh}}

\bigskip
% droit exécution pas donné
% erreur dans le script
% interpréteur indiqué n'existe pas

\subsection{(2 points) Expliquer ce que fait chaque ligne du script, puis globalement ce que le script semble faire.}

\bigskip

\lstset{language=sh}
\begin{lstlisting}[frame=single,numbers=left,title={script.sh}]
#! /bin/sh

OUTFILE=`mktemp fileXXXX`
cat streams.list | sort -t";" -k1 -o $OUTFILE
NB=`wc -l streams.list | cut -c 1`
rm -f $1
for i in `seq 0 $NB`; do
	FILE=`tail -n $i $OUTFILE | head -n 1`
	cat "${FILE}$i" >> $1
done
rm -f $OUTFILE

\end{lstlisting}

\bigskip

%\pagebreak

\subsection{(2 points) \'Ecrire un script qui listera tous les fichiers commençant par le nom "transaction-" modifié durant les 24 dernières heures, puis afficher pour chaque fichier la colonne contenant le montant de chacune des lignes, et renvoyer le tout vers le fichier passé en premier paramètre du script.}

\bigskip

\lstset{language=sh}
\begin{lstlisting}[frame=single,title={Format des fichiers de transactions}]
ID Transaction ; Date ; Montant ; Numero Carte Fidelite ; Remarques
\end{lstlisting}

\lstset{language=sh}
\begin{lstlisting}[frame=single,title={transaction-2017-12-08}]
1445;2017-12-08;42;1001;fait a 11h01
1446;2017-12-08;200;1002;fait a 11h11
1447;2017-12-08;100;1001;fait a 12h04
\end{lstlisting}

\end{document}


Par coeur :
- find ?
- droits ?
- Détaillez les différentes parties de l'espace d'adressage d'un processus.
- Expliquer la différence entre un programme et un processus, et donner les propriétés classiques d'un processus.
- Quelles sont les différence entre processus et threads ?
- Citer 5 informations que contient un i-node.
- Citer et expliquer l'utilité de 4 variables d'environnement

Raisonnement :
- Quels composants manquent pour faire un ordi ? (souris, clavier, ...)
- Quel exemple illustre le mieux le temps partagé, la préemption, l'accès séquentiel, l'accès aléatoire, ...
- Remplissez l'ordonnanceur de façon optimale pour que les lectures/écritures soient le moins en attente, en faisant en sorte que : le processus 687 s'exécute au moins 3 fois, le processus 703 s'exécute 2 fois, le processus 812 s'exécute 3 fois. Prenez en compte ces contraintes : le processus 812 va faire un seul read de 687, 687 va écrire une fois lors d'un quantum puis une deuxième fois lors d'un autre quantum.
- Expliquer pourquoi deux processus sur une même machine, dans une même VM, utilisant la même instance de l'OS, ne peuvent pas échanger de données sans passer par l'OS/le kernel ? Donner en complément des mécanismes bas niveaux qui permettent de déplacer des données, et des mécanismes de l'OS qui peuvent être utilisés pour transmettre des données entre des processus. (pipes, signal/kill, fichier, )
- Donnez toutes les raisons qui peuvent empêcher cette ligne de commande de fonctionner : \TTBF{./script.sh}
- Donnez toutes les raisons qui peuvent empêcher de créer un fichier dans un dossier.