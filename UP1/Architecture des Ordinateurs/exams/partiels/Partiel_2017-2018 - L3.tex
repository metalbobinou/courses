\documentclass[11pt,a4paper]{article}
\usepackage[utf8]{inputenc}
\usepackage[french]{babel}
\usepackage[T1]{fontenc}

\usepackage{amsmath}
\usepackage{amsfonts}
\usepackage{amssymb}

\usepackage[left=2cm,right=2cm,top=2cm,bottom=2cm]{geometry}

\usepackage{courier}

\newcommand{\TTBF}[1]{\texttt{\textbf{#1}}}

\usepackage{fancyhdr}

\pagestyle{fancy}														% definition du style : FancyHDR
\renewcommand{\footrulewidth}{0.4pt}									% Ligne au dessus du footer
\fancyhf{}																% Remise a zero des entetes
\newenvironment{vcenterpage}
{\newpage\vspace*{\fill}}
{\vspace*{\fill}\par\pagebreak}

%%%%%%%%%%%%%%%%%%%%%%%%%%%%%%%%%%%%%%%%%%%%%%%%%%%%%%
% MODIF TITRES ("Exercice 1", "Question 1.1", ...)

\usepackage{titlesec} % Personaliser les titres, sections, ...

\titleformat{\chapter}[display]
  {\normalfont\huge\bfseries}{Partie \thechapter : }  % {\chaptertitlename\ \thechapter -}
  {20pt}{\Huge}

\titleformat{\section}
  {\normalfont\Large\bfseries}{Exercice \thesection : }{1em}{}

\titleformat{\subsection}
  {\normalfont\large\bfseries}{Question \thesubsection : }{1em}{}

% FIN MODIF TITRES
%%%%%%%%%%%%%%%%%%%%%%%%%%%%%%%%%%%%%%%%%%%%%%%%%%%%%%


%%%%%%%%%%%%%%%%%%%%%%%
%Header
%%%%%%%%%%%%%%%%%%%%%%%
\rhead{22 décembre 2017}								%Droite Haut
\chead{Paris 1 - Panthéon Sorbonne}					%Centre Haut
\lhead{Partiel}										%Gauche Haut
\rfoot{}											%Droite Bas
\cfoot{\texttt{Architecture des Ordinateurs et Systèmes d'Exploitation}}	%Centre Bas
\lfoot{}											%Gauche Bas

\usepackage{xcolor}

\definecolor{light-gray}{gray}{0.95}	% Gray model : 1 (white), 0 (dark)
%\definecolor{bg-gray}{gray}{0.97}	% Gray model : 1 (white), 0 (dark)
\definecolor{mauve}{rgb}{0.58,0,0.82}	% RGB model : R (0 - 1), G (0 - 1), B (0 - 1)

\usepackage{listings} % Affichage du code dans le document

\lstset{ %
  backgroundcolor=\color{light-gray},    % choose the background color; should come as last argument
  basicstyle=\ttfamily,            % the size of the fonts that are used for the code (e.g \footnotesize\ttfamily)
  breakatwhitespace=false,         % sets if automatic breaks should only happen at whitespace
  breaklines=true,                 % sets automatic line breaking
  captionpos=b,                    % sets the caption-position to bottom
  commentstyle=\color{red},      % comment style
  deletekeywords={...},            % if you want to delete keywords from the given language
  escapeinside={\%*}{*)},          % if you want to add LaTeX within your code
  extendedchars=true,              % lets you use non-ASCII characters; for 8-bits encodings only, does not work with UTF-8
  frame=single,	                   % adds a frame around the code
  framesep=10pt,                   % separate the code from the border
  keepspaces=true,                 % keeps spaces in text, useful for keeping indentation of code (possibly needs columns=flexible)
  keywordstyle=\color{blue},       % keyword style
  language=Octave,                 % the language of the code
  morekeywords={*,...},            % if you want to add more keywords to the set
  numbers=none,                    % where to put the line-numbers; possible values are (none, left, right)
  numbersep=5pt,                   % how far the line-numbers are from the code
  numberstyle=\tiny\color{gray},   % the style that is used for the line-numbers
  rulecolor=\color{black},         % if not set, the frame-color may be changed on line-breaks within not-black text (e.g. comments (green here))
  showspaces=false,                % show spaces everywhere adding particular underscores; it overrides 'showstringspaces'
  showstringspaces=false,          % underline spaces within strings only
  showtabs=false,                  % show tabs within strings adding particular underscores
  stepnumber=1,                    % the step between two line-numbers. If it's 1, each line will be numbered
  stringstyle=\color{mauve},       % string literal style
  tabsize=2,	                   % sets default tabsize to 2 spaces
  title=\lstname                   % show the filename of files included with \lstinputlisting; also try caption instead of title
}

%%%%%%%%%%%%%%%%%%%%%%%%%%%%%%%%%%%%%%%%%%%%%%%%%%%%%%

\author{Fabrice BOISSIER}
\begin{document}

\setlength{\fboxrule}{2pt}

\noindent \framebox[17cm]{
\begin{minipage}{1\textwidth}
     \begin{center}
          \LARGE\textbf{PARTIEL 2017-2018 - L3 (1h30)} \\
          \normalsize\textbf{Architecture des Ordinateurs et Systèmes d'Exploitation}
     \end{center}
\end{minipage}
}

%\bigskip

%%%%%%%%%%%%%%%%%%%%%%%%%%%%%%%%%%%%%%%%%%%%%%%%

%NOM : \hspace{6.5cm} PR\'ENOM :
%
%\smallskip

%%%%%%%%%%%%%%%%%%%%%%%%%%%%%%%%%%%%%%%%%%%%%%%%

\section{Conversions} % 15 mins

\subsection{(1 point) Convertir ces nombres décimaux en binaires sur 8 bits : \texttt{125}, \texttt{-74}}

\bigskip

\subsection{(2 points) Convertir ces nombres binaires (8 bits signés et non signés) en décimaux : \texttt{\%1011 1100}, \texttt{\%1101 1010}}

\bigskip
\bigskip
\bigskip

\section{Compilation} % 10 mins

\subsection{(2 points) Expliquer succinctement les principales étapes de compilation. (une ou deux phrase(s) par étape).}

\bigskip

\subsection{BONUS : (2 points) \'Ecrivez le code C résultant de l'opération de pré-compilation sur \texttt{main.c}, puis expliquez pourquoi ce code ne pourra pas générer à lui tout seul un exécutable (et comment faire pour corriger cela).}

\bigskip

\lstset{language=c}
\begin{lstlisting}[frame=single,title={header.h}]
#define STDIN 0
#define STDOUT 1
#define STDERR 2

int my_read(int fd, int buf, char *str);
int my_write(int fd, int buf, char *str);
void *malloc(size_t size);
void free(void ptr);
\end{lstlisting}

\lstset{language=c}
\begin{lstlisting}[frame=single,title={main.c}]
#include "header.h"

int   main(void)
{
  char *tab;

  tab = malloc(4 * sizeof(char));
  my_read(STDIN, 3, tab);
  my_write(STDOUT, 4, "Test");
  free(tab);
  return (0);
}
\end{lstlisting}

\bigskip
\bigskip
\bigskip

\pagebreak

\section{Architecture} % 20 mins

\subsection{(1 point) Expliquer succinctement comment le processeur lit une donnée en mémoire (comment les bus sont utilisés).}

\bigskip

\subsection{(1 point) Donner un exemple de code provoquant un overflow (le langage peut être du C, sh, python, ou autre langage de développement).}

\bigskip

\subsection{(2 points) Expliquer succinctement quelles sont les fonctions du processeur, de la mémoire, et des périphériques dans un ordinateur.}

\bigskip
\bigskip
\bigskip

\section{Système d'Exploitation} % 20 mins

\subsection{(2 points) Expliquer la notion de "temps partagé", citer le composant du système qui gère cette notion, et illustrer avec un exemple où le temps est partagé puis un autre exemple où le temps n'est pas partagé.}

\bigskip

\subsection{(2 points) Détailler les différentes parties de l'espace d'adressage d'un processus, puis donner 2 exemples différents où un accès mémoire ne fonctionnera pas (et expliquer pourquoi).}

\bigskip

\subsection{(2 points) Citer 5 informations contenues dans un i-node.}

\bigskip

\subsection{(1 point) Donnez 3 raisons différentes qui peuvent empêcher de créer un fichier dans un dossier.}

\bigskip
% pas le droit de créer fichier
% pas d'espace disque
% périphériques visé pas disponible

\subsection{(2 points) Expliquer ce que fait chaque ligne du script. En admettant qu'une entreprise génère des transactions jusqu'à 20h, indiquez l'heure minimale et l'heure maximale pour lancer ce script et traiter les transactions du jour même.}

\bigskip

\lstset{language=sh}
\begin{lstlisting}[frame=single,numbers=left,title={script.sh}]
#! /bin/sh

OUTFILE=`mktemp fileXXXX`
cat transactions.log | tr "[:upper:]" "[:lower:]" | sort -t";" -k1 -o $OUTFILE
SUFFIXE=`date +%Y-%m-%d`
NAME=comptabilite-${SUFFIXE}
grep "$SUFFIXE" $OUTFILE > $NAME
rm -f $OUTFILE

\end{lstlisting}

\bigskip

\pagebreak

\subsection{(2 points) \'Ecrire un script qui listera tous les fichiers commençant par le nom "comptabilite-" suivi d'une date (format aaaa-mm-jj), puis afficher pour chaque fichier la colonne contenant le montant de chacune des lignes, et renvoyer le tout vers le fichier passé en premier paramètre du script.}

\bigskip

\lstset{language=sh}
\begin{lstlisting}[frame=single,title={Format des fichiers de transactions}]
ID Transaction ; Date ; Montant ; Numero Carte Fidelite ; Remarques
\end{lstlisting}

\lstset{language=sh}
\begin{lstlisting}[frame=single,title={comptabilite-2017-12-08}]
1445;2017-12-08;42;1001;fait a 11h01
1446;2017-12-08;200;1002;fait a 11h11
1447;2017-12-08;100;1001;fait a 12h04
\end{lstlisting}

\end{document}


Par coeur :
- find ?
- droits ?
- Détaillez les différentes parties de l'espace d'adressage d'un processus.
- Expliquer la différence entre un programme et un processus, et donner les propriétés classiques d'un processus.
- Quelles sont les différence entre processus et threads ?
- Citer 5 informations que contient un i-node.
- Citer et expliquer l'utilité de 4 variables d'environnement

Raisonnement :
- Quels composants manquent pour faire un ordi ? (souris, clavier, ...)
- Quel exemple illustre le mieux le temps partagé, la préemption, l'accès séquentiel, l'accès aléatoire, ...
- Remplissez l'ordonnanceur de façon optimale pour que les lectures/écritures soient le moins en attente, en faisant en sorte que : le processus 687 s'exécute au moins 3 fois, le processus 703 s'exécute 2 fois, le processus 812 s'exécute 3 fois. Prenez en compte ces contraintes : le processus 812 va faire un seul read de 687, 687 va écrire une fois lors d'un quantum puis une deuxième fois lors d'un autre quantum.
- Expliquer pourquoi deux processus sur une même machine, dans une même VM, utilisant la même instance de l'OS, ne peuvent pas échanger de données sans passer par l'OS/le kernel ? Donner en complément des mécanismes bas niveaux qui permettent de déplacer des données, et des mécanismes de l'OS qui peuvent être utilisés pour transmettre des données entre des processus. (pipes, signal/kill, fichier, )
- Donnez toutes les raisons qui peuvent empêcher cette ligne de commande de fonctionner : \TTBF{./script.sh}
- Donnez toutes les raisons qui peuvent empêcher de créer un fichier dans un dossier.