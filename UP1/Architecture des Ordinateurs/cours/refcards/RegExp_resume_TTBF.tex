\documentclass[11pt,a4paper]{article}
\usepackage[utf8]{inputenc}
\usepackage[french]{babel}
\usepackage[T1]{fontenc}

\usepackage{amsmath}
\usepackage{amsfonts}
\usepackage{amssymb}

\usepackage{courier}

\usepackage{textcomp}

\usepackage{array}
\newcolumntype{P}[1]{>{\centering\arraybackslash}p{#1}} %Cellule P centrée
\newcolumntype{M}[1]{>{\centering\arraybackslash}m{#1}} %Cellule M centrée
\newcolumntype{B}[1]{>{\centering\arraybackslash}b{#1}} %Cellule B centrée


\usepackage[left=2cm,right=2cm,top=2cm,bottom=2cm]{geometry}

\renewcommand{\thefootnote}{\*} % footnote sans numéro
\newcommand{\TTBF}[1]{\texttt{\textbf{#1}}}	% TTBF : police teletype, fonte bold

\author{Fabrice BOISSIER}

\begin{document}

\section{Expressions Rationnelles / Regular Expressions}

\bigskip

\subsection{Classes de caractères}

\bigskip

%\renewcommand\arraystretch{2.5}

\begin{center}
	\begin{tabular}{ c | M{4cm} | M{7cm} }
  \textbf{Nom de la classe} & \textbf{Signification} & \textbf{ASCII} \\  \hline
	alpha & Lettres alphabétiques & \TTBF{[A-Za-z]} \\ \hline
	digit & Chiffres décimaux & \TTBF{[0-9]} \\ \hline
	xdigit & Chiffres hexadécimaux & \TTBF{[0-9A-Fa-f]} \\ \hline
	alnum & Chiffres ou lettres alphabétiques & \TTBF{[[:alpha:][:digit:]]} \\ \hline
	lower & Lettres minuscules & \TTBF{\textquotesingle[a-z]\textquotesingle} \\ \hline
	upper & Lettres majuscules & \TTBF{[A-Z]} \\ \hline
	blank & Caractères blancs & \TTBF{[ \textbackslash t]} \\ \hline
	space & Caractères d'espacement & \TTBF{[ \textbackslash t\textbackslash n\textbackslash f\textbackslash r]} \\ \hline
	punct & Signes de ponctuation &  \TTBF{[]-!"\#\$\%\&\textquotesingle ()*.,/:+ ;<=>?@\textbackslash \textasciicircum \_\lq \{|\}\textasciitilde } \\ \hline
	graph & Symboles ayant une représentation graphique & \TTBF{[[:alnum:][:punct:]]} \\ \hline
	print & Caractères imprimables & \TTBF{[[:graph:]]} \\ \hline
	cntrl & Caractères de contrôle & Code ASCII inférieur à 31 et caractère de code 127 \\
	\end{tabular}
\end{center}

%\renewcommand\arraystretch{1}

\bigskip

\noindent Exemple : \TTBF{grep "H[ea]llo [[:digit:]]" file.c}

\newpage

\subsection{RegExp}

\bigskip

%\renewcommand\arraystretch{2.5}

\begin{center}
	\begin{tabular}{ M{5cm} | c | c }
	\textbf{Signification} & \textbf{Symbole RegExp simple} & \textbf{Symbole RegExp étendue} \\  \hline
	Un seul caractère générique & \TTBF{\textquotesingle .\textquotesingle} & \TTBF{\textquotesingle .\textquotesingle} \\ \hline
	Début de ligne & \texttt{\textquotesingle \textasciicircum \textquotesingle} & \texttt{\textquotesingle \textasciicircum \textquotesingle} \\ \hline
	Fin de ligne & \TTBF{\textquotesingle \$\textquotesingle} & \TTBF{\textquotesingle \$\textquotesingle} \\ \hline
	Alternative (OU) & \TTBF{\textquotesingle \textbackslash |\textquotesingle} & \TTBF{\textquotesingle |\textquotesingle} \\ \hline
	Liste de caractères & \TTBF{\textquotesingle []\textquotesingle} & \TTBF{\textquotesingle []\textquotesingle} \\ \hline
	Classe de caractères & \TTBF{\textquotesingle [:classe:]\textquotesingle} & \TTBF{\textquotesingle [:classe:]\textquotesingle} \\ \hline
	Zéro ou plusieurs occurrences de l'élément précédent & \TTBF{\textquotesingle *\textquotesingle} & \TTBF{\textquotesingle *\textquotesingle} \\ \hline
	Une ou plusieurs occurrences de l'élément précédent & \TTBF{\textquotesingle \textbackslash +\textquotesingle} & \TTBF{\textquotesingle +\textquotesingle} \\ \hline
	Zéro ou une occurrence de l'élément précédent & \TTBF{\textquotesingle \textbackslash ?\textquotesingle} & \TTBF{\textquotesingle ?\textquotesingle} \\ \hline
	Au moins 'n' et au plus 'm' occurrences de l'élément précédent & \TTBF{\textquotesingle \textbackslash \{n,m\textbackslash \}\textquotesingle} & \TTBF{\textquotesingle \{n,m\}\textquotesingle} \\ \hline
	Au moins 'n' occurrences de l'élément précédent & \TTBF{\textquotesingle \textbackslash \{n,\textbackslash \}\textquotesingle} & \TTBF{\textquotesingle \{n,\}\textquotesingle} \\ \hline
	Au plus 'm' occurrences de l'élément précédent & \TTBF{\textquotesingle \textbackslash \{0,m\textbackslash \}\textquotesingle} & \TTBF{\textquotesingle \{0,m\}\textquotesingle} \\ \hline
	Exactement 'n' occurrences de l'élément précédent & \TTBF{\textquotesingle \textbackslash \{n\textbackslash \}\textquotesingle} & \TTBF{\textquotesingle \{n\}\textquotesingle} \\ \hline
	Groupement de caractères & \TTBF{\textquotesingle \textbackslash ( \textbackslash )\textquotesingle} & \TTBF{\textquotesingle ( )\textquotesingle} \\ \hline
	Référence arrière au num-ième groupe & \TTBF{\textquotesingle \textbackslash num\textquotesingle} & \TTBF{\textquotesingle \textbackslash num\textquotesingle} \\ \hline
	\'Echappement de caractères spéciaux & \TTBF{\textquotesingle \textbackslash \textquotesingle} & \TTBF{\textquotesingle \textbackslash \textquotesingle}
	\end{tabular}
\end{center}

%\renewcommand\arraystretch{1}

\bigskip

\noindent \textbf{Toujours encadrer ses regexp avec des guillemets !}\\

\noindent \TTBF{echo "Hello 42" | grep "H[ea]llo [[:digit:]]\textbackslash \{2\textbackslash \}"}\\
\noindent \textrightarrow Hello 42\\

\noindent Attention aux regexp étendues, et à l'échappement de caractères.\\
\noindent \TTBF{echo "Hello 42" | grep -E "H[ea]llo [[:digit:]]\{2\}"}\\
\noindent \textrightarrow Hello 42\\

\noindent \TTBF{echo "Hello 42*Bla" | grep -E "H[ea]llo [[:digit:]]\{2\}\textbackslash *Bla"}\\
\noindent \textrightarrow Hello 42*Bla\\

\footnotetext{Merci les ACU !}

\end{document}