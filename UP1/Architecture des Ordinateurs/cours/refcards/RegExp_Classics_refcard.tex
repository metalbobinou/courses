\documentclass[10pt,landscape]{article}

\usepackage[utf8]{inputenc}
\usepackage[french]{babel}
\usepackage[T1]{fontenc}

\usepackage{textcomp}

\usepackage{multicol}
\usepackage{calc}
\usepackage{ifthen}
\usepackage[landscape]{geometry}
\usepackage{hyperref}

% This sets page margins to .5 inch if using letter paper, and to 1cm
% if using A4 paper. (This probably isn't strictly necessary.)
% If using another size paper, use default 1cm margins.
\ifthenelse{\lengthtest { \paperwidth = 11in}}
	{ \geometry{top=.5in,left=.5in,right=.5in,bottom=.5in} }
	{\ifthenelse{ \lengthtest{ \paperwidth = 297mm}}
		{\geometry{top=1cm,left=1cm,right=1cm,bottom=1cm} }
		{\geometry{top=1cm,left=1cm,right=1cm,bottom=1cm} }
	}

% Turn off header and footer
\pagestyle{empty}
 

% Redefine section commands to use less space
\makeatletter
\renewcommand{\section}{\@startsection{section}{1}{0mm}%
                                {-1ex plus -.5ex minus -.2ex}%
                                {0.5ex plus .2ex}%x
                                {\normalfont\large\bfseries}}
\renewcommand{\subsection}{\@startsection{subsection}{2}{0mm}%
                                {-1explus -.5ex minus -.2ex}%
                                {0.5ex plus .2ex}%
                                {\normalfont\normalsize\bfseries}}
\renewcommand{\subsubsection}{\@startsection{subsubsection}{3}{0mm}%
                                {-1ex plus -.5ex minus -.2ex}%
                                {1ex plus .2ex}%
                                {\normalfont\small\bfseries}}
\makeatother

% Define BibTeX command
\def\BibTeX{{\rm B\kern-.05em{\sc i\kern-.025em b}\kern-.08em
    T\kern-.1667em\lower.7ex\hbox{E}\kern-.125emX}}

% Don't print section numbers
\setcounter{secnumdepth}{0}


\setlength{\parindent}{0pt}
\setlength{\parskip}{0pt plus 0.5ex}


% -----------------------------------------------------------------------

\begin{document}

\raggedright
\footnotesize
\begin{multicols}{3}


% multicol parameters
% These lengths are set only within the two main columns
%\setlength{\columnseprule}{0.25pt}
\setlength{\premulticols}{1pt}
\setlength{\postmulticols}{1pt}
\setlength{\multicolsep}{1pt}
\setlength{\columnsep}{2pt}

\begin{center}
     \Large{\textbf{Mémo \texttt{expr}, \texttt{grep}, \texttt{sed}, \texttt{awk}}} \\
\end{center}

\section{Expr}
\section{RegExp simples}
\begin{tabular}{@{}ll@{}}
\verb!.!  & 1 caractère quelconque \\
\verb!*!  & 0 à N apparition(s) du caractère précédent \\
\verb!\?! & 0 ou 1 apparition du caractère précédent \\
\verb!\+! & 1 ou N apparition(s) du caractère précédent \\
\verb!.*! & 0 à N caractères quelconques (représente tout) \\
\verb![abc]!  & a ou b ou c (1 seul caractère) \\
\verb![a-d]!  & a ou b ou c ou d (1 seul caractère) \\
\verb![A-Za-z]!  & N'importe quelle majuscule ou minuscule \\
\verb![A-Za-z-0-9]!  & Une lettre ou un chiffre \\
\verb![0-9]*!  & 0 à N chiffres qui se suivent \\
%\verb!^!  & Début de ligne, OU négation d'un ensemble \\
\textbf{\^{}}  & Début de ligne, OU négation d'un ensemble \\
\verb!$!  & Fin de ligne \\
\verb!^[a-z][0-9]$!  & 1 caractère minuscule puis 1 chiffre (ex : a8) \\
\verb![^g-m]!  & N'importe quel caractère sauf g h i j k l m \\
\verb!'\(  \)'!  & Extrait la sous-chaîne entre les parenthèses
\end{tabular}

\subsection{expr : mathématiques}
\# + addition, - soustraction, / division, \textbackslash * multiplication\\
\verb!expr 6 \* 7!\\
\textrightarrow 42\\
\# \% est le modulo\\
\verb!expr 42 % 6!\\
\textrightarrow 0\\
\verb!expr 43 % 6!\\
\textrightarrow 1\\
\# \'Echapper les caractères qui pourraient être interprétés par le shell\\
\verb!expr \( 1 + 2 \) \* \( 3 + 4 \)!\\
\textrightarrow 21\\

\subsection{expr : booléens}
\verb!expr 1 \> 3!\\
\textrightarrow 0  (faux)\\
\verb!expr 1 \< 3!\\
\textrightarrow 1 (vrai)\\
\verb!expr 1 \!= 3!\\
\textrightarrow 1\\
\verb!expr 1 = 3!\\
\textrightarrow 0\\
\verb!expr 1 \<= 1!\\
\textrightarrow 1\\

\verb!expr 1 \< 3  \&  1 \> 3!\\
\verb!expr \( 1 \< 3 \)  \&  \( 1 \> 3 \)!\\
\textrightarrow 0\\

\verb!expr 1 \< 3  \|  1 \> 3!\\
\verb!expr \( 1 \< 3 \)  \|  \( 1 \> 3 \)!\\
\textrightarrow 1\\

\subsection{expr : string}
Longueur d'une chaîne de caractères\\
\verb!expr length Metalman!\\
\textrightarrow 8\\
\verb!expr length Mega\ Man!\\
\textrightarrow 8\\
\verb!expr length "Mega Man"!
\textrightarrow 8\\

Extraction sous-chaîne\\
\verb!expr substr Metalman 1 5!\\
\textrightarrow Metal\\

Première occurrence d'un mot (sinon 0)\\
\verb!expr index Metalman z!\\
\textrightarrow 0\\
\verb!expr index Metalman m!\\
\textrightarrow 6\\
\verb!expr index Metalman a!\\
\textrightarrow 4\\
\verb!expr index Metalman n!\\
\textrightarrow 8\\
\verb!expr index Metalman man!\\
\textrightarrow 4\\
\# 'a' est trouvé en premier, en position 4\\
\# m et n sont plus loin\\

Pattern Matching\\
\verb!expr match 'Chaine 46' '[^0-9]*\([0-9]*\)[^0-9]*'! \\
\textrightarrow 46\\
\# Extrait la première suite de chiffres dans la chaîne\\
\verb!expr match 'Chaine 46. 88' '[^0-9]*\([0-9]*\)[^0-9]*'! \\
\textrightarrow 46\\
\verb!expr match 'Chaine 46. 88.' '[^0-9]*[0-9]*[^0-9]*'! \\
\textrightarrow 11\\
\# Les 11 premiers caractères répondant à la regexp. La regexp n'a PAS de \textbackslash( \textbackslash)\\
\verb!expr match 'Vive 42' '[^0-9]*\([0-9]*$\)'! \\
\textrightarrow 42\\
\# On extrait les chiffres en fin de chaîne de caractères\\
\verb!expr match 'Vive 42.' '[^0-9]*\([0-9]*\.$\)'! \\
\textrightarrow 42.\\
\# On extrait les chiffres suivi d'un . le tout en fin de chaîne de caractères\\
\verb!expr match 'Vive 42.' '[^0-9]*\([0-9]\?\)[^0-9]*'! \\
\textrightarrow 4\\
\# On extrait le premier chiffre disponible\\
\verb!expr match 'Vive 42.' '[^0-9]*\([0-9]\+\)[^0-9]*'! \\
\textrightarrow 42\\
\# On extrait tous les chiffres qui se suivent, si au moins un existe\\


\medskip

\section{Grep}
\subsection{grep : options}
\begin{tabular}{@{}ll@{}}
\texttt{-E}  & Expressions régulières étendues\\
\texttt{-v}  & Affiche lignes différentes du motif\\
\texttt{-i}  & Indifférent aux majuscules/minuscules\\
\texttt{-c}  & Compte les lignes avec le motif\\
\texttt{-x}  & Affiche lignes strictement égales au motif\\
\texttt{-l}  & Nom des fichiers contenant le motif
\end{tabular}

\subsection{grep : exemples}
\verb!echo "Test." | grep  "Test"!\\
\textrightarrow Test.\\
\verb!echo "Test." | grep -x "Test"!\\
\textrightarrow \\
\verb!echo "Test." | grep -v "Test"!\\
\textrightarrow \\
\verb!echo "Test." | grep -v "Testa"!\\
\textrightarrow Test.\\

\medskip

\section{Sed}
\subsection{sed : options}
\begin{tabular}{@{}ll@{}}
\verb!-r! OU \verb!-E!  & Activation des RegExp étendues \\
\verb!-i!  & Remplace dans le(s) fichier(s) le motif \\
\verb!-n!  & N'affiche rien (sauf si /p ajouté) \\
\verb!y/!  & \'Echange des caractères (comme \texttt{tr}) \\
\verb!s/!  & Substitution du motif dans le document \\
\verb!/1!  & Modifie la 1\up{ère} occurrence \\
\verb!/2!  & Modifie la 2\up{nde} occurrence \\
\verb!/g!  & Modifications multiples/dans tous le document \\
\verb!/p!  & Affiche les lignes contenant le motif \\
\verb!a\!  & Ajoute une ligne après chaque changement \\
\verb!i\!  & Ajoute une ligne avant chaque changement \\
\verb!c\!  & Remplace la ligne contenant le motif \\
\verb!d!   & Supprime les lignes contenant le motif \\
\verb!\( \)! & Sauvegarde le motif \\
\verb!\1!  & Réutilise le 1\up{er} motif sauvegardé \\
\end{tabular}

\subsection{sed : exemples}
\verb!sed 's/test/TEST/g' < Filename! \\
\# Affiche le fichier avec les modifications selon le motif\\
\verb!sed -n 's/test/TEST/p' < Filename! \\
\# N'affiche que les lignes où le motif a été appliqué\\

\medskip

\# Créer un fichier exemple\\
\verb!echo -e "Ceci est un test pour\n"\!\\
\verb!"aider a comprendre le\n"\!\\
\verb!"fonctionnement de sed.\n"\!\\
\verb!"Il y aura au moins 3 lignes de\n"\!\\
\verb!"test." > FILE!\\

\smallskip

\# Vérification contenu du fichier\\
\verb!cat FILE!\\
\textrightarrow Ceci est un test pour\\
\textrightarrow aider a comprendre le\\
\textrightarrow fonctionnement de sed.\\
\textrightarrow Il y aura au moins 3 lignes de\\
\textrightarrow test.\\

\smallskip

\# Affiche le texte + ce qui matche\\
\verb!sed 's/test/&/p' FILE!\\
\textrightarrow Ceci est un test pour\\
\textrightarrow Ceci est un test pour\\
\textrightarrow aider a comprendre le\\
\textrightarrow fonctionnement de sed.\\
\textrightarrow Il y aura au moins 3 lignes de\\
\textrightarrow test.\\
\textrightarrow test.\\

\smallskip

\# Affiche seulement les matchs\\
\verb!sed -n 's/test/&/p' FILE!\\
\textrightarrow Ceci est un test pour\\
\textrightarrow test.\\

\smallskip

\# "test" ou "Il" sont matchés (usage du pipe "|")\\
\verb!sed -n 's/test\|Il/&/p' FILE!\\
\textrightarrow Ceci est un test pour\\
\textrightarrow Il y aura au moins 3 lignes de\\
\textrightarrow test.\\

\smallskip

\# Remplacement des occurrences\\
\verb!sed -n 's/test\|Il/MDR/p' FILE!\\
\textrightarrow Ceci est un MDR pour\\
\textrightarrow MDR y aura au moins 3 lignes de\\
\textrightarrow MDR.\\

\smallskip

\# Affiche seulement ce qui est changé\\
\verb!sed -n 's/test/PTDR/p' FILE!\\
\textrightarrow Ceci est un PTDR pour\\
\textrightarrow PTDR.\\

\smallskip

\# Affiche tout le texte changé\\
\verb!sed 's/test/PTDR/g' FILE!\\
\textrightarrow Ceci est un PTDR pour\\
\textrightarrow aider a comprendre le\\
\textrightarrow fonctionnement de sed.\\
\textrightarrow Il y aura au moins 3 lignes de\\
\textrightarrow PTDR.\\

\smallskip

\# Changer les / en \# est possible\\
\verb!sed -n 's#Ceci#&#p' FILE!\\
\textrightarrow Ceci est un test pour\\

\smallskip

\# Travaille sur la ligne 5 seulement\\
\verb!sed -n '5 s/test/&/p' FILE!\\
\textrightarrow test.\\

\smallskip

\# Travaille sur lignes 1 à 4 seulement\\
\verb!sed -n '1,4 s/test/&/p' FILE!\\
\textrightarrow Ceci est un test pour\\

\smallskip

\# Supprime les lignes contenant "test"\\
\verb!sed '/test/ d' FILE!\\
\textrightarrow aider a comprendre le\\
\textrightarrow fonctionnement de sed.\\
\textrightarrow Il y aura au moins 3 lignes de\\

\smallskip

\# !p affiche ce qui ne contient pas "test"\\
\verb|sed -n '/test/!p' FILE|\\
\textrightarrow aider a comprendre le\\
\textrightarrow fonctionnement de sed.\\
\textrightarrow Il y aura au moins 3 lignes de\\

\smallskip

\# Remplacer des caractères par d'autres (comme tr)\\
\verb!sed 'y/abcdefghijk/ABCDEFGHIJK/' FILE!\\
\textrightarrow CECI Est un tEst pour\\
\textrightarrow AIDEr A ComprEnDrE lE\\
\textrightarrow FonCtIonnEmEnt DE sED.\\
\textrightarrow Il y AurA Au moIns 3 lIGnEs DE\\
\textrightarrow tEst.\\

\smallskip

\# Insertion dans le fichier\\
\verb!sed -i 'y/t/T/' FILE!\\
\verb!cat FILE!\\
\textrightarrow Ceci esT un TesT pour\\
\textrightarrow aider a comprendre le\\
\textrightarrow foncTionnemenT de sed.\\
\textrightarrow Il y aura au moins 3 lignes de\\
\textrightarrow TesT.\\

\smallskip

\# Ajoute une ligne après chaque occurrence\\
\# (\textit{a} ajoute après, \textit{i} ajoute avant)\\
\verb!sed '/test/ a LOL' FILE!\\
\textrightarrow Ceci est un test pour\\
\textrightarrow LOL\\
\textrightarrow aider a comprendre le\\
\textrightarrow fonctionnement de sed.\\
\textrightarrow Il y aura au moins 3 lignes de\\
\textrightarrow test.\\
\textrightarrow LOL\\

\smallskip

\# Remplacer une ligne si motif trouvé\\
\verb!sed '/test/c\!\\
\verb!NOP' FILE!\\
\textrightarrow NOP\\
\textrightarrow aider a comprendre le\\
\textrightarrow fonctionnement de sed.\\
\textrightarrow Il y aura au moins 3 lignes de\\
\textrightarrow NOP\\

\smallskip

\# Tester en ligne de commande\\
\verb!echo "Texte2Texte" | sed -nE '/[^0-9]+[0-9]{1}[^0-9]+/p'! \\
\textrightarrow Texte2Texte\\

\smallskip

\# Exemples utiles, supprimer lignes vides\\
\verb!sed '/^$/d'!\\
\verb|sed '/./!d'|\\

\smallskip

\# Réutilisation de motifs \texttt{\textbackslash( \textbackslash)} et \texttt{\textbackslash1}\\
\verb!sed -n 's/.*\(test\).*\(pour\).*/\2 \1/p' FILE!\\
\textrightarrow pour test\\

\medskip

\section{Awk}
\subsection{awk : variables}
\begin{tabular}{@{}lll@{}}
\verb!FS!  & Field Separator & Séparateur de champs/colonnes \\
\verb!NF!  & Number of Fields & Nombre de champs/colonnes\\
\verb!RS!  & Record Separator & Séparateur d'enregistrements/lignes\\
\verb!NR!  & Number of Records & Nombre d'enregistrements/lignes\\
\verb!OFS! & Output FS & Séparateur de colonnes en sortie\\
\verb!ORS! & Output RS & Séparateur de lignes en sortie\\
\end{tabular}
\begin{tabular}{@{}ll@{}}
\verb!FILENAME! & Nom du fichier en cours de traitement\\
\verb!$0! & Ensemble de la ligne sélectionnée\\
\verb!$1! & Première colonne\\
\verb!$2! & Deuxième colonne\\
... & \\
\verb!$NF! & Dernière colonne\\
\end{tabular}

\subsection{awk : exemples}
\# Afficher le nombre de colonnes d'un fichier\\
\# champs/colonnes séparées par des ':'\\
\verb!awk -F":" '{print NF}' /etc/passwd!\\
\textrightarrow 7\\
\textrightarrow 7\\
\textrightarrow 7\\

\# Afficher champs/colonnes 1 et 6\\
\verb!awk -F":" '{print $1,$6}' /etc/passwd!\\
\textrightarrow root /root\\
\textrightarrow metalman /home/metalman\\

\# Traitements avant/après\\
\verb!awk -F":" 'BEGIN { print "Lecture fichier"}!\\
\verb!{print $1,$6,FILENAME }!\\
\verb!END {print "Fin lecture"}' /etc/passwd!\\
\textrightarrow Lecture fichier\\
\textrightarrow root /root /etc/passwd\\
\textrightarrow metalman /home/metalman /etc/passwd\\
\textrightarrow Fin lecture\\

\# Vérifier le contenu de certains champs
\verb!awk 'BEGIN { print "Verification des logins et UID"; FS=":"}!\\
\verb|$1 !~ /^[[:alnum:]]+$/ { print "Bad login line "NR" : \n"$0}|\\
\verb|$3 !~ /^[0-9]+$/ { print "Bad UID line "NR" : \n"$0}|\\
\verb|END { print "Fin verification"}' /etc/passwd|\\
\textrightarrow Lecture fichier\\
\textrightarrow Fin lecture\\

\# (Test du précédent filtre dans un fichier construit)\\
\# (ligne 1 ok, mais ligne 2 a un problème dans UID)\\
\verb!echo -e "metal:*:1234\ntest42:*:1b3" > FILE2!
\verb!awk 'BEGIN { print "Verification des logins et UID"; FS=":"}!\\
\verb|$1 !~ /^[[:alnum:]]+$/ { print "Bad login line "NR" : \n"$0}|\\
\verb|$3 !~ /^[0-9]+$/ { print "Bad UID line "NR" : \n"$0}|\\
\verb!END { print "Fin verification"}' FILE2!\\
\textrightarrow Verification des logins et UID\\
\textrightarrow UID incorrect ligne 2 :\\
\textrightarrow test42:*:1b3\\
\textrightarrow Fin verification\\



\end{multicols}
\end{document}
