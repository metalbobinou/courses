\documentclass[11pt,a4paper]{article}
\usepackage[utf8]{inputenc}
\usepackage[french]{babel}
\usepackage[T1]{fontenc}

\usepackage{amsmath}
\usepackage{amsfonts}
\usepackage{amssymb}

\usepackage{courier}

\usepackage{textcomp}

\usepackage{array}
\newcolumntype{P}[1]{>{\centering\arraybackslash}p{#1}} %Cellule P centrée
\newcolumntype{M}[1]{>{\centering\arraybackslash}m{#1}} %Cellule M centrée
\newcolumntype{B}[1]{>{\centering\arraybackslash}b{#1}} %Cellule B centrée


\usepackage[left=2cm,right=2cm,top=2cm,bottom=2cm]{geometry}

\renewcommand{\thefootnote}{\*} % footnote sans numéro

\author{Fabrice BOISSIER}

\begin{document}

\section{Expressions Rationnelles / Regular Expressions}

\bigskip

\subsection{Classes de caractères}

\bigskip

%\renewcommand\arraystretch{2.5}

\begin{center}
	\begin{tabular}{ c | M{4cm} | M{7cm} }
  \textbf{Nom de la classe} & \textbf{Signification} & \textbf{ASCII} \\  \hline
	alpha & Lettres alphabétiques & \verb![A-Za-z]! \\ \hline
	digit & Chiffres décimaux & \verb![0-9]! \\ \hline
	xdigit & Chiffres hexadécimaux & \verb![0-9A-Fa-f]! \\ \hline
	alnum & Chiffres ou lettres alphabétiques & \verb![[:alpha:][:digit:]]! \\ \hline
	lower & Lettres minuscules & \verb!'[a-z]'! \\ \hline
	upper & Lettres majuscules & \verb![A-Z]! \\ \hline
	blank & Caractères blancs & \verb![ \t]! \\ \hline
	space & Caractères d'espacement & \verb![ \t\n\f\r]! \\ \hline
	punct & Signes de ponctuation & \verb|[]-!"#$%&’()*.,/:+| \verb!;<=>?@\^_‘{|}~! \\ \hline
	graph & Symboles ayant une représentation graphique & \verb![[:alnum:][:punct:]]! \\ \hline
	print & Caractères imprimables & \verb![[:graph:]]! \\ \hline
	cntrl & Caractères de contrôle & Code ASCII inférieur à 31 et caractère de code 127 \\
	\end{tabular}
\end{center}

%\renewcommand\arraystretch{1}

\bigskip

\noindent Exemple : \verb!grep "H[ea]llo [[:digit:]]" file.c!

\newpage

\subsection{RegExp}

\bigskip

%\renewcommand\arraystretch{2.5}

\begin{center}
	\begin{tabular}{ M{5cm} | c | c }
	\textbf{Signification} & \textbf{Symbole RegExp simple} & \textbf{Symbole RegExp étendue} \\  \hline
	Un seul caractère générique & \verb!'.'! & \verb!'.'! \\ \hline
	Début de ligne & \texttt{'\^{}'} & \texttt{'\^{}'} \\ \hline
	Fin de ligne & \verb!'$'! & \verb!'$'! \\ \hline
	Alternative (OU) & \verb!'\|'! & \verb!'|'! \\ \hline
	Liste de caractères & \verb!'[]'! & \verb!'[]'! \\ \hline
	Classe de caractères & \verb!'[:classe:]'! & \verb!'[:classe:]'! \\ \hline
	Zéro ou plusieurs occurrences de l'élément précédent & \verb!'*'! & \verb!'*'! \\ \hline
	Une ou plusieurs occurrences de l'élément précédent & \verb!'\+'! & \verb!'+'! \\ \hline
	Zéro ou une occurrence de l'élément précédent & \verb!'\?'! & \verb!'?'! \\ \hline
	Au moins 'n' et au plus 'm' occurrences de l'élément précédent & \verb!'\{n,m\}'! & \verb!'{n,m}'! \\ \hline
	Au moins 'n' occurrences de l'élément précédent & \verb!'\{n,\}'! & \verb!'{n,}'! \\ \hline
	Au plus 'm' occurrences de l'élément précédent & \verb!'\{0,m\}'! & \verb!'{0,m}'! \\ \hline
	Exactement 'n' occurrences de l'élément précédent & \verb!'\{n\}'! & \verb!'{n}'! \\ \hline
	Groupement de caractères & \verb!'\(\)'! & \verb!'()'! \\ \hline
	Référence arrière au num-ième groupe & \verb!'\num'! & \verb!'\num'! \\ \hline
	\'Echappement de caractères spéciaux & \verb!'\'! & \verb!'\'!
	\end{tabular}
\end{center}

%\renewcommand\arraystretch{1}

\bigskip

\noindent \textbf{Toujours encadrer ses regexp avec des guillemets !}\\

\noindent \verb!echo "Hello 42" | grep "H[ea]llo [[:digit:]]\{2\}"!\\
\noindent \textrightarrow Hello 42\\

\noindent Attention aux regexp étendues, et à l'échappement de caractères.\\
\noindent \verb!echo "Hello 42" | grep -E "H[ea]llo [[:digit:]]{2}"!\\
\noindent \textrightarrow Hello 42\\

\noindent \verb!echo "Hello 42*Bla" | grep -E "H[ea]llo [[:digit:]]{2}\*Bla"!\\
\noindent \textrightarrow Hello 42*Bla\\

\footnotetext{Merci les ACU !}

\end{document}