\documentclass[11pt,a4paper]{article}
\usepackage[utf8]{inputenc}
\usepackage[french]{babel}
\usepackage[T1]{fontenc}

\usepackage{amsmath}
\usepackage{amsfonts}
\usepackage{amssymb}

\newcommand{\NomAuteur}{Fabrice BOISSIER}
\newcommand{\TitreMatiere}{Algo et Structure de Données 2}
\newcommand{\NomUniv}{EPITA - Bachelor Cyber Sécurité}
\newcommand{\NiveauUniv}{CYBER1}
\newcommand{\NumGroupe}{CYBER1}
\newcommand{\AnneeUniv}{2022-2023}
\newcommand{\DateExam}{juillet 2023}
\newcommand{\TypeExam}{Rattrapage}
\newcommand{\TitreExam}{\TitreMatiere}
\newcommand{\DureeExam}{1h30}
\newcommand{\MyWaterMark}{\AnneeUniv} % Watermark de protection

% Ajout de mes classes & definitions
\usepackage{MetalExam} % Appelle un .sty

% "Tableau" et pas "Table"
\addto\captionsfrench{\def\tablename{Tableau}}

%%%%%%%%%%%%%%%%%%%%%%%
%Header
%%%%%%%%%%%%%%%%%%%%%%%
\lhead{\TypeExam}							%Gauche Haut
\chead{\NomUniv}							%Centre Haut
\rhead{\NumGroupe}							%Droite Haut
\lfoot{\DateExam}							%Gauche Bas
\cfoot{\thepage{} / \pageref*{LastPage}}	%Centre Bas
\rfoot{\texttt{\TitreMatiere}}				%Droite Bas

%%%%%

\usepackage{tabularx}

\newlength{\LabelWidth}%
%\setlength{\LabelWidth}{1.3in}%
\setlength{\LabelWidth}{1cm}%
%\settowidth{\LabelWidth}{Employee E-mail:}%  Specify the widest text here.

% Optional first parameter here specifies the alignment of
% the text within the \makebox.  Default is [l] for left
% alignment. Other options are [r] and [c] for right and center
\newcommand*{\AdjustSize}[2][l]{\makebox[\LabelWidth][#1]{#2}}%


\definecolor{mGreen}{rgb}{0,0.6,0}
\definecolor{mGray}{rgb}{0.5,0.5,0.5}
\definecolor{mPurple}{rgb}{0.58,0,0.82}
\definecolor{backgroundColour}{rgb}{0.95,0.95,0.92}

\lstdefinestyle{CStyle}{
    backgroundcolor=\color{backgroundColour},
    commentstyle=\color{mGreen},
    keywordstyle=\color{magenta},
    numberstyle=\tiny\color{mGray},
    stringstyle=\color{mPurple},
    basicstyle=\footnotesize,
    breakatwhitespace=false,
    breaklines=true,
    captionpos=b,
    keepspaces=true,
    numbers=left,
    numbersep=5pt,
    showspaces=false,
    showstringspaces=false,
    showtabs=false,
    tabsize=2,
    language=C
}


\hyphenation{op-tical net-works SIGKILL}


\begin{document}

%\MakeExamTitleDuree     % Pour afficher la duree
\MakeExamTitle                   % Ne pas afficher la duree

%% \MakeStudentName    %% A reutiliser sur chaque nouvelle page

\bigskip
%\bigskip

Vous devez respecter les consignes suivantes, sous peine de 0 :

\begin{enumerate}[label=\Roman*)]
\item Lisez le sujet en entier avec attention
\item Répondez sur le sujet
\item Ne détachez pas les agrafes du sujet
\item \'Ecrivez lisiblement vos réponses (si nécessaire en majuscules)
\item Vous devez écrire dans le langage algorithmique classique ou en C (donc pas de Python ou autre)
\item Ne trichez pas
\end{enumerate}

%\bigskip

\vfillFirst

% Questions cours
\section{Arbres Binaires (10 points)}

\subsection{(3 points) Indiquez toutes les propriétés que possède cet arbre, puis écrivez les clés lors d'un parcours profondeur main gauche de l'arbre dans les 3 ordres ainsi que lors d'un parcours largeur : }

\begin{center}
\begin{tikzpicture}[sibling distance=1.5cm,
  level/.style = {sibling distance = 55mm/#1},
  level 1/.style = {sibling distance = 55mm},
  level 2/.style = {sibling distance = 30mm},
  level 3/.style = {sibling distance = 15mm},
  level 4/.style = {sibling distance = 10mm},
  every node/.style = {minimum width = 2em, draw, circle},
  ]
  \node (nX1) {X}
  child { node (nA2) {A}
          child { node (nO3) {O}
                  child { node (nP4) {P} }
                  child { node (nT5) {T} }
                }
          child { node (nA6) {A}
                  child { node (nE7) {E}
                          child { node (nG8) {G} }
                          child { node [draw=none] (nX) {\phantom{X}} edge from parent [draw=none] }
                        }
                  child { node (nU9) {U} }
                }
         }
   child { node (nU10) {U}
           child { node (nE11) {E}
                   child { node (nL12) {L} }
                   child { node (nG13) {G} }
                 }
           child { node (nE14) {E}
                   child { node (nM15) {M} }
                   child { node (nS16) {S} }
                 }
        };
\end{tikzpicture}
%  child { node [draw=none] (nX) {\phantom{X}} edge from parent [draw=none] }
\end{center}

\begin{tabular}{C{3cm} C{3cm} C{3cm} C{3cm}}
Arité : & Taille : & Hauteur : & Nb feuilles : \\
\end{tabular}

\medskip

\begin{table}[ht!]
  \centering
  \begin{minipage}{0.50\textwidth}
    \centering

\begin{itemize}
  \item[\CaseCoche] Arbre binaire strict / localement complet \phantom{()}
  \item[\CaseCoche] Arbre binaire parfait \phantom{()}
  \item[\CaseCoche] Peigne gauche \phantom{()}
\end{itemize}

  \end{minipage}
  \hfillx
  \begin{minipage}{0.50\textwidth}
    \centering

\begin{itemize}
  \item[\CaseCoche] Arbre binaire (presque) complet \phantom{()}
  \item[\CaseCoche] Arbre filiforme \phantom{()}
  \item[\CaseCoche] Peigne droit \phantom{()}
\end{itemize}

  \end{minipage}
%\stepcounter{figure}
%\caption{Fig.\thefigure : Recherche de la clé 24 dans un ABR}
%\label{fig:example2-BST-succeed-search}
\end{table}

\bigskip

Parcours profondeur :

\medskip

\centerline{
\begin{tabular}{L{2.5cm} C{0.5cm}C{0.5cm}C{0.5cm}C{0.5cm} C{0.5cm}C{0.5cm}C{0.5cm}C{0.5cm} C{0.5cm}C{0.5cm}C{0.5cm}C{0.5cm} C{0.5cm}C{0.5cm}C{0.5cm}}
ordre préfixe : & - & - & - & - & - & - & - & - & - & - & - & - & - & - & - \\
ordre infixe :  & - & - & - & - & - & - & - & - & - & - & - & - & - & - & - \\
ordre suffixe : & - & - & - & - & - & - & - & - & - & - & - & - & - & - & - \\
\end{tabular}
}

\bigskip

Parcours largeur :

\medskip

\centerline{
\begin{tabular}{L{2.5cm} C{0.5cm}C{0.5cm}C{0.5cm}C{0.5cm} C{0.5cm}C{0.5cm}C{0.5cm}C{0.5cm} C{0.5cm}C{0.5cm}C{0.5cm}C{0.5cm} C{0.5cm}C{0.5cm}C{0.5cm}}
ordre : & - & - & - & - & - & - & - & - & - & - & - & - & - & - & - \\
\end{tabular}
}

%%%%%%%%%%%%%%%%%%%%%%%%%%%%%%%%%%%%%%%%%%%%%%%

\vfillLast

\clearpage

%\vfillFirst

%%%%%%%%%%%%%%%%%%%%%%%%%%%%%%%%%%%%%%%%%%%%%%%

\subsection{(4 points) Dessinez le résultat de l'insertion dans cet ordre précis des éléments suivants dans un ABR (insertion en feuille) et dans un AVL : }

\begin{center}

%\'Eléments insérés : 18 - 46 - 55 - 36 - 12 - 38 - 96 - 71
%\'Eléments insérés : 46 - 18 - 55 - 36 - 12 - 38 - 96 - 71
\'Eléments insérés : 32 - 8 - 16 - 96 - 64 - 72 - 24 - 4

\begin{table}[ht!]
  \centering
  \begin{minipage}{0.50\textwidth}
    \centering

\vspace*{6cm}

ABR
  \end{minipage}
  \hfillx
  \begin{minipage}{0.50\textwidth}
    \centering

\vspace*{6cm}

AVL
  \end{minipage}
%\stepcounter{figure}
%\caption{Fig.\thefigure : Recherche de la clé 24 dans un ABR}
%\label{fig:example2-BST-succeed-search}
\end{table}

\end{center}


%%%%%%%%%%%%%%%%%%%%%%%%%%%%%%%%%%%%%%%%%%%%%%%

\subsection{(3 points) \'Ecrivez une fonction récursive \og \textit{parc\_prof\_rec} \fg{} effectuant un parcours profondeur main gauche dans un arbre binaire, et affichant les nœuds dans chacun des ordres : }

%\noindent Pour expliciter les ordres, vous devrez afficher au format : \og Ordre : nœud \fg{} (exemple : \og Préfixe : node \fg{})
\noindent Il faut expliciter les éventuels ordres au format : \og Ordre : nœud \fg{} (exemple : \og Préfixe : 42 \fg{})

\begin{center}
\GrilleReponseN{14}
\end{center}

%\bigskip

%\vfillLast

\clearpage



%%%%%%%%%%%%%%%%%%%%%%%%%%%%%%%%%%%%%%%%%%%%%%%%%%%%%%%%%%%%%%%%%
%%%%%%%%%%%%%%%%%%%%%%%%%%%%%%%%%%%%%%%%%%%%%%%%%%%%%%%%%%%%%%%%%
%%%%%%%%%%%%%%%%%%%%%%%%%%%%%%%%%%%%%%%%%%%%%%%%%%%%%%%%%%%%%%%%%

\section{Arbres Binaires : Parcours Largeur (10 points)}


%\noindent Afin de tester l'ensemble des compétences acquises au cours de cette année, vous allez maintenant toutes les exploiter pour interpréter des données et des structures.
%Le but de ces exercices est de vous faire changer de point de vue : vous avez construit des structures durant l'année, vous allez maintenant analyser des structures existantes.

%%%%%%%%%%%%%%%%%%%%%%%%%%%%%%%%%%%%%%%%%%%%%%%

\subsection{(2 points) Effectuez les opérations suivantes et affichez la structure dans son état final : }


\begin{center}

\begin{table}[ht!]
  \centering
  \begin{minipage}{0.5\textwidth}
    \centering
Pile

  \end{minipage}
  \hfillx
  \begin{minipage}{0.5\textwidth}
    \centering

File

  \end{minipage}
%\stepcounter{figure}
%\caption{Fig.\thefigure : Recherche de la clé 24 dans un ABR}
%\label{fig:example2-BST-succeed-search}
\end{table}



\begin{table}[ht!]
  \centering
  \begin{minipage}{0.25\textwidth}
    \centering

\begin{enumerate}[label=\roman*]
\item push 5
\item push 12
\item push 14
\item pop
\item push 7
\item push 6
\item pop
\item pop
\end{enumerate}

\vspace*{2cm}

  \end{minipage}
  \hfillx
  \begin{minipage}{0.25\textwidth}

\begin{enumerate}[label=\roman*]
\setcounter{enumi}{8}
\item pop
\item push 2
\item push 8
\item push 11
\item pop
\end{enumerate}

\vspace*{1.75cm}

\vspace*{2cm}

  \end{minipage}
  \hfillx
  \begin{minipage}{0.01\textwidth}

\begin{tikzpicture}
\draw (0,0) -- (0,7.5);
\end{tikzpicture}

  \end{minipage}
  \hfillx
  \begin{minipage}{0.25\textwidth}
    \centering

\begin{enumerate}[label=\roman*]
\item enqueue 5
\item enqueue 12
\item enqueue 14
\item dequeue
\item enqueue 7
\item enqueue 6
\item dequeue
\item dequeue
\end{enumerate}

\vspace*{2cm}

  \end{minipage}
  \hfillx
  \begin{minipage}{0.25\textwidth}
    \centering

\begin{enumerate}[label=\roman*]
\setcounter{enumi}{8}
\item dequeue
\item enqueue 2
\item enqueue 5
\item enqueue 11
\item dequeue
\end{enumerate}

\vspace*{1.75cm}

\vspace*{2cm}

  \end{minipage}
%\stepcounter{figure}
%\caption{Fig.\thefigure : Recherche de la clé 24 dans un ABR}
%\label{fig:example2-BST-succeed-search}
\end{table}



\begin{table}[ht!]
  \centering
  \begin{minipage}{0.5\textwidth}
    \centering
1) Quelle est la spécificité d'une pile concernant l'ordre d'entrée et de sortie des éléments ?

  \end{minipage}
  \hfillx
  \begin{minipage}{0.5\textwidth}
    \centering
2) Quelle est la spécificité d'une file concernant l'ordre d'entrée et de sortie des éléments ?

  \end{minipage}
%\stepcounter{figure}
%\caption{Fig.\thefigure : Recherche de la clé 24 dans un ABR}
%\label{fig:example2-BST-succeed-search}
\end{table}

\vspace*{1cm}

\end{center}


\bigskip

%%%%%%%%%%%%%%%%%%%%%%%%%%%%%%%%%%%%%%%%%%%%%%%

\subsection{(1 point) En admettant que l'on dispose d'une pile et que l'on insère les données \og 1 2 3 4 5 6 \fg{} dans cet ordre exclusivement, décrivez les scénarios permettant d'obtenir les sorties suivantes : }
%%% PREFERER CE TEXTE :
%\subsection{(3 points) En admettant que l'on dispose d'une pile vide et que les éléments \og 1 2 3 4 5 6 \fg{} arrivent en entrée dans cet ordre exclusivement, décrivez les scénarios permettant d'obtenir les sorties suivantes : }

\bigskip

\begin{center}
\noindent \textit{exemple : pour \og A B C \fg{} en entrée, on peut obtenir \og B C A \fg{} en sortie en faisant : \linebreak
\og push A \fg, \og push B \fg, \og pop \fg, \og push C \fg, \og pop \fg, \og pop \fg }

\textit{On a bien inséré A, puis B, puis C, mais l'ordre de sortie est différent suivant les \og pop \fg}
\end{center}

\medskip


\begin{center}

\begin{large}
2, 4, 3, 5, 6, 1
\end{large}

\begin{center}
\GrilleReponseN{5}
% push 1, push 2, pop, push 3, push 4, pop, pop, push 5, pop, push 6, pop, pop
\end{center}

\end{center}


%%%%%%%%%%%%%%%%%%%%%%%%%%%%%%%%%%%%%%%%%%%%%%%
\clearpage
%%%%%%%%%%%%%%%%%%%%%%%%%%%%%%%%%%%%%%%%%%%%%%%

\subsection{(2 points) \`A partir de l'arbre affiché, répondez aux questions et effectuez le parcours largeur : }

\vspace*{-0.5cm}

\begin{center}

\begin{table}[ht!]
  \centering
  \begin{minipage}{0.60\textwidth}
    \centering

\begin{tikzpicture}[sibling distance=1.5cm,
  level/.style = {sibling distance = 55mm/#1},
  level 1/.style = {sibling distance = 55mm},
  level 2/.style = {sibling distance = 30mm},
  level 3/.style = {sibling distance = 15mm},
  level 4/.style = {sibling distance = 10mm},
  every node/.style = {minimum width = 2em, draw, circle},
  ]
  \node (n3) {3}
  child { node (n8) {8}
          child { node (n2) {2}
                  child { node [draw=none] (nX) {\phantom{X}} edge from parent [draw=none] }
                  child { node (n15) {15} }
                }
          child { node (n20) {20} }
         }
   child { node (n42) {42}
           child { node (n11) {11}
                   child { node (n9) {9} }
                   child { node (n4) {4} }
                 }
           child { node (n21) {21}
                   child { node (n36) {36} }
                   child { node [draw=none] (nX) {\phantom{X}} edge from parent [draw=none] }
                 }
        };
\end{tikzpicture}

  \end{minipage}
  \hfillx
  \begin{minipage}{0.30\textwidth}

1) Quelle structure est requise pour effectuer un parcours largeur ?

\vspace*{2cm}

  \end{minipage}
\end{table}

\end{center}

\bigskip
\vspace*{-1cm}


2) Effectuez le parcours largeur de l'arbre en détaillant pas à pas l'état de la structure associée.

\vspace*{-0.5cm}

\begin{center}

\begin{table}[ht!]
  \centering
  \begin{minipage}{0.45\textwidth}
%    \centering

\begin{tabular}{ L{4cm} L{2.5cm} }
Structure : & N\oe{}ud traité : \\
\end{tabular}

%\bigskip
\smallskip

\begin{tabular}{ |C{0.5cm}| C{3cm} L{2cm} }
\cline{1-1}
 & & \textit{Ø} \\
\cline{1-1}
\end{tabular}

%\bigskip
\medskip

\begin{tabular}{ |C{0.5cm}| C{3cm} L{2cm} }
\cline{1-1}
3 & & \textit{Ø} \\
\cline{1-1}
\end{tabular}

%\bigskip
\medskip

\begin{tabular}{ |C{0.5cm}| C{3cm} L{2cm} }
\cline{1-1}
 & & 3 \\
\cline{1-1}
\end{tabular}

\vspace*{7.5cm}

  \end{minipage}
  \hfillx
  \begin{minipage}{0.01\textwidth}

\begin{tikzpicture}
\draw (0,0) -- (0,10.85);
\end{tikzpicture}

  \end{minipage}
  \hfillx
  \begin{minipage}{0.45\textwidth}
    \centering

\begin{tabular}{ L{4cm} L{2.5cm} }
Structure : & N\oe{}ud traité : \\
\end{tabular}

\vspace*{9.65cm}

  \end{minipage}
%\stepcounter{figure}
%\caption{Fig.\thefigure : Recherche de la clé 24 dans un ABR}
%\label{fig:example2-BST-succeed-search}
\end{table}

\end{center}


\vspace*{-1.45cm}


%%%%%%%%%%%%%%%%%%%%%%%%%%%%%%%%%%%%%%%%%%%%%%%
%\clearpage
%%%%%%%%%%%%%%%%%%%%%%%%%%%%%%%%%%%%%%%%%%%%%%%

\subsection{(1 point) \`A partir du tableau et de l'arbre affiché, répondez à la question suivante : }

\vspace*{-0.5cm}

\begin{center}

\begin{table}[ht!]
  \centering
  \begin{minipage}{0.50\textwidth}
    \centering

\begin{tikzpicture}[sibling distance=1cm,
  level/.style = {sibling distance = 35mm/#1},
  every node/.style = {minimum width = 2em, draw, circle},
  ]
  \node (n21) {21}
  child { node (n42) {42}
          child { node (n14) {14} }
          child { node (n20) {20}
                  child { node (n17) {17} }
                  child { node [draw=none] (nX) {\phantom{X}} edge from parent [draw=none] }
                }
        }
  child { node (n55) {55}
          child { node (n6) {6}
                  child { node (n9) {9} }
                  child { node [draw=none] (nX) {\phantom{X}} edge from parent [draw=none] }
                }
          child { node (n15) {15}
                  child { node [draw=none] (nX) {\phantom{X}} edge from parent [draw=none] }
                  child { node (n13) {13} }
                }
        };
\end{tikzpicture}

  \end{minipage}
  \hfillx
  \begin{minipage}{0.40\textwidth}


\begin{center}

\begin{tabular}{ |C{0.5cm}|C{0.5cm}|C{0.5cm}|C{0.5cm}| }
\hline
15 & 6 & 20 & 14 \\
\hline
\end{tabular}

\end{center}

\bigskip

Dans le cas d'un parcours largeur, quel n\oe{}ud est actuellement traité d'après l'état de la structure ?

\vspace*{2cm}

  \end{minipage}
\end{table}

\end{center}


%%%%%%%%%%%%%%%%%%%%%%%%%%%%%%%%%%%%%%%%%%%%%%%
\clearpage
%%%%%%%%%%%%%%%%%%%%%%%%%%%%%%%%%%%%%%%%%%%%%%%

\subsection{(4 points) \'Ecrivez une fonction itérative \og \textit{parc\_larg} \fg{} effectuant un parcours largeur dans un arbre binaire, et affichant chacun des nœuds dans l'ordre hiérarchique : }

%\noindent Il faut expliciter les éventuels ordres au format : \og Ordre : nœud \fg{} (exemple : \og Préfixe : 42 \fg{})

%\medskip

\noindent \textit{Vous pouvez utiliser les structures externes :}

\noindent \textit{stack\_t (create, push, head, pop, delete) \hfill queue\_t (create, enqueue, head, dequeue, delete) }

\begin{center}
\GrilleReponseN{22}
\end{center}


%%%%%%%%%%%%%%%%%%%%%%%%%%%%%%%%%%%%%%%%%%%%%%%

\clearpage


%\thispagestyle{empty}

\vfillFirst

\begin{center}

\begin{LARGE}
\textbf{RATTRAPAGE ALGORITHMIQUE ET STRUCTURES DE DONN\'EES 2}
\end{LARGE}

\end{center}

\vfillLast

\end{document}
