\documentclass[11pt,a4paper]{article}
\usepackage[utf8]{inputenc}
\usepackage[french]{babel}
\usepackage[T1]{fontenc}

\usepackage{amsmath}
\usepackage{amsfonts}
\usepackage{amssymb}

\newcommand{\TitreMatiere}{Architecture des Ordinateurs}
\newcommand{\TitreSeance}{TD1 - Conversions}
\newcommand{\NumeroTD}{Entiers \& Flottants}
\newcommand{\DateCours}{Novembre 2022}
\newcommand{\AnneeScolaire}{2022-2023}
\newcommand{\Organisation}{EPITA}
\newcommand{\NomAuteurA}{Fabrice BOISSIER}
\newcommand{\MailAuteurA}{fabrice.boissier@epita.fr}
\newcommand{\NomAuteurB}{ }
\newcommand{\MailAuteurB}{ }
\newcommand{\DocKeywords}{Architecture}
\newcommand{\DocLangue}{fr} % "en", "fr", ...

\usepackage{MetalQuickLabs}

% Babel ne traduit pas toujours bien les tableaux et autres
\renewcommand*\frenchfigurename{%
    {\scshape Figure}%
}
\renewcommand*\frenchtablename{%
    {\scshape Tableau}%
}

% Ne pas afficher le numéro de la légende sur tableaux et figures
\captionsetup{format=sanslabel}


\begin{document}

\EncadreTitre

\bigskip


%\begin{center}
%\begin{tabular}{p{5cm} p{11cm}}
%\textbf{Commandes étudiées :} & \texttt{sh}, \texttt{bash}, \texttt{man}, \texttt{ls}, \texttt{mkdir}, \texttt{touch}, \texttt{chmod}, \texttt{mv}, \texttt{rm}, \texttt{rmdir}, \texttt{cat}, \texttt{file}, \texttt{which}, \texttt{which}\\
%
%\textbf{Builtins étudiées :} & \texttt{pwd}, \texttt{cd}, \texttt{exit}, \texttt{logout}, \texttt{echo}, \texttt{umask}, \texttt{type}, \texttt{>}, \texttt{>{}>}, \texttt{<}, \texttt{<{}<}, \texttt{|}\\
%
%\textbf{Notions étudiées :} & Shell, Manuels, Fichiers, Répertoires, Droits, Redirections\\
%\end{tabular}
%\end{center}

\bigskip


Ce document a pour objectif de vous familiariser avec les conversions entre plusieurs bases dans le cas des entiers, mais également dans le cas des flottants en respectant la norme IEEE 754.

\bigskip

La plupart des conversions que nous effectuerons seront entre les bases 2, 8, 10, et 16 pour les entiers, et entre les bases 10 et 2 pour les flottants IEEE 754.

Pour rappel, plusieurs notations pour représenter les bases existent : celles où l'on explicite par un indice en suffixe de nombre dans quelle base celui-ci a été écrit, ou par un caractère en préfixe du nombre.

$ 42_{(10)} $ indique l'on écrit le nombre \og $42$ \fg{} en base $ 10 $.

$ 1010_{(2)} $ indique que l'on écrit le nombre \og $6$ \fg{} en base $ 2 $, c'est-à-dire $ 1010 $.

\medskip

Parmi les caractères servant de préfixe, on retrouvera : $ \%   \; \;  0o  \; \;  \$  $

$ \% 00101111 $ le pourcentage indique que l'on manipule un nombre binaire (ici $ 47 $)

$ 0o 42 $ le zéro suivi d'un O minuscule indiquent que l'on manipule un nombre octal (ici $ 34 $)

$ \$ 15AB $ le dollar indique l'on manipule un nombre hexadécimal (ici $ 5547 $)

Un nombre sans préfixe est considéré par défaut comme un nombre décimal.

\bigskip

%%%%%%%%%%%%%%%%%%%%%%%%%%%%%%%%%%%%%%

\section{Entiers}

\bigskip

On appelle \textit{entier signé} (qui dispose d'un signe) un entier pouvant être positif, négatif, ou nul.
\`A l'inverse, un \textit{entier non-signé} est un entier positif ou nul.

\medskip

Pour représenter les entiers signés, on réserve le bit de poids fort pour représenter le signe.
$ 0 $ signifiant $ + $, et $ 1 $ signifiant $ - $.

\medskip

N'oubliez pas que les nombres négatifs sont construits à partir du complètement à 2 : on transforme tous les $ 0 $ en $ 1 $, et tous les $ 1 $ en $ 0 $, puis, on ajoute $ 1 $.

Une autre façon de représenter ce concept est tout simplement de partir sur la notion d'\textit{overflow}/dépassement :  $ 0 $ en décimal correspond à $ 0000 $ en binaire, donc lui appliquer $ -1 $ revient à faire un overflow et obtenir $ 1111 $.

\bigskip

%%%%%%%%%%%%%%%%%%%

\subsection{Base 2}

\bigskip

Rappelez tout d'abord les bornes inférieures et supérieures (le plus petit/grand nombre) pour les différents cas : \textit{[au delà de 1 million, arrondissez à un chiffre et une puissance de 10]}


\begin{center}
\centerline{
\begin{tabular}{| c | C{3.5cm} | C{3.5cm} || c | C{3.5cm} | C{3.5cm} |}
\hline
  &  entier signé  &  entier non-signé  & &  entier signé  &  entier non-signé \\
\hline
 & & & & & \\
 4 bits & & &   32 bits & & \\
 & & & & & \\
\hline
 & & & & & \\
 8 bits & & &   48 bits & & \\
 & & & & & \\
\hline
 & & & & & \\
16 bits & & &   64 bits & & \\
 & & & & & \\
\hline
 & & & & & \\
24 bits & & &   128 bits & & \\
 & & & & & \\
\hline
\end{tabular}
}
\end{center}

\bigskip

Convertissez vers la base 2 sur 8 bits ces valeurs :

\bigskip


\begin{table}[h!]
  \centering
  \begin{minipage}{0.45\textwidth}
$ 0_{(10)} $

\bigskip

$ 42_{(10)} $

\bigskip

$ 10_{(10)} $

\bigskip

$ 142_{(10)} $

\bigskip

$ 143_{(10)} $

\bigskip

$ 88_{(10)} $

\bigskip

$ 255_{(10)} $

\bigskip

$ 203_{(10)} $
  \end{minipage}
  \hfillx
  \begin{minipage}{0.5\textwidth}
$ -1_{(10)} $

\bigskip

$ -42_{(10)} $

\bigskip

$ -10_{(10)} $

\bigskip

$ -113_{(10)} $

\bigskip

$ -112_{(10)} $

\bigskip

$ -88_{(10)} $

\bigskip

$ -127_{(10)} $

\bigskip

$ 127_{(10)} $

  \end{minipage}
%  \caption{Algorithme de la somme des N premiers entiers}
%  \label{somme-n-premiers-entiers}
\end{table}






\bigskip




\begin{center}
\textit{Ce document et ses illustrations ont été réalisés par Fabrice BOISSIER en novembre 2022}
\end{center}

\end{document}
