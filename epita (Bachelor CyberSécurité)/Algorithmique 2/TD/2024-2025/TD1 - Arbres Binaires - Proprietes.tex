\documentclass[11pt,a4paper]{article}
\usepackage[utf8]{inputenc}
\usepackage[french]{babel}
\usepackage[T1]{fontenc}

\usepackage{amsmath}
\usepackage{amsfonts}
\usepackage{amssymb}

\newcommand{\TitreMatiere}{Algorithmique 2}
\newcommand{\TitreSeance}{Arbres Binaires}
\newcommand{\SousTitreSeance}{Propriétés, Typologies, Parcours}
\newcommand{\DateCours}{Mars 2025}
\newcommand{\AnneeScolaire}{2024-2025}
\newcommand{\Organisation}{EPITA}
\newcommand{\NomAuteurA}{Fabrice BOISSIER}
\newcommand{\MailAuteurA}{fabrice.boissier@epita.fr}
\newcommand{\NomAuteurB}{ }
\newcommand{\MailAuteurB}{ }
\newcommand{\DocKeywords}{Algorithmique ; Algorithmes ; Arbres ; Arbres Binaires ; Propriétés Arbres ; Typologie Arbres Binaires ; Parcours Profondeur ; Parcours Largeur ; Trees ; Binary Trees ; Trees Properties ; Binary Trees Typology ; Depth-First Search ; DFS ; Breadth-First Search ; BFS}
\newcommand{\DocLangue}{fr} % "en", "fr", ...

\usepackage{MetalQuickLabs}

% Babel ne traduit pas toujours bien les tableaux et autres
\renewcommand*\frenchfigurename{%
    {\scshape Figure}%
}
\renewcommand*\frenchtablename{%
    {\scshape Tableau}%
}

% Ne pas afficher le numéro de la légende sur tableaux et figures
\captionsetup{format=sanslabel}


\begin{document}

\EncadreTitre

\bigskip


%\begin{center}
%\begin{tabular}{p{5cm} p{11cm}}
%\textbf{Commandes étudiées :} & \texttt{sh}, \texttt{bash}, \texttt{man}, \texttt{ls}, \texttt{mkdir}, \texttt{touch}, \texttt{chmod}, \texttt{mv}, \texttt{rm}, \texttt{rmdir}, \texttt{cat}, \texttt{file}, \texttt{which}, \texttt{which}\\
%
%\textbf{Builtins étudiées :} & \texttt{pwd}, \texttt{cd}, \texttt{exit}, \texttt{logout}, \texttt{echo}, \texttt{umask}, \texttt{type}, \texttt{>}, \texttt{>{}>}, \texttt{<}, \texttt{<{}<}, \texttt{|}\\
%
%\textbf{Notions étudiées :} & Shell, Manuels, Fichiers, Répertoires, Droits, Redirections\\
%\end{tabular}
%\end{center}

\bigskip


Ce document a pour objectif de vous familiariser avec les arbres binaires.
Vous devrez y utiliser une pile et une file à un moment donné pour les algorithmes itératifs appliqués aux arbres.

\bigskip

Pour rappel, les arbres binaires sont constitués de \textit{nœuds} stockant une \textit{clé} (l'élément ou l'identifiant de l'élément), et chaque nœud dispose de liens vers un \textit{fils gauche} et un \textit{fils droit}.

\medskip

\textit{Dans l'ensemble des exercices, toutes les clés qui seront stockées seront strictement supérieures à 0.}

\bigskip

%%%%%%%%%%%%%%%%%%%%%%%%%%%%%%%%%%%%%%

\section{Propriétés, Typologie, Parcours}

\bigskip


\question{Donnez les propriétés des arbres suivants, à quelle(s) éventuelle(s) typologie(s) ceux-ci correspondent, et enfin, indiquez les parcours largeur et profondeur dans les 3 ordres.}


\begin{table}[ht!]
  \centering
  \begin{minipage}{0.45\textwidth}
    \centering

% sibling distance=0cm, level distance=15mm,
\begin{tikzpicture}[sibling distance=0cm, level distance=12mm,
  leaf/.style = {circle, forestgreen(traditional), draw=green(htmlcssgreen), very thick},
  root/.style = {circle, harvardcrimson, draw=red, very thick},
  internal/.style = {circle, auburn, draw=auburn, very thick},
  level/.style = {sibling distance = 35mm/#1},
  level 3/.style={sibling distance = 8mm},
  every node/.style = {minimum width = 2em, draw, circle},
  ]
  \node {42}
  child { node {21}
          child { node {8} }
          child { node {24} }
         }
  child { node {64}
          child { node {48} }
          child [missing] { }
        };
\end{tikzpicture}

\medskip

Arbre 1

  \end{minipage}
  \hfillx
  \begin{minipage}[t]{0.50\textwidth}
    %\centering
    \raggedright

    \begin{tabular}{l p{0.5cm} l p{1.00cm} l p{1.00cm}}
    Arité : & & Hauteur : & & Taille : \\
     & & & & \\
    \multicolumn{6}{ l }{Parcours Largeur :} \\
     & & & & \\
     & & & & \\
    \multicolumn{6}{ l }{Parcours Profondeur (ordre préfixe) :} \\
     & & & & \\
     & & & & \\
    \multicolumn{6}{ l }{Parcours Profondeur (ordre infixe) :} \\
     & & & & \\
     & & & & \\
    \multicolumn{6}{ l }{Parcours Profondeur (ordre suffixe) :} \\
     & & & & \\
     & & & & \\
    \multicolumn{6}{ l }{Typologie(s) :} \\
    \end{tabular}

  \end{minipage}
\end{table}

%%%%%%%%%%%%%%%%%%%%%%%%%%%%%%%%%%%%%%%%%%%%%%%%%%%%%%%%%%%%%
\centerline{ \rule{1.0\linewidth}{0.50pt} }
%\clearpage
%%%%%%%%%%%%%%%%%%%%%%%%%%%%%%%%%%%%%%%%%%%%%%%%%%%%%%%%%%%%%

\smallskip


\begin{table}[ht!]
  \centering
  \begin{minipage}{0.45\textwidth}
    \centering

% sibling distance=0cm, level distance=15mm,
\begin{tikzpicture}[sibling distance=0cm, level distance=12mm,
  leaf/.style = {circle, forestgreen(traditional), draw=green(htmlcssgreen), very thick},
  root/.style = {circle, harvardcrimson, draw=red, very thick},
  internal/.style = {circle, auburn, draw=auburn, very thick},
  level/.style = {sibling distance = 35mm/#1},
  level 3/.style={sibling distance = 8mm},
  every node/.style = {minimum width = 2em, draw, circle},
  ]
  \node {42}
  child { node {21}
          child { node {8} }
          child { node {24} }
         }
  child { node {64}
          child { node {48} }
          child { node {96} }
        };
\end{tikzpicture}

\medskip

Arbre 2

  \end{minipage}
  \hfillx
  \begin{minipage}[t]{0.50\textwidth}
    %\centering
    \raggedright

    \begin{tabular}{l p{0.5cm} l p{1.00cm} l p{1.00cm}}
    Arité : & & Hauteur : & & Taille : \\
     & & & & \\
    \multicolumn{6}{ l }{Parcours Largeur :} \\
     & & & & \\
     & & & & \\
    \multicolumn{6}{ l }{Parcours Profondeur (ordre préfixe) :} \\
     & & & & \\
     & & & & \\
    \multicolumn{6}{ l }{Parcours Profondeur (ordre infixe) :} \\
     & & & & \\
     & & & & \\
    \multicolumn{6}{ l }{Parcours Profondeur (ordre suffixe) :} \\
     & & & & \\
     & & & & \\
    \multicolumn{6}{ l }{Typologie(s) :} \\
    \end{tabular}

  \end{minipage}
\end{table}


%%%%%%%%%%%%%%%%%%%%%%%%%%%%%%%%%%%%%%%%%%%%%%%%%%%%%%%%%%%%%
%\centerline{ \rule{1.0\linewidth}{0.50pt} }
\clearpage
%%%%%%%%%%%%%%%%%%%%%%%%%%%%%%%%%%%%%%%%%%%%%%%%%%%%%%%%%%%%%

%\smallskip
%\vfillFirst


\begin{table}[ht!]
  \centering
  \begin{minipage}{0.45\textwidth}
    \centering

\begin{tikzpicture}[
  leaf/.style = {circle, forestgreen(traditional), draw=green(htmlcssgreen), very thick},
  root/.style = {circle, harvardcrimson, draw=red, very thick},
  internal/.style = {circle, auburn, draw=auburn, very thick},
  level/.style = {sibling distance = 30mm/#1},
  every node/.style = {minimum width = 2em, draw, circle},
  ]

  \node {A}
  child {node {B}
         child {node {J} }
        }
  child {node {C}
         child [missing] {}
        }
  child {node {D}
         child {node {E}
                child {node {F}
                       child {node {G} }
                       child {node {H} }
                      }
               }
        };
\end{tikzpicture}

\medskip

Arbre 3

  \end{minipage}
  \hfillx
  \begin{minipage}[t]{0.50\textwidth}
    %\centering
    \raggedright

    \begin{tabular}{l p{0.5cm} l p{1.00cm} l p{1.00cm}}
    Arité : & & Hauteur : & & Taille : \\
     & & & & \\
    \multicolumn{6}{ l }{Parcours Largeur :} \\
     & & & & \\
     & & & & \\
    \multicolumn{6}{ l }{Parcours Profondeur (ordre préfixe) :} \\
     & & & & \\
     & & & & \\
    \multicolumn{6}{ l }{Parcours Profondeur (ordre infixe) :} \\
     & & & & \\
     & & & & \\
    \multicolumn{6}{ l }{Parcours Profondeur (ordre suffixe) :} \\
     & & & & \\
     & & & & \\
    \multicolumn{6}{ l }{Typologie(s) :} \\
    \end{tabular}

  \end{minipage}
\end{table}


%%%%%%%%%%%%%%%%%%%%%%%%%%%%%%%%%%%%%%%%%%%%%%%%%%%%%%%%%%%%%
\centerline{ \rule{1.0\linewidth}{0.50pt} }
%\clearpage
%%%%%%%%%%%%%%%%%%%%%%%%%%%%%%%%%%%%%%%%%%%%%%%%%%%%%%%%%%%%%

\smallskip


\begin{table}[ht!]
  \centering
  \begin{minipage}{0.45\textwidth}
    \centering

\begin{tikzpicture}[
  leaf/.style = {circle, forestgreen(traditional), draw=green(htmlcssgreen), very thick},
  root/.style = {circle, harvardcrimson, draw=red, very thick},
  internal/.style = {circle, auburn, draw=auburn, very thick},
  level/.style = {sibling distance = 30mm/#1},
  every node/.style = {minimum width = 2em, draw, circle},
  ]

  \node {42}
  child {node {21}
         child {node {8}
                child {node {2}}
                child {node {16}}
               }
         child {node {24}}
        }
  child {node {64}
         child {node {48}
                child [missing] {}
                child {node {56}}
               }
         child {node {96}}
        };
\end{tikzpicture}

\medskip

Arbre 4

  \end{minipage}
  \hfillx
  \begin{minipage}[t]{0.50\textwidth}
    %\centering
    \raggedright

    \begin{tabular}{l p{0.5cm} l p{1.00cm} l p{1.00cm}}
    Arité : & & Hauteur : & & Taille : \\
     & & & & \\
    \multicolumn{6}{ l }{Parcours Largeur :} \\
     & & & & \\
     & & & & \\
    \multicolumn{6}{ l }{Parcours Profondeur (ordre préfixe) :} \\
     & & & & \\
     & & & & \\
    \multicolumn{6}{ l }{Parcours Profondeur (ordre infixe) :} \\
     & & & & \\
     & & & & \\
    \multicolumn{6}{ l }{Parcours Profondeur (ordre suffixe) :} \\
     & & & & \\
     & & & & \\
    \multicolumn{6}{ l }{Typologie(s) :} \\
    \end{tabular}

  \end{minipage}
\end{table}


%%%%%%%%%%%%%%%%%%%%%%%%%%%%%%%%%%%%%%%%%%%%%%%%%%%%%%%%%%%%%
\centerline{ \rule{1.0\linewidth}{0.50pt} }
%\clearpage
%%%%%%%%%%%%%%%%%%%%%%%%%%%%%%%%%%%%%%%%%%%%%%%%%%%%%%%%%%%%%

\smallskip


\begin{table}[ht!]
  \centering
  \begin{minipage}{0.45\textwidth}
    \centering

\begin{tikzpicture}[
  leaf/.style = {circle, forestgreen(traditional), draw=green(htmlcssgreen), very thick},
  root/.style = {circle, harvardcrimson, draw=red, very thick},
  internal/.style = {circle, auburn, draw=auburn, very thick},
  level/.style = {sibling distance = 30mm/#1},
  every node/.style = {minimum width = 2em, draw, circle},
  ]

  \node {42}
  child {node {21}
         child [missing] {}
         child {node {24}
                child {node {22}}
                child {node {32}}
               }
        }
  child {node {64}
         child {node {48}
                child [missing] {}
                child {node {56}}
               }
         child {node {96}}
        };
\end{tikzpicture}

\medskip

Arbre 5

  \end{minipage}
  \hfillx
  \begin{minipage}[t]{0.50\textwidth}
    %\centering
    \raggedright

    \begin{tabular}{l p{0.5cm} l p{1.00cm} l p{1.00cm}}
    Arité : & & Hauteur : & & Taille : \\
     & & & & \\
    \multicolumn{6}{ l }{Parcours Largeur :} \\
     & & & & \\
     & & & & \\
    \multicolumn{6}{ l }{Parcours Profondeur (ordre préfixe) :} \\
     & & & & \\
     & & & & \\
    \multicolumn{6}{ l }{Parcours Profondeur (ordre infixe) :} \\
     & & & & \\
     & & & & \\
    \multicolumn{6}{ l }{Parcours Profondeur (ordre suffixe) :} \\
     & & & & \\
     & & & & \\
    \multicolumn{6}{ l }{Typologie(s) :} \\
    \end{tabular}

  \end{minipage}
\end{table}


%\vfillLast

%%%%%%%%%%%%%%%%%%%%%%%%%%%%%%%%%%%%%%%%%%%%%%%%%%%%%%%%%%%%%
%\centerline{ \rule{1.0\linewidth}{0.50pt} }
\clearpage
%%%%%%%%%%%%%%%%%%%%%%%%%%%%%%%%%%%%%%%%%%%%%%%%%%%%%%%%%%%%%

%\smallskip


\begin{table}[ht!]
  \centering
  \begin{minipage}{0.45\textwidth}
    \centering

% sibling distance=0cm, level distance=15mm,
\begin{tikzpicture}[sibling distance=0cm, level distance=12mm,
  leaf/.style = {circle, forestgreen(traditional), draw=green(htmlcssgreen), very thick},
  root/.style = {circle, harvardcrimson, draw=red, very thick},
  internal/.style = {circle, auburn, draw=auburn, very thick},
  level/.style = {sibling distance = 35mm/#1},
  level 3/.style={sibling distance = 8mm},
  every node/.style = {minimum width = 2em, draw, circle},
  ]
  \node {42}
  child { node {21}
          child [missing] {}
          child { node {24}
                  child { node {22} }
                  child [missing] {}
                }
         }
  child [missing] {};
\end{tikzpicture}

Arbre 6

  \end{minipage}
  \hfillx
  \begin{minipage}[t]{0.50\textwidth}
    %\centering
    \raggedright

    \begin{tabular}{l p{0.5cm} l p{1.00cm} l p{1.00cm}}
    Arité : & & Hauteur : & & Taille : \\
     & & & & \\
    \multicolumn{6}{ l }{Parcours Largeur :} \\
     & & & & \\
     & & & & \\
    \multicolumn{6}{ l }{Parcours Profondeur (ordre préfixe) :} \\
     & & & & \\
     & & & & \\
    \multicolumn{6}{ l }{Parcours Profondeur (ordre infixe) :} \\
     & & & & \\
     & & & & \\
    \multicolumn{6}{ l }{Parcours Profondeur (ordre suffixe) :} \\
     & & & & \\
     & & & & \\
    \multicolumn{6}{ l }{Typologie(s) :} \\
    \end{tabular}

  \end{minipage}
\end{table}


%%%%%%%%%%%%%%%%%%%%%%%%%%%%%%%%%%%%%%%%%%%%%%%%%%%%%%%%%%%%%
\centerline{ \rule{1.0\linewidth}{0.50pt} }
%\clearpage
%%%%%%%%%%%%%%%%%%%%%%%%%%%%%%%%%%%%%%%%%%%%%%%%%%%%%%%%%%%%%

\smallskip


\begin{table}[ht!]
  \centering
  \begin{minipage}{0.45\textwidth}
    \centering

% sibling distance=0cm, level distance=15mm,
\begin{tikzpicture}[sibling distance=0cm, level distance=12mm,
  leaf/.style = {circle, forestgreen(traditional), draw=green(htmlcssgreen), very thick},
  root/.style = {circle, harvardcrimson, draw=red, very thick},
  internal/.style = {circle, auburn, draw=auburn, very thick},
  level/.style = {sibling distance = 35mm/#1},
  level 3/.style={sibling distance = 8mm},
  every node/.style = {minimum width = 2em, draw, circle},
  ]
  \node {42}
  child [missing] {}
  child { node {64}
          child [missing] {}
          child { node {96}
                  child [missing] {}
                  child { node {98} }
                }
        };
\end{tikzpicture}

Arbre 7

  \end{minipage}
  \hfillx
  \begin{minipage}[t]{0.50\textwidth}
    %\centering
    \raggedright

    \begin{tabular}{l p{0.5cm} l p{1.00cm} l p{1.00cm}}
    Arité : & & Hauteur : & & Taille : \\
     & & & & \\
    \multicolumn{6}{ l }{Parcours Largeur :} \\
     & & & & \\
     & & & & \\
    \multicolumn{6}{ l }{Parcours Profondeur (ordre préfixe) :} \\
     & & & & \\
     & & & & \\
    \multicolumn{6}{ l }{Parcours Profondeur (ordre infixe) :} \\
     & & & & \\
     & & & & \\
    \multicolumn{6}{ l }{Parcours Profondeur (ordre suffixe) :} \\
     & & & & \\
     & & & & \\
    \multicolumn{6}{ l }{Typologie(s) :} \\
    \end{tabular}

  \end{minipage}
\end{table}


%%%%%%%%%%%%%%%%%%%%%%%%%%%%%%%%%%%%%%%%%%%%%%%%%%%%%%%%%%%%%
\centerline{ \rule{1.0\linewidth}{0.50pt} }
%\clearpage
%%%%%%%%%%%%%%%%%%%%%%%%%%%%%%%%%%%%%%%%%%%%%%%%%%%%%%%%%%%%%

\smallskip


\begin{table}[ht!]
  \centering
  \begin{minipage}{0.45\textwidth}
    \centering

% sibling distance=0cm, level distance=15mm,

%% Peigne gauche
%\begin{tikzpicture}[
%  every node/.style = {minimum width = 2em, draw, circle},
%  ]
%  \node {42}
%  child { node {21}
%		  child { node {8}
%                  child { node {2}
%                          child { node {1} }
%                          child { node {3} }
%                        }
%                  child { node {16} }
%                }
%          child { node {24} }
%        }
%  child { node {64} }
%  ;
%\end{tikzpicture}

%% Peigne droit
\begin{tikzpicture}[
  every node/.style = {minimum width = 2em, draw, circle},
  ]
  \node {42}
  child { node {21} }
  child { node {64}
          child { node {48} }
          child { node {96}
                  child { node {72} }
                  child { node {98}
                          child { node {97} }
                          child { node {99} }
                        }
                }
        };
\end{tikzpicture}

Arbre 8

  \end{minipage}
  \hfillx
  \begin{minipage}[t]{0.50\textwidth}
    %\centering
    \raggedright

    \begin{tabular}{l p{0.5cm} l p{1.00cm} l p{1.00cm}}
    Arité : & & Hauteur : & & Taille : \\
     & & & & \\
    \multicolumn{6}{ l }{Parcours Largeur :} \\
     & & & & \\
     & & & & \\
    \multicolumn{6}{ l }{Parcours Profondeur (ordre préfixe) :} \\
     & & & & \\
     & & & & \\
    \multicolumn{6}{ l }{Parcours Profondeur (ordre infixe) :} \\
     & & & & \\
     & & & & \\
    \multicolumn{6}{ l }{Parcours Profondeur (ordre suffixe) :} \\
     & & & & \\
     & & & & \\
    \multicolumn{6}{ l }{Typologie(s) :} \\
    \end{tabular}

  \end{minipage}
\end{table}


%%%%%%%%%%%%%%%%%%%%%%%%%%%%%%%%%%%%%%%%%%%%%%%%%%%%%%%%%%%%%
%\centerline{ \rule{1.0\linewidth}{0.50pt} }
\clearpage
%%%%%%%%%%%%%%%%%%%%%%%%%%%%%%%%%%%%%%%%%%%%%%%%%%%%%%%%%%%%%

%\smallskip
%\vfillLast


\begin{table}[ht!]
  \centering
  \begin{minipage}{0.45\textwidth}
    \centering

% sibling distance=0cm, level distance=15mm,
\begin{tikzpicture}[sibling distance=0cm, level distance=12mm,
  leaf/.style = {circle, forestgreen(traditional), draw=green(htmlcssgreen), very thick},
  root/.style = {circle, harvardcrimson, draw=red, very thick},
  internal/.style = {circle, auburn, draw=auburn, very thick},
  level/.style = {sibling distance = 35mm/#1},
  level 3/.style={sibling distance = 8mm},
  every node/.style = {minimum width = 2em, draw, circle},
  ]
  \node {24}
  child { node {32}
          child { node {17} }
          child { node {23}
                  child { node {15} }
                  child [missing] {}
                }
        }
  child { node {48}
          child [missing] {}
          child { node {36}
                  child { node {19} }
                  child { node {20}
                          child [missing] {}
                          child { node {42} }
                        }
                }
        };
\end{tikzpicture}

Arbre 9

  \end{minipage}
  \hfillx
  \begin{minipage}[t]{0.50\textwidth}
    %\centering
    \raggedright

    \begin{tabular}{l p{0.5cm} l p{1.00cm} l p{1.00cm}}
    Arité : & & Hauteur : & & Taille : \\
     & & & & \\
    \multicolumn{6}{ l }{Parcours Largeur :} \\
     & & & & \\
     & & & & \\
    \multicolumn{6}{ l }{Parcours Profondeur (ordre préfixe) :} \\
     & & & & \\
     & & & & \\
    \multicolumn{6}{ l }{Parcours Profondeur (ordre infixe) :} \\
     & & & & \\
     & & & & \\
    \multicolumn{6}{ l }{Parcours Profondeur (ordre suffixe) :} \\
     & & & & \\
     & & & & \\
    \multicolumn{6}{ l }{Typologie(s) :} \\
    \end{tabular}

  \end{minipage}
\end{table}


%%%%%%%%%%%%%%%%%%%%%%%%%%%%%%%%%%%%%%%%%%%%%%%%%%%%%%%%%%%%%
\centerline{ \rule{1.0\linewidth}{0.50pt} }
%\clearpage
%%%%%%%%%%%%%%%%%%%%%%%%%%%%%%%%%%%%%%%%%%%%%%%%%%%%%%%%%%%%%

\smallskip


\begin{table}[ht!]
  \centering
  \begin{minipage}{0.45\textwidth}
    \centering

\begin{tikzpicture}[
  leaf/.style = {circle, forestgreen(traditional), draw=green(htmlcssgreen), very thick},
  root/.style = {circle, harvardcrimson, draw=red, very thick},
  internal/.style = {circle, auburn, draw=auburn, very thick},
  level/.style = {sibling distance = 30mm/#1},
  every node/.style = {minimum width = 2em, draw, circle},
  ]

  \node {42}
  child {node {21}
         child {node {8}
                child {node {2}}
                child {node {16}}
               }
         child {node {24}}
        }
  child {node {64}
         child {node {48}
                child [missing] {}
                child {node {56}}
               }
         child {node {96}}
        };
\end{tikzpicture}

\medskip

Arbre 10

  \end{minipage}
  \hfillx
  \begin{minipage}[t]{0.50\textwidth}
    %\centering
    \raggedright

    \begin{tabular}{l p{0.5cm} l p{1.00cm} l p{1.00cm}}
    Arité : & & Hauteur : & & Taille : \\
     & & & & \\
    \multicolumn{6}{ l }{Parcours Largeur :} \\
     & & & & \\
     & & & & \\
    \multicolumn{6}{ l }{Parcours Profondeur (ordre préfixe) :} \\
     & & & & \\
     & & & & \\
    \multicolumn{6}{ l }{Parcours Profondeur (ordre infixe) :} \\
     & & & & \\
     & & & & \\
    \multicolumn{6}{ l }{Parcours Profondeur (ordre suffixe) :} \\
     & & & & \\
     & & & & \\
    \multicolumn{6}{ l }{Typologie(s) :} \\
    \end{tabular}

  \end{minipage}
\end{table}


%%%%%%%%%%%%%%%%%%%%%%%%%%%%%%%%%%%%%%%%%%%%%%%%%%%%%%%%%%%%%
\centerline{ \rule{1.0\linewidth}{0.50pt} }
%\clearpage
%%%%%%%%%%%%%%%%%%%%%%%%%%%%%%%%%%%%%%%%%%%%%%%%%%%%%%%%%%%%%

\smallskip


\begin{table}[ht!]
  \centering
  \begin{minipage}{0.45\textwidth}
    \centering

\begin{tikzpicture}[
  leaf/.style = {circle, forestgreen(traditional), draw=green(htmlcssgreen), very thick},
  root/.style = {circle, harvardcrimson, draw=red, very thick},
  internal/.style = {circle, auburn, draw=auburn, very thick},
  level/.style = {sibling distance = 30mm/#1},
  every node/.style = {minimum width = 2em, draw, circle},
  ]

  \node {42}
  child {node {21}
         child [missing] {}
         child {node {24}
                child {node {22}}
                child {node {32}}
               }
        }
  child {node {64}
         child {node {48}
                child [missing] {}
                child {node {56}}
               }
         child {node {96}}
        };
\end{tikzpicture}

\medskip

Arbre 11

  \end{minipage}
  \hfillx
  \begin{minipage}[t]{0.50\textwidth}
    %\centering
    \raggedright

    \begin{tabular}{l p{0.5cm} l p{1.00cm} l p{1.00cm}}
    Arité : & & Hauteur : & & Taille : \\
     & & & & \\
    \multicolumn{6}{ l }{Parcours Largeur :} \\
     & & & & \\
     & & & & \\
    \multicolumn{6}{ l }{Parcours Profondeur (ordre préfixe) :} \\
     & & & & \\
     & & & & \\
    \multicolumn{6}{ l }{Parcours Profondeur (ordre infixe) :} \\
     & & & & \\
     & & & & \\
    \multicolumn{6}{ l }{Parcours Profondeur (ordre suffixe) :} \\
     & & & & \\
     & & & & \\
    \multicolumn{6}{ l }{Typologie(s) :} \\
    \end{tabular}

  \end{minipage}
\end{table}


%\vfillLast

%%%%%%%%%%%%%%%%%%%%%%%%%%%%%%%%%%%%%%%%%%%%%%%%%%%%%%%%%%%%%
%\centerline{ \rule{1.0\linewidth}{0.50pt} }
%\clearpage
%%%%%%%%%%%%%%%%%%%%%%%%%%%%%%%%%%%%%%%%%%%%%%%%%%%%%%%%%%%%%


%%%%%%%%%%%%%%%%%%%%%%%%%%%%%%%%%%%%%%
\clearpage

\section{Numéros hiérarchiques et représentation tableaux}

\subsection{Numéros hiérarchiques 1}

Indiquez le numéro hiérarchique et la profondeur des nœuds suivants parmi les précédents arbres.

\begin{itemize}
\item \textbf{Arbre 1 : } 42, 24
\item \textbf{Arbre 2 : } 8, 48
\item \textbf{Arbre 4 : } 16, 56
\item \textbf{Arbre 5 : } 32, 96
\item \textbf{Arbre 6 : } 22
\item \textbf{Arbre 8 : } 72, 97
\item \textbf{Arbre 9 : } 19, 23, 42
\end{itemize}

\bigskip

%%%%%%%%%%%%%%%%%%%%%%%%%%%%%%%%%%%

\subsection{Numéros hiérarchiques 2}

Indiquez maintenant le nœud et le niveau associés à chaque numéro hiérarchique parmi les précédents arbres.
Indiquez \textit{NULL} si le nœud n'existe pas, mais indiquez le niveau où il devrait se trouver

\begin{itemize}
\item \textbf{Arbre 1 : } 3
\item \textbf{Arbre 2 : } 4
\item \textbf{Arbre 4 : } 7, 12, 14, 13
\item \textbf{Arbre 5 : } 5, 6, 12, 13, 11
\item \textbf{Arbre 6 : } [Citez plutôt les numéros hiérarchiques qui existent]
\item \textbf{Arbre 7 : } [Citez plutôt les numéros hiérarchiques qui existent]
\item \textbf{Arbre 8 : } 6, 13, 15, 30
\item \textbf{Arbre 9 : } 31, 10, 5, 13
\end{itemize}

%%%%%%%%%%%%%%%%%%%%%%%%%%%%%%%%%%%

\bigskip

\subsection{Représentation tableaux}

Représentez les précédents arbres sous forme de tableaux, indiquez \og \textit{Impossible} \fg{} lorsqu'il n'est pas possible de remplir le tableau à cause de son arité/degré.
Concernant les arbres trop grands, commencez à remplir les premiers niveaux, puis, indiquez à quels numéros hiérarchiques placer les nœuds les plus profonds.

\bigskip

\centerline{
\begin{tabular}{ C{0.51cm}C{0.51cm}C{0.51cm}C{0.51cm}C{0.51cm} C{0.51cm}C{0.51cm}C{0.51cm}C{0.51cm}C{0.51cm} C{0.51cm}C{0.51cm}C{0.51cm}C{0.51cm}C{0.51cm}  C{0.51cm} }
1 & 2 & 3 & 4 & 5 & 6 & 7 & 8 & 9 & 10 & 11 & 12 & 13 & 14 & 15 & ... \\
\end{tabular}
}

\centerline{
\begin{tabular}{ |C{0.50cm}|C{0.50cm}|C{0.50cm}|C{0.50cm}|C{0.50cm} |C{0.50cm}|C{0.50cm}|C{0.50cm}|C{0.50cm}|C{0.50cm} |C{0.50cm}|C{0.50cm}|C{0.50cm}|C{0.50cm}|C{0.50cm}|  C{0.50cm} }
\hline
& & & & &   & & & & &    & & & & &     \\
& & & & &   & & & & &    & & & & & ... \\
\hline
\end{tabular}
}


\bigskip


\centerline{
\begin{tabular}{ C{0.51cm}  C{0.51cm}C{0.51cm}C{0.51cm}C{0.51cm}C{0.51cm} C{0.51cm}C{0.51cm}C{0.51cm}C{0.51cm}C{0.51cm} C{0.51cm}C{0.51cm}C{0.51cm}C{0.51cm}C{0.51cm} C{0.51cm} }
... & 16 & 17 & 18 & 19 & 20 & 21 & 22 & 23 & 24 & 25 & 26 & 27 & 28 & 29 & 30 & 31 \\
\end{tabular}
}

\centerline{
\begin{tabular}{ C{0.50cm} |C{0.50cm}|C{0.50cm}|C{0.50cm}|C{0.50cm}|C{0.50cm} |C{0.50cm}|C{0.50cm}|C{0.50cm}|C{0.50cm}|C{0.50cm} |C{0.50cm}|C{0.50cm}|C{0.50cm}|C{0.50cm}|C{0.50cm}| C{0.50cm}| }
\hline
    &  & & & & &   & & & & &    & & & & & \\
... &  & & & & &   & & & & &    & & & & & \\
\hline
\end{tabular}
}




%%%%%%%%%%%%%%%%%%%%%%%%%%%%%%%%%%%%%%


\bigskip

\vfillFirst

\vfillLast


\begin{center}
\textit{Ce document et ses illustrations ont été réalisés par Fabrice BOISSIER en mars 2025.
Certains exercices sont inspirés des supports de cours de Nathalie "Junior" BOUQUET, et Christophe "Krisboul" BOULLAY.}
\end{center}

\end{document}
