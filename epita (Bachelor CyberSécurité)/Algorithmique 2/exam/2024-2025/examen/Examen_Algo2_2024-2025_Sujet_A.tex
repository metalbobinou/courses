\documentclass[11pt,a4paper]{article}
\usepackage[utf8]{inputenc}
\usepackage[french]{babel}
\usepackage[T1]{fontenc}

\usepackage{amsmath}
\usepackage{amsfonts}
\usepackage{amssymb}

\newcommand{\NomAuteur}{Fabrice BOISSIER}
\newcommand{\TitreMatiere}{Algorithmique 2}
\newcommand{\NomUniv}{EPITA - Bachelor Cyber Sécurité}
\newcommand{\NiveauUniv}{CYBER1}
\newcommand{\NumGroupe}{CYBER1}
\newcommand{\AnneeUniv}{2024-2025}
\newcommand{\DateExam}{avril 2025}
%\newcommand{\TypeExam}{Examen - SUJET 2}
\newcommand{\TypeExam}{Examen (Sujet A)}
\newcommand{\TitreExam}{\TitreMatiere}
\newcommand{\DureeExam}{2h00}
\newcommand{\MyWaterMark}{\AnneeUniv} % Watermark de protection


% Ajout de mes classes & definitions
\usepackage{MetalExam} % Appelle un .sty

% "Tableau" et pas "Table"
\addto\captionsfrench{\def\tablename{Tableau}}

%%%%%%%%%%%%%%%%%%%%%%%
%Header
%%%%%%%%%%%%%%%%%%%%%%%
\lhead{\TypeExam}							%Gauche Haut
\chead{\NomUniv}							%Centre Haut
\rhead{\NumGroupe}							%Droite Haut
\lfoot{\DateExam}							%Gauche Bas
\cfoot{\thepage{} / \pageref*{LastPage}}	%Centre Bas
\rfoot{\texttt{\TitreMatiere}}				%Droite Bas

%%%%%

\usepackage{tabularx}

\newlength{\LabelWidth}%
%\setlength{\LabelWidth}{1.3in}%
\setlength{\LabelWidth}{1cm}%
%\settowidth{\LabelWidth}{Employee E-mail:}%  Specify the widest text here.

% Optional first parameter here specifies the alignment of
% the text within the \makebox.  Default is [l] for left
% alignment. Other options are [r] and [c] for right and center
\newcommand*{\AdjustSize}[2][l]{\makebox[\LabelWidth][#1]{#2}}%


\definecolor{mGreen}{rgb}{0,0.6,0}
\definecolor{mGray}{rgb}{0.5,0.5,0.5}
\definecolor{mPurple}{rgb}{0.58,0,0.82}
\definecolor{backgroundColour}{rgb}{0.95,0.95,0.92}

\lstdefinestyle{CStyle}{
    backgroundcolor=\color{backgroundColour},
    commentstyle=\color{mGreen},
    keywordstyle=\color{magenta},
    numberstyle=\tiny\color{mGray},
    stringstyle=\color{mPurple},
    basicstyle=\footnotesize,
    breakatwhitespace=false,
    breaklines=true,
    captionpos=b,
    keepspaces=true,
    numbers=left,
    numbersep=5pt,
    showspaces=false,
    showstringspaces=false,
    showtabs=false,
    tabsize=2,
    language=C
}


\hyphenation{op-tical net-works SIGKILL}


\begin{document}

%\MakeExamTitleDuree     % Pour afficher la duree
\MakeExamTitle                   % Ne pas afficher la duree

%% \MakeStudentName    %% A reutiliser sur chaque nouvelle page

\bigskip
%\bigskip

Vous devez respecter les consignes suivantes, sous peine de 0 :

\begin{table}[ht!]
  \begin{minipage}{0.45\textwidth}

\begin{enumerate}[label=\Roman*)]
\item Lisez le sujet en entier avec attention
\item Répondez sur le sujet
\item Ne trichez pas
\item Ne détachez pas les agrafes du sujet
\end{enumerate}

  \end{minipage}
  \hfillx
  \begin{minipage}{0.55\textwidth}

\begin{enumerate}[label=\Roman*),start=5]
\item \'Ecrivez lisiblement vos réponses (si nécessaire en majuscules)
%\item Vous devez écrire dans le langage algorithmique classique ou en C (donc pas de Python ou autre)
\item Vous devez écrire les algorithmes et structures en langage C (donc pas de Python ou autre)
\end{enumerate}

  \end{minipage}
\end{table}

%\begin{enumerate}[label=\Roman*)]
%\item Lisez le sujet en entier avec attention
%\item Répondez sur le sujet
%\item Ne détachez pas les agrafes du sujet
%\item \'Ecrivez lisiblement vos réponses (si nécessaire en majuscules)
%%\item Vous devez écrire dans le langage algorithmique classique ou en C (donc pas de Python ou autre)
%\item Vous devez écrire les algorithmes et structures en langage C (donc pas de Python ou autre)
%\item Ne trichez pas
%\end{enumerate}

%\bigskip

%\vfillFirst

%\vspace*{-0.5cm}
\vspace*{-0.75cm}


% Listes chainees
\section{Listes chaînées (9 points)}

%\noindent Dans ces exercices, vous devrez implémenter une liste chaînée ainsi qu'une pile s'appuyant sur cette même liste chaînée.
%Vous serez évaluez autant sur l'implémentation que sur le prototype de chaque fonction (indiquer trop ou trop peu de paramètres ne donnera pas tous les points : limitez-vous aux seuls paramètres essentiels pour chaque fonction).

%\bigskip
%\vspace*{-0.25cm}
\vspace*{-0.50cm}

\begin{table}[ht!]
  \centering
  \begin{minipage}{0.50\textwidth}
    \centering

    \subsection{\'Ecrivez la structure d'une liste chaînée d'entiers (0,5 point) }

    \begin{center}
    \GrilleReponseXY{8}{8}
    \end{center}

  \end{minipage}
  \hfillx
  \begin{minipage}{0.50\textwidth}
    \centering

%    \subsection{\'Ecrivez la fonction \textit{is\_empty\_list} renvoyant \textit{vrai} si la liste est vide, et \textit{faux} dans le cas inverse (1 point)}
    \subsection{\'Ecrivez la fonction \textit{IsEmpty} testant si une liste est vide (0,5 point) }

    \begin{center}
    \GrilleReponseXY{8}{8}
    \end{center}

  \end{minipage}
\end{table}


\vspace*{-0.5cm}

\subsection{(1 point) \'Ecrivez la fonction \textit{LengthList} retournant la longueur d'une liste }

\GrilleReponseN{8}


%\vfillLast

\clearpage

%%%%%%%%%%%%%%%%%%%%%%%%%%%%%%%%%%%%%%%

%\subsection{\'Ecrivez la fonction \textit{length\_list} retournant la longueur d'une liste (1 point) }
%
%\GrilleReponseN{8}
%
%\bigskip


\subsection{(5 points) \'Ecrivez une fonction \textit{insert\_list} insérant un élément \textit{elt} à la position \textit{pos} dans une liste chaînée \textit{L} et respectant les exigences suivantes }

\begin{itemize}
\item La fonction doit renvoyer la tête de la liste (éventuellement la nouvelle tête)
\item Les entiers insérés doivent être positifs, sinon la fonction ne fait rien et retourne \textit{NULL}
\item Le premier élément est considéré comme étant en position $ 1 $
\item Si la liste est vide, l'élément sera inséré en première position
\item Lors de l'ajout, si un élément est déjà présent à la position \textit{pos} donnée en paramètre, alors il faut pousser l'élément existant en position $ pos + 1 $
\item Si la position \textit{pos} donnée en paramètre est supérieure à la longueur, alors on doit insérer l'élément en dernière position de la liste
\item Si la position \textit{pos} donnée en paramètre est inférieure ou égale à 1, alors on doit insérer en première position et décaler l'élément déjà présent s'il y en a un
\end{itemize}

\GrilleReponseN{19}

\clearpage

%%%%%%%%%%%%%%%%%%%%%%%%%%%%%%%%%%%%%%%%%%%%%%%%%%%%%%%%%%%%%%%%%

\GrilleReponseN{9}

\subsection{En réutilisant les fonctions précédentes, et en considérant que vous disposez de la fonction \textit{remove\_list} qui supprime l'élément à une position donnée d'une liste chaînée, réécrivez les fonctions \textit{push} et \textit{pop} d'une pile }

\subsubsection{(1 point) Push }

\GrilleReponseN{6}


\subsubsection{(1 point) Pop }

\GrilleReponseN{6}

\clearpage

%%%%%%%%%%%%%%%%%%%%%%%%%%%%%%%%%%%%%%%%%%%%%%%%%%%%%%%%%%%%%%%%%

% Questions cours
\section{Arbres Binaires (11 points)}

\subsection{Répondez aux différentes questions concernant l'arbre suivant (4 points) }

\bigskip

\begin{center}
\begin{tikzpicture}[sibling distance=1.5cm,
  level/.style = {sibling distance = 55mm/#1},
  level 1/.style = {sibling distance = 55mm},
  level 2/.style = {sibling distance = 30mm},
  level 3/.style = {sibling distance = 15mm},
  level 4/.style = {sibling distance = 10mm},
  every node/.style = {minimum width = 2em, draw, circle},
  ]
  \node (n1S) {S}
  child { node (n2E) {E}
          child { node (n3E) {E}
                  child { node (n4O) {O}
                          child { node [draw=none] (nX) {\phantom{X}} edge from parent [draw=none] }
                          child { node (n5P) {P} }
                        }
                  child { node (n6U) {U} }
                }
          child { node (n7U) {U}
                  child { node (n8T) {T} }
                  child { node (n9O) {O}
                          child { node (n10P) {P} }
                          child { node [draw=none] (nY) {\phantom{Y}} edge from parent [draw=none] }
                        }
                }
         }
   child { node (n11T) {T}
           child { node (n12E) {E}
                   child { node (n13T) {T} }
                   child { node (n14M) {M}
                           child { node (n15C) {C} }
                           child { node (n16A) {A} }
                         }
                 }
           child { node (n17R) {R}
                   child { node (n18B) {B}
                           child { node [draw=none] (nZ) {\phantom{Z}} edge from parent [draw=none] }
                           child { node (n19M) {M} }
                         }
                   child { node (n20E) {E} }
                 }
        };
\end{tikzpicture}
%  child { node [draw=none] (nX) {\phantom{X}} edge from parent [draw=none] }
\end{center}


\subsubsection{(1,5 point) Indiquez toutes les propriétés que possède cet arbre : }

\bigskip

\begin{tabular}{L{2cm}C{1.5cm} L{2cm}C{1.5cm} L{2cm}C{1.5cm} L{2cm}C{1.5cm}}
Arité : & & Taille : & & Hauteur : & & Nb feuilles : & \\
\end{tabular}

\medskip

\begin{table}[ht!]
  \centering
  \begin{minipage}{0.50\textwidth}
    \centering

\begin{itemize}
  \item[\CaseCoche] Arbre binaire strict / localement complet \phantom{()}
  \item[\CaseCoche] Arbre binaire parfait \phantom{()}
  \item[\CaseCoche] Peigne gauche \phantom{()}
\end{itemize}

  \end{minipage}
  \hfillx
  \begin{minipage}{0.50\textwidth}
    \centering

\begin{itemize}
%  \item[\checkmark] Arbre binaire (presque) complet \phantom{()}
  \item[\CaseCoche] Arbre binaire (presque) complet \phantom{()}
  \item[\CaseCoche] Arbre filiforme \phantom{()}
  \item[\CaseCoche] Peigne droit \phantom{()}
\end{itemize}

  \end{minipage}
%\stepcounter{figure}
%\caption{Fig.\thefigure : Recherche de la clé 24 dans un ABR}
%\label{fig:example2-BST-succeed-search}
\end{table}


%\bigskip

\subsubsection{(2 points) \'Ecrivez les clés lors d'un parcours profondeur main gauche de l'arbre dans les 3 ordres ainsi que lors d'un parcours largeur : }

%\medskip

Parcours profondeur :

\medskip

\centerline{
%\begin{tabular}{L{2.5cm} C{0.5cm}C{0.5cm}C{0.5cm}C{0.5cm}C{0.5cm} C{0.5cm}C{0.5cm}C{0.5cm}C{0.5cm} C{0.5cm}C{0.5cm}C{0.5cm}C{0.5cm} C{0.5cm}C{0.5cm}C{0.5cm} C{0.5cm}C{0.5cm}C{0.5cm}}
\begin{tabular}{L{2.5cm} C{0.35cm}C{0.35cm}C{0.35cm}C{0.35cm}C{0.35cm} C{0.35cm}C{0.35cm}C{0.35cm}C{0.35cm} C{0.35cm}C{0.35cm}C{0.35cm}C{0.35cm} C{0.3cm}C{0.35cm}C{0.35cm} C{0.35cm}C{0.35cm}C{0.35cm} C{0.35cm}}
 & & & & & & & & & & & & & & & & & & & \\
ordre préfixe : & \_ & \_ & \_ & \_ & \_ & \_ & \_ & \_ & \_ & \_ & \_ & \_ & \_ & \_ & \_ & \_ & \_ & \_ & \_ & \_ \\
 & & & & & & & & & & & & & & & & & & & \\
ordre infixe :  & \_ & \_ & \_ & \_ & \_ & \_ & \_ & \_ & \_ & \_ & \_ & \_ & \_ & \_ & \_ & \_ & \_ & \_ & \_ & \_ \\
 & & & & & & & & & & & & & & & & & & & \\
ordre suffixe : & \_ & \_ & \_ & \_ & \_ & \_ & \_ & \_ & \_ & \_ & \_ & \_ & \_ & \_ & \_ & \_ & \_ & \_ & \_ & \_ \\
\end{tabular}
}

%\bigskip
\vspace*{0.75cm}

Parcours largeur :

\medskip

\centerline{
%\begin{tabular}{L{2.5cm} C{0.5cm}C{0.5cm}C{0.5cm}C{0.5cm}C{0.5cm} C{0.5cm}C{0.5cm}C{0.5cm}C{0.5cm} C{0.5cm}C{0.5cm}C{0.5cm}C{0.5cm} C{0.5cm}C{0.5cm}C{0.5cm} C{0.5cm}C{0.5cm}C{0.5cm}}
\begin{tabular}{L{2.5cm} C{0.35cm}C{0.35cm}C{0.35cm}C{0.35cm}C{0.35cm} C{0.35cm}C{0.35cm}C{0.35cm}C{0.35cm} C{0.35cm}C{0.35cm}C{0.35cm}C{0.35cm} C{0.3cm}C{0.35cm}C{0.35cm} C{0.35cm}C{0.35cm}C{0.35cm} C{0.35cm}}
 & & & & & & & & & & & & & & & & & & & \\
ordre :  & \_ & \_ & \_ & \_ & \_ & \_ & \_ & \_ & \_ & \_ & \_ & \_ & \_ & \_ & \_ & \_ & \_ & \_ & \_ & \_ \\
\end{tabular}
}

\medskip

%\subsubsection{(0,5 point) Indiquez la profondeur des nœuds suivants, ainsi que leur numéro hiérarchique : }
\subsubsection{(0,5 point) Indiquez la profondeur et le numéro hiérarchique des nœuds suivants : }

\medskip

\centerline{
\begin{tabular}{ | C{1cm}|C{1.75cm}|C{2.65cm} |   p{2.0cm}   | C{1cm}|C{1.75cm}|C{2.65cm} | }
\cline{1-3}\cline{5-7}
%\hline
 & Profondeur & N\textsuperscript{o} hiérarchique &
% & Profondeur & N\textsuperscript{o} hiérarchique &*
 &
 & Profondeur & N\textsuperscript{o} hiérarchique \\
\cline{1-3}\cline{5-7}
%\hline
\multirow{3}{*}{B} & & &
 &
%\multirow{3}{*}{X} & & &
\multirow{3}{*}{C} & & \\
% & & & & & & & & \\
% & & & & & & & & \\
 & & & & & & \\
 & & & & & & \\
%\hline
\cline{1-3}\cline{5-7}
\end{tabular}
}

\clearpage

%%%%%%%%%%%%%%%%%%%%%%%%%%%%%%%%%%%%%%%%%%%%%%%

\subsection*{Algorithmes (7 points) }

%\begin{table}[ht!]
%  \centering
%  \begin{minipage}{0.50\textwidth}
%    \centering
%
    \subsection{(0,5 point) \'Ecrivez la structure récursive \textit{node} permettant de représenter des arbres binaires contenant des nombres entiers : }

    \begin{center}
    %\GrilleReponseXY{8}{8}
    \GrilleReponseN{8}
    \end{center}

%  \end{minipage}
%  \hfillx
%  \begin{minipage}{0.50\textwidth}
%    \centering

%    \subsection{(0,5 point) \'Ecrivez la fonction récursive \textit{Height} retournant la hauteur d'un arbre binaire (l'arbre est de type \textit{node*}) : }
%
%    \begin{center}
%    %\GrilleReponseXY{8}{8}
%    \GrilleReponseN{10}
%    \end{center}
%
%  \end{minipage}
%\end{table}


%%%%%%%%%%%%%%%%%%%%%%%%%%%%%%%%%%%%%%%%%%%%%%%

\subsection{(2 points) \'Ecrivez une fonction récursive \og \textit{parc\_prof\_rec} \fg{} effectuant un parcours profondeur main gauche dans un arbre binaire, et affichant les nœuds dans l'ordre \textit{suffixe} (l'arbre est de type \textit{node*}) : }

\begin{center}
\GrilleReponseN{13}
\end{center}


\clearpage

%%%%%%%%%%%%%%%%%%%%%%%%%%%%%%%%%%%%%%%%%%%%%%%

\subsection{(2 points) \'Ecrivez une fonction itérative \og \textit{parc\_larg\_iter} \fg{} effectuant un parcours largeur dans un arbre binaire, et affichant les nœuds (l'arbre est de type \textit{node*}) : }

\begin{center}

Vous pouvez utiliser les conteneurs externes suivants avec leurs opérations :

\medskip

\begin{tabular}{ |c|c|c| }
\hline
\textbf{Liste}    & \textbf{File}     & \textbf{Pile} \\ \hline
\textit{list\_p}  & \textit{queue\_p} & \textit{stack\_p} \\ \hline
Create  & Create  & Create  \\ \hline
Length  & Length  & Length  \\ \hline
IsEmpty & IsEmpty & IsEmpty \\ \hline
Insert  & Enqueue & Push    \\ \hline
Remove  & Dequeue & Pop     \\ \hline
Clear   & Clear   & Clear   \\ \hline
Delete  & Delete  & Delete  \\ \hline
\end{tabular}

\medskip

%\begin{center}
\GrilleReponseN{18}
\end{center}

\clearpage

%%%%%%%%%%%%%%%%%%%%%%%%%%%%%%%%%%%%%%%%%%%%%%%

\subsection{(2,5 points) \'Ecrivez une fonction \og \textit{node\_to\_array} \fg{} transformant un arbre au format \textit{node*} vers le format tableau \textit{int*} : }

\noindent Le tableau est donné en paramètre et est déjà alloué avec la bonne taille : votre fonction ne doit \textit{que} le remplir avec les bonnes valeurs.
La taille du tableau est évidemment fournie en paramètre.
Un nœud vide doit être représenté par \og $-1$ \fg.

\begin{center}
\GrilleReponseN{22}
\end{center}


%\bigskip

%\vfillLast

\clearpage


%%%%%%%%%%%%%%%%%%%%%%%%%%%%%%%%%%%%%%%%%%%%%%%%%%%%%%%%

%\thispagestyle{empty}

\vfillFirst

\begin{center}

\begin{LARGE}
\textbf{SUJET A}

\bigskip

\textbf{\MakeUppercase{\TitreMatiere}}
\end{LARGE}

\end{center}

\vfillLast

\end{document}
