\documentclass[11pt,a4paper]{article}
\usepackage[utf8]{inputenc}
\usepackage[french]{babel}
\usepackage[T1]{fontenc}

\usepackage{amsmath}
\usepackage{amsfonts}
\usepackage{amssymb}

\newcommand{\NomAuteur}{Fabrice BOISSIER}
\newcommand{\TitreMatiere}{Algorithmique 2}
\newcommand{\NomUniv}{EPITA - Bachelor Cyber Sécurité}
\newcommand{\NiveauUniv}{CYBER1}
\newcommand{\NumGroupe}{CYBER1}
\newcommand{\AnneeUniv}{2024-2025}
\newcommand{\DateExam}{juin 2025}
%\newcommand{\TypeExam}{Rattrapage}
\newcommand{\TypeExam}{CORRECTION Rattrapage}
\newcommand{\TitreExam}{\TitreMatiere}
\newcommand{\DureeExam}{2h00}
\newcommand{\MyWaterMark}{\AnneeUniv} % Watermark de protection

% Ajout de mes classes & definitions
\usepackage{MetalExam} % Appelle un .sty

% "Tableau" et pas "Table"
\addto\captionsfrench{\def\tablename{Tableau}}

%%%%%%%%%%%%%%%%%%%%%%%
%Header
%%%%%%%%%%%%%%%%%%%%%%%
\lhead{\TypeExam}							%Gauche Haut
\chead{\NomUniv}							%Centre Haut
\rhead{\NumGroupe}							%Droite Haut
\lfoot{\DateExam}							%Gauche Bas
\cfoot{\thepage{} / \pageref*{LastPage}}	%Centre Bas
\rfoot{\texttt{\TitreMatiere}}				%Droite Bas

%%%%%

\usepackage{tabularx}

\newlength{\LabelWidth}%
%\setlength{\LabelWidth}{1.3in}%
\setlength{\LabelWidth}{1cm}%
%\settowidth{\LabelWidth}{Employee E-mail:}%  Specify the widest text here.

% Optional first parameter here specifies the alignment of
% the text within the \makebox.  Default is [l] for left
% alignment. Other options are [r] and [c] for right and center
\newcommand*{\AdjustSize}[2][l]{\makebox[\LabelWidth][#1]{#2}}%


\definecolor{mGreen}{rgb}{0,0.6,0}
\definecolor{mGray}{rgb}{0.5,0.5,0.5}
\definecolor{mPurple}{rgb}{0.58,0,0.82}
\definecolor{backgroundColour}{rgb}{0.95,0.95,0.92}

\lstdefinestyle{CStyle}{
    backgroundcolor=\color{backgroundColour},
    commentstyle=\color{mGreen},
    keywordstyle=\color{magenta},
    numberstyle=\tiny\color{mGray},
    stringstyle=\color{mPurple},
    basicstyle=\footnotesize,
    breakatwhitespace=false,
    breaklines=true,
    captionpos=b,
    keepspaces=true,
    numbers=left,
    numbersep=5pt,
    showspaces=false,
    showstringspaces=false,
    showtabs=false,
    tabsize=2,
    language=C
}


\hyphenation{op-tical net-works SIGKILL}


\begin{document}

%\MakeExamTitleDuree     % Pour afficher la duree
\MakeExamTitle                   % Ne pas afficher la duree

%% \MakeStudentName    %% A reutiliser sur chaque nouvelle page

\bigskip
%\bigskip

Vous devez respecter les consignes suivantes, sous peine de 0 :

\begin{table}[ht!]
  \begin{minipage}{0.45\textwidth}

\begin{enumerate}[label=\Roman*)]
\item Lisez le sujet en entier avec attention
\item Répondez sur le sujet
\item Ne trichez pas
\item Ne détachez pas les agrafes du sujet
\end{enumerate}

  \end{minipage}
  \hfillx
  \begin{minipage}{0.55\textwidth}

\begin{enumerate}[label=\Roman*),start=5]
\item \'Ecrivez lisiblement vos réponses (si nécessaire en majuscules)
%\item Vous devez écrire dans le langage algorithmique classique ou en C (donc pas de Python ou autre)
\item Vous devez écrire les algorithmes et structures en langage C (donc pas de Python ou autre)
\end{enumerate}

  \end{minipage}
\end{table}

%\begin{enumerate}[label=\Roman*)]
%\item Lisez le sujet en entier avec attention
%\item Répondez sur le sujet
%\item Ne détachez pas les agrafes du sujet
%\item \'Ecrivez lisiblement vos réponses (si nécessaire en majuscules)
%%\item Vous devez écrire dans le langage algorithmique classique ou en C (donc pas de Python ou autre)
%\item Vous devez écrire les algorithmes et structures en langage C (donc pas de Python ou autre)
%\item Ne trichez pas
%\end{enumerate}

%\bigskip

%\vfillFirst

%\vspace*{-0.5cm}
\vspace*{-0.75cm}


% Arbres Binaires
\section{Arbres Binaires (8 points)}

\subsection*{Questions}


\subsection{Dessinez un arbre répondant aux critères imposés (2 points) }

%\begin{center}
%\textit{Les arbres ne doivent pas être des arbres vides}
%\end{center}
%
%\vspace{-0.5cm}

\begin{table}[ht!]
  \centering
  \begin{minipage}{0.50\textwidth}
    \centering

\subsubsection{(0,5 point) Arbre binaire localement complet}

\textit{(hauteur minimale : 3)}

\begin{tikzpicture}[
  leaf/.style = {circle, forestgreen(traditional), draw=green(htmlcssgreen), very thick},
  root/.style = {circle, harvardcrimson, draw=red, very thick},
  internal/.style = {circle, auburn, draw=auburn, very thick},
  level/.style = {sibling distance = 55mm/#1},
  level 1/.style = {sibling distance = 38mm},
  level 2/.style = {sibling distance = 19mm},
  level 3/.style = {sibling distance = 9mm},
  every node/.style = {minimum width = 2em, draw, circle},
  ]
  \node {42}
  child { node {38} }
  child { node {64}
          child { node {56}
                  child { node {48} }
                  child { node {60} }
                }
          child { node {80} }
        };
\end{tikzpicture}


  \end{minipage}
  \hfillx
  \begin{minipage}{0.50\textwidth}
    \centering

\subsubsection{(0,5 point) Arbre filiforme}

\textit{(hauteur minimale : 4)}

\begin{tikzpicture}[
  leaf/.style = {circle, forestgreen(traditional), draw=green(htmlcssgreen), very thick},
  root/.style = {circle, harvardcrimson, draw=red, very thick},
  internal/.style = {circle, auburn, draw=auburn, very thick},
  level/.style = {sibling distance = 55mm/#1},
  level 1/.style = {sibling distance = 38mm},
  level 2/.style = {sibling distance = 19mm},
  level 3/.style = {sibling distance = 9mm},
  every node/.style = {minimum width = 2em, draw, circle},
  ]
  \node {42}
  child { edge from parent[draw = none] }
  child { node {64}
          child { edge from parent[draw = none] }
          child { node {72}
                  child { edge from parent[draw = none] }
                  child { node {96}
                          child { edge from parent[draw = none] }
                          child { node {98} }
                        }
                }
        };
\end{tikzpicture}


  \end{minipage}
\end{table}

%%%%%%%%%%%%%%%%
\vspace*{-0.5cm}
\rule{1.0\linewidth}{0.75pt}
%%%%%%%%%%%%%%%%

\begin{table}[ht!]
  \centering
  \begin{minipage}{0.50\textwidth}
    \centering

\subsubsection{(0,5 point) Arbre binaire parfait}

\textit{(hauteur minimale : 2)}

\vspace*{1cm}

\begin{tikzpicture}[
  leaf/.style = {circle, forestgreen(traditional), draw=green(htmlcssgreen), very thick},
  root/.style = {circle, harvardcrimson, draw=red, very thick},
  internal/.style = {circle, auburn, draw=auburn, very thick},
  level/.style = {sibling distance = 55mm/#1},
  level 1/.style = {sibling distance = 38mm},
  level 2/.style = {sibling distance = 19mm},
  level 3/.style = {sibling distance = 9mm},
  every node/.style = {minimum width = 2em, draw, circle},
  ]
  \node {42}
  child { node {21}
          child { node {8}  }
          child { node {24} }
         }
  child { node {64}
          child { node {48} }
          child { node {96} }
        };
\end{tikzpicture}

  \end{minipage}
  \hfillx
  \begin{minipage}{0.50\textwidth}
    \centering

\subsubsection{(0,5 point) Arbre binaire presque complet}

\textit{(hauteur minimale : 3)}

\begin{tikzpicture}[
  leaf/.style = {circle, forestgreen(traditional), draw=green(htmlcssgreen), very thick},
  root/.style = {circle, harvardcrimson, draw=red, very thick},
  internal/.style = {circle, auburn, draw=auburn, very thick},
  level/.style = {sibling distance = 55mm/#1},
  level 1/.style = {sibling distance = 38mm},
  level 2/.style = {sibling distance = 19mm},
  level 3/.style = {sibling distance = 9mm},
  every node/.style = {minimum width = 2em, draw, circle},
  ]
  \node {42}
  child { node {38}
          child { node {24}
                  child { node {18} }
                  child { node {32} }
                }
          child { node {40}
                  child { node {39} }
                  child { node {41} }
                }
        }
  child { node {64}
          child { node {56}
                  child { node {48} }
                  child { edge from parent[draw = none] }
                }
          child { node {80} }
        };
\end{tikzpicture}

  \end{minipage}
\end{table}



\clearpage

%%%%%%%%%%%%%%%%%%%%%%%%%%%%%%%%%%%%%%%%%%%%%%%

\subsection{Répondez aux différentes questions concernant l'arbre suivant (4 points) }

\bigskip

% lvl1 = 55mm, lvl2 = 30mm, lvl3 = 15mm, lvl4 = 10mm
\begin{center}
\begin{tikzpicture}[sibling distance=1.5cm,
  level/.style = {sibling distance = 55mm/#1},
  level 1/.style = {sibling distance = 70mm},
  level 2/.style = {sibling distance = 35mm},
  level 3/.style = {sibling distance = 18mm},
  level 4/.style = {sibling distance = 8mm},
  every node/.style = {minimum width = 2em, draw, circle},
  ]
  \node (n1R) {R}
  child { node (n2A) {A}
          child { node (n3T) {T}
                  child { node (n4T) {T} }
                  child { node (n5R) {R}
                          child { node (n6A) {A} }
                          child { node (n7P) {P} }
                        }
                }
          child { node (n8E) {E}
                  child { node (n9Z) {Z} }
                  child { node (n10L) {L}
                          child { node (n11Z) {E} }
                          child { node (n12S) {S} }
                        }
                }
         }
   child { node (n13T) {T}
           child { node (n14O) {O} }
           child { node (n15U) {U}
                   child { node (n16A) {S} }
                   child { node (n17I) {!}
                           child { node [draw=none] (nX) {\phantom{X}} edge from parent [draw=none] }
                           child { node [draw=none] (nX) {\phantom{X}} edge from parent [draw=none] }
                         }
                 }
        };
\end{tikzpicture}
%  child { node [draw=none] (nX) {\phantom{X}} edge from parent [draw=none] }
\end{center}


\subsubsection{(1,5 point) Indiquez toutes les propriétés que possède cet arbre : }

\bigskip

\begin{tabular}{L{2cm}C{1.5cm} L{2cm}C{1.5cm} L{2cm}C{1.5cm} L{2cm}C{1.5cm}}
Arité : 2 & & Taille : 17 & & Hauteur : 4 & & Nb feuilles : 9 & \\
\end{tabular}

\medskip

\begin{table}[ht!]
  \centering
  \begin{minipage}{0.50\textwidth}
    \centering

\begin{itemize}
  \item[\checkmark] Arbre binaire strict / localement complet \phantom{()}
  \item[\CaseCoche] Arbre binaire parfait \phantom{()}
  \item[\CaseCoche] Peigne gauche \phantom{()}
\end{itemize}

  \end{minipage}
  \hfillx
  \begin{minipage}{0.50\textwidth}
    \centering

\begin{itemize}
%  \item[\checkmark] Arbre binaire (presque) complet \phantom{()}
  \item[\CaseCoche] Arbre binaire (presque) complet \phantom{()}
  \item[\CaseCoche] Arbre filiforme \phantom{()}
  \item[\CaseCoche] Peigne droit \phantom{()}
\end{itemize}

  \end{minipage}
\end{table}


%\bigskip

\subsubsection{(2 points) \'Ecrivez les clés lors d'un parcours profondeur main gauche de l'arbre dans les 3 ordres ainsi que lors d'un parcours largeur : }

%\medskip

Parcours profondeur :

\medskip

\centerline{
%\begin{tabular}{L{2.5cm} C{0.5cm}C{0.5cm}C{0.5cm}C{0.5cm}C{0.5cm} C{0.5cm}C{0.5cm}C{0.5cm}C{0.5cm} C{0.5cm}C{0.5cm}C{0.5cm}C{0.5cm} C{0.5cm}C{0.5cm}C{0.5cm} C{0.5cm}C{0.5cm}C{0.5cm}}
\begin{tabular}{L{2.5cm} C{0.35cm}C{0.35cm}C{0.35cm}C{0.35cm}C{0.35cm} C{0.35cm}C{0.35cm}C{0.35cm}C{0.35cm} C{0.35cm}C{0.35cm}C{0.35cm}C{0.35cm} C{0.3cm}C{0.35cm}C{0.35cm} C{0.35cm}C{0.35cm}C{0.35cm} C{0.35cm}}
 & & & & & & & & & & & & & & & & & & & \\
ordre préfixe : & R & A & T & T & R & A & P & E & Z & L & E & S & T & O & U & S & ! & & & \\
 & & & & & & & & & & & & & & & & & & & \\
ordre infixe :  & T & T & A & R & P & A & Z & E & E & L & S & R & O & T & S & U & ! & & & \\
 & & & & & & & & & & & & & & & & & & & \\
ordre suffixe : & T & A & P & R & T & Z & E & S & L & E & A & O & S & ! & U & T & R & & & \\
\end{tabular}
}

%\bigskip
\vspace*{0.75cm}

Parcours largeur :

\medskip

\centerline{
%\begin{tabular}{L{2.5cm} C{0.5cm}C{0.5cm}C{0.5cm}C{0.5cm}C{0.5cm} C{0.5cm}C{0.5cm}C{0.5cm}C{0.5cm} C{0.5cm}C{0.5cm}C{0.5cm}C{0.5cm} C{0.5cm}C{0.5cm}C{0.5cm} C{0.5cm}C{0.5cm}C{0.5cm}}
\begin{tabular}{L{2.5cm} C{0.35cm}C{0.35cm}C{0.35cm}C{0.35cm}C{0.35cm} C{0.35cm}C{0.35cm}C{0.35cm}C{0.35cm} C{0.35cm}C{0.35cm}C{0.35cm}C{0.35cm} C{0.3cm}C{0.35cm}C{0.35cm} C{0.35cm}C{0.35cm}C{0.35cm} C{0.35cm}}
 & & & & & & & & & & & & & & & & & & & \\
ordre :  & R & A & T & T & E & O & U & T & R & Z & L & S & ! & A & P & E & S & & & \\
\end{tabular}
}

\medskip

\subsubsection{(0,5 point) Indiquez la profondeur et le numéro hiérarchique des nœuds suivants : }

\medskip

\centerline{
\begin{tabular}{ | C{1cm}|C{1.75cm}|C{2.65cm} |   p{2.0cm}   | C{1cm}|C{1.75cm}|C{2.65cm} | }
\cline{1-3}\cline{5-7}
 & Profondeur & N\textsuperscript{o} hiérarchique &
 &
 & Profondeur & N\textsuperscript{o} hiérarchique \\
\cline{1-3}\cline{5-7}
\multirow{3}{*}{L} & \multirow{3}{*}{3} & \multirow{3}{*}{11} &
 &
\multirow{3}{*}{P} & \multirow{3}{*}{4} & \multirow{3}{*}{19} \\
 & & & & & & \\
 & & & & & & \\
\cline{1-3}\cline{5-7}
\end{tabular}
}


\clearpage

%%%%%%%%%%%%%%%%%%%%%%%%%%%%%%%%%%%%%%%%%%%%%%%

\subsection*{Algorithmes}


\subsection{(0,5 point) Rappelez sur quelle structure doit s'appuyer chaque implémentation itérative des parcours d'un arbre binaire : }


\begin{center}
\begin{table}[ht!]
%  \centering
  \begin{minipage}{0.50\textwidth}
  \centering

\begin{tabular}{|L{5cm}|L{2cm}|}
\hline
\multirow{3}{*}[0pt]{\begin{minipage}{4.85cm} Parcours Profondeur \end{minipage}}
 & \\
 & PILE \\
 & \\
\hline
\end{tabular}

  \end{minipage}
  \hfillx
  \begin{minipage}{0.50\textwidth}
  \centering

\begin{tabular}{|L{5cm}|L{2cm}|}
\hline
\multirow{3}{*}[0pt]{\begin{minipage}{4.85cm} Parcours Largeur \end{minipage}}
 & \\
 & FILE \\
 & \\
\hline
\end{tabular}

  \end{minipage}
\end{table}
\end{center}


%%%%%%%%%%%%%%%%%%%%%%%%%%%%%%%%%%%%%%%%%%%%%%%

\subsection{(1,5 point) \'Ecrivez une procédure \og \textit{ParcProf} \fg{} effectuant un parcours profondeur dans un arbre binaire, et affichant les nœuds uniquement dans \textit{un} ordre de votre choix que vous devrez indiquer (l'arbre est de type \textit{node*}) : }

\begin{center}
\GrilleReponseN{15}

\vspace*{1cm}

\begin{tabular}{|L{3cm}|L{5cm}|}
\hline
\multirow{3}{*}[0pt]{\begin{minipage}{2.85cm} Ordre choisi : \end{minipage}}
 & \\
 & \\
 & \\
\hline
\end{tabular}
\end{center}


\clearpage

%%%%%%%%%%%%%%%%%%%%%%%%%%%%%%%%%%%%%%%%%%%%%%%%%%%%%%%%%%%%%%%%%

% Arbres Binaires de Recherche
\section{Arbres Binaires de Recherche (6 points)}

\subsection*{Questions}


\subsection{(1 point) Indiquez pour chacun des cas suivants s'il s'agit d'un arbre binaire de recherche ou non : }


\begin{table}[ht!]
  \centering
  \begin{minipage}{0.48\textwidth}
    \centering

\begin{tikzpicture}[
  leaf/.style = {circle, forestgreen(traditional), draw=green(htmlcssgreen), very thick},
  root/.style = {circle, harvardcrimson, draw=red, very thick},
  internal/.style = {circle, auburn, draw=auburn, very thick},
  level/.style = {sibling distance = 55mm/#1},
  level 1/.style = {sibling distance = 38mm},
  level 2/.style = {sibling distance = 19mm},
  level 3/.style = {sibling distance = 9mm},
  every node/.style = {minimum width = 2em, draw, circle},
  ]
  \node {42}
  child { node {21}
          child { node {8}
                  child { node {2} }
                  child { node {16} }
                }
          child { node {24}
                  child { node {22} }
                  child { node {32} }
                }
         }
  child { node {64}
          child { node[root] {48}
                  child { node[root] {40} }
                  child { node {56} }
                }
          child { node {96}
                  child { node {72} }
                  child { node {98} }
                }
        };
\end{tikzpicture}

\begin{center}
\CaseCocheTexte{\sout{Est un ABR}}  \hspace*{0.5cm}  \checkmark{N'est pas un ABR}
\end{center}

  \end{minipage}
    \hfillx
  \vrule\begin{minipage}{0.02\textwidth}

  \end{minipage}
  \hfillx
  \begin{minipage}{0.48\textwidth}
    \centering

\begin{tikzpicture}[
  leaf/.style = {circle, forestgreen(traditional), draw=green(htmlcssgreen), very thick},
  root/.style = {circle, harvardcrimson, draw=red, very thick},
  internal/.style = {circle, auburn, draw=auburn, very thick},
  level/.style = {sibling distance = 55mm/#1},
  level 1/.style = {sibling distance = 38mm},
  level 2/.style = {sibling distance = 19mm},
  level 3/.style = {sibling distance = 9mm},
  every node/.style = {minimum width = 2em, draw, circle},
  ]
  \node {78}
  child { node {20}
          child { edge from parent[draw = none] }
          child { node {40}
                  child { node {22} }
                  child { node {56} }
                }
         }
  child { edge from parent[draw = none] };
\end{tikzpicture}

\begin{center}
\checkmark{Est un ABR}  \hspace*{0.5cm}  \CaseCocheTexte{\sout{N'est pas un ABR}}
\end{center}

  \end{minipage}
\end{table}

%%%%%%%%%%%%%%%%%%%%%%%%%%%%%
\rule{1.0\linewidth}{0.25pt}
%%%%%%%%%%%%%%%%%%%%%%%%%%%%%
\vspace*{0.10cm}

\begin{table}[ht!]
  \centering
  \begin{minipage}{0.48\textwidth}
    \centering

\begin{tikzpicture}[
  leaf/.style = {circle, forestgreen(traditional), draw=green(htmlcssgreen), very thick},
  root/.style = {circle, harvardcrimson, draw=red, very thick},
  internal/.style = {circle, auburn, draw=auburn, very thick},
  level/.style = {sibling distance = 55mm/#1},
  level 1/.style = {sibling distance = 38mm},
  level 2/.style = {sibling distance = 19mm},
  level 3/.style = {sibling distance = 9mm},
  every node/.style = {minimum width = 2em, draw, circle},
  ]
  \node {50}
  child { node {25}
          child { node {18} }
          child { node {36}
                  child { node {28} }
                  child { node {40} }
                }
         }
  child { node {70}
          child { node {60} }
          child { node {96}
                  child { node {80} }
                  child { edge from parent[draw = none] }
                }
        };
\end{tikzpicture}

\begin{center}
\checkmark{Est un ABR}  \hspace*{0.5cm}  \CaseCocheTexte{\sout{N'est pas un ABR}}
\end{center}

  \end{minipage}
    \hfillx
  \vrule\begin{minipage}{0.02\textwidth}

  \end{minipage}
  \hfillx
  \begin{minipage}{0.48\textwidth}
    \centering

\begin{tikzpicture}[
  leaf/.style = {circle, forestgreen(traditional), draw=green(htmlcssgreen), very thick},
  root/.style = {circle, harvardcrimson, draw=red, very thick},
  internal/.style = {circle, auburn, draw=auburn, very thick},
  level/.style = {sibling distance = 55mm/#1},
  level 1/.style = {sibling distance = 38mm},
  level 2/.style = {sibling distance = 19mm},
  level 3/.style = {sibling distance = 9mm},
  every node/.style = {minimum width = 2em, draw, circle},
  ]
  \node {38}
  child { node {23}
          child { node {15}
                  child { node {15} }
                  child { node {20} }
                }
          child { node {25} }
         }
  child { node {50}
          child { node {50}
                  child { node {50} }
                  child { edge from parent[draw = none] }
                }
          child { node {75}
                  child { node {75} }
                  child { edge from parent[draw = none] }
                }
        };
\end{tikzpicture}

\begin{center}
\checkmark{Est un ABR}  \hspace*{0.5cm}  \CaseCocheTexte{\sout{N'est pas un ABR}}
\end{center}

  \end{minipage}
\end{table}


%%%%%%%%%%%%%%%%%%%%%%%%%%%%%%%%%%%%%%%%%%%%%%%%%%%%%%%%%%%%%%%%%%%%%%%


\subsection{(3 points) Dessinez le résultat des opérations successives : }

\begin{itemize}
\item \textit{InsertLeaf(node *R, int val)} insère l'élément \og \textit{val} \fg{} en feuille dans l'ABR \og \textit{R} \fg{}
\item \textit{InsertRoot(node *R, int vat)} insère l'élément \og \textit{val} \fg{} en racine dans l'ABR \og \textit{R} \fg{}
\item \textit{RemoveFromBST(node *R, int val)} supprime l'élément \og \textit{val} \fg{} de l'ABR \og \textit{R} \fg{}
\end{itemize}

\vspace*{-0.5cm}

\begin{center}

%\'Eléments insérés : 18 - 46 - 55 - 36 - 12 - 38 - 96 - 71
%\'Eléments insérés : 46 - 18 - 55 - 36 - 12 - 38 - 96 - 71
%\'Eléments insérés : 32 - 24 - 18 - 30 - 31 - 42 - 36 - 50
%\'Eléments insérés : 32 - 24 - 18 - 30 - 31 - 42 - 36
%\'Eléments insérés : F42, F24, F64, F32, F56, F36, R34, F72, R38, -42, -38, R60


% 1-2-3 : F50, F30, F80

\begin{table}[ht!]
  \centering
  \begin{minipage}{0.33\textwidth}
    \centering

\textit{\'Etape 1}

R = InsertLeaf(NULL, 50)

\medskip

\vspace*{0.75cm}

\begin{tikzpicture}[
  leaf/.style = {circle, forestgreen(traditional), draw=green(htmlcssgreen), very thick},
  root/.style = {circle, harvardcrimson, draw=red, very thick},
  internal/.style = {circle, auburn, draw=auburn, very thick},
  level/.style = {sibling distance = 55mm/#1},
  level 1/.style = {sibling distance = 38mm},
  level 2/.style = {sibling distance = 19mm},
  level 3/.style = {sibling distance = 9mm},
  every node/.style = {minimum width = 2em, draw, circle},
  ]
  \node {50};
\end{tikzpicture}

\vspace*{0.75cm}

  \end{minipage}
  \hfillx
  \begin{minipage}{0.33\textwidth}
    \centering

\textit{\'Etape 2}

R = InsertLeaf(R, 30)

\medskip

%\vspace*{0.5cm}

\begin{tikzpicture}[
  level 1/.style = {sibling distance = 20mm},
  every node/.style = {minimum width = 2em, draw, circle},
  ]
  \node {50}
  child { node {30} }
  child { edge from parent[draw = none]
        };
\end{tikzpicture}

%\vspace*{0.5cm}

  \end{minipage}
  \hfillx
  \begin{minipage}{0.33\textwidth}
    \centering

\textit{\'Etape 3}

R = InsertLeaf(R, 80)

\medskip

%\vspace*{0.5cm}

\begin{tikzpicture}[
  level 1/.style = {sibling distance = 20mm},
  every node/.style = {minimum width = 2em, draw, circle},
  ]
  \node {50}
  child { node {30} }
  child { node {80}
        };
\end{tikzpicture}

%\vspace*{0.5cm}

  \end{minipage}
\end{table}

%%%%%%%%%%
\clearpage
%%%%%%%%%%

% 4-5-6 : F20, F60, F40

\begin{table}[ht!]
  \centering
  \begin{minipage}{0.33\textwidth}
    \centering

\textit{\'Etape 4}

R = InsertLeaf(R, 20)

\medskip

\vspace*{0.5cm}

\begin{tikzpicture}[
  level 1/.style = {sibling distance = 30mm},
  level 2/.style = {sibling distance = 15mm},
  every node/.style = {minimum width = 2em, draw, circle},
  ]
  \node {50}
  child { node {30}
          child { node {20} }
          child { edge from parent[draw = none] }
        }
  child { node {80}
        };
\end{tikzpicture}

\vspace*{1.5cm}

  \end{minipage}
  \hfillx
  \begin{minipage}{0.33\textwidth}
    \centering

\textit{\'Etape 5}

R = InsertLeaf(R, 60)

\medskip

\vspace*{0.5cm}

\begin{tikzpicture}[
  level 1/.style = {sibling distance = 30mm},
  level 2/.style = {sibling distance = 15mm},
  every node/.style = {minimum width = 2em, draw, circle},
  ]
  \node {50}
  child { node {30}
          child { node {20} }
          child { edge from parent[draw = none] }
        }
  child { node {80}
          child { node {60} }
          child { edge from parent[draw = none] }
        };
\end{tikzpicture}

\vspace*{1.5cm}

  \end{minipage}
  \hfillx
  \begin{minipage}{0.33\textwidth}
    \centering

\textit{\'Etape 6}

R = InsertLeaf(R, 40)

\vspace*{0.5cm}

\begin{tikzpicture}[
  level 1/.style = {sibling distance = 30mm},
  level 2/.style = {sibling distance = 15mm},
  every node/.style = {minimum width = 2em, draw, circle},
  ]
  \node {50}
  child { node {30}
          child { node {20} }
          child { node {40} }
        }
  child { node {80}
          child { node {60} }
          child { edge from parent[draw = none] }
        };
\end{tikzpicture}

\vspace*{1.5cm}

  \end{minipage}
\end{table}

%%%%%%%%%%%%%%%%
\vspace*{-0.5cm}
\rule{1.0\linewidth}{0.75pt}
%%%%%%%%%%%%%%%%

% 7-8 : F25, F70

\begin{table}[ht!]
  \centering
  \begin{minipage}{0.50\textwidth}
    \centering

\textit{\'Etape 7}

R = InsertLeaf(R, 25)

\vspace*{0.5cm}

\begin{tikzpicture}[
  level 1/.style = {sibling distance = 30mm},
  level 2/.style = {sibling distance = 15mm},
  level 3/.style = {sibling distance = 9mm},
  every node/.style = {minimum width = 2em, draw, circle},
  ]
  \node {50}
  child { node {30}
          child { node {20}
                  child { edge from parent[draw = none] }
                  child { node {25} }
                }
          child { node {40} }
        }
  child { node {80}
          child { node {60} }
          child { edge from parent[draw = none] }
        };
\end{tikzpicture}

\vspace*{1.5cm}

  \end{minipage}
  \hfillx
  \begin{minipage}{0.50\textwidth}
    \centering

\textit{\'Etape 8}

R = InsertLeaf(R, 70)

\vspace*{0.5cm}

\begin{tikzpicture}[
  level 1/.style = {sibling distance = 30mm},
  level 2/.style = {sibling distance = 15mm},
  level 3/.style = {sibling distance = 9mm},
  every node/.style = {minimum width = 2em, draw, circle},
  ]
  \node {50}
  child { node {30}
          child { node {20}
                  child { edge from parent[draw = none] }
                  child { node {25} }
                }
          child { node {40} }
        }
  child { node {80}
          child { node {60}
                  child { edge from parent[draw = none] }
                  child { node {70} }
                }
          child { edge from parent[draw = none] }
        };
\end{tikzpicture}

\vspace*{1.5cm}

  \end{minipage}
\end{table}


%%%%%%%%%%%%%%%%
\vspace*{-0.5cm}
\rule{1.0\linewidth}{0.75pt}
%%%%%%%%%%%%%%%%

% 9-10 : -30, -80

\begin{table}[ht!]
  \centering
  \begin{minipage}{0.50\textwidth}
    \centering

\textit{\'Etape 9}

R = RemoveFromBST(R, 30)

\vspace*{0.5cm}

\begin{tikzpicture}[
  level 1/.style = {sibling distance = 30mm},
  level 2/.style = {sibling distance = 15mm},
  level 3/.style = {sibling distance = 9mm},
  every node/.style = {minimum width = 2em, draw, circle},
  ]
  \node {50}
  child { node {25}
          child { node {20} }
          child { node {40} }
        }
  child { node {80}
          child { node {60}
                  child { edge from parent[draw = none] }
                  child { node {70} }
                }
          child { edge from parent[draw = none] }
        };
\end{tikzpicture}

\vspace*{0.5cm}

  \end{minipage}
  \hfillx
  \begin{minipage}{0.50\textwidth}
    \centering

\textit{\'Etape 10}

R = RemoveFromBST(R, 80)

\vspace*{0.5cm}

\begin{tikzpicture}[
  level 1/.style = {sibling distance = 30mm},
  level 2/.style = {sibling distance = 15mm},
  level 3/.style = {sibling distance = 9mm},
  every node/.style = {minimum width = 2em, draw, circle},
  ]
  \node {50}
  child { node {25}
          child { node {20} }
          child { node {40} }
        }
  child { node {60}
          child { edge from parent[draw = none] }
          child { node {70} }
        };
\end{tikzpicture}

\vspace*{0.5cm}

  \end{minipage}
\end{table}

%%%%%%%%%%
\clearpage
%%%%%%%%%%

% 11-12 : R30, R35

\begin{table}[ht!]
  \centering
  \begin{minipage}{0.50\textwidth}
    \centering

\textit{\'Etape 11}

R = \textit{InsertRoot}(R, 30)

\vspace*{0.5cm}

\begin{tikzpicture}[
  level 1/.style = {sibling distance = 30mm},
  level 2/.style = {sibling distance = 15mm},
  level 3/.style = {sibling distance = 9mm},
  every node/.style = {minimum width = 2em, draw, circle},
  ]
  \node {30}
  child { node {25}
          child { node {20} }
          child { edge from parent[draw = none] }
        }
  child { node {50}
          child { node {40} }
          child { node {60}
                  child { edge from parent[draw = none] }
                  child { node {70} }
                }
        };
\end{tikzpicture}

\vspace*{0.5cm}

  \end{minipage}
  \hfillx
  \begin{minipage}{0.50\textwidth}
    \centering

\textit{\'Etape 12}

R = \textit{InsertRoot}(R, 35)

\vspace*{0.5cm}

\begin{tikzpicture}[
  level 1/.style = {sibling distance = 30mm},
  level 2/.style = {sibling distance = 15mm},
  level 3/.style = {sibling distance = 9mm},
  every node/.style = {minimum width = 2em, draw, circle},
  ]
  \node {35}
  child { node {30}
          child { node {25}
                  child { node {20} }
                  child { edge from parent[draw = none] }
                }
          child { edge from parent[draw = none] }
        }
  child { node {50}
          child { node {40} }
          child { node {60}
                  child { edge from parent[draw = none] }
                  child { node {70} }
                }
        };
\end{tikzpicture}

\vspace*{0.5cm}

  \end{minipage}
\end{table}

%%%%%%%%%%%%%%%%
\vspace*{-0.5cm}
\rule{1.0\linewidth}{0.75pt}
%%%%%%%%%%%%%%%%

%Indiquez combien de rotations droite gauche et gauche droite ont été effectuées dans l'insertion en AVL des précédents éléments (1 point) :
%
%\begin{table}[ht!]
%  \centering
%  \begin{minipage}{0.50\textwidth}
%
%RDG :
%
%  \end{minipage}
%  \hfillx
%  \begin{minipage}{0.50\textwidth}
%
%RGD :
%
%  \end{minipage}
%\end{table}

\end{center}

%%%%%%%%%%%%%%%%%%%%%%%%%%%%%%%%%%%%%%%%%%%%%%%%%%%%%%%%

\subsection*{Algorithmes}

%\subsection{(2 points) \'Ecrivez une fonction \textit{RechercheABR} recherchant récursivement un entier dans un arbre binaire de recherche et renvoyant l'adresse de son nœud. Si l'élément n'est pas trouvé, la fonction doit renvoyer \TTBF{NULL} : }
%
%\begin{center}
%\GrilleReponseN{8}
%\end{center}
%
%%%%%%%%%%%%%%%%%%%%%%%%%%%%%%%%%%%%%%%

\subsection{(2 points) \'Ecrivez une fonction \textit{AjoutFeuille} ajoutant récursivement en feuille un élément dans un arbre binaire de recherche : }

\begin{center}
%\GrilleReponseN{13}
\GrilleReponseN{14.5}
\end{center}

\clearpage


%%%%%%%%%%%%%%%%%%%%%%%%%%%%%%%%%%%%%%%%%%%%%%%%%%%%%%%%%%%%%%%%%
%%%%%%%%%%%%%%%%%%%%%%%%%%%%%%%%%%%%%%%%%%%%%%%%%%%%%%%%%%%%%%%%%
%%%%%%%%%%%%%%%%%%%%%%%%%%%%%%%%%%%%%%%%%%%%%%%%%%%%%%%%%%%%%%%%%

\clearpage

%%%%%%%%%%%%%%%%%%%%%%%%%%%%%%%%%%%%%%%%%%%%%%%%%%%%%%%%%%%%%%%%%
%%%%%%%%%%%%%%%%%%%%%%%%%%%%%%%%%%%%%%%%%%%%%%%%%%%%%%%%%%%%%%%%%
%%%%%%%%%%%%%%%%%%%%%%%%%%%%%%%%%%%%%%%%%%%%%%%%%%%%%%%%%%%%%%%%%

\section{Problème (6 points)}


%\noindent Afin de tester l'ensemble des compétences acquises au cours de cette année, vous allez maintenant toutes les exploiter pour interpréter des données et des structures.
%Le but de ces exercices est de vous faire changer de point de vue : vous avez construit des structures durant l'année, vous allez maintenant analyser des structures existantes.

\noindent En mathématiques, vous avez appris tout au long de vos études primaires et secondaires à utiliser plusieurs opérateurs communs : \og $ + - \times \div $ \fg{}.
Ces opérateurs sont tous des opérateurs \textit{binaires}, c'est-à-dire qu'ils prennent deux paramètres : \og $ 3 + 5 $ \fg{} peut également s'écrire \textit{addition(3, 5)}.

\noindent \og $ + $ \fg{} et \og $ - $ \fg{} sont également des opérateurs \textit{unaires} lorsqu'ils représentent le signe de la valeur absolue qui suit : \og $ - 8 $ \fg{} peut aussi s'écrire \textit{négatif(8)}.

\medskip

\noindent Dans cet exercice, vous allez transformer des arbres en formules mathématiques, puis l'inverse, et enfin, écrire l'algorithme de parcours et d'exécution de ces opérations.

%%%%%%%%%%%%%%%%%%%%%%%%%%%%%%%%%%%%%%%%%%%%%%%

\subsection{(2 points) Transformez les arbres suivants en leurs formules mathématiques associées : }

\begin{table}[ht!]
  \centering
  \begin{minipage}{0.30\textwidth}
    \centering

\begin{center}
\begin{tikzpicture}[sibling distance=1.5cm,
  level/.style = {sibling distance = 26mm/#1},
  every node/.style = {minimum width = 2em, draw, circle}
  ]
  \node (nO1) {$\times$}
  child { node (nV1) {3} }
  child { node (nV2) {5} };
\end{tikzpicture}
%  child { node [draw=none] (nX) {\phantom{X}} edge from parent [draw=none] }
\end{center}

  \end{minipage}
  \hfillx
  \begin{minipage}{0.30\textwidth}
    \centering

\begin{center}
\begin{tikzpicture}[sibling distance=1.5cm,
  level/.style = {sibling distance = 26mm/#1},
  every node/.style = {minimum width = 2em, draw, circle}
  ]
  \node (nO1) {$\div$}
  child { node (nV1) {8} }
  child {
          node (nO2) {$-$}
          child { node (nV2) {4} }
          child { node [draw=none] (nX) {\phantom{X}} edge from parent [draw=none] }
        };
\end{tikzpicture}
%  child { node [draw=none] (nX) {\phantom{X}} edge from parent [draw=none] }
\end{center}

  \end{minipage}
  \hfillx
  \begin{minipage}{0.40\textwidth}
    \centering

\begin{center}
\begin{tikzpicture}[sibling distance=1.5cm,
  level/.style = {sibling distance = 40mm/#1},
  level 1/.style = {sibling distance = 40mm},
  level 2/.style = {sibling distance = 20mm},
  level 3/.style = {sibling distance = 15mm},
  every node/.style = {minimum width = 2em, draw, circle},
  ]
  \node (nO1) {+}
  child { node (nO2) {$\times$}
          child { node (nV1) {4} }
          child { node (nV2) {2} }
         }
   child { node (nO3) {$-$}
           child { node (nV3) {6} }
           child { node (nV4) {8} }
        };
\end{tikzpicture}
%  child { node [draw=none] (nX) {\phantom{X}} edge from parent [draw=none] }
\end{center}

  \end{minipage}
\end{table}


\begin{table}[ht!]
  \centering
  \begin{minipage}{0.30\textwidth}
    \centering

\textit{(0,5 pt)} Formule :

\medskip

$ 3 \times 5 $

  \end{minipage}
  \hfillx
  \begin{minipage}{0.30\textwidth}
    \centering

\textit{(0,5 pt)} Formule :

\medskip

$ 8 \div (-4) $

  \end{minipage}
  \hfillx
  \begin{minipage}{0.40\textwidth}
    \centering

\textit{(0.5 pt)} Formule :

\medskip

$ (4 \times 2) + (6 - 8) $

  \end{minipage}
\end{table}

%\vspace*{1cm}

\noindent \textit{(0.5 point)} Que remarquez-vous concernant les opérateurs et les nombres par rapport à leur placement dans les arbres ?

\medskip

Les nœuds internes de l'arbre ne contiennent que des opérateurs, et les feuilles/nœuds externes ne contiennent que des nombres.

%\bigskip

%%%%%%%%%%%%%%%%%%%%%%%%%%%%%%%%%%%%%%%%%%%%%%%

\subsection{(2 points) Transformez chacune des formules suivantes en un arbre binaire : }

\vspace*{-0.5cm}

\begin{table}[ht!]
  \centering
  \begin{minipage}{0.25\textwidth}
    \centering

\textit{(0,5 point)}

$ 3 \times (4 - 5) $

  \end{minipage}
  \hfillx
  \begin{minipage}{0.25\textwidth}
    \centering

\textit{(0,5 point)}

$ (-5) \div (8 + 3) $

  \end{minipage}
  \hfillx
  \begin{minipage}{0.50\textwidth}
    \centering

\textit{(1 point)}

$ (6 \times (-4)) + ((-8) \div (5 - 3)) $

  \end{minipage}
\end{table}

\begin{table}[ht!]
  \centering
  \begin{minipage}{0.25\textwidth}
    \centering

\begin{center}
\begin{tikzpicture}[sibling distance=1.5cm,
  level/.style = {sibling distance = 26mm/#1},
  every node/.style = {minimum width = 2em, draw, circle}
  ]
  \node (nO1) {$\times$}
  child { node (nV1) {3} }
  child {
          node (nO2) {$-$}
          child { node (nV2) {4} }
          child { node (nV3) {5} }
        };
\end{tikzpicture}
%  child { node [draw=none] (nX) {\phantom{X}} edge from parent [draw=none] }
\end{center}

  \end{minipage}
  \hfillx
  \begin{minipage}{0.25\textwidth}
    \centering

\begin{center}
\begin{tikzpicture}[sibling distance=1.5cm,
  level/.style = {sibling distance = 26mm/#1},
  every node/.style = {minimum width = 2em, draw, circle}
  ]
  \node (nO1) {$\div$}
  child {
          node (nO2) {$-$}
          child { node (nV1) {5} }
        }
  child {
          node (nO2) {$+$}
          child { node (nV2) {8} }
          child { node (nV3) {3} }
        };
\end{tikzpicture}
%  child { node [draw=none] (nX) {\phantom{X}} edge from parent [draw=none] }
\end{center}

  \end{minipage}
  \hfillx
  \begin{minipage}{0.50\textwidth}
    \centering

\begin{center}
\begin{tikzpicture}[sibling distance=1.5cm,
  level/.style = {sibling distance = 40mm/#1},
  level 1/.style = {sibling distance = 40mm},
  level 2/.style = {sibling distance = 20mm},
  level 3/.style = {sibling distance = 15mm},
  every node/.style = {minimum width = 2em, draw, circle},
  ]
  \node (nO1) {+}
  child { node (nO2) {$\times$}
          child { node (nV1) {6} }
          child {
                  node (nO3) {$-$}
                  child { node (nV2) {4} }
                  child { node [draw=none] (nX) {\phantom{X}} edge from parent [draw=none] }
                }
         }
   child { node (nO4) {$\div$}
           child {
                   node (nO5) {$-$}
                   child { node (nV3) {8} }
                   child { node [draw=none] (nX) {\phantom{X}} edge from parent [draw=none] }
                 }
           child {
                   node (nO6) {$-$}
                   child { node (nV4) {5} }
                   child { node (nV5) {3} }
                 }
        };
\end{tikzpicture}
%  child { node [draw=none] (nX) {\phantom{X}} edge from parent [draw=none] }
\end{center}

  \end{minipage}
\end{table}

\vspace*{3cm}

%\bigskip

\clearpage

%%%%%%%%%%%%%%%%%%%%%%%%%%%%%%%%%%%%%%%%%%%%%%%

\subsection{(2 points) \'Ecrivez une fonction \og \textit{exec\_maths\_tree} \fg{} exécutant l'expression représentée par l'arbre binaire donné en paramètre, et renvoyant le résultat : }

\noindent Chaque nœud de cet arbre est une structure contenant du texte en tant que valeur, vous pouvez utiliser les fonctions suivantes pour tester ou extraire le contenu de la valeur.
Les nœuds ne contiendront jamais autre chose que l'un des 4 opérateurs ou un nombre, et l'arbre ne sera jamais vide.

\vspace*{-0.5cm}

\begin{center}

\begin{table}[ht!]
  \centering
  \begin{minipage}{0.38\textwidth}
    \centering

\begin{lstlisting}[language=C]
typedef struct math_node
{
  char *value;
  struct math_node lc;
  struct math_node rc;
} math_node; \end{lstlisting}

  \end{minipage}
  \hfillx
  \begin{minipage}{0.57\textwidth}

\noindent \TTBF{int UnaryOp(char *op, int val)} : fonction exécutant l'opérateur fournit en paramètre sur la valeur.

\smallskip

\noindent \TTBF{int BinaryOp(char *op, int v1, int v2)} : fonction exécutant l'opérateur fournit en paramètre sur les 2 valeurs.

\smallskip

\noindent \TTBF{int atoi(char *text)} : fonction transformant le texte donné en paramètre en un entier.

  \end{minipage}
\end{table}

\end{center}

\vspace*{-1.5cm}

\begin{center}
%\GrilleReponseN{22}
\GrilleReponseTextUp{18}{0}{\TTBF{\textcolor{blue}{int} exec\_maths\_tree(\textcolor{blue}{math\_node} *root)}}
\end{center}


%%%%%%%%%%%%%%%%%%%%%%%%%%%%%%%%%%%%%%%%%%%%%%%%%%%%%%%%
%%%%%%%%%%%%%%%%%%%%%%%%%%%%%%%%%%%%%%%%%%%%%%%%%%%%%%%%
%%%%%%%%%%%%%%%%%%%%%%%%%%%%%%%%%%%%%%%%%%%%%%%%%%%%%%%%

%\clearpage
%
%%\thispagestyle{empty}
%
%\vfillFirst
%
%\begin{center}
%
%\begin{LARGE}
%\textbf{SUJET RATTRAPAGE}
%
%\bigskip
%
%\textbf{\MakeUppercase{\TitreMatiere}}
%\end{LARGE}
%
%\end{center}
%
%\vfillLast

\end{document}
