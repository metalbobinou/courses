\documentclass[11pt,a4paper]{article}
\usepackage[utf8]{inputenc}
\usepackage[french]{babel}
\usepackage[T1]{fontenc}

\usepackage{amsmath}
\usepackage{amsfonts}
\usepackage{amssymb}

\newcommand{\TitreMatiere}{Algorithmique 1}
\newcommand{\TitreSeance}{[PROF] Listes, Piles, Files}
\newcommand{\SousTitreSeance}{Fiche 5}
\newcommand{\DateCours}{Décembre 2024}
\newcommand{\AnneeScolaire}{2024-2025}
\newcommand{\Organisation}{EPITA}
\newcommand{\NomAuteurA}{Fabrice BOISSIER}
\newcommand{\MailAuteurA}{fabrice.boissier@epita.fr}
\newcommand{\NomAuteurB}{ }
\newcommand{\MailAuteurB}{ }
\newcommand{\DocKeywords}{Algorithmique ; Algorithmes ; Liste ; List ; Pile ; Stack ; File ; Queue ; API}
\newcommand{\DocLangue}{fr} % "en", "fr", ...

\usepackage{MetalCourseBooklet}

% Babel ne traduit pas toujours bien les tableaux et autres
\renewcommand*\frenchfigurename{%
    {\scshape Figure}%
}
\renewcommand*\frenchtablename{%
    {\scshape Tableau}%
}

% Ne pas afficher le numéro de la légende sur tableaux et figures
\captionsetup{format=sanslabel}


\begin{document}

\EncadreTitre

\bigskip


%\begin{center}
%\begin{tabular}{p{5cm} p{11cm}}
%\textbf{Commandes étudiées :} & \texttt{sh}, \texttt{bash}, \texttt{man}, \texttt{ls}, \texttt{mkdir}, \texttt{touch}, \texttt{chmod}, \texttt{mv}, \texttt{rm}, \texttt{rmdir}, \texttt{cat}, \texttt{file}, \texttt{which}, \texttt{which}\\
%
%\textbf{Builtins étudiées :} & \texttt{pwd}, \texttt{cd}, \texttt{exit}, \texttt{logout}, \texttt{echo}, \texttt{umask}, \texttt{type}, \texttt{>}, \texttt{>{}>}, \texttt{<}, \texttt{<{}<}, \texttt{|}\\
%
%\textbf{Notions étudiées :} & Shell, Manuels, Fichiers, Répertoires, Droits, Redirections\\
%\end{tabular}
%\end{center}

\bigskip


Ce document a pour objectif de guider les enseignants pour le cours d'algorithmique.
Il est déconseillé de le fournir aux étudiants, car il vise surtout à vous permettre de guider la séance.
Vous n'êtes évidemment pas obligé de le suivre à la lettre (c'est même déconseillé, car cela va autant vous perturber vous que la classe : suivez votre chemin de pensée et/ou celui de la classe, et ensuite vérifiez que vous n'avez rien oublié).

\medskip

La cinquième partie vise à découvrir les structures de données fondamentales (listes, piles, files) et les notions d'encapsulation ainsi que d'API qui en découlent.
L'implémentation proposée dans le cours pour chaque structure n'est pas la plus classique, mais elle a le mérite d'être logique et complète.


\bigskip

%%%%%%%%%%%%%%%%%%%%%%%%%%%%%%%%%%%%%%

\section{Introduction}

Présentez cette partie en détails \textit{si et seulement si} il n'y a pas eu de cours de C présentant les structures de données.
Sinon, faites simplement une révision de la syntaxe et des mots clés.

\medskip

Révisions :

\begin{itemize}
\item Présentez le mot clé \TTBF{struct}
\item Présentez l'accès aux \textit{champs} d'une structure allouée localement comme une variable
\item Présentez l'accès aux \textit{champs} d'une structure allouée en mémoire avec \textit{malloc(3)}
\end{itemize}


%%%%%%%%%%%%%%%%%%%%%%%%%%%%%%%%%%%%%

\bigskip

\vfillFirst

\vfillLast

\begin{center}
\textit{Ce document et ses illustrations ont été réalisés par Fabrice BOISSIER en décembre 2024}

\textit{(dernière mise à jour décembre 2024)}
\end{center}

\end{document}