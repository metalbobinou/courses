\documentclass[11pt,a4paper]{article}
\usepackage[utf8]{inputenc}
\usepackage[french]{babel}
\usepackage[T1]{fontenc}

\usepackage{amsmath}
\usepackage{amsfonts}
\usepackage{amssymb}

\newcommand{\TitreMatiere}{Algorithmique 1}
\newcommand{\TitreSeance}{Algorithmique 1}
\newcommand{\NumeroTD}{Syllabus}
\newcommand{\DateCours}{Septembre 2023}
\newcommand{\AnneeScolaire}{2023-2024}
\newcommand{\Organisation}{EPITA}
\newcommand{\NomAuteurA}{Fabrice BOISSIER}
\newcommand{\MailAuteurA}{fabrice.boissier@epita.fr}
\newcommand{\NomAuteurB}{ }
\newcommand{\MailAuteurB}{ }
\newcommand{\DocKeywords}{Algorithmique}
\newcommand{\DocLangue}{fr} % "en", "fr", ...

\usepackage{MetalCourseBooklet}

% Babel ne traduit pas toujours bien les tableaux et autres
\renewcommand*\frenchfigurename{%
    {\scshape Figure}%
}
\renewcommand*\frenchtablename{%
    {\scshape Tableau}%
}

% Ne pas afficher le numéro de la légende sur tableaux et figures
\captionsetup{format=sanslabel}


\begin{document}

\EncadreTitre

\bigskip


%\begin{center}
%\begin{tabular}{p{5cm} p{11cm}}
%\textbf{Commandes étudiées :} & \texttt{sh}, \texttt{bash}, \texttt{man}, \texttt{ls}, \texttt{mkdir}, \texttt{touch}, \texttt{chmod}, \texttt{mv}, \texttt{rm}, \texttt{rmdir}, \texttt{cat}, \texttt{file}, \texttt{which}, \texttt{which}\\
%
%\textbf{Builtins étudiées :} & \texttt{pwd}, \texttt{cd}, \texttt{exit}, \texttt{logout}, \texttt{echo}, \texttt{umask}, \texttt{type}, \texttt{>}, \texttt{>{}>}, \texttt{<}, \texttt{<{}<}, \texttt{|}\\
%
%\textbf{Notions étudiées :} & Shell, Manuels, Fichiers, Répertoires, Droits, Redirections\\
%\end{tabular}
%\end{center}

\bigskip


Ce document a pour objectif de lister les notions à voir / vues dans le cours.

\bigskip


\begin{enumerate}
\item Introduction à l'algorithmique \textit{[22/09/2023]}\\
  CM :
  \begin{itemize}
  \item \textit{(arrivée au campus cyber)}
  \item Qu'est-ce qu'un algorithme, paramètres et valeur retournée
  \item Types de données (entiers, flottants, carcatère, chaîne de caractères, booléens, ...)
  \item While, For, Foreach
  \item Fonctions et Procédures
  \item Paramètres locaux/globaux (Portée des variables)
  \item Effets de bord
  \end{itemize}
  TD :
  \begin{itemize}
  \item Exécution d'algorithmes simples
  \item Traces des variables
  \item Différence $ > $ et $ >= $
  \end{itemize}

\medskip

\item Récursivité \textit{[29/09/2023]}\\
  CM :
  \begin{itemize}
  \item Modulo
  \item Logique (conditions ET OU)
  \item (exemple : test de parité)
  \item Récursivité (cas d'arrêt \& cas récursif)
  \item Pile d'appels
  \item Récursivité Terminale (accumulateur \& fonction chapeau)
  \item (exemple : somme des N premiers entiers)
  \end{itemize}
  TD :
  \begin{itemize}
  \item \'Ecriture d'algorithmes
  \item min/max entre 3 paramètres, somme des carrés des 2 plus grands paramètres
  \item \'Ecriture d'algorithmes récursifs \& récursifs terminaux
  \item factorielle, suite géométrique, Fibonacci, Ackermann, nombre miroir (print \& return)
  \item (Procédure récursive == récursive terminale ?)
  \item Comparaison versions itératives et récursives
  \end{itemize}

\medskip

\clearpage

\item Tableaux \textit{[06/10/2023]}\\
  CM :
  \begin{itemize}
  \item Rappels : While, For, Foreach
  \item Continue \& Break
  \item Tableaux
  \item Chaînes de caractères
  \item Taille max d'un tableau et d'une chaîne
  \end{itemize}
  TD :
  \begin{itemize}
  \item \'Ecriture d'algorithmes
  \item Recherche dans un tableau (iter \& rec)
  \item Min/Max tableau (iter \& rec), Somme des éléments d'un tableau (iter \& rec), strlen (iter \& rec)
  \item Comparaison de tableaux (iter \& rec), strcmp (iter \& rec)
  \item Inversion d'un tableau
  \item Test tableau croissant/décroissant (iter \& rec)
  \item Préfixe/Suffixe chaîne (iter \& rec)
  \item {} [BONUS] Substr
  \end{itemize}

\medskip

\item Tris \textit{[13/10/2023]}\\
  CM :
  \begin{itemize}
  \item Tri à bulles
  \item Tri par sélection
  \item (Preprocessing du Tri par sélection)
  \end{itemize}
  TD :
  \begin{itemize}
  \item \'Ecriture d'algorithmes
  \item Swap(p1, p2, tab)
  \item Tri à bulles
  \item Tri par sélection
  \item (Preprocessing du Tri par sélection)
  \end{itemize}

\medskip

\item Structures de Données et Listes \textit{[20/10/2023]}\\
  CM :
  \begin{itemize}
  \item Structures de données (champs, types, ...)
  \item Listes (tableaux)
  \item API
  \item Ajout Elt, Suppression Elt, EstVide, EstPleine, Acces, Creer, Vider, Supprimer
  \item Statique vs Dynamique
  \end{itemize}
  TD :
  \begin{itemize}
  \item \'Ecriture d'algorithmes
  \item Ajout Elt en fin, Suppression Elt en fin
  \item Acces, Creer, Vider, Supprimer
  \item Print, Taille, Taille Max, EstVide, EstPleine
  \item Ajout Elt en tête/milieu, Suppression Elt en tête/milieu
  \end{itemize}

\medskip

\clearpage

\item Pile \textit{[27/10/2023]}\\
  CM :
  \begin{itemize}
  \item Piles (tableaux)
  \item Usages LIFO (parcours, tâches à gérer, paramètres, ...)
  \item Pile par dessus l'API Listes
  \item Ajout Elt, Suppression Elt, EstVide, EstPleine, Sommet, Creer, Vider, Supprimer
  \end{itemize}
  TD :
  \begin{itemize}
  \item Exercice d'utilisation
  \item Empiler "1 2 3 4 5 6" dans cet ordre avec des pushs, et écrire le scénario pour sortir avec pop dans ces ordres si cela est possible :\\
  \item 6 5 4 3 2 1
  \item 1 2 3 4 5 6
  \item 2 3 1 6 5 4
  \item 1 4 3 2 6 5
  \item 4 6 5 1 2 3  % (impossible)
  \item 2 4 3 1 6 5
  \item 3 2 5 4 6 1
  \item 1 3 5 4 2 6
  \item 5 6 1 2 4 3  % (impossible)
  \item \'Ecriture d'algorithmes
  \item Ajout Elt, Suppression Elt, EstVide, EstPleine, Sommet, Creer, Vider, Supprimer
  \end{itemize}

\medskip

\item File \textit{[10/11/2023]}\\
  CM :
  \begin{itemize}
  \item File (tableaux)
  \item Usages FIFO (parcours, traitement clients, ...)
  \item File par dessus l'API Listes
  \item Ajout Elt, Suppression Elt, EstVide, EstPleine, Tete, Creer, Vider, Supprimer
  \end{itemize}
  TD :
  \begin{itemize}
  \item \'Ecriture d'algorithmes
  \item Ajout Elt, Suppression Elt, EstVide, EstPleine, Tete, Creer, Vider, Supprimer
  \end{itemize}

\medskip

\clearpage

\item Pointeurs \textit{[17/11/2023]}\\
  CM :
  \begin{itemize}
  \item Mémoire, tableaux, accès
  \item Pointeurs
  \item Chaînes de caractères en C
  \item Variables "globales" (avec pointeur/référence)
  \end{itemize}
  TD :
  \begin{itemize}
  \item Exercices de prise en main
  \item Accès à des entiers, pointeurs d'entiers, char, char*, ...
  \item \'Ecriture d'algorithmes :
  \item Adresse du premier caractère trouvé
  \item Recherche (avec reprise) jusqu'à la dernière occurrence trouvée dans un texte
  \item MyStrTok (à la sauce Fabrice : transformer une chaîne en un tableau de chaînes NULL-terminated)
  \end{itemize}

\medskip

\item Listes Chaînées \textit{[24/11/2023]}\\
  CM :
  \begin{itemize}
  \item Listes chaînées \& Maillons
  \item Déclaration structures
  \end{itemize}
  TD :
  \begin{itemize}
  \item \'Ecriture d'algorithmes :
  \item Ajout Elt en fin, Suppression Elt en fin
  \item Acces, Creer, Vider, Supprimer
  \item Print, Taille, Taille Max, EstVide, EstPleine
  \item Ajout Elt en tête/milieu, Suppression Elt en tête/milieu
  \end{itemize}

\medskip

\item Piles (pointeurs) \textit{[01/12/2023]}\\
  CM :
  \begin{itemize}
  \item Pile \& Maillons
  \end{itemize}
  TD :
  \begin{itemize}
  \item \'Ecriture d'algorithmes
  \item Ajout Elt, Suppression Elt, EstVide, EstPleine, Sommet, Creer, Vider, Supprimer
  \end{itemize}

\medskip

\item Files (pointeurs) \textit{[08/12/2023]}\\
  CM :
  \begin{itemize}
  \item File \& Maillons
  \end{itemize}
  TD :
  \begin{itemize}
  \item \'Ecriture d'algorithmes
  \item Ajout Elt, Suppression Elt, EstVide, EstPleine, Tete, Creer, Vider, Supprimer
  \end{itemize}

\medskip

\item ??? \textit{[15/12/2023]}\\
  CM :
  \begin{itemize}
  \item ???
  \item (liste doublement chaînée ?)
  \end{itemize}
  TD :
  \begin{itemize}
  \item ???
  \end{itemize}

\medskip

\item ??? \textit{[22/12/2023]}\\
  CM :
  \begin{itemize}
  \item ???
  \end{itemize}
  TD :
  \begin{itemize}
  \item ???
  \end{itemize}

\end{enumerate}

\bigskip

Mode d'évaluation :

\begin{itemize}
\item 2 examens papiers (80\% note finale)
\item 2 QCM (20\% note finale)
\end{itemize}




\bigskip

\vfillFirst

\vfillLast


\begin{center}
\textit{Ce document et ses illustrations ont été réalisés par Fabrice BOISSIER en septembre 2023}
\end{center}

\end{document}
