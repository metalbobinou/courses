\documentclass[11pt,a4paper]{article}
\usepackage[utf8]{inputenc}
\usepackage[french]{babel}
\usepackage[T1]{fontenc}

\usepackage{amsmath}
\usepackage{amsfonts}
\usepackage{amssymb}

\newcommand{\TitreMatiere}{Algorithmique 1}
\newcommand{\TitreSeance}{Récursivité}
\newcommand{\NumeroTD}{Récursivité \& Récursivité Terminale}
\newcommand{\DateCours}{Septembre 2023}
\newcommand{\AnneeScolaire}{2023-2024}
\newcommand{\Organisation}{EPITA}
\newcommand{\NomAuteurA}{Fabrice BOISSIER}
\newcommand{\MailAuteurA}{fabrice.boissier@epita.fr}
\newcommand{\NomAuteurB}{ }
\newcommand{\MailAuteurB}{ }
\newcommand{\DocKeywords}{Algorithmique}
\newcommand{\DocLangue}{fr} % "en", "fr", ...

\usepackage{MetalCourseBooklet}

% Babel ne traduit pas toujours bien les tableaux et autres
\renewcommand*\frenchfigurename{%
    {\scshape Figure}%
}
\renewcommand*\frenchtablename{%
    {\scshape Tableau}%
}

% Ne pas afficher le numéro de la légende sur tableaux et figures
\captionsetup{format=sanslabel}


\begin{document}

\EncadreTitre

\bigskip


%\begin{center}
%\begin{tabular}{p{5cm} p{11cm}}
%\textbf{Commandes étudiées :} & \texttt{sh}, \texttt{bash}, \texttt{man}, \texttt{ls}, \texttt{mkdir}, \texttt{touch}, \texttt{chmod}, \texttt{mv}, \texttt{rm}, \texttt{rmdir}, \texttt{cat}, \texttt{file}, \texttt{which}, \texttt{which}\\
%
%\textbf{Builtins étudiées :} & \texttt{pwd}, \texttt{cd}, \texttt{exit}, \texttt{logout}, \texttt{echo}, \texttt{umask}, \texttt{type}, \texttt{>}, \texttt{>{}>}, \texttt{<}, \texttt{<{}<}, \texttt{|}\\
%
%\textbf{Notions étudiées :} & Shell, Manuels, Fichiers, Répertoires, Droits, Redirections\\
%\end{tabular}
%\end{center}

\bigskip


Ce document a pour objectif de vous familiariser avec la récursivité et les notions associées telles que la récursivité terminale, les accumulateurs, et les fonctions chapeaux.

\bigskip

Définition informelle de la récursivité~\footnote{Wikipedia : \href{https://fr.wikipedia.org/wiki/R\%C3\%A9cursivit\%C3\%A9}{Récursivité}} : \og \textit{La récursivité est une démarche qui fait référence à l'objet même de la démarche à un moment du processus} \fg{}, ou encore \og \textit{C'est une démarche dont la description mène à la répétition d'une même règle} \fg{} (Edgar Morin, \textit{La méthode : L'éthique} 2004).

\medskip

En pratique cela implique qu'une fonction se rappelle elle-même, ou qu'une structure contient une référence vers elle-même.
Parmi les termes synonymes, on retrouve \textit{récursion} et d'une certaine manière la \textit{récurrence}.

\medskip

Parmi les exemples les plus classiques de fonctions récursives, on retrouve \textit{factorielle}, \textit{la somme des N premiers entiers}, \textit{le N\up{ième} terme de la suite de fibonacci}, ...
Pour rappel, la factorielle se calcule à partir des entiers précédents :
%
\begin{center}
\[ facto(n) = n! = \prod_{i=1}^{n} i = 1 \times 2 \times 3 \times \text{...} \times n \]
\end{center}


\begin{center}
\begin{lstlisting}[style=algorithmique]
algorithme fonction Factorielle : entier
  parametres locaux
    entier   n

debut
si (n == 0) ou (n == 1) alors
  retourne (1)
sinon
  retourne (n * Factorielle(n - 1))
fin si
fin algorithme fonction Factorielle \end{lstlisting}
\end{center}


Pour déclarer une fonction récursive, il est nécessaire de tout d'abord définir le ou les cas d'arrêt, et ensuite seulement les cas récursifs.

Dans le code de la factorielle, on remarque les deux cas d'arrêt : $ n == 0 $ et $ n == 1 $.




\bigskip

\vfillFirst

\vfillLast


\begin{center}
\textit{Ce document et ses illustrations ont été réalisés par Fabrice BOISSIER en septembre 2023}
\end{center}

\end{document}
