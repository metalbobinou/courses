\documentclass[11pt,a4paper]{article}
\usepackage[utf8]{inputenc}
\usepackage[french]{babel}
\usepackage[T1]{fontenc}

\usepackage{amsmath}
\usepackage{amsfonts}
\usepackage{amssymb}

\newcommand{\NomAuteur}{Fabrice BOISSIER}
\newcommand{\TitreMatiere}{Algorithmique 1}
\newcommand{\NomUniv}{EPITA - Bachelor Cyber Sécurité}
\newcommand{\NiveauUniv}{CYBER1}
\newcommand{\NumGroupe}{CYBER1}
\newcommand{\AnneeUniv}{2023-2024}
\newcommand{\DateExam}{juillet 2024}
%\newcommand{\TypeExam}{Rattrapage}
\newcommand{\TypeExam}{CORRECTION Rattrap}
\newcommand{\TitreExam}{\TitreMatiere}
\newcommand{\DureeExam}{2h00}
\newcommand{\MyWaterMark}{\AnneeUniv} % Watermark de protection

% Ajout de mes classes & definitions
\usepackage{MetalExam} % Appelle un .sty

% "Tableau" et pas "Table"
\addto\captionsfrench{\def\tablename{Tableau}}

%%%%%%%%%%%%%%%%%%%%%%%
%Header
%%%%%%%%%%%%%%%%%%%%%%%
\lhead{\TypeExam}							%Gauche Haut
\chead{\NomUniv}							%Centre Haut
\rhead{\NumGroupe}							%Droite Haut
\lfoot{\DateExam}							%Gauche Bas
\cfoot{\thepage{} / \pageref*{LastPage}}	%Centre Bas
\rfoot{\texttt{\TitreMatiere}}				%Droite Bas

%%%%%

\usepackage{tabularx}

\newlength{\LabelWidth}%
%\setlength{\LabelWidth}{1.3in}%
\setlength{\LabelWidth}{1cm}%
%\settowidth{\LabelWidth}{Employee E-mail:}%  Specify the widest text here.

% Optional first parameter here specifies the alignment of
% the text within the \makebox.  Default is [l] for left
% alignment. Other options are [r] and [c] for right and center
\newcommand*{\AdjustSize}[2][l]{\makebox[\LabelWidth][#1]{#2}}%


\definecolor{mGreen}{rgb}{0,0.6,0}
\definecolor{mGray}{rgb}{0.5,0.5,0.5}
\definecolor{mPurple}{rgb}{0.58,0,0.82}
\definecolor{backgroundColour}{rgb}{0.95,0.95,0.92}

\lstdefinestyle{CStyle}{
    backgroundcolor=\color{backgroundColour},
    commentstyle=\color{mGreen},
    keywordstyle=\color{magenta},
    numberstyle=\tiny\color{mGray},
    stringstyle=\color{mPurple},
    basicstyle=\footnotesize,
    breakatwhitespace=false,
    breaklines=true,
    captionpos=b,
    keepspaces=true,
    numbers=left,
    numbersep=5pt,
    showspaces=false,
    showstringspaces=false,
    showtabs=false,
    tabsize=2,
    language=C
}


\hyphenation{op-tical net-works SIGKILL}


\begin{document}

%\MakeExamTitleDuree     % Pour afficher la duree
\MakeExamTitle                   % Ne pas afficher la duree

%% \MakeStudentName    %% A reutiliser sur chaque nouvelle page

\bigskip
%\bigskip

Vous devez respecter les consignes suivantes, sous peine de 0 :

\begin{enumerate}[label=\Roman*)]
\item Lisez le sujet en entier avec attention
\item Répondez sur le sujet
\item Ne détachez pas les agrafes du sujet
\item \'Ecrivez lisiblement vos réponses (si nécessaire en majuscules)
%\item Vous devez écrire dans le langage algorithmique classique ou en C (donc pas de Python ou autre)
\item Ne trichez pas
\end{enumerate}

%\bigskip

\vfillFirst

% Questions cours
\section{Questions (5 points)}

%\subsection{(1 point) \'Ecrivez l'état d'une file après avoir effectué ces opérations (la file est considérée comme initialement vide) : }
\subsection{(1 point) \'Ecrivez l'état d'une file après avoir effectué ces opérations (la file est considérée comme initialement vide), puis, indiquez quel élément sortira de la file lors du prochain \og dequeue \fg{}, ainsi que celui qui sortira en dernier : }

\bigskip

\begin{center}

\begin{large}
enfiler 2, enfiler 4, défiler, enfiler 3, enfiler 1, défiler, défiler, enfiler 7, enfiler 5, défiler, enfiler 6
\end{large}

\bigskip

\begin{tabular}{ | C{1cm} | C{1cm} | C{1cm} | C{1cm} | C{1cm} | C{1cm} | }
  \hline
  \cellcolor{black!0}   & \cellcolor{black!0}   & \cellcolor{black!0}   & \cellcolor{black!0} & \cellcolor{black!0} & \cellcolor{black!0} \\
  \cellcolor{black!0} 7 & \cellcolor{black!0} 5 & \cellcolor{black!0} 6 & \cellcolor{black!0} & \cellcolor{black!0} & \cellcolor{black!0} \\
  \cellcolor{black!0}   & \cellcolor{black!0}   & \cellcolor{black!0}   & \cellcolor{black!0} & \cellcolor{black!0} & \cellcolor{black!0} \\
  \hline
\end{tabular}

\smallskip

\begin{tabular}{   C{1cm}   C{1cm}   C{1cm}   C{1cm}   C{1cm}   C{1cm}   }
  \tikzarrowup{middarkgreen}{middarkgreen}{\phantom{cf}} &  &  &  &  &  \\
  Head &  &  &  &  &  \\
\end{tabular}

%\end{center}

%\bigskip
%\vspace*{-0.75cm}

%%%%%%%%%%%%%%%%%%%%%%%%%%%%%%%%%%%%%%%%%%%%%%%%%%%%%%%%%%%%%%%%%%%%%%%

%\subsection{(1 point) Quel élément sortira de la file lors du prochain \og dequeue \fg{} ? Quel élément sortira en dernier de la file ? }

%\bigskip
%\bigskip

%\begin{center}
\begin{table}[ht!]
%  \centering
  \begin{minipage}{0.50\textwidth}

Prochain élément qui sortira : 7

  \end{minipage}
  \hfillx
  \begin{minipage}{0.50\textwidth}

Dernier élément qui sortira : 6

  \end{minipage}
\end{table}
\end{center}

\bigskip

%%%%%%%%%%%%%%%%%%%%%%%%%%%%%%%%%%%%%%%%%%%%%%%%%%%%%%%%%%%%%%%%%%%%%%%
%%%%%%%%%%%%%%%%%%%%%%%%%%%%%%%%%%%%%%%%%%%%%%%%%%%%%%%%%%%%%%%%%%%%%%%

%\subsection{(1 point) \'Ecrivez l'état d'une pile après avoir effectué ces opérations (la pile est considérée comme initialement vide) : }
\subsection{(1 point) \'Ecrivez l'état d'une pile après avoir effectué ces opérations (la pile est considérée comme initialement vide), puis, indiquez quel élément sortira de la pile lors du prochain \og pop \fg{}, ainsi que celui qui sortira en dernier : }

\bigskip

\begin{center}

\begin{large}
empiler 2, empiler 4, dépiler, empiler 3, empiler 1, dépiler, dépiler, empiler 7, empiler 5, dépiler, empiler 6
\end{large}

%\bigskip

\bigskip

\begin{tabular}{ | C{1cm} | C{1cm} | C{1cm} | C{1cm} | C{1cm} | C{1cm} | }
  \hline
  \cellcolor{black!0}   & \cellcolor{black!0}   & \cellcolor{black!0}   & \cellcolor{black!0} & \cellcolor{black!0} & \cellcolor{black!0} \\
  \cellcolor{black!0} 6 & \cellcolor{black!0} 7 & \cellcolor{black!0} 2 & \cellcolor{black!0} & \cellcolor{black!0} & \cellcolor{black!0} \\
  \cellcolor{black!0}   & \cellcolor{black!0}   & \cellcolor{black!0}   & \cellcolor{black!0} & \cellcolor{black!0} & \cellcolor{black!0} \\
  \hline
\end{tabular}

\smallskip

\begin{tabular}{   C{1cm}   C{1cm}   C{1cm}   C{1cm}   C{1cm}   C{1cm}   }
  \tikzarrowup{middarkgreen}{middarkgreen}{\phantom{cf}} &  &  &  &  &  \\
  Head &  &  &  &  &  \\
\end{tabular}

%\end{center}

%\bigskip
%\vspace*{-0.75cm}

%%%%%%%%%%%%%%%%%%%%%%%%%%%%%%%%%%%%%%%%%%%%%%%%%%%%%%%%%%%%%%%%%%%%%%%

%\subsection{(1 point) Quel élément sortira de la pile lors du prochain \og pop \fg{} ? Quel élément sortira en dernier de la pile ? }

%\bigskip
%\bigskip

%\begin{center}
\begin{table}[ht!]
%  \centering
  \begin{minipage}{0.50\textwidth}

Prochain élément qui sortira : 6

  \end{minipage}
  \hfillx
  \begin{minipage}{0.50\textwidth}

Dernier élément qui sortira : 2

  \end{minipage}
\end{table}
\end{center}

\bigskip
\bigskip


%%%%%%%%%%%%%%%%%%%%%%%%%%%%%%%%%%%%%%%%%%%%%%%%%%%%%%%%%%%%%%%%%%%%%%%

\vfillLast

\clearpage

\subsection{(1,5 point) En admettant que l'on dispose d'une pile vide et que les éléments \og 1 2 3 4 5 6 \fg{} arrivent en entrée dans cet ordre exclusivement, décrivez les scénarios permettant d'obtenir les sorties suivantes : }

%\vfillFirst

%\bigskip
\medskip

\begin{center}
\noindent \textit{exemple : pour \og A B C \fg{} en entrée, on peut obtenir \og B C A \fg{} en sortie en faisant : \linebreak
\og push A \fg, \og push B \fg, \og pop \fg, \og push C \fg, \og pop \fg, \og pop \fg }

\noindent \textit{On a bien inséré A, puis B, puis C, mais l'ordre de sortie est différent suivant les \og pop \fg}
\end{center}

\medskip

%\vfillLast


\begin{center}

\begin{large}
2, 1, 3, 6, 5, 4
\end{large}

\begin{center}
%\GrilleReponseN{6}
 push 1, push 2, pop, pop, push 3, pop, push 4, push 5, push 6, pop, pop, pop
\end{center}


\begin{large}
4, 3, 5, 2, 1, 6
\end{large}

\begin{center}
%\GrilleReponseN{6}
 push 1, push 2, push 3, push 4, pop, pop, push 5, pop, pop, pop, push 6, pop
\end{center}


\begin{large}
2, 4, 3, 5, 6, 1
\end{large}

\begin{center}
%\GrilleReponseN{6}
 push 1, push 2, pop, push 3, push 4, pop, pop, push 5, pop, push 6, pop, pop
\end{center}

\end{center}

%\vfillLast
\clearpage

%%%%%%%%%%%%%%%%%%%%%%%%%%%%%%%%%%%%%%%%%%%%%%%%%%%%%%%%%%%%%%%%%%%%%%%%%%%%%%%%%%%%

\subsection{(1,5 point) Exécutez l'algorithme suivant, et notez l'état d'avancement des variables (inscrivez l'état initial dans la ligne prévue à cet effet) : }

\bigskip

\begin{center}
\textit{Vous exécuterez la fonction suivante avec comme paramètres a = 16435 et b = 4242 : }
\end{center}


\vfillFirst


%int FunctionXYZ(int a, int b)
%{
%  if (a <= 0)
%    return (b);
%  return (FunctionXYZ(a / 10, b + 1);
%}


\begin{table}[h!]
  \centering
  \begin{minipage}{0.59\textwidth}
    \centering
% %*   *)
\begin{lstlisting}[language=C]
int FunctionXYZ(int a, int b)
{
  int num = 0;

  if (a <= 0)
    return (-1);

  while ((a > 0) && (b > 0))
  {
    num += 1;
    a = a / 10;
    b = b / 10;
  }
  return (num);
}
\end{lstlisting}
%    \caption{Algorithme de la somme des N premiers entiers}
%    \label{algo-somme-n-premiers-entiers}
  \end{minipage}
  \hfillx
  \begin{minipage}{0.4\textwidth}
    \centering
    \begin{tabular}{| C{1cm} | C{1cm} | C{1cm} | C{1cm} |}
        \hline
          \cellcolor{black!45} tour  &  \cellcolor{black!15} a  &  \cellcolor{black!15} b  &  \cellcolor{black!15} num  \\
        \hline
        \multirow{2}{*}[0pt]{\textit{\'Etat}}  &       &      &   \\
        \multirow{2}{*}[0pt]{\textit{initial}} & 16435 & 4242 & 0 \\
             &     &     &       \\
        \hline
             &      &     &   \\
         1   & 1643 & 424 & 1 \\
             &      &     &   \\
        \hline
             &     &    &   \\
         2   & 164 & 42 & 2 \\
             &     &    &   \\
        \hline
             &    &   &   \\
         3   & 16 & 4 & 3 \\
             &    &   &   \\
        \hline
             &     &   &   \\
         4   & 1   & 0 & 4 \\
             &     &   &   \\
        \hline
        \cellcolor{black!5}                 & \cellcolor{black!5} & \cellcolor{black!5} & \cellcolor{black!5}   \\
        \cellcolor{black!5} \textit{return} & \cellcolor{black!5} & \cellcolor{black!5} & \cellcolor{black!5} 4 \\
        \cellcolor{black!5}                 & \cellcolor{black!5} & \cellcolor{black!5} & \cellcolor{black!5}   \\
        \hline
        \cellcolor{black!15} & \cellcolor{black!15} & \cellcolor{black!15} & \cellcolor{black!15} \\
        \cellcolor{black!15} & \cellcolor{black!15} & \cellcolor{black!15} & \cellcolor{black!15} \\
        \cellcolor{black!15} & \cellcolor{black!15} & \cellcolor{black!15} & \cellcolor{black!15} \\
        \hline
    \end{tabular}
%    \caption{Tableau d'exécution pas à pas}
%    \label{table-somme-n-premiers-entiers-execution}
  \end{minipage}
%  \caption{Algorithme de la somme des N premiers entiers}
%  \label{somme-n-premiers-entiers}
\end{table}


\bigskip

\vfillLast
\clearpage

%%%%%%%%%%%%%%%%%%%%%%%%%%%%%%%%%%%%%%%%%%%%%%%%%%%%%%%%%%%%%%%%%%%%%%%%%%%%%%%%%%%%
\section{Algorithmes (15 points)}

\subsection{(2 points) \'Ecrivez une fonction récursive \og \textit{ProductNInt} \fg{} calculant le produit des N premiers entiers }

\bigskip

\begin{center}
\GrilleReponseN{10}
\end{center}

\bigskip


\subsection{(2 points) \'Ecrivez une fonction itérative \og \textit{my\_strlen} \fg{} calculant la taille d'une chaîne de caractères }

\bigskip

\begin{center}
\GrilleReponseN{10}
\end{center}


\clearpage

%%%%%%%%%%%%%%%%%%%

\subsection{(0,5 point) \'Ecrivez une structure de données \og \textit{my\_stack} \fg{} pouvant servir de pile (à base de pointeurs ou de tableaux) }

\bigskip

\begin{center}
\GrilleReponseN{10}
\end{center}

\bigskip


\subsection{(0,5 point) \'Ecrivez une structure de données \og \textit{my\_queue} \fg{} pouvant servir de file (à base de pointeurs ou de tableaux) }

\bigskip

\begin{center}
\GrilleReponseN{10}
\end{center}


\clearpage

%%%%%%%%%%%%%%%%%%%

\subsection{(2,5 points) \'Ecrivez une fonction \og \textit{push} \fg{} pouvant servir à empiler un élément dans votre précédente structure \og \textit{my\_stack} \fg{} }

\bigskip

\begin{center}
%%\LigneReponseQuarante
%\LigneReponseTrente
%\LigneReponseCinq
%\LigneReponseTrois

\GrilleReponseN{24}
\end{center}


\newpage

%%%%%%%%%%%%%%%%%%%

\subsection{(2,5 points) \'Ecrivez une fonction \og \textit{pop} \fg{} pouvant servir à dépiler un élément dans votre précédente structure \og \textit{my\_stack} \fg{} }

\bigskip

\begin{center}
\GrilleReponseN{24}
\end{center}

\bigskip



\newpage

%%%%%%%%%%%%%%%%%%%

\subsection{(2,5 points) \'Ecrivez une fonction \og \textit{enqueue} \fg{} pouvant servir à enfiler un élément dans votre précédente structure \og \textit{my\_queue} \fg{} }

\bigskip

\begin{center}

\GrilleReponseN{24}
\end{center}


\newpage

%%%%%%%%%%%%%%%%%%%

\subsection{(2,5 points) \'Ecrivez une fonction \og \textit{dequeue} \fg{} pouvant servir à défiler un élément dans votre précédente structure \og \textit{my\_queue} \fg{} }

\bigskip

\begin{center}
\GrilleReponseN{24}
\end{center}

\bigskip

%%%%%%%%%%%%%%%%%%%%%%%%%%%%%%%%%%%%%%%%%%%%%%%%%%%%%%%%%%%%%

\clearpage


%\thispagestyle{empty}

\vfillFirst

\begin{center}

\begin{LARGE}
\textbf{\MakeUppercase{RATTRAPAGE \TitreMatiere}}
\end{LARGE}

\end{center}

\vfillLast

\end{document}
