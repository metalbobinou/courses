\documentclass[11pt,a4paper]{article}
\usepackage[utf8]{inputenc}
\usepackage[french]{babel}
\usepackage[T1]{fontenc}

\usepackage{amsmath}
\usepackage{amsfonts}
\usepackage{amssymb}

\newcommand{\NomAuteur}{Fabrice BOISSIER}
\newcommand{\TitreMatiere}{Algorithmique 1}
\newcommand{\NomUniv}{EPITA - Bachelor Cyber Sécurité}
\newcommand{\NiveauUniv}{CYBER1}
\newcommand{\NumGroupe}{CYBER1}
\newcommand{\AnneeUniv}{2023-2024}
\newcommand{\DateExam}{novembre 2023}
\newcommand{\TypeExam}{CORRECTION Examen 1 - Décalé}
\newcommand{\TitreExam}{\TitreMatiere}
\newcommand{\DureeExam}{1h30}
\newcommand{\MyWaterMark}{\AnneeUniv} % Watermark de protection

% Ajout de mes classes & definitions
\usepackage{MetalExam} % Appelle un .sty

% "Tableau" et pas "Table"
\addto\captionsfrench{\def\tablename{Tableau}}

%%%%%%%%%%%%%%%%%%%%%%%
%Header
%%%%%%%%%%%%%%%%%%%%%%%
\lhead{\TypeExam}							%Gauche Haut
\chead{\NomUniv}							%Centre Haut
\rhead{\NumGroupe}							%Droite Haut
\lfoot{\DateExam}							%Gauche Bas
\cfoot{\thepage{} / \pageref*{LastPage}}	%Centre Bas
\rfoot{\texttt{\TitreMatiere}}				%Droite Bas

%%%%%

\usepackage{tabularx}

\newlength{\LabelWidth}%
%\setlength{\LabelWidth}{1.3in}%
\setlength{\LabelWidth}{1cm}%
%\settowidth{\LabelWidth}{Employee E-mail:}%  Specify the widest text here.

% Optional first parameter here specifies the alignment of
% the text within the \makebox.  Default is [l] for left
% alignment. Other options are [r] and [c] for right and center
\newcommand*{\AdjustSize}[2][l]{\makebox[\LabelWidth][#1]{#2}}%


\definecolor{mGreen}{rgb}{0,0.6,0}
\definecolor{mGray}{rgb}{0.5,0.5,0.5}
\definecolor{mPurple}{rgb}{0.58,0,0.82}
\definecolor{backgroundColour}{rgb}{0.95,0.95,0.92}

\lstdefinestyle{CStyle}{
    backgroundcolor=\color{backgroundColour},
    commentstyle=\color{mGreen},
    keywordstyle=\color{magenta},
    numberstyle=\tiny\color{mGray},
    stringstyle=\color{mPurple},
    basicstyle=\footnotesize,
    breakatwhitespace=false,
    breaklines=true,
    captionpos=b,
    keepspaces=true,
    numbers=left,
    numbersep=5pt,
    showspaces=false,
    showstringspaces=false,
    showtabs=false,
    tabsize=2,
    language=C
}


\hyphenation{op-tical net-works SIGKILL}


\begin{document}

%\MakeExamTitleDuree     % Pour afficher la duree
\MakeExamTitle                   % Ne pas afficher la duree

%% \MakeStudentName    %% A reutiliser sur chaque nouvelle page

\bigskip
%\bigskip

Vous devez respecter les consignes suivantes, sous peine de 0 :

\begin{itemize}
\item Lisez le sujet en entier avec attention
\item Répondez sur le sujet
\item Ne détachez pas les agrafes du sujet
\item \'Ecrivez lisiblement vos réponses (si nécessaire en majuscules)
%\item Vous devez écrire dans le langage algorithmique classique ou en C (donc pas de Python ou autre)
\item Ne trichez pas
\end{itemize}

%\bigskip

%\vfillFirst

% Questions cours
\section{Questions (6 points)}

\subsection{(2 points) Sélectionnez les conditions vraies pour A = 8 et B = 5 : }

% Compter pour chaque réponse 0,5 points : si bonne ou pas bonne
\bigskip

\begin{itemize}
  \item[\CaseCoche] (A > B) et ((A - 3) > B)  % Faux
  \item[\checkmark] (A <= B) ou ((B + 3) <= A)  % Vrai
  \item[\CaseCoche] ((A - 2) < (B + 2)) et (A > (B + 3))  % Faux
  \item[\checkmark] ((A - 3) <= B) et ((A / 2) < B)  % Vrai
\end{itemize}


%\bigskip
%\medskip


\subsection{(4 points) Exécutez cet algorithme avec les valeurs (x = 3) (y = 2) et (z = 4) en remplissant le tableau, puis donnez les caractéristiques de cet algorithme : }

%\bigskip

\begin{table}[!ht]
  \centering
  \begin{minipage}{0.59\textwidth}
    \centering

% %*   *)
\begin{lstlisting}[style=algorithmique]
algorithme fonction Calc : entier
  parametres locaux
    entier    x, y, z
debut
si (y == 1)
  retourne (1)
sinon
  si ((x %*\texttt{\%}*) y) == 0)
    retourne (y + Calc(x, (y - 1), z))
  sinon
    retourne (Calc(x, (y - 1), z))
  fin si
fin si
fin algorithme fonction Calc \end{lstlisting}

\begin{itemize}
  \item[\checkmark] Il est récursif  \phantom{lfg}\\
  \item[\CaseCoche] Il est même récursif terminal  \phantom{lfg}\\
  \item[\checkmark] Il s'agit d'une fonction  \phantom{lfg}\\
  \item[\CaseCoche] Il s'agit d'une procédure  \phantom{lfg}\\
\end{itemize}


  \end{minipage}
  \hfillx
  \begin{minipage}{0.4\textwidth}
    \centering

    \begin{tabular}{| C{1cm} | C{1cm} | C{1cm} | C{1cm} |}
        \hline
   \textit{tour}  &  \textbf{x}  &  \textbf{y}  &  \textbf{z}    \\
        \hline
                  &     &     &       \\
           0      &  4  &  3  &  5    \\
                  &     &     &       \\
        \hline
                  &     &     &       \\
           1      &  4  &  2  &  5    \\
                  &     &     &       \\
        \hline
                  &     &     &       \\
           2      &  4  &  1  &  5    \\
                  &     &     &       \\
        \hline
                  &     &     &       \\
          (3)     &     &     &       \\
                  &     &     &       \\
        \hline
                  &     &     &       \\
                  &     &     &       \\
                  &     &     &       \\
        \hline
                  &     &     &       \\
                  &     &     &       \\
                  &     &     &       \\
        \hline
                  &     &     &       \\
                  &     &     &       \\
                  &     &     &       \\
        \hline
    \end{tabular}
  \end{minipage}

\end{table}


%\vfillLast

%%%%%%%%%%%%%%%%%%%%%%%%%%%%%%%%%%%%%%%%%%%%%%%%%%%%%%%%%%%%%%%%

\clearpage

\section{Algorithmes (15 points)}

\subsection{(2 points) \'Ecrivez une fonction \og \textit{Factorielle} \fg{} récursive terminale. (Vous décrirez d'abord le ou les cas d'arrêts, puis le cas général) }

%\bigskip
\vspace*{-0.5cm}

\begin{center}

\begin{equation*}
n! = \prod^{n}_{i = 1} i = 1 \times 2 \times ... \times N
\end{equation*}
%
\vspace*{-0.5cm}
%
\begin{equation*}
0! = 1
\end{equation*}

Explications du/des cas d'arrêts, puis du cas général :
\GrilleReponseN{6}

\bigskip

Algorithme :
\GrilleReponseN{14}
\end{center}


\clearpage


\subsection{(2 points) \'Ecrivez une fonction \og \textit{SommePow} \fg{} récursive calculant la somme des N premières puissances d'un nombre M. (Vous décrirez d'abord le ou les cas d'arrêts, puis le cas général) }

%\bigskip
\vspace*{-0.5cm}

\begin{center}

\begin{equation*}
\sum^{N}_{i = 0} M^{i} = M^{0} + M^{1} + M^{2} + ... + M^{N}
\end{equation*}

Explications du/des cas d'arrêts, puis du cas général :
\GrilleReponseN{6}

\bigskip

Algorithme :
\GrilleReponseN{13}
\end{center}

\smallskip
%\medskip
%\bigskip


%\clearpage


%\subsection{(2 points) \'Ecrivez une fonction \og \textit{RechercheTab} \fg{} cherchant l'emplacement d'un élément dans un tableau d'entiers. Si l'élément n'est pas trouvé, vous renverrez \textit{-1}. (Vous décrirez d'abord le fonctionnement général) }
%
%%\smallskip
%\bigskip
%
%\begin{center}
%Explications du/des cas d'arrêts, puis du cas général :
%\GrilleReponseN{6}
%
%\bigskip
%
%Algorithme :
%\GrilleReponseN{16}
%\end{center}


\clearpage


\subsection{(3 points) \'Ecrivez une procédure \og \textit{NbPairsTab} \fg{} itérative affichant la quantité de nombres pairs d'un tableau d'entiers. (Vous décrirez d'abord le fonctionnement général) }

%\bigskip

\begin{center}
Explications du fonctionnement :
\GrilleReponseN{6.5}

Algorithme :
\GrilleReponseN{16}
\end{center}


%\clearpage


\subsection{(3 points) \'Ecrivez une fonction \og \textit{TabToInt} \fg{} transformant un tableau d'entiers en un unique entier, et l'affichant à l'écran. Chaque case contient un nombre positif entre 10 et 99. Si une case possède un nombre plus petit que 10 ou plus grand que 99, la fonction renverra -1. (Vous décrirez d'abord le fonctionnement général) }

%\medskip

\begin{center}
  \begin{tabular}{| c | c | c | c |}
    \hline
    42 & 10 & 21 & 32 \\
    \hline
  \end{tabular}

  \smallskip

  Ce tableau doit devenir $ 42102132 $ sous forme d'entier
\end{center}

%\bigskip

\begin{center}
Explications du fonctionnement :
\GrilleReponseN{5}

Algorithme :
\GrilleReponseN{14}
\end{center}
% tabtoint(tab, len, acc, mul)
% si len == 0
%  retourner (acc + (mul * tab[0]))
% sinon
%  retourner (f(tab, len - 1, acc + (mul * tab[len]), mul * 10)


\subsection{(5 points) \'Ecrivez une fonction \og \textit{Swap} \fg{} qui échange la position de deux valeurs dans un tableau, puis, écrivez une procédure de tri. (Vous décrirez d'abord le fonctionnement général de l'algorithme de tri) }

\vfillFirst

\begin{center}
Swap :
\GrilleReponseN{12}
\end{center}

\vfillLast

\clearpage

\vfillFirst

\begin{center}
Explications du fonctionnement :
\GrilleReponseN{6}

\medskip

Algorithme de tri :
\GrilleReponseN{15}
\end{center}

\vfillLast

\newpage



%\thispagestyle{empty}

\vfillFirst

\begin{center}

\begin{LARGE}
\textbf{\MakeUppercase{Sujet Décalé}}

\bigskip

\textbf{\MakeUppercase{\TitreMatiere}}
\end{LARGE}

\end{center}

\vfillLast

\end{document}
