\documentclass[11pt,a4paper]{article}
\usepackage[utf8]{inputenc}
\usepackage[french]{babel}
\usepackage[T1]{fontenc}

\usepackage{amsmath}
\usepackage{amsfonts}
\usepackage{amssymb}

\newcommand{\TitreMatiere}{Langage C}
\newcommand{\TitreSeance}{Bases de C}
\newcommand{\NumeroTD}{TP 0 - Préparatifs}
\newcommand{\DateCours}{23 novembre 2022}
\newcommand{\AnneeScolaire}{2022-2023}
\newcommand{\Organisation}{EPITA}
\newcommand{\NomAuteurA}{Fabrice BOISSIER}
\newcommand{\MailAuteurA}{fabrice.boissier@epita.fr}
\newcommand{\NomAuteurB}{ }
\newcommand{\MailAuteurB}{ }
\newcommand{\DocKeywords}{C, Langage}
\newcommand{\DocLangue}{fr} % "en", "fr", ...

\usepackage{MetalQuickLabs}

\begin{document}

\EncadreTitre

\bigskip


%\begin{center}
%\begin{tabular}{p{5cm} p{11cm}}
%\textbf{Commandes étudiées :} & \texttt{sh}, \texttt{bash}, \texttt{man}, \texttt{ls}, \texttt{mkdir}, \texttt{touch}, \texttt{chmod}, \texttt{mv}, \texttt{rm}, \texttt{rmdir}, \texttt{cat}, \texttt{file}, \texttt{which}, \texttt{which}\\
%
%\textbf{Builtins étudiées :} & \texttt{pwd}, \texttt{cd}, \texttt{exit}, \texttt{logout}, \texttt{echo}, \texttt{umask}, \texttt{type}, \texttt{>}, \texttt{>{}>}, \texttt{<}, \texttt{<{}<}, \texttt{|}\\
%
%\textbf{Notions étudiées :} & Shell, Manuels, Fichiers, Répertoires, Droits, Redirections\\
%\end{tabular}
%\end{center}

\bigskip


Ce TP a pour objectif de vous faire découvrir le langage C et les outils nécessaires pour le faire fonctionner.
Vous démarrerez depuis l'écriture de quelques lignes de code que vous compilerez, pour finalement réaliser un petit programme de statistiques.
Durant tout l'exercice, vous utiliserez au fur et à mesure quelques notions essentielles du langage C.

\bigskip

\section{Prise en main de C}

\bigskip

Avant de pouvoir utiliser C, vous aurez besoin d'installer l'environnement de développement : le compilateur C, et plusieurs outils pour débugger et automatiser une partie de la chaîne de compilation.
De plus, vous devez choisir un éditeur texte qui vous permettra de développer facilement (grâce à la coloration syntaxique, notamment).

Selon votre système d'exploitation, vous pouvez trouver quelques éditeurs texte comme : \textit{notepad++}, \textit{sublimtext}, \textit{vim}, \textit{emacs}, ...

\bigskip

Pour pouvoir \textit{compiler} du code (c'est-à-dire transformer un code écrit dans un langage haut niveau en un code bas niveau/compréhensible par votre processeur), il vous faut disposer d'un programme faisant ce travail : le compilateur.

Plusieurs compilateurs pour le C existent : \textit{visual studio}, \textit{icc}, \textit{gcc}, \textit{clang}, ...

Nous utiliserons dans ce cours \textit{gcc} et \textit{clang} indifférement.

\bigskip

Si vous êtes sur linux, sur des distributions dérivées de Debian (comme Ubuntu), vous devrez installer tous ces programmes :

\begin{itemize}
\item gcc
\item g++
\item make
\item gdb
\item valgrind
\item m4
\item bash
\item tar
\item openssl
\end{itemize}

Certains sont déjà inclus dans un pack de programmes extrêmement utile : \textit{build-essential}

\bigskip

Pour installer ces programmes sur un linux dérivé de Debian, vous devez entrer dans le terminal :

\medskip

\TTBF{apt-get update}

\TTBF{apt-get install build-essential}

\TTBF{apt-get install gcc g++ make gdb valgrind m4 bash}

\medskip

En cas d'échec pour manque de permissions, vous devrez précéder chacune de vos commandes par un \TTBF{sudo}, ce qui donnera par exemple :

\medskip

\TTBF{sudo apt-get update}

\medskip

Si vous avez installé un autre linux, ou un autre dérivé d'UNIX, vous devrez trouver vous-même comment installer ces programmes (et éventuellement leurs dépendances).

Si vous êtes sur macOS, vous devrez probablement installer une VM (comme pour Windows), ou alors utiliser un gestionnaire de paquets externe (comme \textit{MacPorts}, ou \textit{HomeBrew}) et \textit{Xcode} que vous trouverez dans l'app store officiel d'Apple.

\medskip

Une fois installés, vous pouvez fermer votre terminal, et en rouvrir un nouveau dans lequel vous pourrez taper la commande suivante :

\medskip

\TTBF{gcc}

\medskip

Si tout est correctement installé, vous devriez obtenir comme réponse dans le terminal :

\medskip

\begin{lstlisting}[style=sh,morekeywords={floor,ceil}]
gcc: fatal error: no input files
compilation terminated. \end{lstlisting}


Si vous utilisez \TTBF{clang} malgré vous (par exemple sur macOS), vous devriez obtenir ce message :

\medskip

\begin{lstlisting}[style=sh,morekeywords={floor,ceil}]
clang: error: no input files \end{lstlisting}

\bigskip

Pour accéder à la documentation des fonctions et appels systèmes, vous pouvez la commande \TTBF{man}.
La section numéro 2 sert aux appels système (\TTBF{open}, \TTBF{read}, \TTBF{write}, \TTBF{close}, ...), et la section numéro 3 sert aux fonctions des bibliothèques (\TTBF{printf}, \TTBF{malloc}, \TTBF{memset}, \TTBF{fopen}, \TTBF{fread}, ...).

\medskip

Pour lire la documentation de la fonction \TTBF{malloc(3)}, il suffit donc de taper dans le shell :

\medskip

\begin{lstlisting}[style=sh,morekeywords={floor,ceil}]
man 3 malloc\end{lstlisting}

Pour avancer dans le manuel, vous pouvez utiliser les flèches et la touche espace.
Pour quitter le manuel, il suffit d'appuyer sur la touche \TTBF{q} (\TTBF{Q} minuscule).

\end{document}
