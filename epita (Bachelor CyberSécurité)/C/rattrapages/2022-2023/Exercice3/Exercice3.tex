%% Exercice 3

%\ExoSpecs{\TTBF{CalculTVA.sh}}{\TTBF{\RenduDir/src/exo1/}}{750}{640}{\TTBF{write}}
\ExoSpecsSimple{\TTBF{my\_transpose.c}}{\TTBF{\RenduDir/src/my\_transpose.c}}{750}{640}

\vspace*{0.7cm}

\noindent \ExoObjectif{Le but de l'exercice est de transposer une matrice.}

\bigskip

%\noindent Une matrice est donnée dans un fichier en paramètre, il faut simplement l'affichée dans sa version transposée.

\noindent Vous devez écrire une fonction nommée \TTBF{my\_transpose} qui prendra en paramètre un fichier contenant une matrice, et affichera dans le terminal la version transposée.

%\bigskip

%\vfill
%\hspace{0pt}

%\smallskip

\bigskip

\lstset{language=C}
\begin{lstlisting}[frame=single,title={Prototype de la fonction}]
int my_transpose(char *filename)
\end{lstlisting}

%\bigskip

\vfillFirst

\lstset{language=sh}
\begin{lstlisting}[frame=single,title={Exemples de fichiers d'entrée}]
$ cat file1.txt
123
456
789
$ cat file2.txt
0123
4567
$ cat file3.txt
01
23
45
67
$ cat empty.txt
$ cat file0.txt
0
\end{lstlisting}

%\hspace{0pt}
%\vfill

%\bigskip

\vfillLast

\clearpage

\vfillFirst

%\smallskip

\lstset{language=sh}
\begin{lstlisting}[frame=single,title={Cas général}]
$ ./my_transpose file1.txt
147
258
369
$ ./my_transpose file2.txt
04
15
26
37
$ ./my_transpose file3.txt
0246
1357
$ ./my_transpose file0.txt
0
\end{lstlisting}

\bigskip

\noindent Deux cas d'erreur doivent être gérés avant tout affichage : si la matrice est vide (ou n'existe pas), et si la matrice contient autre chose que des nombres.

\noindent Si la matrice est vide ou que le fichier n'existe pas, il faut indiquer le message suivant et retourner 1.

\bigskip

\noindent \TTBF{Empty\textvisiblespace matrix}

\bigskip

\lstset{language=sh}
\begin{lstlisting}[frame=single,title={Cas d'erreur 1},morekeywords={rm}]
$ ./my_transpose empty.txt
Empty matrix
$ echo $?
1
$ rm -f non-existent
$ ./my_transpose non-existent
Empty matrix
$ echo $?
1
\end{lstlisting}

\noindent Si la matrice contient des caractères autres que des nombres, il faut indiquer le message suivant et renvoyer 2.

\bigskip

\noindent \TTBF{Incorrect\textvisiblespace matrix}

\vfillLast

%\bigskip

\lstset{language=sh}
\begin{lstlisting}[frame=single,title={Cas d'erreur 2}]
$ cat file_error.txt
123
abc
789
$ ./my_transpose file_error.txt
Incorrect matrix
$ echo $?
2
\end{lstlisting}

%\bigskip
%
%\begin{RedBoxTitle}{ATTENTION}
%    Les retours à la ligne ne doivent pas être faits avec la balise \TTBF{"<br />"}, mais avec \TTBF{"\textbackslash n"}.
%    (se référer à la section \hyperref[sec:AideMemoire]{Aide Mémoire})
%\end{RedBoxTitle}
