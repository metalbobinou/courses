%% Exercice 3

%\ExoSpecs{\TTBF{CalculTVA.sh}}{\TTBF{\RenduDir/src/exo3/}}{750}{640}{\TTBF{write}}
\ExoSpecsSimple{\TTBF{my\_pintree.c}}{\TTBF{\RenduDir/src/my\_pintree.c}}{750}{640}

\vspace*{0.7cm}

\noindent \ExoObjectif{Le but de l'exercice est d'afficher un sapin en ASCII art, dont la taille varie selon le paramètre donné.}

\bigskip

\noindent Vous devez écrire un programme qui prendra un paramètre, et affichera selon ce paramètre un sapin.
\noindent Dans le cas général, affichez le sapin, et renvoyez 0.

\bigskip

\lstset{language=sh}
\begin{lstlisting}[frame=single,title={Cas général}]
$ ./my_pintree 1
 /\
/  \
____
 ||
$ echo $?
0
$ ./my_pintree 4
    /\
   /  \
  /    \
 /      \
/        \
__________
    ||
$ echo $?
0
$ ./my_pintree 5
     /\
    /  \
   /    \
  /      \
 /        \
/          \
____________
     ||
$ echo $?
0
\end{lstlisting}

\newpage

\noindent De façon précise, voici les spécifications pour les cas 1, 4, et 5 :

\bigskip

\hspace*{-\parindent} %% BIDOUILLE
\begin{minipage}{15.85cm} %% BIDOUILLE
\lstset{language=sh}
\begin{lstlisting}[frame=single,title={Cas général 1}]
 /\     1 espace / 0 espace  \
/  \    0 espace / 2 espaces \
____    4 _
 ||     1 espace 2 |
\end{lstlisting}
\end{minipage} %% FIN BIDOUILLE

\hspace*{-\parindent} %% BIDOUILLE
\begin{minipage}{15.85cm} %% BIDOUILLE
\lstset{language=sh}
\begin{lstlisting}[frame=single,title={Cas général 4}]
    /\          4 espaces / 0 espace  \
   /  \         3 espaces / 2 espaces \
  /    \        2 espaces / 4 espaces \
 /      \       1 espace  / 6 espaces \
/        \      0 espace  / 8 espaces \
__________      10 _
    ||          4 espaces 2 |
\end{lstlisting}
\end{minipage} %% FIN BIDOUILLE

\hspace*{-\parindent} %% BIDOUILLE
\begin{minipage}{15.85cm} %% BIDOUILLE
\lstset{language=sh}
\begin{lstlisting}[frame=single,title={Cas général 5}]
     /\         5 espaces / 0 espace   \
    /  \        4 espaces / 2 espaces  \
   /    \       3 espaces / 4 espaces  \
  /      \      2 espaces / 6 espaces  \
 /        \     1 espace  / 8 espaces  \
/          \    0 espace  / 10 espaces \
____________    12 _
     ||         5 espaces 2 |
\end{lstlisting}
\end{minipage} %% FIN BIDOUILLE

\bigskip

\noindent Plusieurs cas spéciaux sont à prendre en compte.\\

\noindent Tout d'abord, dans le cas où le paramètre 0 est donné, vous retournerez 0 et afficherez :

\bigskip

\lstset{language=sh}
\begin{lstlisting}[frame=single,title={Cas 0}]
$ ./my_pintree 0
/\
||
$ echo $?
0
\end{lstlisting}

\bigskip

\noindent Si aucun ou plus de 1 paramètre est donné, vous devrez afficher le message d'erreur suivant, et retourner 1 : \\

\bigskip

\noindent \TTBF{Usage:\textvisiblespace ./my\_pintree\textvisiblespace number}

\bigskip

\lstset{language=sh}
\begin{lstlisting}[frame=single,title={Cas d'erreur}]
$ ./my_pintree
Usage: ./my_pintree number
$ echo $?
1
$ ./my_pintree 0 4
Usage: ./my_pintree number
$ echo $?
1
\end{lstlisting}


%%%%%

\noindent Si un paramètre texte est donné, il sera interprété comme 0 :

\bigskip

\lstset{language=sh}
\begin{lstlisting}[frame=single,title={Cas texte}]
$ ./my_pintree blob
/\
||
$ echo $?
0
\end{lstlisting}

\bigskip

\noindent Si plusieurs paramètres sont donnés, qu'ils soient texte ou nombre, vous devez afficher le message d'erreur suivant et retourner 1.

\bigskip

\lstset{language=sh}
\begin{lstlisting}[frame=single,title={Cas texte d'erreur}]
$ ./my_pintree plop plop
Usage: ./my_pintree number
$ echo $?
1
$ ./my_pintree 3 plop
Usage: ./my_pintree number
$ echo $?
1
\end{lstlisting}
