%% Exercice 1

%\ExoSpecs{\TTBF{CalculTVA.sh}}{\TTBF{\RenduDir/src/exo1/}}{750}{640}{\TTBF{write}}
\ExoSpecsCustom{\TTBF{StrBasics.c} \TTBF{StrBasics.h}}{\TTBF{\RenduDir/src/}}{750}{640}{Fonctions autorisées}{\TTBF{malloc(3)}, \TTBF{free(3)}}

\vspace*{0.7cm}

\noindent \ExoObjectif{Le but de l'exercice est de recoder les fonctions essentielles permettant de manipuler des chaînes de caractères : \TTBF{my\_strlen}, \TTBF{my\_strdup}, et \TTBF{my\_strcmp} }

\bigskip

\noindent Vous devez implémenter ces fonctions en respectant strictement les spécifications suivantes.
Vous ne devez ni rendre de fonction \TTBF{main} ni vos tests : vous ne devez rendre \textit{que} les fonctions demandées (et éventuellement celles vous permettant de les faire fonctionner) dans le fichier \texttt{StrBasics.c}, ainsi que leurs prototypes dans le fichier \texttt{StrBasics.h}.

\bigskip

%\noindent Vous devez implémenter les X fonctions suivantes :
%\begin{itemize}
%\item \TTBF{int my\_strlen(char *str)}
%\item \TTBF{char *my\_strdup(char *str)}
%\item \TTBF{int my\_strcmp(char *s1, char *s2)}
%\end{itemize}

\noindent Vous devez implémenter les fonctions suivantes :

\bigskip

\lstset{language=C}
%\begin{lstlisting}[frame=single,title={Liste des fonctions pour l'exercice 1}]
\begin{lstlisting}[frame=single]
int my_strlen(char *str);
char *my_strdup(char *str);
int my_strcmp(char *s1, char *s2);
\end{lstlisting}


\subsubsection*{\TTBF{int my\_strlen(char *str)}}

\noindent Cette fonction calcule la taille d'une chaîne de caractères.
La fonction doit renvoyer un entier qui correspond au nombre de caractères dans la chaîne (sans compter le '\TTBF{\textbackslash{}0}' final).
Si la chaîne de caractère est vide (elle ne contient qu'un seul caractère : '\TTBF{\textbackslash{}0}'), alors la fonction doit renvoyer $ 0 $.
Si le paramètre donné est \TTBF{NULL}, alors la fonction renverra $ -1 $.


\subsubsection*{\TTBF{char *my\_strdup(char *str)}}

\noindent Cette fonction duplique la chaîne de caractère donnée en paramètre.
La fonction doit renvoyer l'adresse d'un espace mémoire distinct contenant une copie de la chaîne donnée en paramètre.
Cet espace mémoire doit avour été alloué avec \TTBF{malloc(3)} (afin que l'utilisateur de votre fonction puisse le libérer avec \TTBF{free(3)}).
Si la chaîne de caractères donnée en paramètre est vide (le premier caractère est un '\TTBF{\textbackslash{}0}'), alors la fonction doit renvoyer \TTBF{NULL}.
Si \TTBF{malloc(3)} n'arrive pas à allouer assez de mémoire pour stocker la nouvelle chaîne, alors la fonction doit renvoyer \TTBF{NULL}.


\subsubsection*{\TTBF{int my\_strcmp(char *s1, char *s2)}}

\noindent Cette fonction compare deux chaînes de caractères entre elles, caractère par caractère.
La fonction retourne $ 0 $ si les deux chaînes de caractères sont égales.
En cas d'inégalité, il s'agit de calculer la différence lexicographique entre \TTBF{s1} et \TTBF{s2}.

\noindent Ainsi, \TTBF{strcmp("ABCDE", "ABCDG")} renverra $ -2 $, car '\TTBF{E}' est deux crans avant '\TTBF{G}', mais,  \TTBF{strcmp("ABCDG", "ABCDE")} renverra $ 2 $, car '\TTBF{G}' est deux crans après '\TTBF{E}'.

\noindent \textit{N'oubliez pas que les caractères sont déjà considérés comme des nombres.}

% AJOUTER LES TESTS ET PRECISIONS SI CHAINE 1 EST PLUS PETITE QUE CHAINE 2
