%% Exercice 4

%\ExoSpecs{\TTBF{CalculTVA.sh}}{\TTBF{\RenduDir/src/exo1/}}{750}{640}{\TTBF{write}}
%\ExoSpecsCustom{\TTBF{check.sh} \TTBF{tests.c} \textit{[\TTBF{test_XYZ.c}]}}{\TTBF{\RenduDir/check/}}{750}{640}{Fonctions autorisées}{\textit{(toutes)}}
\ExoSpecsCustom{\TTBF{check.sh}, \TTBF{tests.c}, \textit{[\TTBF{test\_XYZ.c}]}}{\TTBF{\RenduDir/check/}}{750}{640}{Fonctions autorisées}{\textit{(toutes)}}

\vspace*{0.7cm}

\noindent \ExoObjectif{Le but de l'exercice est de fournir un programme et un script qui testeront votre bibliothèque d'arbres.}

\bigskip

\noindent Vous devez écrire une suite de tests vérifiant le bon fonctionnement de votre bibliothèque \TTBF{libmybintree}.
L'appel à la commande \texttt{make} à la racine du projet avec la cible \textit{check} doit compiler un (ou des) fichiers C se trouvant dans le dossier \TTBF{check} afin qu'ils utilisent la bibliothèque que vous avez écrite préalablement, puis le script \TTBF{check.sh} doit lancer les tests.

Attention : la cible \TTBF{check} doit avoir comme dépendance la cible \TTBF{libmybintree} afin que les bibliothèques soient générées \textit{avant} de compiler et lier les tests.
Inclure les fichiers C des exercices précédents ne donnera pas tous les points/limitera la note de cet exercice : seuls les fichiers \TTBF{bt\_basics.h} et \TTBF{bt\_bst.h} doivent être utilisés dans un \TTBF{include} pour cet exercice, mais pas les \TTBF{.c} !

\bigskip

\noindent Vous devez donc écrire plusieurs scénarios de test qui construisent des arbres, les parcourent, et suppriment des nœuds (en re-faisant le parcours par la suite).
Comme vous l'aurez constaté, le format de sortie texte est très précis pour justement permettre des comparaisons simples et aisées.

\noindent Une façon de rédiger les tests consiste à préparer un scénario sur papier en amont, d'écrire le code C réalisant ce scénario et écrivant en sortie standard à chaque étape l'état de l'arbre, puis, d'écrire dans un script shell les sorties attendues et les comparer.

\medskip

\noindent Vous pouvez tout à fait écrire plusieurs scénarios dans plusieurs fichiers C et sh distincts.
Il faut néanmoins que le script \TTBF{check.sh} existe et lance les tests.

\bigskip

\noindent Les points attribués dans cet exercice dépendront de l'exhaustivité des tests, mais également de la façon de tester : comme indiqué plus haut, si vous n'utilisez pas l'une des bibliothèques résultantes, mais directement les sources de votre projet (\TTBF{bt\_basics.c}, \TTBF{bt\_bst.c}, et vos éventuels autres fichiers sources, ou vous copiez/collez le code, ou d'autres techniques encore...), alors vous n'aurez pas le maximum de points.
Les tests doivent être exhaustifs pour ne rater aucun cas parmi ceux spécificiés dans ce sujet.
