%% Format de Rendu

%\ResponsablesProjet{Metal Man/metalman@example.org, Damdoshi/damdoshi@example.org, Tayst/tayst@tayst.org}
%\begin{tabular}{p{7cm} p{10cm}}
\begin{tabular}{p{7cm} p{8.5cm}}
	%\ResponsablesProjetRow{Fabrice BOISSIER/fabrice.boissier@epita.fr, Mark ANGOUSTURES/mark.angoustures@epita.fr}
	\ResponsablesProjetRow{Fabrice BOISSIER/fabrice.boissier@epita.fr}
	& \\
%	\RenduSpecsGenerales{[PHP][DM]}{2}{Envoi par mail}{\RenduDir}{\RenduTarball}{10/02/2020 23h42}{2 semaines}
%	\RenduSpecsGenerales{[CAV][TP1]}{1}{Pas de rendu}{\RenduDir}{\RenduTarball}{Pas de rendu}{Pas de rendu}
%	\RenduSpecsGenerales{[ALG][BT]}{1}{Devoir/Assignment sur Teams}{\RenduDir}{\RenduTarball}{02/05/2023 23h42}{1 mois}
	\RenduSpecsGenerales{[ALG][LPF]}{1}{Devoir sur Moodle}{\RenduDir}{\RenduTarball}{28/01/2024 23h42}{1,5 semaine}
	& \\
%	\RenduSpecsTechniques{WAMP ou MAMP}{PHP}{Apache/PHP}{ }
	\RenduSpecsTechniques{Linux - Ubuntu (x86\_64)}{C}{/usr/bin/gcc}{-W -Wall -Werror -std=c99 -pedantic}
%	& \\
%	Fonctions autorisées : & malloc(3), free(3), printf(3)
\end{tabular}


\vspace*{1cm}


\noindent Les fichiers suivants sont requis :

\medskip

\begin{tabular}{l p{12cm}}
\texttt{AUTHORS} & contient le(s) nom(s) et prénom(s) de(s) auteur(s).\\
\texttt{Makefile} & le Makefile principal.\\
\texttt{README} & contient la description du projet et des exercices, ainsi que la fa\c con d'utiliser le projet.\\
\texttt{configure} & le script shell de configuration pour l'environnement de compilation.\\
\end{tabular}


\vspace*{1cm}


\noindent Un fichier \TTBF{Makefile} doit être présent à la racine du dossier, et doit obligatoirement proposer ces règles :

\medskip

\begin{tabular}{l p{13cm}}
\texttt{all} & \textit{[Première règle]} lance la règle \texttt{libmylpf}.\\
\texttt{clean} & supprime tous les fichiers temporaires et ceux créés par le compilateur.\\
\texttt{dist} & crée une archive propre, valide, et répondant aux exigences de rendu.\\
\texttt{distclean} & lance la règle \texttt{clean}, puis supprime les binaires et bibliothèques.\\
\texttt{check} & lance le(s) script(s) de test.\\
\texttt{libmylpf} & lance les règles \texttt{shared} et \texttt{static} \\
\texttt{shared} & compile l'ensemble du projet avec les options de compilations exigées et génère une bibliothèque dynamique.\\
\texttt{static} & compile l'ensemble du projet avec les options de compilations exigées et génère une bibliothèque statique.\\
\end{tabular}


\vspace*{1cm}
\newpage

\noindent Votre code sera testé automatiquement, vous devez donc scrupuleusement respecter les spécifications pour pouvoir obtenir des points en validant les exercices.
%
%Votre code sera testé en l'intégrant à une série de tests automatisés qui seront fournis un peu plus tard.
%N'attendez SURTOUT PAS que ces tests soient envoyés pour commencer à produire vos fonctions et vos propres tests.
%
Votre code sera testé en générant un exécutable ou des bibliothèques avec les commandes suivantes :

\medskip

\begin{tabular}{l}
\texttt{./configure}\\
\texttt{make}\\
\end{tabular}

\bigskip

\noindent Suite à cette étape de génération, les exécutables ou bibliothèques doivent être placés à ces endroits :

\medskip

\begin{tabular}{l}
\TTBF{\RenduDir/libmylpf.a}\\
\TTBF{\RenduDir/libmylpf.so}\\
\end{tabular}

\bigskip

\noindent L'arborescence attendue pour le projet est la suivante :

\medskip

\begin{tabular}{l}
\TTBF{\RenduDir/}\\
\TTBF{\RenduDir/AUTHORS}\\
\TTBF{\RenduDir/README}\\
\TTBF{\RenduDir/Makefile}\\
\TTBF{\RenduDir/configure}\\
%\TTBF{\RenduDir/check/}\\
%\TTBF{\RenduDir/check/check.sh}\\
\TTBF{\RenduDir/src/}\\
%\TTBF{\RenduDir/src/list\_array.c}\\
%\TTBF{\RenduDir/src/list\_array.h}\\
\TTBF{\RenduDir/src/list\_linked.c}\\
\TTBF{\RenduDir/src/list\_linked.h}\\
%\TTBF{\RenduDir/src/queue\_array.c}\\
%\TTBF{\RenduDir/src/queue\_array.h}\\
\TTBF{\RenduDir/src/queue\_linked.c}\\
\TTBF{\RenduDir/src/queue\_linked.h}\\
%\TTBF{\RenduDir/src/stack\_array.c}\\
%\TTBF{\RenduDir/src/stack\_array.h}\\
\TTBF{\RenduDir/src/stack\_linked.c}\\
\TTBF{\RenduDir/src/stack\_linked.h}\\
\end{tabular}


\vspace*{1cm}


\noindent \textit{Vous ne serez jamais pénalisés pour la présence de makefiles ou de fichiers sources (code et/ou headers) dans les différents dossiers du projet tant que leur existence peut être justifiée (des makefiles vides ou jamais utilisés sont néanmoins pénalisés).}

%\noindent \textit{Vous ne serez jamais pénalisés pour la présence de fichiers de différentes natures dans le dossier \texttt{check} tant que leur existence peut être justifiée (des fichiers de test jamais utilisés sont pénalisés).}

\vspace*{1cm}

%Les points de la notation sont distribués ainsi :
%
%\medskip
%
%\begin{itemize}
%\item Exercice 1 - Arbres Binaires : 5 points
%\item Exercice 2 - Bibliothèques statique et dynamique : 4 points
%\item Exercice 3 - Arbres Binaires de Recherche : 5 points
%\item Exercice 4 - Tests  : 6 points
%\end{itemize}