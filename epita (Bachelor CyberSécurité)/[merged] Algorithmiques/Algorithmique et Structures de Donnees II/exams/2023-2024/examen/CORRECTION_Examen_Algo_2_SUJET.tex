\documentclass[11pt,a4paper]{article}
\usepackage[utf8]{inputenc}
\usepackage[french]{babel}
\usepackage[T1]{fontenc}

\usepackage{amsmath}
\usepackage{amsfonts}
\usepackage{amssymb}

\newcommand{\NomAuteur}{Fabrice BOISSIER}
\newcommand{\TitreMatiere}{Algorithmique 2}
\newcommand{\NomUniv}{EPITA - Bachelor Cyber Sécurité}
\newcommand{\NiveauUniv}{CYBER1}
\newcommand{\NumGroupe}{CYBER1}
\newcommand{\AnneeUniv}{2023-2024}
\newcommand{\DateExam}{avril 2024}
%\newcommand{\TypeExam}{Examen - SUJET 2}
\newcommand{\TypeExam}{CORRECTION Examen}
\newcommand{\TitreExam}{\TitreMatiere}
\newcommand{\DureeExam}{2h00}
\newcommand{\MyWaterMark}{\AnneeUniv} % Watermark de protection

% Ajout de mes classes & definitions
\usepackage{MetalExam} % Appelle un .sty

% "Tableau" et pas "Table"
\addto\captionsfrench{\def\tablename{Tableau}}

%%%%%%%%%%%%%%%%%%%%%%%
%Header
%%%%%%%%%%%%%%%%%%%%%%%
\lhead{\TypeExam}							%Gauche Haut
\chead{\NomUniv}							%Centre Haut
\rhead{\NumGroupe}							%Droite Haut
\lfoot{\DateExam}							%Gauche Bas
\cfoot{\thepage{} / \pageref*{LastPage}}	%Centre Bas
\rfoot{\texttt{\TitreMatiere}}				%Droite Bas

%%%%%

\usepackage{tabularx}

\newlength{\LabelWidth}%
%\setlength{\LabelWidth}{1.3in}%
\setlength{\LabelWidth}{1cm}%
%\settowidth{\LabelWidth}{Employee E-mail:}%  Specify the widest text here.

% Optional first parameter here specifies the alignment of
% the text within the \makebox.  Default is [l] for left
% alignment. Other options are [r] and [c] for right and center
\newcommand*{\AdjustSize}[2][l]{\makebox[\LabelWidth][#1]{#2}}%


\definecolor{mGreen}{rgb}{0,0.6,0}
\definecolor{mGray}{rgb}{0.5,0.5,0.5}
\definecolor{mPurple}{rgb}{0.58,0,0.82}
\definecolor{backgroundColour}{rgb}{0.95,0.95,0.92}

\lstdefinestyle{CStyle}{
    backgroundcolor=\color{backgroundColour},
    commentstyle=\color{mGreen},
    keywordstyle=\color{magenta},
    numberstyle=\tiny\color{mGray},
    stringstyle=\color{mPurple},
    basicstyle=\footnotesize,
    breakatwhitespace=false,
    breaklines=true,
    captionpos=b,
    keepspaces=true,
    numbers=left,
    numbersep=5pt,
    showspaces=false,
    showstringspaces=false,
    showtabs=false,
    tabsize=2,
    language=C
}


\hyphenation{op-tical net-works SIGKILL}


\begin{document}

%\MakeExamTitleDuree     % Pour afficher la duree
\MakeExamTitle                   % Ne pas afficher la duree

%% \MakeStudentName    %% A reutiliser sur chaque nouvelle page

\bigskip
%\bigskip

Vous devez respecter les consignes suivantes, sous peine de 0 :

\begin{enumerate}[label=\Roman*)]
\item Lisez le sujet en entier avec attention
\item Répondez sur le sujet
\item Ne détachez pas les agrafes du sujet
\item \'Ecrivez lisiblement vos réponses (si nécessaire en majuscules)
%\item Vous devez écrire dans le langage algorithmique classique ou en C (donc pas de Python ou autre)
\item Ne trichez pas
\end{enumerate}

%\bigskip

\vfillFirst

% Questions cours
\section{Arbres Binaires (10 points)}

\subsection{(3 points) Indiquez toutes les propriétés que possède cet arbre, puis écrivez les clés lors d'un parcours profondeur main gauche de l'arbre dans les 3 ordres ainsi que lors d'un parcours largeur : }

\begin{center}
\begin{tikzpicture}[sibling distance=1.5cm,
  level/.style = {sibling distance = 55mm/#1},
  level 1/.style = {sibling distance = 55mm},
  level 2/.style = {sibling distance = 30mm},
  level 3/.style = {sibling distance = 15mm},
  level 4/.style = {sibling distance = 10mm},
  every node/.style = {minimum width = 2em, draw, circle},
  ]
  \node (n1A) {A}
  child { node (n2R) {R}
          child { node (n3V) {V}
                  child { node (n4E) {E}
                          child { node (n5N) {N} }
                          child { node [draw=none] (nX) {\phantom{X}} edge from parent [draw=none] }
                        }
                  child { node (n6E) {E} }
                }
          child { node (n7N) {N}
                  child { node (n8G) {G}
                          child { node [draw=none] (nY) {\phantom{Y}} edge from parent [draw=none] }
                          child { node (n9O) {O} }
                        }
                  child { node (n10N) {N} }
                }
         }
   child { node (n11O) {O}
           child { node (n12E) {E}
                  child { node (n13I) {I}
                          child { node (n14G) {G} }
                          child { node (n15V) {V} }
                        }
                   child { node (n16Y) {Y} }
                 }
           child { node (n17U) {U}
                   child { node (n18U) {U} }
                   child { node (n19P) {P} }
                 }
        };
\end{tikzpicture}
%  child { node [draw=none] (nX) {\phantom{X}} edge from parent [draw=none] }
\end{center}

\begin{tabular}{C{3cm} C{3cm} C{3cm} C{3cm}}
Arité : 2 & Taille : 19 & Hauteur : 4 & Nb feuilles : 9 \\
\end{tabular}

\medskip

\begin{table}[ht!]
  \centering
  \begin{minipage}{0.50\textwidth}
    \centering

\begin{itemize}
  \item[\CaseCoche] Arbre binaire strict / localement complet \phantom{()}
  \item[\CaseCoche] Arbre binaire parfait \phantom{()}
  \item[\CaseCoche] Peigne gauche \phantom{()}
\end{itemize}

  \end{minipage}
  \hfillx
  \begin{minipage}{0.50\textwidth}
    \centering

\begin{itemize}
%  \item[\checkmark] Arbre binaire (presque) complet \phantom{()}
  \item[\CaseCoche] Arbre binaire (presque) complet \phantom{()}
  \item[\CaseCoche] Arbre filiforme \phantom{()}
  \item[\CaseCoche] Peigne droit \phantom{()}
\end{itemize}

  \end{minipage}
%\stepcounter{figure}
%\caption{Fig.\thefigure : Recherche de la clé 24 dans un ABR}
%\label{fig:example2-BST-succeed-search}
\end{table}

\bigskip

Parcours profondeur :

\medskip

\centerline{
%\begin{tabular}{L{2.5cm} C{0.5cm}C{0.5cm}C{0.5cm}C{0.5cm}C{0.5cm} C{0.5cm}C{0.5cm}C{0.5cm}C{0.5cm} C{0.5cm}C{0.5cm}C{0.5cm}C{0.5cm} C{0.5cm}C{0.5cm}C{0.5cm} C{0.5cm}C{0.5cm}C{0.5cm}}
\begin{tabular}{L{2.5cm} C{0.35cm}C{0.35cm}C{0.35cm}C{0.35cm}C{0.35cm} C{0.35cm}C{0.35cm}C{0.35cm}C{0.35cm} C{0.35cm}C{0.35cm}C{0.35cm}C{0.35cm} C{0.3cm}C{0.35cm}C{0.35cm} C{0.35cm}C{0.35cm}C{0.35cm}}
ordre préfixe : & A & R & V & E & N & E & N & G & O & N & O & E & I & G & V & Y & U & U & P \\
ordre infixe :  & N & E & V & E & R & G & O & N & N & A & G & I & V & E & Y & O & U & U & P \\
ordre suffixe : & N & E & E & V & O & G & N & N & R & G & V & I & Y & E & U & P & U & O & A \\
\end{tabular}
}

\bigskip

Parcours largeur :

\medskip

\centerline{
%\begin{tabular}{L{2.5cm} C{0.5cm}C{0.5cm}C{0.5cm}C{0.5cm}C{0.5cm} C{0.5cm}C{0.5cm}C{0.5cm}C{0.5cm} C{0.5cm}C{0.5cm}C{0.5cm}C{0.5cm} C{0.5cm}C{0.5cm}C{0.5cm} C{0.5cm}C{0.5cm}C{0.5cm}}
\begin{tabular}{L{2.5cm} C{0.35cm}C{0.35cm}C{0.35cm}C{0.35cm}C{0.35cm} C{0.35cm}C{0.35cm}C{0.35cm}C{0.35cm} C{0.35cm}C{0.35cm}C{0.35cm}C{0.35cm} C{0.3cm}C{0.35cm}C{0.35cm} C{0.35cm}C{0.35cm}C{0.35cm}}
ordre :  & A & R & O & V & N & E & U & E & E & G & N & I & Y & U & P & N & O & G & V \\
\end{tabular}
}


\vfillLast

\clearpage

%\vfillFirst

%%%%%%%%%%%%%%%%%%%%%%%%%%%%%%%%%%%%%%%%%%%%%%%

\subsection{(5 points) Dessinez le résultat de l'insertion dans cet ordre précis des éléments suivants dans un ABR (insertion en feuille et en racine) et dans un AVL : }

\begin{center}

%\'Eléments insérés : 18 - 46 - 55 - 36 - 12 - 38 - 96 - 71
%\'Eléments insérés : 46 - 18 - 55 - 36 - 12 - 38 - 96 - 71
%\'Eléments insérés : 32 - 24 - 18 - 30 - 31 - 42 - 36 - 50
\'Eléments insérés : 32 - 24 - 18 - 30 - 31 - 42 - 36

\begin{table}[ht!]
  \centering
  \begin{minipage}{0.33\textwidth}
    \centering

\begin{tikzpicture}[
  level/.style = {sibling distance = 26mm/#1},
  every node/.style = {minimum width = 2em, draw, circle}
  ]
  \node (n32) {32}
  child { node (n24) {24}
          child { node (n18) {18}
                  child { node [draw=none] (nA) {\phantom{A}} edge from parent [draw=none] }
                }
          child { node (n30) {30}
                  child { node [draw=none] (nB) {\phantom{B}} edge from parent [draw=none] }
                  child { node (n31) {31} }
                }
        }
  child { node (n42) {42}
          child { node (n36) {36}
                  child { node [draw=none] (nC) {\phantom{C}} edge from parent [draw=none] }
                }
          child { node [draw=none] (n50) {\phantom{50}} edge from parent [draw=none] }
        };
\end{tikzpicture}

ABR

\textit{insertion en feuille}

(1 point)
  \end{minipage}
  \hfillx
  \begin{minipage}{0.33\textwidth}
    \centering

%\begin{tikzpicture}[
%  level/.style = {sibling distance = 26mm/#1},
%  every node/.style = {minimum width = 2em, draw, circle},
%  level distance = 11mm
%  ]
%  \node (n50) {50}
%  child { node (n36) {36}
%          child { node (n31) {31}
%                  child { node (n30) {30}
%                          child { node (n18) {18}
%                                  child { node [draw=none] (nA) {\phantom{A}} edge from parent [draw=none] }
%                                  child { node (n24) {24} }
%                                }
%                          child { node [draw=none] (nA) {\phantom{A}} edge from parent [draw=none] }
%                        }
%                  child { node (n32) {32} }
%                }
%          child { node (n42) {42} }
%        }
%  child {
%          node [draw=none] (nC) {\phantom{C}} edge from parent [draw=none]
%        };
%\end{tikzpicture}

\begin{tikzpicture}[
  level/.style = {sibling distance = 26mm/#1},
  every node/.style = {minimum width = 2em, draw, circle},
  level distance = 11mm
  ]
  \node (n36) {36}
  child { node (n31) {31}
          child { node (n30) {30}
                  child { node (n18) {18}
                          child { node [draw=none] (nA) {\phantom{A}} edge from parent [draw=none] }
                          child { node (n24) {24} }
                        }
                  child { node [draw=none] (nA) {\phantom{A}} edge from parent [draw=none] }
                }
          child { node (n32) {32} }
        }
  child { node (n42) {42}
        };
\end{tikzpicture}

ABR

\textit{insertion en racine}

(2 points)
  \end{minipage}
  \hfillx
  \begin{minipage}{0.33\textwidth}
    \centering

\begin{tikzpicture}[
  level/.style = {sibling distance = 26mm/#1},
  every node/.style = {minimum width = 2em, draw, circle}
  ]
  \node (n31) {31}
  child { node (n24) {24}
          child { node (n18) {18}
                  child { node [draw=none] (nA) {\phantom{A}} edge from parent [draw=none] }
                }
          child { node (n30) {30}
                  child { node [draw=none] (nB) {\phantom{B}} edge from parent [draw=none] }
                }
        }
  child { node (n36) {36}
          child { node (n32) {32} }
          child { node (n42) {42}
                  child { node [draw=none] (nC) {\phantom{C}} edge from parent [draw=none] }
                  child { node [draw=none] (n60) {\phantom{50}} edge from parent [draw=none] }
                }
        };
\end{tikzpicture}

AVL

(2 points)
  \end{minipage}
%\stepcounter{figure}
%\caption{Fig.\thefigure : Recherche de la clé 24 dans un ABR}
%\label{fig:example2-BST-succeed-search}
\end{table}


%Indiquez combien de rotations droite gauche et gauche droite ont été effectuées dans l'insertion en AVL des précédents éléments (1 point) :
%
%\begin{table}[ht!]
%  \centering
%  \begin{minipage}{0.50\textwidth}
%
%RDG :
%
%  \end{minipage}
%  \hfillx
%  \begin{minipage}{0.50\textwidth}
%
%RGD :
%
%  \end{minipage}
%\end{table}

\end{center}

%%%%%%%%%%%%%%%%%%%%%%%%%%%%%%%%%%%%%%%%%%%%%%%

\subsection{(2 points) \'Ecrivez une fonction récursive \og \textit{parc\_prof\_rec} \fg{} effectuant un parcours profondeur main gauche dans un arbre binaire, et affichant les nœuds dans chacun des ordres : }

%\noindent Pour expliciter les ordres, vous devrez afficher au format : \og Ordre : nœud \fg{} (exemple : \og Préfixe : node \fg{})
\noindent Il faut expliciter les éventuels ordres au format : \og Ordre : nœud \fg{} (exemple : \og Préfixe : 42 \fg{})

\begin{center}
\GrilleReponseN{12}
\end{center}



%\bigskip

%\vfillLast

\clearpage


%%%%%%%%%%%%%%%%%%%%%%%%%%%%%%%%%%%%%%%%%%%%%%%%%%%%%%%%%%%%%%%%%
%%%%%%%%%%%%%%%%%%%%%%%%%%%%%%%%%%%%%%%%%%%%%%%%%%%%%%%%%%%%%%%%%
%%%%%%%%%%%%%%%%%%%%%%%%%%%%%%%%%%%%%%%%%%%%%%%%%%%%%%%%%%%%%%%%%

\section{Micro Projet (10 points)}

\noindent Afin de tester l'ensemble des compétences acquises au cours de cette année, vous allez maintenant toutes les exploiter pour traiter un problème simple.
La question s'intéresse au nombre et à la façon de construire les chemins entre deux nœuds d'un même arbre.
Vous devez mobiliser absolument toutes les connaissances vues en cours pour pouvoir réussir cet exercice.
La macro \textit{sizeof} est autorisée.

\noindent Vous devez écrire les fonctions et procédures suivantes en C.
Aucune fonction de la bibliothèque C ne pourra être utilisée exceptées \textit{malloc} et \textit{free} : vous devez réécrire toute fonction utile, sauf si une consigne précise le contraire.

\medskip

\noindent Un chemin est une liste de nœuds reliés les uns après les autres par des liens.
Dans le cas des arbres, la racine est reliée à ses fils, et vice-versa.

\smallskip

\noindent Par exemple, dans le schéma suivant, il existe un chemin entre B et I constitué de : B - E - I

\begin{center}
\begin{tikzpicture}[
  level/.style = {sibling distance = 26mm/#1},
  every node/.style = {minimum width = 2em, draw, circle}
  ]
  \node (nA) {A}
  child { node (nB) {B}
          child { node (nD) {D}
                  child { node (nH) {H} }
                  child { node [draw=none] (nZ) {\phantom{Z}} edge from parent [draw=none] }
                }
          child { node (nE) {E}
                  child { node [draw=none] (nY) {\phantom{Y}} edge from parent [draw=none] }
                  child { node (nI) {I} }
                }
        }
  child { node (nC) {C}
          child { node (nF) {F} }
          child { node (nG) {G} }
        };
\end{tikzpicture}
\end{center}

\subsection{Questions préalables [2 points]}

\subsubsection{(1,5 point) Indiquez les chemins entre les nœuds suivants : }

\begin{table}[ht!]
  \centering
  \begin{minipage}{0.50\textwidth}
    \centering

%\centerline{
%\begin{tabular}{ | c || C{0.33cm}|C{0.33cm}|C{0.33cm}|C{0.33cm}|C{0.33cm}|C{0.33cm}| }
%\hline
%\multirow[c]{2}{*}[0in]{B $ \rightarrow $ I}  & & & & & & \\
%                                              & & & & & & \\
%\hline
%\hline
%\multirow[c]{2}{*}[0in]{A $ \rightarrow $ B}  & & & & & & \\
%                                              & & & & & & \\
%\hline
%\multirow[c]{2}{*}[0in]{B $ \rightarrow $ A}  & & & & & & \\
%                                              & & & & & & \\
%\hline
%\multirow[c]{2}{*}[0in]{A $ \rightarrow $ I}  & & & & & & \\
%                                              & & & & & & \\
%\hline
%\end{tabular}
%}

\centerline{
\begin{tabular}{ | c || C{0.33cm}|C{0.33cm}|C{0.33cm}|C{0.33cm}|C{0.33cm}|C{0.33cm}| }
\hline
\multirow[c]{2}{*}[0in]{B $ \rightarrow $ I}  & \multirow[c]{2}{*}[0in]{B} & \multirow[c]{2}{*}[0in]{E} & \multirow[c]{2}{*}[0in]{I} & \cellcolor{black!15} & \cellcolor{black!15} & \cellcolor{black!15} \\
                                              & & & & \cellcolor{black!15} & \cellcolor{black!15} & \cellcolor{black!15} \\
\hline
\hline
\multirow[c]{2}{*}[0in]{A $ \rightarrow $ B}  & \multirow[c]{2}{*}[0in]{A} & \multirow[c]{2}{*}[0in]{B} & \cellcolor{black!15} & \cellcolor{black!15} & \cellcolor{black!15} & \cellcolor{black!15} \\
                                              & & & \cellcolor{black!15} & \cellcolor{black!15} & \cellcolor{black!15} & \cellcolor{black!15} \\
\hline
\multirow[c]{2}{*}[0in]{B $ \rightarrow $ A}  & \multirow[c]{2}{*}[0in]{B} & \multirow[c]{2}{*}[0in]{A} & \cellcolor{black!15} & \cellcolor{black!15} & \cellcolor{black!15} & \cellcolor{black!15} \\
                                              & & & \cellcolor{black!15} & \cellcolor{black!15} & \cellcolor{black!15} & \cellcolor{black!15} \\
\hline
\multirow[c]{2}{*}[0in]{A $ \rightarrow $ I}  & \multirow[c]{2}{*}[0in]{A} & \multirow[c]{2}{*}[0in]{B} & \multirow[c]{2}{*}[0in]{E} & \multirow[c]{2}{*}[0in]{I} & \cellcolor{black!15} & \cellcolor{black!15} \\
                                              & & & & & \cellcolor{black!15} & \cellcolor{black!15} \\
\hline
\end{tabular}
}

  \end{minipage}
  \hfillx
  \begin{minipage}{0.50\textwidth}
    \centering

%\centerline{
%\begin{tabular}{ | c || C{0.33cm}|C{0.33cm}|C{0.33cm}|C{0.33cm}|C{0.33cm}|C{0.33cm}| }
%\hline
%\multirow[c]{2}{*}[0in]{D $ \rightarrow $ E}  & & & & & & \\
%                                              & & & & & & \\
%\hline
%\hline
%\multirow[c]{2}{*}[0in]{A $ \rightarrow $ D}  & & & & & & \\
%                                              & & & & & & \\
%\hline
%\multirow[c]{2}{*}[0in]{D $ \rightarrow $ A}  & & & & & & \\
%                                              & & & & & & \\
%\hline
%\multirow[c]{2}{*}[0in]{A $ \rightarrow $ E}  & & & & & & \\
%                                              & & & & & & \\
%\hline
%\end{tabular}
%}

\centerline{
\begin{tabular}{ | c || C{0.33cm}|C{0.33cm}|C{0.33cm}|C{0.33cm}|C{0.33cm}|C{0.33cm}| }
\hline
\multirow[c]{2}{*}[0in]{D $ \rightarrow $ E}  & \multirow[c]{2}{*}[0in]{D} & \multirow[c]{2}{*}[0in]{B} & \multirow[c]{2}{*}[0in]{E} & \cellcolor{black!15} & \cellcolor{black!15} & \cellcolor{black!15} \\
                                              & & & & \cellcolor{black!15} & \cellcolor{black!15} & \cellcolor{black!15} \\
\hline
\hline
\multirow[c]{2}{*}[0in]{A $ \rightarrow $ D}  & \multirow[c]{2}{*}[0in]{A} & \multirow[c]{2}{*}[0in]{B} & \multirow[c]{2}{*}[0in]{D} & \cellcolor{black!15} & \cellcolor{black!15} & \cellcolor{black!15} \\
                                              & & & & \cellcolor{black!15} & \cellcolor{black!15} & \cellcolor{black!15} \\
\hline
\multirow[c]{2}{*}[0in]{D $ \rightarrow $ A}  & \multirow[c]{2}{*}[0in]{D} & \multirow[c]{2}{*}[0in]{B} & \multirow[c]{2}{*}[0in]{A} & \cellcolor{black!15} & \cellcolor{black!15} & \cellcolor{black!15} \\
                                              & & & & \cellcolor{black!15} & \cellcolor{black!15} & \cellcolor{black!15} \\
\hline
\multirow[c]{2}{*}[0in]{A $ \rightarrow $ E}  & \multirow[c]{2}{*}[0in]{A} & \multirow[c]{2}{*}[0in]{B} & \multirow[c]{2}{*}[0in]{E} & \cellcolor{black!15} & \cellcolor{black!15} & \cellcolor{black!15} \\
                                              & & & & \cellcolor{black!15} & \cellcolor{black!15} & \cellcolor{black!15} \\
\hline
\end{tabular}
}

  \end{minipage}
\end{table}


%\centerline{
%\begin{tabular}{ | c || C{0.33cm}|C{0.33cm}|C{0.33cm}|C{0.33cm}|C{0.33cm}|C{0.33cm}| }
%\hline
%\multirow[c]{2}{*}[0in]{I $ \rightarrow $ F}  & & & & & & \\
%                                              & & & & & & \\
%\hline
%\hline
%\multirow[c]{2}{*}[0in]{A $ \rightarrow $ I}  & & & & & & \\
%                                              & & & & & & \\
%\hline
%\multirow[c]{2}{*}[0in]{I $ \rightarrow $ A}  & & & & & & \\
%                                              & & & & & & \\
%\hline
%\multirow[c]{2}{*}[0in]{A $ \rightarrow $ F}  & & & & & & \\
%                                              & & & & & & \\
%\hline
%\end{tabular}
%}

\centerline{
\begin{tabular}{ | c || C{0.33cm}|C{0.33cm}|C{0.33cm}|C{0.33cm}|C{0.33cm}|C{0.33cm}| }
\hline
\multirow[c]{2}{*}[0in]{I $ \rightarrow $ F}  & \multirow[c]{2}{*}[0in]{I} & \multirow[c]{2}{*}[0in]{E} & \multirow[c]{2}{*}[0in]{B} & \multirow[c]{2}{*}[0in]{A} & \multirow[c]{2}{*}[0in]{C} & \multirow[c]{2}{*}[0in]{F} \\
                                              & & & & & & \\
\hline
\hline
\multirow[c]{2}{*}[0in]{A $ \rightarrow $ I}  & \multirow[c]{2}{*}[0in]{A} & \multirow[c]{2}{*}[0in]{B} & \multirow[c]{2}{*}[0in]{E} & \multirow[c]{2}{*}[0in]{I} & \cellcolor{black!15} & \cellcolor{black!15} \\
                                              & & & & & \cellcolor{black!15} & \cellcolor{black!15} \\
\hline
\multirow[c]{2}{*}[0in]{I $ \rightarrow $ A}  & \multirow[c]{2}{*}[0in]{I} & \multirow[c]{2}{*}[0in]{E} & \multirow[c]{2}{*}[0in]{B} & \multirow[c]{2}{*}[0in]{A} & \cellcolor{black!15} & \cellcolor{black!15} \\
                                              & & & & & \cellcolor{black!15} & \cellcolor{black!15} \\
\hline
\multirow[c]{2}{*}[0in]{A $ \rightarrow $ F}  & \multirow[c]{2}{*}[0in]{A} & \multirow[c]{2}{*}[0in]{C} & \multirow[c]{2}{*}[0in]{F} & \cellcolor{black!15} & \cellcolor{black!15} & \cellcolor{black!15} \\
                                              & & & & \cellcolor{black!15} & \cellcolor{black!15} & \cellcolor{black!15} \\
\hline
\end{tabular}
}

\subsubsection{(0,5 point) Combien existe-t-il de chemin(s) le(s) plus court(s) entre deux nœuds quelconques d'un arbre ?}

1 : Il n'existe qu'un seul chemin entre deux nœuds quelconques d'un arbre.


%\subsubsection{De quel algorithme récursif peut-on dériver la recherche d'un chemin dans un arbre entre la racine et un nœud ?}
%
%Calcul de la hauteur d'un arbre / Profondeur d'un nœud


%\bigskip
\clearpage
%%%%%%%%%%%%%%%%%%%%%%%%%%%%%%%%%%%%%%%%%%%%%%%


\subsection{Résolution générale du problème [2 points]}

\noindent Dans les exercices suivants, nous considèrerons les arbres binaires comme cette structure \TTBF{node} :

\begin{center}
% %*   *)
\begin{lstlisting}[language=C,commentstyle=\color{teal}\ttfamily,morekeywords={node}]
typedef struct node
{
  int         key;
  struct node *lc;  // Left child
  struct node *rc;  // Right child
} node;
\end{lstlisting}
\end{center}

\noindent En admettant que vous disposez des fonctions suivantes exclusivement pour cette question, écrivez l'algorithme général de résolution du problème sous la forme d'une fonction C.
(Les tableaux de node* sont tous \textit{NULL-terminated}).

\medskip

\noindent \textit{Rappel : un tableau NULL-terminated est un tableau dont la dernière case contient la valeur NULL.}

\medskip

\setlength\fboxrule{0.2pt}
\noindent Un tableau NULL-terminated contenant A, B, et C sera de la forme suivante : \fbox{A}\fbox{B}\fbox{C}\fbox{\textit{NULL}}
\setlength\fboxrule{0.4pt}

\begin{center}
% %*   *)
\begin{lstlisting}[language=C,commentstyle=\color{teal}\ttfamily,morekeywords={node}]
// Longueur du chemin (nombre de noeuds) entre la racine et un noeud
//      (si le chemin n'existe pas, la fonction renvoie -1)
int count_path_root_to_node(node *root, node *end);

// Generation du tableau (NULL-terminated) de node* contenant le
//   chemin entre la racine et un noeud
node **build_path_root_to_node(node *root, node *end);

// Longueur d'un tableau de node*
int array_length(node **in_array);

// Inversion en place d'un tableau de node*
//   (en place = sans le reallouer)
void invert_array_in_place(node **in_array);

// Longueur du prefixe commun entre 2 tableaux
int common_prefix_length(node **tab1, node **tab2);

// Fusion de 2 tableaux de node* vers un nouveau tableau
node **merge_arrays_new(node **tab1, node **tab2);
\end{lstlisting}
\end{center}


\noindent Vous pouvez également appeler les fonctions classiques des arbres : \textit{hauteur(arbre)}, \textit{profondeur(nœud)}, \textit{taille(arbre)}, \textit{parc\_prof(arbre)}, ... mais vous devrez les réimplémenter si nécessaire.


%\bigskip
\clearpage


\subsubsection{(2 points) \'Ecrivez la fonction produisant un tableau \textit{NULL-terminated} du chemin entre deux nœuds quelconques d'un arbre binaire}

\begin{center}
% %*   *)
\begin{lstlisting}[language=C,morekeywords={node}]
node **build_path(node *root, node *start, node *end)
{
  node **root_path1, **root_path2;
  node **path1, **path2;
  node **final_path;
  int prefix_end;

  root_path1 = build_path_root_to_node(root, start);
  root_path2 = build_path_root_to_node(root, end);

  prefix_end = common_prefix_next_index(root_path1, root_path2);
  path1 = &root_path1[prefix_end];
  path2 = &root_path2[prefix_end - 1];

  path1 = invert_array_in_place(path1);

  final_path = merge_arrays_copy(path1, path2);

  free(root_path1)
  free(root_path2);

  return (final_path);
}
\end{lstlisting}
\end{center}


%\bigskip
\clearpage
%%%%%%%%%%%%%%%%%%%%%%%%%%%%%%%%%%%%%%%%%%%%%%%


\subsection{Génération des tableaux de chemins [2 points]}

%\subsubsection{Fonction calculant la longueur du chemin de la racine vers un nœud d'un arbre binaire}
%
%\subsubsection{Fonction produisant un tableau \textit{NULL-terminated} du chemin de la racine vers un nœud d'un arbre binaire}


\subsubsection{(2 points) \'Ecrivez une fonction comptant le nombre de nœuds sur le chemin entre la racine d'un arbre et un nœud. Si aucun chemin n'existe vers ce nœud, vous devez renvoyer -1 : }

\noindent \textit{Exemple : entre le nœud B et I du schéma illustratif en page 3, il y a 3 nœuds : B - E - I.}

\begin{center}
% %*   *)
\begin{lstlisting}[language=C]
int count_path_nodes_rec(node *root, node *end, int level)
{
  int left, right;

  if (root == NULL)
    return (-1);

  if (root == end)
    return (level);

  left = count_path_nodes_rec(root->lc, end, level + 1);
  right = count_path_nodes_rec(root->rc, end, level + 1);

  if (left > right)
    return (left);
  else
    return (right);
}
\end{lstlisting}

% %*   *)
\begin{lstlisting}[language=C]
int count_path_nodes(node *root, node *end)
{
  return (count_path_nodes_rec(root, end, 1));
}
\end{lstlisting}
\end{center}


%\bigskip
\clearpage
%%%%%%%%%%%%%%%%%%%%%%%%%%%%%%%%%%%%%%%%%%%%%%%


%\subsection{(2 points) \'Ecrivez une fonction remplissant un tableau \textit{NULL-terminated} de node* avec l'adresse de chaque nœud d'un chemin entre une racine et un nœud parmi ses sous-arbres. Si aucun chemin n'existe, renvoyez simplement NULL. Considérez que le tableau est déjà alloué avec suffisament de place : }

%%\noindent \textit{Rappel : un tableau NULL-terminated est un tableau dont la dernière case contient la valeur NULL, '\textbackslash{}0', ou 0.}
%\noindent \textit{Rappel : un tableau NULL-terminated est un tableau dont la dernière case contient la valeur NULL.}

%\begin{center}
%% %*   *)
%\begin{lstlisting}[language=C]
%int fill_path_nodes_rec(node *root, node *end, int level, node **tab)
%{
%  int left, right;
%
%  if (root == NULL)
%    return (-1);
%
%  if (root == end)
%    return (level);
%
%  left = fill_path_nodes_rec(root->lc, end, level + 1, tab);
%  right = fill_path_nodes_rec(root->rc, end, level + 1, tab);
%
%  if ((left > 0) || (right > 0))
%    tab[level - 1] = root;
%
%  if (left > right)
%    return (left);
%  else
%    return (right);
%}
%\end{lstlisting}

%% %*   *)
%\begin{lstlisting}[language=C]
%node **fill_path_nodes(node *start, node *end, node **tab)
%{
%  int level;
%
%  level = fill_path_nodes_rec(start, end, 1, tab);
%
%  if (level == -1)
%    return (NULL);
%
%  tab[level] = NULL;
%  return (tab);
%}
%\end{lstlisting}
%\end{center}


%\bigskip
%\clearpage
%%%%%%%%%%%%%%%%%%%%%%%%%%%%%%%%%%%%%%%%%%%%%%%

\subsection{Gestion des tableaux [4 points]}

%\subsubsection{Fonction calculant la taille des préfixes communs de deux tableaux \textit{NULL-terminated}}
%
%\subsubsection{Fonction inversant un tableau \textit{NULL-terminated} en place}
%
%\subsubsection{Fonction fusionnant deux tableaux \textit{NULL-terminated}}


%\subsubsection{(1 point) \'Ecrivez une fonction calculant la taille du préfixe commun entre deux tableaux \textit{NULL-terminated}. Si les deux tableaux sont des pointeurs NULL, renvoyez -1 : }
%
%\setlength\fboxrule{0.2pt}
%\noindent Exemple : Un tableau contenant \fbox{A}\fbox{B}\fbox{C}\fbox{\textit{NULL}} et un tableau contenant \fbox{A}\fbox{B}\fbox{D}\fbox{\textit{NULL}} auront un préfixe commun de taille 2 (A et B).
%\setlength\fboxrule{0.4pt}
%
%\smallskip
%
%\noindent \textit{Rappel : un tableau NULL-terminated est un tableau dont la dernière case contient la valeur NULL.}
%
%\begin{center}
%% %*   *)
%\begin{lstlisting}[language=C]
%int common_prefix_length(node **tab1, node **tab2)
%{
%  int index = 0;
%
%  if ((tab1 == NULL) && (tab2 == NULL))
%    return (-1);
%
%  if ((tab1 == NULL) || (tab2 == NULL))
%    return (0);
%
%  while (tab1[index] == tab2[index]) &&
%        (tab1[index] != NULL) && (tab2[index] != NULL)
%    index++;
%
%  return (index);
%}
%\end{lstlisting}
%\end{center}


%%\bigskip
%\clearpage
%%%%%%%%%%%%%%%%%%%%%%%%%%%%%%%%%%%%%%%%%%%%%%%


\subsubsection{(2 points) \'Ecrivez une fonction inversant l'ordre des éléments d'un tableau \textit{NULL-terminated} contenant des node*. Si le tableau est un pointeur NULL, renvoyez NULL. L'inversion doit se faire en place, c'est-à-dire qu'il ne faut pas allouer de nouveau tableau : }

\setlength\fboxrule{0.2pt}
\noindent Exemple : Un tableau contenant \fbox{A}\fbox{B}\fbox{C}\fbox{\textit{NULL}} sera inversé en \fbox{C}\fbox{B}\fbox{A}\fbox{\textit{NULL}}.
\setlength\fboxrule{0.4pt}

%\smallskip
%
%\noindent \textit{Rappel : un tableau NULL-terminated est un tableau dont la dernière case contient la valeur NULL.}

\begin{center}
% %*   *)
\begin{lstlisting}[language=C]
void invert_array_in_place(node **in_tab)
{
  node *tmp;
  int max = 0;
  int i = 0;

  if (in_tab != NULL)
  {
    while (in_tab[max] != NULL)
      max++;

    while (i < max)
    {
      tmp = in_tab[i];
      in_tab[i] = in_tab[max];
      in_tab[max] = tmp;
      i++;
      max--;
    }
  }
}
\end{lstlisting}
\end{center}


%\bigskip
\clearpage
%%%%%%%%%%%%%%%%%%%%%%%%%%%%%%%%%%%%%%%%%%%%%%%


\subsubsection{(2 points) \'Ecrivez une fonction prenant en paramètres deux tableaux \textit{NULL-terminated} de node* et les fusionnant en un seul tableau \textit{NULL-terminated}. Le tableau de sortie doit être un nouveau tableau alloué. Si les tableaux sont des pointeurs NULL, renvoyez NULL : }

\begin{center}
% %*   *)
\begin{lstlisting}[language=C]
int tab_len(node *tab)
{
  int i = 0;

  if (tab == NULL)
    return (0);

  while (tab[i] != NULL)
    i++;
  return (i);
}
\end{lstlisting}

% %*   *)
\begin{lstlisting}[language=C]
node **merge_arrays(node **tab1, node **tab2)
{
  node **out = NULL;
  int len1, len2, i = 0;

  if ((tab1 == NULL) && (tab2 == NULL))
    return (NULL);

  len1 = tab_len(tab1);
  len2 = tab_len(tab2);
  out = malloc((len1 + len2 + 1) * sizeof (node*));
  while (i < len1)
  {
    out[i] = tab1[i];
    i++;
  }

  i = 0;
  while (i < len2)
  {
    out[len1 + i] = tab2[i];
    i++;
  }

  out[len1 + len2] = NULL;

  return (out);
}
\end{lstlisting}
\end{center}


\end{document}
