\documentclass[11pt,a4paper]{article}
\usepackage[utf8]{inputenc}
\usepackage[french]{babel}
\usepackage[T1]{fontenc}

\usepackage{amsmath}
\usepackage{amsfonts}
\usepackage{amssymb}

\newcommand{\NomAuteur}{Fabrice BOISSIER}
\newcommand{\TitreMatiere}{Algo et Structure de Données 1}
\newcommand{\NomUniv}{EPITA - Bachelor Cyber Sécurité}
\newcommand{\NiveauUniv}{CYBER1}
\newcommand{\NumGroupe}{CYBER1}
\newcommand{\AnneeUniv}{2022-2023}
\newcommand{\DateExam}{Novembre 2022}
\newcommand{\TypeExam}{QCM 2}
\newcommand{\TitreExam}{\TitreMatiere}
\newcommand{\DureeExam}{20 min}
\newcommand{\MyWaterMark}{\AnneeUniv} % Watermark de protection

% Ajout de mes classes & definitions
\usepackage{MetalExam} % Appelle un .sty

% "Tableau" et pas "Table"
\addto\captionsfrench{\def\tablename{Tableau}}

%%%%%%%%%%%%%%%%%%%%%%%
%Header
%%%%%%%%%%%%%%%%%%%%%%%
\lhead{\TypeExam}							%Gauche Haut
\chead{\NomUniv}							%Centre Haut
\rhead{\NumGroupe}							%Droite Haut
\lfoot{\DateExam}							%Gauche Bas
\cfoot{\thepage{} / \pageref*{LastPage}}	%Centre Bas
\rfoot{\texttt{\TitreMatiere}}				%Droite Bas

%%%%%

\usepackage{tabularx}

\newlength{\LabelWidth}%
%\setlength{\LabelWidth}{1.3in}%
\setlength{\LabelWidth}{1cm}%
%\settowidth{\LabelWidth}{Employee E-mail:}%  Specify the widest text here.

% Optional first parameter here specifies the alignment of
% the text within the \makebox.  Default is [l] for left
% alignment. Other options are [r] and [c] for right and center
\newcommand*{\AdjustSize}[2][l]{\makebox[\LabelWidth][#1]{#2}}%


\definecolor{mGreen}{rgb}{0,0.6,0}
\definecolor{mGray}{rgb}{0.5,0.5,0.5}
\definecolor{mPurple}{rgb}{0.58,0,0.82}
\definecolor{backgroundColour}{rgb}{0.95,0.95,0.92}

\lstdefinestyle{CStyle}{
    backgroundcolor=\color{backgroundColour},
    commentstyle=\color{mGreen},
    keywordstyle=\color{magenta},
    numberstyle=\tiny\color{mGray},
    stringstyle=\color{mPurple},
    basicstyle=\footnotesize,
    breakatwhitespace=false,
    breaklines=true,
    captionpos=b,
    keepspaces=true,
    numbers=left,
    numbersep=5pt,
    showspaces=false,
    showstringspaces=false,
    showtabs=false,
    tabsize=2,
    language=C
}


\hyphenation{op-tical net-works SIGKILL}


\begin{document}

% \MakeExamTitleDuree     % Pour afficher la duree
\MakeExamTitle                   % Ne pas afficher la duree

%% \MakeStudentName    %% A reutiliser sur chaque nouvelle page


% \setcounter{section}{1}  % Ne pas faire une liste de 0.1, 0.2, ... mais 1.1, 1.2, ...

\renewcommand{\thesubsection}{\arabic{subsection}} % Subsection sans 0.x, mais juste x

% \newcommand{\thesubsection}{\thesection.\arabic{subsection}} % Subsection avec numero section (0.x)


\vfillFirst

\subsection{(4 points) Cochez la (ou les) affirmation(s) vraie(s) : }

\begin{itemize}
  \item[\CaseCoche] Les piles sont des structures LOFI \\
  \item[\checkmark] Les piles sont des structures LIFO \\
  \item[\CaseCoche] Les files sont des structures FOFI \\
  \item[\checkmark] Les files sont des structures FIFO \\
  \item[\CaseCoche] Les listes sont des structures POPI \\
  \item[\CaseCoche] Les listes sont des structures PIPO \\
\end{itemize}


\bigskip
\vspace*{1cm}

\subsection{(4 points) Cocher la (ou les) structure(s) pouvant implémenter une file : }


\begin{table}[ht!]
  \centering
  \begin{minipage}{0.28\textwidth}
    \centering
% %*   *)
\begin{lstlisting}[style=algorithmique]
struct A
  struct A next
  entier nb_elt
  entier head
  entier tail
fin struct \end{lstlisting}
    % \caption{Algorithme de la somme des N premiers entiers}
    % \label{algo-somme-n-premiers-entiers}
  \CaseCoche
  \end{minipage}
  \hfillx
  \begin{minipage}{0.28\textwidth}
    \centering
% %*   *)
\begin{lstlisting}[style=algorithmique]
struct B
  struct B *next
  entier head
fin struct \end{lstlisting}
    % \caption{Algorithme de la somme des N premiers entiers}
    % \label{algo-somme-n-premiers-entiers}
  \checkmark
  \end{minipage}
  \hfillx
  \begin{minipage}{0.28\textwidth}
    \centering
% %*   *)
\begin{lstlisting}[style=algorithmique]
struct C
  entier[] tab
  entier head
  entier tail
fin struct \end{lstlisting}
    % \caption{Algorithme de la somme des N premiers entiers}
    % \label{algo-somme-n-premiers-entiers}
  \checkmark
  \end{minipage}
%  \caption{Algorithme de la somme des N premiers entiers}
%  \label{somme-n-premiers-entiers}
\end{table}


\bigskip
\vspace*{1cm}


\subsection{(4 points) On peut empiler dans l'ordre 1, 2, 3, 4, 5, 6 et les dépiler dans cet (ou ces) ordre(s) : }


\begin{table}[ht!]
  \centering
  \begin{minipage}{0.45\textwidth}
    \centering
\begin{itemize}
  \item[\checkmark] 1, 2, 3, 4, 5, 6
  \item[\checkmark] 6, 5, 4, 3, 2, 1
  \item[\CaseCoche] 6, 5, 3, 1, 2, 4
\end{itemize}
  \end{minipage}
  \hfillx
  \begin{minipage}{0.45\textwidth}
    \centering
\begin{itemize}
  \item[\checkmark] 3, 2, 4, 1, 5, 6
  \item[\checkmark] 2, 3, 1, 5, 4, 6
  \item[\CaseCoche] 4, 3, 1, 2, 5, 6
\end{itemize}
  \end{minipage}
%  \caption{Algorithme de la somme des N premiers entiers}
%  \label{somme-n-premiers-entiers}
\end{table}


\vfillLast

%\bigskip
\newpage

\vfillFirst


\subsection{(4 points) On peut accéder à un entier en utilisant cette (ou ces) expression(s) : }

\begin{center}
% %*   *)
\begin{lstlisting}[style=algorithmique]
struct MyStruct1
  entier elt
  struct MyStruct1 *next
fin struct 

struct MyStruct1 var1
struct MyStruct1 *var2 \end{lstlisting}
\end{center}



\begin{table}[ht!]
  \centering
  \begin{minipage}{0.45\textwidth}
    \centering
\begin{itemize}
  \item[\checkmark] var1.elt    \\
  \item[\CaseCoche] (*var1).elt \\
  \item[\CaseCoche] (*var1.elt) \\
\end{itemize}
  \end{minipage}
  \hfillx
  \begin{minipage}{0.45\textwidth}
    \centering
\begin{itemize}
  \item[\CaseCoche] var2.elt    \\
  \item[\checkmark] (*var2).elt \\
  \item[\CaseCoche] (*var2.elt) \\
\end{itemize}
  \end{minipage}
%  \caption{Algorithme de la somme des N premiers entiers}
%  \label{somme-n-premiers-entiers}
\end{table}

% DIFFERENCE ENTRE (*ptr).next, (*(*ptr).next), (*ptr.next), ptr.next

\bigskip
%\vspace*{1cm}

\subsection{(4 points) Qu'affichera le programme suivant ? }

\begin{lstlisting}[style=algorithmique]
algorithme fonction CalculPointeurs2 : entier
  parametres locaux
    entier    i, j
  variables
    entier    *ptr1, **ptr2
debut
  i = 42
  j = 1337
  ptr1 = &j
  ptr2 = &ptr1
  (*ptr1) = 3615
  ecrire((**ptr2) + i)
fin algorithme fonction CalculPointeurs2
\end{lstlisting}

\begin{table}[ht!]
  \centering
  \begin{minipage}{0.45\textwidth}
    \centering
\begin{itemize}
  \item[\CaseCoche] 42 \\  
  \item[\CaseCoche] 1337 \\
  \item[\CaseCoche] 1379 \\
\end{itemize}
  \end{minipage}
  \hfillx
  \begin{minipage}{0.45\textwidth}
    \centering
\begin{itemize}
  \item[\CaseCoche] 3615 \\
  \item[\checkmark] 3657 \\  
  \item[\CaseCoche] Rien, le programme va crasher \\
\end{itemize}
  \end{minipage}
%  \caption{Algorithme de la somme des N premiers entiers}
%  \label{somme-n-premiers-entiers}
\end{table}


\bigskip
%\vspace*{1cm}

\subsection{[BONUS] (0 point) Le fossile nautile dans Pokémon, est connu pour : }


\begin{table}[ht!]
  \centering
  \begin{minipage}{0.40\textwidth}
    \centering
\begin{itemize}
  \item[\checkmark] obtenir Amonita \\
  \item[\CaseCoche] obtenir Kabuto \\
  \item[\CaseCoche] obtenir Ptéra \\
\end{itemize}
  \end{minipage}
  \hfillx
  \begin{minipage}{0.50\textwidth}
    \centering
\begin{itemize}
  \item[\checkmark] avoir généré des memes \\
  \item[\checkmark] être l'objet d'un culte quasi-religieux \\
  \item[\CaseCoche] être le sous-marin du livre de Jules Vernes \\
\end{itemize}
  \end{minipage}
%  \caption{Algorithme de la somme des N premiers entiers}
%  \label{somme-n-premiers-entiers}
\end{table}

\vfillLast

\end{document}
