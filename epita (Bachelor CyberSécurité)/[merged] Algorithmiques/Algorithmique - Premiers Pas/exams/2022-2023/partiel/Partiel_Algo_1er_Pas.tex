\documentclass[11pt,a4paper]{article}
\usepackage[utf8]{inputenc}
\usepackage[french]{babel}
\usepackage[T1]{fontenc}

\usepackage{amsmath}
\usepackage{amsfonts}
\usepackage{amssymb}

\newcommand{\NomAuteur}{Fabrice BOISSIER}
\newcommand{\TitreMatiere}{Algorithmique - Premiers Pas}
\newcommand{\NomUniv}{EPITA - Bachelor Cyber Sécurité}
\newcommand{\NiveauUniv}{CYBER1}
\newcommand{\NumGroupe}{CYBER1}
\newcommand{\AnneeUniv}{2022-2023}
\newcommand{\DateExam}{Janvier 2023}
\newcommand{\TypeExam}{Partiel (Sujet 1)}
\newcommand{\TitreExam}{\TitreMatiere}
\newcommand{\DureeExam}{2h00}
\newcommand{\MyWaterMark}{\AnneeUniv} % Watermark de protection

% Ajout de mes classes & definitions
\usepackage{MetalExam} % Appelle un .sty

% "Tableau" et pas "Table"
\addto\captionsfrench{\def\tablename{Tableau}}

%%%%%%%%%%%%%%%%%%%%%%%
%Header
%%%%%%%%%%%%%%%%%%%%%%%
\lhead{\TypeExam}							%Gauche Haut
\chead{\NomUniv}							%Centre Haut
\rhead{\NumGroupe}							%Droite Haut
\lfoot{\DateExam}							%Gauche Bas
\cfoot{\thepage{} / \pageref*{LastPage}}	%Centre Bas
\rfoot{\texttt{\TitreMatiere}}				%Droite Bas

%%%%%

\usepackage{tabularx}

\newlength{\LabelWidth}%
%\setlength{\LabelWidth}{1.3in}%
\setlength{\LabelWidth}{1cm}%
%\settowidth{\LabelWidth}{Employee E-mail:}%  Specify the widest text here.

% Optional first parameter here specifies the alignment of
% the text within the \makebox.  Default is [l] for left
% alignment. Other options are [r] and [c] for right and center
\newcommand*{\AdjustSize}[2][l]{\makebox[\LabelWidth][#1]{#2}}%


\definecolor{mGreen}{rgb}{0,0.6,0}
\definecolor{mGray}{rgb}{0.5,0.5,0.5}
\definecolor{mPurple}{rgb}{0.58,0,0.82}
\definecolor{backgroundColour}{rgb}{0.95,0.95,0.92}

\lstdefinestyle{CStyle}{
    backgroundcolor=\color{backgroundColour},
    commentstyle=\color{mGreen},
    keywordstyle=\color{magenta},
    numberstyle=\tiny\color{mGray},
    stringstyle=\color{mPurple},
    basicstyle=\footnotesize,
    breakatwhitespace=false,
    breaklines=true,
    captionpos=b,
    keepspaces=true,
    numbers=left,
    numbersep=5pt,
    showspaces=false,
    showstringspaces=false,
    showtabs=false,
    tabsize=2,
    language=C
}


\hyphenation{op-tical net-works SIGKILL}


\begin{document}

%\MakeExamTitleDuree     % Pour afficher la duree
\MakeExamTitle                   % Ne pas afficher la duree

%% \MakeStudentName    %% A reutiliser sur chaque nouvelle page

\bigskip
%\bigskip

Vous devez respecter les consignes suivantes, sous peine de 0 :

\begin{itemize}
\item Lisez le sujet en entier avec attention
\item Répondez sur le sujet
\item Ne détachez pas les agrafes du sujet
\item \'Ecrivez lisiblement vos réponses (si nécessaire en majuscules)
\item Vous devez écrire dans le langage algorithmique classique ou en C (donc pas de Python ou autre)
\item Ne trichez pas
\end{itemize}

%\bigskip


% Questions cours
\section{Questions (6 points)}

\subsection{(2 points) Sélectionnez les conditions vraies pour A = 6 et B = 9 : }

% Compter pour chaque réponse 0,5 points : si bonne ou pas bonne
\bigskip

\begin{itemize}
  \item[\CaseCoche] ((A <= B) et (non (B == A)) et (A - 2 > B - 5)) ou ((A != B) et (B - 1 == A + 3)) \\ % Faux
  \item[\CaseCoche] ((non (A > B)) et (A != B) et (non (A == B)) ou ((B != A + 3) et (B - 3 == A)) \\ % Vrai
  \item[\CaseCoche] (non ((A > B) et (B > A))) et ((B == A + 3) et (A == B - 3)) \\ % Vrai
  \item[\CaseCoche] (non ((A >= B - 4) ou (B >= A + 3))) ou ((B <= A + 4) et (A <= B - 3)) \\ % Vrai
\end{itemize}

%\bigskip

\subsection{(1 point) Lors de la déclaration d'une fonction/procédure et dans son implémentation, il faut déclarer : }

\bigskip

\begin{itemize}
  \item[\CaseCoche] Les paramètres d'entrée
  \item[\CaseCoche] Le nom des variables locales utilisées
  \item[\CaseCoche] Le type des variables locales utilisées
  \item[\CaseCoche] Le type de la valeur de retour
  \item[\CaseCoche] Les instructions à exécuter
\end{itemize}

%\bigskip

\subsection{(1 point) Rappelez les trois types de base/Donnez un exemple pour chaque type de base : }

\bigskip

\begin{itemize}
  \item[\CaseCoche] Entier/Int : 42
  \item[\CaseCoche] Flottant/Float : 3.14
  \item[\CaseCoche] Caractère/Char : 'a'
\end{itemize}

%\bigskip
\newpage
\vfillFirst

% x = 10 ou 11   y = 4 ou 5
\subsection{(2 points) Exécutez l'algorithme suivant, et écrivez l'évolution des variables pour \textit{x = 10} et \textit{y = 4} }

%\bigskip

\begin{table}[h!]
  \centering
  \begin{minipage}{0.59\textwidth}
    \centering
% %*   *)
% Modulo : %*\texttt{\%}*)
% Recoit : %*$\leftarrow$*)
\begin{lstlisting}[style=algorithmique]
algorithme fonction CalculXYZ2 : entier
  parametres locaux
    entier    x, y
  variables
    entier    i, j
debut
i %*$\leftarrow$*) 0
j %*$\leftarrow$*) 0
tant que (x > 1)
  si ((x %*\texttt{\%}*) 2) == 1)
    i %*$\leftarrow$*) x - 3
    j %*$\leftarrow$*) j + y
    y %*$\leftarrow$*) y - 1
  sinon
    i %*$\leftarrow$*) x - 1
    j %*$\leftarrow$*) j - y
    y %*$\leftarrow$*) y / 2
  fin si
  x %*$\leftarrow$*) ((x + i) / 2)
fin tant que
retourner (i + j + x + y)
fin algorithme fonction CalculXYZ2 \end{lstlisting}
%    \caption{Algorithme de la somme des N premiers entiers}
%    \label{algo-somme-n-premiers-entiers}
  \end{minipage}
  \hfillx
  \begin{minipage}{0.4\textwidth}
    \centering
    \begin{tabular}{| C{1cm} | C{1cm} | C{1cm} | C{1cm} |}
        \hline
          x  &  y  &  i  &  j    \\
        \hline
             &     &     &       \\
             &     &     &       \\
             &     &     &       \\
        \hline
             &     &     &       \\
             &     &     &       \\
             &     &     &       \\
        \hline
             &     &     &       \\
             &     &     &       \\
             &     &     &       \\
        \hline
             &     &     &       \\
             &     &     &       \\
             &     &     &       \\
        \hline
             &     &     &       \\
             &     &     &       \\
             &     &     &       \\
        \hline
             &     &     &       \\
             &     &     &       \\
             &     &     &       \\
        \hline
             &     &     &       \\
             &     &     &       \\
             &     &     &       \\
        \hline
    \end{tabular}
%    \caption{Tableau d'exécution pas à pas}
%    \label{table-somme-n-premiers-entiers-execution}
  \end{minipage}
%  \caption{Algorithme de la somme des N premiers entiers}
%  \label{somme-n-premiers-entiers}
\end{table}


\vfillLast
\newpage
%\vfillFirst

\section{Algorithmes (14 points)}

\subsection{(2 points) \'Ecrivez une procédure \og \textit{AfficheMul2} \fg{} itérative/récursive affichant les N premiers entiers multiples de 2 (0 inclus). }

%\smallskip
\bigskip

\begin{center}
\GrilleReponseN{10}
\end{center}

\smallskip
%\medskip
%\bigskip


%\smallskip

%\vfillLast
\newpage

\vfillFirst

\subsection{(2 points) \'Ecrivez une fonction \og \textit{SearchElt} \fg{} itérative/récursive cherchant un élément dans un tableau et renvoyant l'index de la case contenant l'élément s'il est trouvé. Si l'élément n'est pas trouvé, la fonction renverra \textit{-1}. }

%\smallskip
\bigskip

\begin{center}
\GrilleReponseN{10}
\end{center}

\bigskip

\subsection{(2 points) \'Ecrivez une fonction récursive terminale \og \textit{Fibonacci} \fg{} calculant le $n^{ème}$ terme de la suite de Fibonacci / suite géométrique }

Fibo :
$ fibo(0) = fibo(1) = 1 $
$ fibo(n) = fibo(n − 1) + fibo(n − 2) $


Géom :
$ u_{n} = u_{0} × q^{n} $

%\smallskip
\bigskip

\begin{center}
\GrilleReponseN{10}
\end{center}




\subsection{(2 points) \'Ecrivez une fonction \og \textit{Miroir} \fg{} itérative renvoyant un entier qui est le miroir de l'entier donné en entrée (afficher ne suffit pas) / récursive affichant le miroir de l'entier donné en paramètre. }

\bigskip

\begin{center}
\GrilleReponseN{20}
\end{center}

\bigskip

\subsection{(2 points) \'Ecrivez une fonction \og \textit{strlen} \fg{} itérative/récursive renvoyant la taille d'une chaîne caractères. }

%\smallskip
\bigskip

\begin{center}
\GrilleReponseN{10}
\end{center}

\bigskip

\subsection{(2 points) \'Ecrivez une fonction \og \textit{SuffixStrIter} \fg{} / \og \textit{PrefixStrIter} \fg{} itérative vérifiant si une chaîne de caractères est bien un suffixe/préfixe d'une autre chaîne de carcatères. }

\bigskip

\begin{center}
\GrilleReponseN{20}
\end{center}


%\bigskip

\vfillLast
\newpage

\subsection{(4 points) \'Ecrivez une procédure respectant l'algorithme de tri par insertion/fusion \og \textit{TriInsertion/Fusion} \fg{} qui permet de trier en place un tableau. }

\bigskip

\begin{center}
\GrilleReponseN{22}
\end{center}

\end{document}
