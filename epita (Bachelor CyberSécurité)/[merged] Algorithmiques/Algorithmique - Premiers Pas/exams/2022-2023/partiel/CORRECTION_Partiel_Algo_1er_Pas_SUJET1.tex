\documentclass[11pt,a4paper]{article}
\usepackage[utf8]{inputenc}
\usepackage[french]{babel}
\usepackage[T1]{fontenc}

\usepackage{amsmath}
\usepackage{amsfonts}
\usepackage{amssymb}

\newcommand{\NomAuteur}{Fabrice BOISSIER}
\newcommand{\TitreMatiere}{Algorithmique - Premiers Pas}
\newcommand{\NomUniv}{EPITA - Bachelor Cyber Sécurité}
\newcommand{\NiveauUniv}{CYBER1}
\newcommand{\NumGroupe}{CYBER1}
\newcommand{\AnneeUniv}{2022-2023}
\newcommand{\DateExam}{Janvier 2023}
\newcommand{\TypeExam}{Partiel (Sujet 1)}
\newcommand{\TitreExam}{\TitreMatiere}
\newcommand{\DureeExam}{2h00}
\newcommand{\MyWaterMark}{\AnneeUniv} % Watermark de protection

% Ajout de mes classes & definitions
\usepackage{MetalExam} % Appelle un .sty

% "Tableau" et pas "Table"
\addto\captionsfrench{\def\tablename{Tableau}}

%%%%%%%%%%%%%%%%%%%%%%%
%Header
%%%%%%%%%%%%%%%%%%%%%%%
\lhead{\TypeExam}							%Gauche Haut
\chead{\NomUniv}							%Centre Haut
\rhead{\NumGroupe}							%Droite Haut
\lfoot{\DateExam}							%Gauche Bas
\cfoot{\thepage{} / \pageref*{LastPage}}	%Centre Bas
\rfoot{\texttt{\TitreMatiere}}				%Droite Bas

%%%%%

\usepackage{tabularx}

\newlength{\LabelWidth}%
%\setlength{\LabelWidth}{1.3in}%
\setlength{\LabelWidth}{1cm}%
%\settowidth{\LabelWidth}{Employee E-mail:}%  Specify the widest text here.

% Optional first parameter here specifies the alignment of
% the text within the \makebox.  Default is [l] for left
% alignment. Other options are [r] and [c] for right and center
\newcommand*{\AdjustSize}[2][l]{\makebox[\LabelWidth][#1]{#2}}%


\definecolor{mGreen}{rgb}{0,0.6,0}
\definecolor{mGray}{rgb}{0.5,0.5,0.5}
\definecolor{mPurple}{rgb}{0.58,0,0.82}
\definecolor{backgroundColour}{rgb}{0.95,0.95,0.92}

\lstdefinestyle{CStyle}{
    backgroundcolor=\color{backgroundColour},
    commentstyle=\color{mGreen},
    keywordstyle=\color{magenta},
    numberstyle=\tiny\color{mGray},
    stringstyle=\color{mPurple},
    basicstyle=\footnotesize,
    breakatwhitespace=false,
    breaklines=true,
    captionpos=b,
    keepspaces=true,
    numbers=left,
    numbersep=5pt,
    showspaces=false,
    showstringspaces=false,
    showtabs=false,
    tabsize=2,
    language=C
}


\hyphenation{op-tical net-works SIGKILL}


\begin{document}

%\MakeExamTitleDuree     % Pour afficher la duree
\MakeExamTitle                   % Ne pas afficher la duree

%% \MakeStudentName    %% A reutiliser sur chaque nouvelle page

\bigskip
%\bigskip

Vous devez respecter les consignes suivantes, sous peine de 0 :

\begin{itemize}
\item Lisez le sujet en entier avec attention
\item Répondez sur le sujet
\item Ne détachez pas les agrafes du sujet
\item \'Ecrivez lisiblement vos réponses (si nécessaire en majuscules)
\item Vous devez écrire dans le langage algorithmique ou en C (donc pas de Python ou autre)
\item Ne trichez pas
\end{itemize}

%\bigskip

%%%%%%%%%%%%%%%%%%%%%%%%%%%%%%%%%%% CENTRAGE
\vfillFirst


% Questions cours
\section{Questions (4 points)}

\subsection{(2 points) Sélectionnez les conditions vraies pour A = 6 et B = 9 : }

% Compter pour chaque réponse 0,5 points : si bonne ou pas bonne
\bigskip

\begin{itemize}
  \item[\CaseCoche] ((A <= B) et (non (B == A)) et (A - 2 > B - 5)) ou ((A != B) et (B - 1 == A + 3)) \\ % Faux
  \item[\checkmark] ((non (A > B)) et (A != B) et (non (A == B)) ou ((B != A + 3) et (B - 3 == A)) \\ % Vrai
  \item[\checkmark] (non ((A > B) et (B > A))) et ((B == A + 3) et (A == B - 3)) \\ % Vrai
  \item[\checkmark] (non ((A >= B - 4) ou (B >= A + 3))) ou ((B <= A + 4) et (A <= B - 3)) \\ % Vrai
\end{itemize}

\bigskip

\subsection{(1 point) Lors de la déclaration d'une fonction et dans son implémentation, il faut déclarer : }

\bigskip

\begin{itemize}
  \item[\checkmark] Les paramètres d'entrée   \phantom{(}
  \item[\checkmark] Le nom des variables locales utilisées   \phantom{(}
  \item[\checkmark] Le type des variables locales utilisées   \phantom{(}
  \item[\checkmark] Le type de la valeur de retour   \phantom{(}
  \item[\checkmark] Les instructions à exécuter   \phantom{(}
\end{itemize}

\bigskip

\subsection{(1 point) Indiquez à quels types de base ces valeurs peuvent être associées : }

\bigskip

\begin{itemize}
  \item[$\bullet$] 'b'   : caractère ou entier \phantom{(} \\
  \item[$\bullet$] 16.64 : flottant            \phantom{(} \\
  \item[$\bullet$] 42    : entier ou caractère \phantom{(} \\
\end{itemize}

\vfillLast
%%%%%%%%%%%%%%%%%%%%%%%%%%%%%%%%%%% CENTRAGE

\newpage


\section{Algorithmes (16 points)}

\subsection{(2 points) \'Ecrivez une procédure \og \textit{AfficheMul2} \fg{} itérative affichant les N premiers entiers multiples de 2 (0 inclus). }

\bigskip

\begin{center}
\GrilleReponseN{10}
\end{center}

\bigskip

\subsection{(2 points) \'Ecrivez une fonction \og \textit{SearchElt} \fg{} récursive cherchant un élément dans un tableau et renvoyant l'index de la case contenant l'élément s'il est trouvé. Si l'élément n'est pas trouvé, la fonction renverra \textit{-1}. }

\bigskip

\begin{center}
\GrilleReponseN{10}
\end{center}

\bigskip

\subsection{(2 points) \'Ecrivez une fonction récursive \og \textit{Fibonacci} \fg{} calculant le n\up{ème} terme de la suite de Fibonacci. }

\begin{itemize}
\item[$\bullet$] $ fibo(0) = fibo(1) = 1 $
\item[$\bullet$] $ fibo(n) = fibo(n - 1) + fibo(n - 2) $
\end{itemize}

\bigskip

\begin{center}
\GrilleReponseN{10}
\end{center}

\bigskip

\subsection{(2 points) \'Ecrivez une fonction \og \textit{strlen} \fg{} récursive renvoyant la taille d'une chaîne caractères. }

\bigskip

\begin{center}
\GrilleReponseN{10}
\end{center}

\bigskip

\subsection{(2 points) \'Ecrivez une fonction \og \textit{Miroir} \fg{} itérative renvoyant un entier qui est le miroir de l'entier donné en entrée (afficher ne suffit pas). }

\bigskip

\begin{center}
\GrilleReponseN{24}
\end{center}

\bigskip

\subsection{(2 points) \'Ecrivez une fonction \og \textit{SuffixStrIter} \fg{} itérative vérifiant si une chaîne de caractères est bien un suffixe d'une autre chaîne de caractères. }

\bigskip

\begin{center}
\GrilleReponseN{24}
\end{center}

\bigskip

\subsection{(4 points) \'Ecrivez une fonction ou une procédure respectant l'algorithme de tri par insertion \og \textit{TriInsertion} \fg{} qui permet de trier en place un tableau. }

\bigskip

\begin{center}
\GrilleReponseN{24}
\end{center}

\end{document}
