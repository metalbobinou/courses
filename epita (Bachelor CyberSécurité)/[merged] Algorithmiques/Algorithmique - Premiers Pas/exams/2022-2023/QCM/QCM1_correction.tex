\documentclass[11pt,a4paper]{article}
\usepackage[utf8]{inputenc}
\usepackage[french]{babel}
\usepackage[T1]{fontenc}

\usepackage{amsmath}
\usepackage{amsfonts}
\usepackage{amssymb}

\newcommand{\NomAuteur}{Fabrice BOISSIER}
\newcommand{\TitreMatiere}{Algorithmique - Premiers Pas}
\newcommand{\NomUniv}{EPITA - Bachelor Cyber Sécurité}
\newcommand{\NiveauUniv}{CYBER1}
\newcommand{\NumGroupe}{CYBER1}
\newcommand{\AnneeUniv}{2022-2023}
\newcommand{\DateExam}{Septembre 2022}
\newcommand{\TypeExam}{QCM}
\newcommand{\TitreExam}{\TitreMatiere}
\newcommand{\DureeExam}{20 min}
\newcommand{\MyWaterMark}{\AnneeUniv} % Watermark de protection

% Ajout de mes classes & definitions
\usepackage{MetalExam} % Appelle un .sty

% "Tableau" et pas "Table"
\addto\captionsfrench{\def\tablename{Tableau}}

%%%%%%%%%%%%%%%%%%%%%%%
%Header
%%%%%%%%%%%%%%%%%%%%%%%
\lhead{\TypeExam}							%Gauche Haut
\chead{\NomUniv}							%Centre Haut
\rhead{\NumGroupe}							%Droite Haut
\lfoot{\DateExam}							%Gauche Bas
\cfoot{\thepage{} / \pageref*{LastPage}}	%Centre Bas
\rfoot{\texttt{\TitreMatiere}}				%Droite Bas

%%%%%

\usepackage{tabularx}

\newlength{\LabelWidth}%
%\setlength{\LabelWidth}{1.3in}%
\setlength{\LabelWidth}{1cm}%
%\settowidth{\LabelWidth}{Employee E-mail:}%  Specify the widest text here.

% Optional first parameter here specifies the alignment of
% the text within the \makebox.  Default is [l] for left
% alignment. Other options are [r] and [c] for right and center
\newcommand*{\AdjustSize}[2][l]{\makebox[\LabelWidth][#1]{#2}}%


\definecolor{mGreen}{rgb}{0,0.6,0}
\definecolor{mGray}{rgb}{0.5,0.5,0.5}
\definecolor{mPurple}{rgb}{0.58,0,0.82}
\definecolor{backgroundColour}{rgb}{0.95,0.95,0.92}

\lstdefinestyle{CStyle}{
    backgroundcolor=\color{backgroundColour},
    commentstyle=\color{mGreen},
    keywordstyle=\color{magenta},
    numberstyle=\tiny\color{mGray},
    stringstyle=\color{mPurple},
    basicstyle=\footnotesize,
    breakatwhitespace=false,
    breaklines=true,
    captionpos=b,
    keepspaces=true,
    numbers=left,
    numbersep=5pt,
    showspaces=false,
    showstringspaces=false,
    showtabs=false,
    tabsize=2,
    language=C
}


\hyphenation{op-tical net-works SIGKILL}


\begin{document}

% \MakeExamTitleDuree     % Pour afficher la duree
\MakeExamTitle                   % Ne pas afficher la duree

%% \MakeStudentName    %% A reutiliser sur chaque nouvelle page


% \setcounter{section}{1}  % Ne pas faire une liste de 0.1, 0.2, ... mais 1.1, 1.2, ...

\renewcommand{\thesubsection}{\arabic{subsection}} % Subsection sans 0.x, mais juste x

% \newcommand{\thesubsection}{\thesection.\arabic{subsection}} % Subsection avec numero section (0.x)



\subsection{(4 points) Sélectionnez le ou les types algorithmiques de base/fondamentaux (qui ne dépendent pas d'autres) }

\begin{itemize}
  \item[\checkmark] Entier \\ % OK
  \item[\CaseCoche] Relatif \\
  \item[\checkmark] Flottant \\ % OK
  \item[\checkmark] Caractère \\ % OK
\end{itemize}


\bigskip


\subsection{(4 points) Ce qui définit précisément les algorithmes récursifs terminaux est : }

\begin{itemize}
  \item[\checkmark] On ne fait que retourner une valeur sans rien exécuter lors du retour récursif \\ % OK
  \item[\CaseCoche] La fonction se rappelle elle-même \\
  \item[\CaseCoche] L'utilisation d'un accumulateur \\
  \item[\CaseCoche] L'utilisation d'une fonction chapeau \\
\end{itemize}


\bigskip


\subsection{(4 points) Cet algorithme est-il une procédure ou une fonction ? }

\begin{lstlisting}[style=algorithmique]
algorithme *** Fibonacci ***
  parametres locaux
    entier    n
debut
  si (n != 0)
    var1 = 0
    var2 = 1
    tant que (n > 0)
      total = var1 + var2
      var1 = var2
      var2 = total
      ecrire (total)
      fin tant que
  fin si
fin algorithme *** Fibonacci ***
\end{lstlisting}

\begin{itemize}
  \item[\checkmark] Procédure \\ % OK
  \item[\CaseCoche] Fonction \\
\end{itemize}


\bigskip


\subsection{(4 points) En donnant 5 et 4 en paramètre, que renverra cet algorithme ? }

\begin{lstlisting}[style=algorithmique]
algorithme fonction Calcul1 : entier
  parametres locaux
    entier    a, b
debut
  si (b == 1)
    retourner(1)
  sinon si (b <= 0)
    retourner(-a + Calcul1(a, (b + 1))
  sinon
    retourner(a + Calcul1(a, (b - 1))
  fin si
fin algorithme fonction Calcul1
\end{lstlisting}


\begin{itemize}
  \item[\CaseCoche] 15 \\
  \item[\checkmark] 16 \\ % OK
  \item[\CaseCoche] 20 \\
  \item[\CaseCoche] 21 \\
\end{itemize}


\bigskip


\subsection{(4 points) Qu'est-ce qui définit au mieux les effets de bord ? }

\begin{itemize}
  \item[\CaseCoche] L'exécution d'une instruction impacte les variables de la fonction courante \\
  \item[\checkmark] L'exécution d'une instruction impacte des variables hors de la fonction \\ % OK
  \item[\CaseCoche] L'exécution d'une instruction impacte les paramètres locaux \\
  \item[\CaseCoche] L'exécution d'une instruction n'a aucun impact sur quoique ce soit \\
\end{itemize}


\bigskip


\subsection{[BONUS] (0 point) "Attrapez-les tous" provient de : }

\begin{itemize}
  \item[\checkmark] Pokémon \\ % OK
  \item[\CaseCoche] Digimon \\
  \item[\CaseCoche] Medabots \\
  \item[\CaseCoche] Yu-Gi-Oh! \\
  \item[\CaseCoche] La réponse D \\
\end{itemize}

\end{document}
