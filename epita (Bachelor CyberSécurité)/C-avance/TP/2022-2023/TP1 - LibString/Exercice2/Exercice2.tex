%% Exercice 2

%\ExoSpecs{\TTBF{CalculTVA.sh}}{\TTBF{\RenduDir/src/exo1/}}{750}{640}{\TTBF{write}}
\ExoSpecsCustom{\TTBF{StrPart1.c} \TTBF{StrPart1.h}}{\TTBF{\RenduDir/src/}}{750}{640}{Fonctions autorisées}{\TTBF{malloc(3)}, \TTBF{free(3)}}

\vspace*{0.7cm}

\noindent \ExoObjectif{Le but de l'exercice est de recoder les fonctions essentielles permettant de manipuler des chaînes de caractères : \TTBF{my\_strchr}, \TTBF{my\_strrchr}, et \TTBF{my\_strcpy} }

\bigskip

%\noindent Vous devez implémenter ces fonctions en respectant strictement les spécifications suivantes.
%Vous ne devez ni rendre de fonction \TTBF{main} ni vos tests : vous ne devez rendre \textit{que} les fonctions demandées (et éventuellement celles vous permettant de les faire fonctionner) dans le fichier \texttt{StrPart1.c}, ainsi que leurs prototypes dans le fichier \texttt{StrPart1.h}.

\bigskip

%\noindent Vous devez implémenter les X fonctions suivantes :
%\begin{itemize}
%\item \TTBF{char *my\_strchr(char *s, int c)}
%\item \TTBF{char *my\_strrchr(char *s, int c)}
%\item \TTBF{char *my\_strcpy(char *dest, char *src)}
%\end{itemize}

\noindent Vous devez implémenter les fonctions suivantes :

\bigskip

\lstset{language=C}
%\begin{lstlisting}[frame=single,title={Liste des fonctions pour l'exercice 1}]
\begin{lstlisting}[frame=single]
char *my_strchr(char *s, int c);
char *my_strrchr(char *s, int c);
char *my_strcpy(char *dest, char *src);
\end{lstlisting}


\subsubsection*{\TTBF{char *my\_strchr(char *s, int c)}}

\noindent Cette fonction renvoie un pointeur vers la première occurrence d'un caractère dans une chaîne.
La fonction doit renvoyer un pointeur vers le premier caractère retrouvé dans la chaîne \TTBF{s}.
Le caractère recherché est donné en paramètre en tant que \TTBF{c}.
Si le caractère n'est pas trouvé, la fonction doit renvoyer \TTBF{NULL}.


\subsubsection*{\TTBF{char *my\_strrchr(char *s, int c)}}

\noindent Cette fonction renvoie un pointeur vers la dernière occurrence d'un caractère dans une chaîne.
La fonction doit renvoyer un pointeur vers le dernier caractère retrouvé dans la chaîne \TTBF{s}.
Le caractère recherché est donné en paramètre en tant que \TTBF{c}.
Si le caractère n'est pas trouvé, la fonction doit renvoyer \TTBF{NULL}.


\subsubsection*{\TTBF{char *my\_strcpy(char *dest, char *src)}}

\noindent Cette fonction copie une chaîne de caractères vers un espace de taille suffisant déjà alloué.
La fonction lit les caractères de la chaîne \TTBF{src} pour les copier dans l'espace mémoire pointé par \TTBF{dest}.
L'espace mémoire au bout de \TTBF{dest} doit avoir été préalablement réservé par l'utilisateur appelant, afin de pouvoir y mettre tous les caractères de la chaîne contenue dans \TTBF{src} ainsi que le '\TTBF{\textbackslash{}0}' final.
La fonction doit renvoyer un pointeur vers la chaîne \TTBF{dest}.
\textit{Si l'espace mémoire dans \TTBF{dest} n'est pas assez grand, vous ne pouvez pas le savoir à l'avance, donc, vous devez essayer de copier normalement les caractères dedans. Si l'utilisateur ne respecte l'exigence de taille, ce n'est pas votre faute en tant que développeur : votre fonction peut échouer dans ce cas.}
