%% Exercice 5

%\ExoSpecs{\TTBF{CalculTVA.sh}}{\TTBF{\RenduDir/src/exo1/}}{750}{640}{\TTBF{write}}
\ExoSpecsCustom{\TTBF{queue\_static.c}}{\TTBF{\RenduDir/src/}}{750}{640}{Fonctions autorisées}{\TTBF{write(2)}}

\vspace*{0.7cm}

\noindent \ExoObjectif{Le but de l'exercice est d'implémenter une file en C utilisant un tableau statique et sans jamais modifier le tas du processus (c'est-à-dire sans utiliser \texttt{malloc(3)}, \texttt{mmap(2)}, ou tout autre appel système ou fonction modifiant le tas ou réservant des pages mémoire lors de l'exécution).}

\bigskip

\noindent Les fonctions demandées dans cet exercice devront également se trouver dans la bibliothèque nommée \TTBF{libmyqueue}.

\bigskip

\noindent Vous devez écrire plusieurs fonctions permettant de créer, utiliser, vider, et libérer une file.
Un fichier \TTBF{queue\_static.h} contenant toutes les fonctions exportables à implémenter vous est fourni en annexe.
Vous devez déclarer une structure \TTBF{queue\_static} et l'ajouter dans \TTBF{queue\_static.h}.
N'oubliez pas qu'un tableau statique est généré par le compilateur : vous devrez indiquer dans la constante de pré-compilation \TTBF{QUEUE\_STATIC\_MAX\_LEN} la taille maximum.
La file étant statique, il vous faudra déclarer une variable globale statique nommée \TTBF{g\_queue\_static} qui pointera vers la structure (elle-même déclarée dans la variable globale nommée \TTBF{g\_s\_queue\_static}).
Pour les premières étapes, vous devrez implémenter une version simplifiée de la file qui ne prend en charge que des entiers positifs.

\noindent \textit{Attention : étant donné qu'il s'agit d'une version statique, vous ne devez \textbf{JAMAIS} utiliser \TTBF{malloc}, \TTBF{free}, ou toute autre fonction ou appel système visant à réserver de la mémoire.}

%\noindent Vous devez implémenter les 11 fonctions suivantes :
%\begin{itemize}
%\item \TTBF{void queue\_static\_create(void)}
%%%\item \TTBF{void queue\_static\_delete(void)}
%\item \TTBF{int queue\_static\_length(void)}
%\item \TTBF{int queue\_static\_max\_length(void)
%\item \TTBF{int queue\_static\_enqueue(int elt)}
%\item \TTBF{int queue\_static\_dequeue(void)}
%\item \TTBF{int queue\_static\_head(void)}
%\item \TTBF{int queue\_static\_tail(void)}
%\item \TTBF{int queue\_static\_clear(void)}
%\item \TTBF{int queue\_static\_is\_empty(void)}
%\item \TTBF{int queue_static_insert(int elt, int pos)}
%\item \TTBF{int queue_static_replace(int elt, int pos)}
%\item \TTBF{int queue\_static\_search(int elt)}
%\item \TTBF{void queue\_static\_reverse(void)}
%\item \TTBF{void queue\_static\_print(void)}
%\end{itemize}

\noindent Vous devez implémenter les fonctions suivantes :

\bigskip

\lstset{language=C}
\begin{lstlisting}[frame=single,title={Liste des fonctions pour une file avec liste chaînée}]
void queue_static_create(void);
void queue_static_delete(void);

int queue_static_length(void);
int queue_static_max_length(void);

int queue_static_enqueue(int elt);
int queue_static_dequeue(void);
int queue_static_head(void);
int queue_static_tail(void);

int queue_static_clear(void);
int queue_static_is_empty(void);

int queue_static_insert(int elt, int pos);
int queue_static_replace(int elt, int pos);

int queue_static_search(int elt);
void queue_static_reverse(void);
void queue_static_print(void);
\end{lstlisting}


\subsubsection*{\TTBF{void queue\_static\_create(void)}}

\noindent Cette fonction initialise les valeurs de la structure.


\subsubsection*{\TTBF{void queue\_static\_delete(void)}}

\noindent Cette fonction vide une file de l'ensemble de ses éléments.


\subsubsection*{\TTBF{int queue\_static\_max\_length(void)}}

\noindent Cette fonction renvoie la longueur du tableau contenant la file.


\subsubsection*{\TTBF{int queue\_static\_length(void)}}

\noindent Cette fonction renvoie la longueur de la file (c'est-à-dire le nombre d'éléments actuellement dans la file).


\subsubsection*{\TTBF{int queue\_static\_enqueue(int elt)}}

\noindent Cette fonction enfile un élément dans une file, c'est-à-dire qu'elle ajoute un élément en queue.
En cas de succès, la fonction renvoie $ 0 $.
Si le nombre donné en paramètre est inférieur à $ 0 $, la fonction renvoie $ -4 $.
Si le tableau est déjà plein, la fonction renvoie $ -3 $.


\subsubsection*{\TTBF{int queue\_static\_dequeue(void)}}

\noindent Cette fonction défile un élément d'une file, c'est-à-dire qu'elle supprime l'élément en tête.
En cas de succès, la fonction renvoie $ 0 $.
Si la file est vide, la fonction renvoie $ -2 $.


\subsubsection*{\TTBF{int queue\_static\_head(void)}}

\noindent Cette fonction renvoie l'élément en tête de file.
Si la file est vide, la fonction renvoie $ -2 $.


\subsubsection*{\TTBF{int queue\_static\_tail(void)}}

\noindent Cette fonction renvoie l'élément en queue de file.
Si la file est vide, la fonction renvoie $ -2 $.


\subsubsection*{\TTBF{int queue\_static\_clear(void)}}

\noindent Cette fonction vide une file de l'ensemble de ses éléments, sans détruire la structure de la file.
La fonction renvoie le nombre d'éléments supprimés de la mémoire.
Si la file est vide, la fonction renvoie $ 0 $.


\subsubsection*{\TTBF{int queue\_static\_is\_empty(void)}}

\noindent Cette fonction teste si une file est vide ou non.
Si la file est vide, la fonction renvoie $ 1 $.
Si la file n'est pas vide, la fonction renvoie $ 0 $.


\subsubsection*{\TTBF{int queue\_static\_insert(int elt, int pos)}}

\noindent Cette fonction ajoute un élément dans la file à l'emplacement \textit{pos}.
La première position est celle où l'élément le plus ancien a été placé (c'est-à-dire la tête de la file), cette position sera numérotée $ 0 $.
L'ancien élément qui était présent à cette position est décalé d'un cran en arrière (vers la queue).
Si le nombre donné en paramètre est inférieur à $ 0 $, la fonction renvoie $ -4 $.
Si l'emplacement n'existe pas et est positif, on ajoute l'élément en queue.
Si l'emplacement n'existe pas et est négatif, on ajoute l'élément en tête.
En cas de succès, la fonction renvoie $ 0 $.
%Si la file est vide, la fonction renvoie $ -2 $.
Si le tableau est déjà plein, la fonction renvoie $ -3 $.


\subsubsection*{\TTBF{int queue\_static\_replace(int elt, int pos)}}

\noindent Cette fonction remplace un élément dans la file à l'emplacement \textit{pos}, et renvoie l'élément qui était présent à cet endroit.
La première position est celle où l'élément le plus ancien a été placé (c'est-à-dire la tête de la file), cette position sera numérotée $ 0 $.
Si le nombre donné en paramètre est inférieur à $ 0 $, la fonction renvoie $ -4 $.
Si l'emplacement n'existe pas et est positif, on remplace l'élément en queue.
Si l'emplacement n'existe pas et est négatif, on remplace l'élément en tête.
%En cas de succès, la fonction renvoie $ 0 $.
Si la file est vide, la fonction renvoie $ -2 $.


\subsubsection*{\TTBF{int queue\_static\_search(int elt)}}

\noindent Cette fonction recherche un élément dans la file et renvoie sa position dans le tableau.
La première position est celle où l'élément le plus ancien a été placé (c'est-à-dire la tête de la file), cette position sera numérotée $ 0 $.
Si l'élément n'est pas trouvé, la fonction renvoie $ -4 $.


\subsubsection*{\TTBF{void queue\_static\_reverse(void)}}

\noindent Cette fonction inverse la position de tous les éléments de la file.
Le premier élément devient le dernier, l'avant dernier devient le deuxième, etc.
\textit{Attention : vous ne devez pas utiliser de \TTBF{malloc} ou de tableau temporaire pour effectuer cette fonction.}

\subsubsection*{\TTBF{void queue\_static\_print(void)}}

\noindent Cette fonction affiche le contenu de la file.
Le format d'affichage attendu implique d'afficher un seul élément par ligne, suivi d'un retour à la ligne.
L'élément en tête de file sera affiché en premier.
Si la file donnée en paramètre est vide, seul un retour à la ligne est affiché.
\textit{Attention, en version statique, vous devrez utiliser \texttt{write(2)} et non pas \texttt{printf(3)}.}

\bigskip

\lstset{language=sh}
\begin{lstlisting}[frame=single,title={Exemple d'affichage du cas normal : file contenant 42, 5, 13}]
$ ./my_queue_static
42
5
13

$
\end{lstlisting}

\bigskip

\lstset{language=sh}
\begin{lstlisting}[frame=single,title={Exemple d'affichage d'une file vide}]
$ ./my_queue_static

$
\end{lstlisting}
