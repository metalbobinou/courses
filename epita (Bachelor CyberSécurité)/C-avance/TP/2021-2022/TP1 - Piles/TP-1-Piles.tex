\documentclass[12pt,a4paper]{article}

\usepackage[utf8]{inputenc}
\usepackage[french]{babel}
\usepackage[T1]{fontenc}

\usepackage{amsmath}
\usepackage{amsfonts}
\usepackage{amssymb}

% Preparation des infos generales
\newcommand{\TitreMatiere}{C avancé}
\newcommand{\DateExo}{18 mars 2022}
\newcommand{\TypeExo}{TP}
\newcommand{\TitreExercice}{Piles}
\newcommand{\NomAuteurA}{Fabrice BOISSIER}
\newcommand{\MailAuteurA}{fabrice.boissier@epita.fr}
\newcommand{\NomAuteurB}{Mark ANGOUSTURES}
\newcommand{\MailAuteurB}{mark.angoustures@epita.fr}
\newcommand{\VersionExo}{Version 10}
\newcommand{\LoginEtudiant}{2021-2022} % Watermark de protection

%\newcommand{\RenduTarball}{nom1-nom2-TP1.tar.bz2}
%\newcommand{\RenduDir}{nom1-nom2-TP1}
\newcommand{\RenduTarball}{login.x-TP1.tar.bz2}
\newcommand{\RenduDir}{login.x-TP1}


% Ajout de mes classes & definitions
\usepackage{MetalExo} % Appelle un .sty


% Redefinition headers
\lhead{\TitreExercice}		%Gauche Haut
\chead{\TypeExo}			%Centre Haut
\rhead{\thepage}			%Droite Haut
\lfoot{}					%Gauche Bas
\cfoot{\TitreMatiere}		%Centre Bas
\rfoot{}					%Droite Bas


\definecolor{mygreen}{rgb}{0,0.6,0}		% RGB model
\definecolor{mygray}{rgb}{0.5,0.5,0.5}
\definecolor{mymauve}{rgb}{0.58,0,0.82}


\usepackage{array}
\newcolumntype{P}[1]{>{\raggedright\arraybackslash }m{#1}}

\newcolumntype{R}[1]{>{\raggedleft\arraybackslash }b{#1}}
\newcolumntype{L}[1]{>{\raggedright\arraybackslash }b{#1}}
\newcolumntype{C}[1]{>{\centering\arraybackslash }b{#1}}

\newcolumntype{M}[1]{>{\raggedright\let\newline\\\arraybackslash\hspace{0pt}}m{#1}} % another L
\newcolumntype{D}[1]{>{\centering\let\newline\\\arraybackslash\hspace{0pt}}m{#1}}   % another C
\newcolumntype{S}[1]{>{\raggedleft\let\newline\\\arraybackslash\hspace{0pt}}m{#1}}  % another R

\begin{document}

%% Titre
\maketitle

%\newpage

%% Copyright
\pagenumbering{Roman}

%% Copyright

{\Large \textbf{Copyright}}

\vspace{30px}

Ce document est destiné à une utilisation interne à EPITA.\\

Copyright \space \copyright \space{} 2022/2023 Fabrice BOISSIER\\

\bigskip

\begin{center}
	\fcolorbox{black}{white}{\makebox[14cm]{
	\begin{minipage}[l]{12cm}
		\vspace*{10px}
		\textbf{La copie de ce document est soumise à conditions :}\\
		\medskip
		\begin{itemize}
%		\item[$\vartriangleright$] Vous devez avoir téléchargé votre copie de ce document depuis Moodle. <\url{https://www.acu.epita.fr/intra/}>.\\
		\item[$\vartriangleright$] Il est interdit de partager ce document avec d'autres personnes.\\
		\item[$\vartriangleright$] Vérifiez que vous disposez de la dernière révision de ce document.\\
		\end{itemize}
	\vspace*{10px}
	\end{minipage}
	}}
\end{center}


\newpage

%% Table des matieres
\tableofcontents

\newpage

%% Consignes generales
\section{Consignes Générales}

\bigskip

%% Consignes generales
\noindent \textit{Les informations suivantes sont très importantes :}

\bigskip

\noindent \textit{Le non-respect d'une des consignes suivantes entraînera des sanctions pouvant aller jusqu'à la multiplication de la note finale par 0.}

\bigskip

\noindent \textit{Ces consignes sont claires, non-ambiguës, et ont un objectif précis. En outre, elles ne sont pas négociables.}

\bigskip

\noindent N'hésitez pas à demander si vous ne comprenez pas une des règles.

\bigskip

\newcounter{My_Counter}

\ConsigneGenerale{My_Counter}{Vous devez lire le sujet.}
\ConsigneGenerale{My_Counter}{Vous devez respecter les consignes.}
\ConsigneGenerale{My_Counter}{Vous devez rendre le travail dans les délais prévus.}

\medskip

\ConsigneGenerale{My_Counter}{Le travail doit être rendu dans le format décrit à la section \hyperref[sec:FormatDeRendu]{Format de Rendu}.}

\ConsigneGenerale{My_Counter}{Le travail rendu ne doit pas contenir de fichiers binaires, temporaires, ou d'erreurs (\texttt{\textbf{*\textasciitilde}}, \texttt{\textbf{*.o}}, \texttt{\textbf{*.a}}, \texttt{\textbf{*.so}}, \texttt{\textbf{*\#*}}, \texttt{\textbf{*core}}, \texttt{\textbf{*.log}}, \texttt{\textbf{*.exe}}, binaires, ...).}

\ConsigneGenerale{My_Counter}{Dans l'ensemble de ce document, la casse (caractères majuscules et minuscules) est très importante. Vous devez strictement respecter les majuscules et minuscules imposées dans les messages et noms de fichiers du sujet.}

\ConsigneGenerale{My_Counter}{Dans l'ensemble de ce document, \TTBF{\Login} \space correspond à votre login.}

\ConsigneGenerale{My_Counter}{Dans l'ensemble de ce document, \TTBF{nom1-nom2} \space correspond à la combinaison des deux noms de votre binôme (par exemple pour Fabrice BOISSIER et Mark ANGOUSTURES, cela donnera \TTBF{boissier-angoustures}).}

\ConsigneGenerale{My_Counter}{Dans l'ensemble de ce document, le caractère \TTBF{\textvisiblespace } correspond à une espace (s'il vous est demandé d'afficher \TTBF{\textvisiblespace \textvisiblespace \textvisiblespace }, vous devez afficher trois espaces consécutives).}

\ConsigneGenerale{My_Counter}{Tout retard, même d'une seconde, entraîne la note non négociable de 0.}

\ConsigneGenerale{My_Counter}{La triche (échange de code, copie de code ou de texte, ...) entraîne \textbf{au mieux} la note non négociable de 0.}

\ConsigneGenerale{My_Counter}{En cas de problème avec le projet, vous devez contacter le plus tôt possible les responsables du sujet aux adresses mail indiquées.}

\bigskip

\noindent \textbf{Conseil :} N'attendez pas la dernière minute pour commencer à travailler sur le sujet.


\newpage

%% Format de Rendu
\section{Format de Rendu}
\label{sec:FormatDeRendu}

\vspace*{1cm}

%% Format de Rendu

%\ResponsablesProjet{Metal Man/metalman@example.org, Damdoshi/damdoshi@example.org, Tayst/tayst@tayst.org}
\begin{tabular}{p{7cm} p{10cm}}
	%\ResponsablesProjetRow{Fabrice BOISSIER/fabrice.boissier@univ-paris1.fr, Ali JAFFAL/ali.jaffal@univ-paris1.fr}
	\ResponsablesProjetRow{Fabrice BOISSIER/fabrice.boissier@univ-paris1.fr}
	& \\
%	\RenduSpecsGenerales{[PHP][DM]}{2}{Envoi par mail}{\RenduDir}{\RenduTarball}{10/02/2020 23h42}{2 semaines}	
	\RenduSpecsGenerales{[OS][TP1]}{1}{Pas de rendu}{\RenduDir}{\RenduTarball}{Pas de rendu}{Pas de rendu}
	& \\
%	\RenduSpecsTechniques{WAMP ou MAMP}{PHP}{Apache/PHP}{ }
	\RenduSpecsTechniques{Linux - Ubuntu (x86\_64)}{sh}{/bin/bash}{ }
\end{tabular}

\vspace*{1cm}


\noindent L'arborescence attendue pour le projet est la suivante :

\medskip

\begin{tabular}{l}
\TTBF{\RenduDir/}\\
\TTBF{\RenduDir/AUTHORS}\\
\TTBF{\RenduDir/README}\\
\TTBF{\RenduDir/}\\
\TTBF{\RenduDir/src/}\\
%\TTBF{\RenduDir/src/exo1/}\\
%\TTBF{\RenduDir/src/exo1/exo1\_test.sh}\\
%\TTBF{\RenduDir/src/exo2/}\\
%\TTBF{\RenduDir/src/exo2/exo2.sh}\\
%\TTBF{\RenduDir/src/exo3/}\\
%\TTBF{\RenduDir/src/exo3/exo3.sh}\\
%\TTBF{\RenduDir/src/exo4/}\\
%\TTBF{\RenduDir/src/exo4/exo4.sh}\\
%\TTBF{\RenduDir/src/exo5/}\\
%\TTBF{\RenduDir/src/exo5/exo5.sh}\\
\end{tabular}


\vspace*{1cm}


\noindent Les fichiers suivants sont requis :

\medskip

\begin{tabular}{l p{12cm}}
\texttt{AUTHORS} & contient le(s) nom(s) et prénom(s) de(s) auteur(s).\\
\texttt{README} & contient la description du projet et des exercices, ainsi que la fa\c con d'utiliser le projet.\\
\end{tabular}


%%%%%%%%%%%%%%%%%%%%%%%%%%%%
%% AIDE MEMOIRE AU CAS OU %%
%%%%%%%%%%%%%%%%%%%%%%%%%%%%
\newpage

\section{Aide Mémoire}
\label{sec:AideMemoire}

\vspace*{1cm}

%%%%%%%%%%%%%%%%%%%%%%%%%%%%
%% AIDE MEMOIRE AU CAS OU %%
%%%%%%%%%%%%%%%%%%%%%%%%%%%%
%\newpage

%{\Large \textbf{Aide Mémoire}}

%\vspace{30px}

%\noindent Le travail doit être rendu au format \textbf{\textit{.zip}}, c'est-à-dire une archive \textbf{zip} compressée avec un outil adapté (les logiciels \textit{7zip} ou \textit{Keka} sont gratuits et adaptés).
\noindent Le travail doit être rendu au format \textbf{\textit{.tar.bz2}}, c'est-à-dire une archive \textbf{bz2} compressée avec un outil adapté (voir \TTBF{man 1 tar} et \TTBF{man 1 bz2}).

%\noindent Tout autre format d'archive (rar, 7zip, gz, gzip, bzip, ...) ne sera pas pris en compte, et votre travail ne sera pas corrigé (entraînant la note de 0).
\noindent Tout autre format d'archive (zip, rar, 7zip, gz, gzip, ...) ne sera pas pris en compte, et votre travail ne sera pas corrigé (entraînant la note de 0).

\bigskip

\noindent Pour générer une archive \textit{tar} en y mettant les dossiers \textit{folder1} et \textit{folder2}, vous devez taper :

\begin{tabular}{l}
\TTBF{tar cvf MyTarball.tar folder1 folder2}\\
\end{tabular}

\bigskip

\noindent Pour générer une archive \textit{tar} et la compresser avec GZip, vous devez taper :

\begin{tabular}{l}
\TTBF{tar cvzf MyTarball.tar.gz folder1 folder2}\\
\end{tabular}

\bigskip

\noindent Pour générer une archive \textit{tar} et la compresser avec BZip2, vous devez taper :

\begin{tabular}{l}
\TTBF{tar cvjf MyTarball.tar.bz2 folder1 folder2}\\
\end{tabular}

\bigskip

\noindent Pour lister le contenu d'une archive \textit{tar}, vous devez taper :

\begin{tabular}{l}
\TTBF{tar tf MyTarball.tar.bz2}\\
\end{tabular}

\bigskip

\noindent Pour extraire le contenu d'une archive \textit{tar}, vous devez taper :

\begin{tabular}{l}
\TTBF{tar xvf MyTarball.tar.bz2}\\
\end{tabular}


\vspace*{1cm}


%\noindent Dans ce sujet précis, vous ferez du code en C et des appels à des scripts shell qui afficheront les résultats dans le terminal (donc des flux de sortie qui pourront être redirigés vers un fichier texte).

%\noindent Dans ce sujet précis, vous ferez du code en script shell, qui affichera les résultats dans le terminal (donc des flux de sortie qui pourront être redirigés vers un fichier texte).

%\noindent Dans ce sujet précis, vous ferez du code en PHP, qui affichera les résultats dans une page HTML. Les valeurs seront affichées dans une \textit{textarea} dont le texte est généré par des outils multiplateformes supportant les retours à la ligne UNIX (\textbf{\textbackslash n}). Il ne faut donc pas inclure de balise \TTBF{"<br />"} pour retourner à la ligne, mais un \TTBF{"\textbackslash n"}.

\noindent Dans ce sujet précis, vous ferez du code en Python, qui affichera les résultats dans le terminal (donc des flux de sortie qui pourront être redirigés vers un fichier texte).

%\vspace*{1cm}

%\noindent Pour réaliser le travail demandé, nous vous fournirons pour chaque exercice au moins 2 fichiers : \TTBF{exoN\_res.php} (le fichier qui sera appelé pour voir le résultat de votre travail), et \TTBF{exoN\_fun.php} (le fichier contenant la fonction que vous devez coder dans chaque exercice).
%Optionnellement, un fichier \TTBF{exoN\_data.php} peut être fourni pour indiquer le format de données en entrée.
%Le \TTBF{N} correspond au numéro de l'exercice.

%\medskip

%\noindent Dans tous les cas, vous ne devez rendre que le fichier \TTBF{exoN\_fun.php} avec au moins la fonction demandée remplie (qui peut faire appel à d'autres fonctions que vous définirez dans le \textbf{même} fichier). Les autres fichiers seront générés par nos soins pour tester vos fonctions.

%\vspace*{1cm}

%\noindent Vous ne devez \textbf{PAS} utiliser la fonction \TTBF{echo} pour écrire !
%Il faut retourner une chaîne de caractères correctement formattée.


\newpage

\pagenumbering{arabic}

%%%%%%%%%%%%%%%
%%   COURS   %%
%%%%%%%%%%%%%%%
\section{Cours}

\vspace*{0.7cm}

\newcommand{\defaultparindent}{\parindent}
\setlength{\parindent}{0pt}				% \noindent partout
% \parindent in one-column documents is :
% 15pt when the default text size is 10pt,
% 17pt for 11pt,
% and 1.5em for 12pt.
% In two-column documents it is 1em

%\begin{center}
%\begin{tabular}{p{5cm} p{11cm}}
%\textbf{Commandes étudiées :} & \texttt{sh}, \texttt{bash}, \texttt{man}, \texttt{ls}, \texttt{mkdir}, \texttt{touch}, \texttt{chmod}, \texttt{mv}, \texttt{rm}, \texttt{rmdir}, \texttt{cat}, \texttt{file}, \texttt{whereis}, \texttt{which}\\

%\textbf{Builtins étudiées :} & \texttt{pwd}, \texttt{cd}, \texttt{exit}, \texttt{logout}, \texttt{echo}, \texttt{umask}, \texttt{type}, \texttt{>}, \texttt{>{}>}, \texttt{<}, \texttt{<{}<}, \texttt{|}\\

%\textbf{Notions étudiées :} & Tableaux, Pointeurs, Piles\\
%\end{tabular}
%\end{center}

\textbf{Notions étudiées :} Tableaux, Pointeurs, Piles\\

%\bigskip

\subsection{Rappel sur les piles}

\bigskip

Les \textbf{piles}, ou \textbf{stacks} en anglais, sont des structures visant à stocker les données dans l'ordre d'arrivée, mais ne permettant leur récupération uniquement dans l'ordre inverse.
Dans une pile, on ne peut accéder qu'à la dernière donnée stockée, celle se situant au \textit{sommet} de la pile.
Ces structures sont aussi appelées \textbf{LIFO} (\textit{Last In First Out}).\\

\begin{center}
\includegraphics[scale=0.75]{Cours/Piles_1_Structure_Generale_centered.png}
\end{center}

\smallskip

Deux opérations permettent d'utiliser une pile :
\begin{itemize}
\item \TTBF{PUSH} : permettant d'\textit{empiler} une donnée supplémentaire dans la pile
\item \TTBF{POP} : permettant de \textit{dépiler} une donnée depuis la pile
\end{itemize}
On ajoute donc une donnée en l'empilant avec un \TTBF{PUSH}, et il est possible de directement y accéder, car elle est au sommet de la pile.
À l'inverse, pour accéder à une donnée tout au fond de la pile, il est nécessaire de dépiler autant d'éléments que nécessaire avec un \TTBF{POP}.\\

Voici un exemple où l'on crée une pile, puis on empile successivement $ 42 $, $ 5 $, et $ 13 $, puis, on dépile une fois (pour récupérer $ 13 $), et enfin, on empile successivement $ 37 $, $ 10 $, $ 24 $.\\

\begin{center}
\includegraphics[scale=0.5]{Cours/Piles_2_Structure_Generale_Usage_pack_1.png}
\end{center}

\begin{center}
\includegraphics[scale=0.5]{Cours/Piles_2_Structure_Generale_Usage_pack_2.png}
\end{center}

\smallskip

Les piles, et surtout la contrainte d'accès aux objets, sont couramment utilisées : un camion de livraison sera d'abord rempli avec les paquets à livrer en dernier/le camion sera rempli dans l'ordre inverse de livraison (on accède d'abord aux derniers éléments chargés).

En informatique, on utilisera les piles dans certains \textit{parsers} (analyse grammaticale) pour connaître en premier l'opérateur à exécuter (opérateur binaire ? unaire ?) et dépiler par la suite le nombre exact de paramètres.
La \textit{pile d'appels} est également une convention fondamentale partagée par les processeurs et les systèmes d'exploitation permettant de passer des paramètres (et d'autres informations de contexte) aux fonctions appelées par les programmes, ou en cas d'interruption pour sauvegarder l'adresse de l'instruction qui était en cours d'exécution.\\

Afin d'implémenter une pile, il est donc nécessaire d'avoir un espace de stockage ordonné (un tableau numéroté ou une liste chaînée), et un indicateur de l'élément en haut de la pile.
Nous allons maintenant voir comment implémenter une pile avec des listes chaînées et un tableau de taille fixe.

\bigskip

%%%%%%%%%%%%%%%%%%%%%%%%%%%%%%%%%%%%%%%%%%%%%%%%%%%%%%%%%%%%
%%%%%%%%%%%%%%%%%%%%%%%%%%%%%%%%%%%%%%%%%%%%%%%%%%%%%%%%%%%%
%%%%%%%%%%%%%%%%%%%%%%%%%%%%%%%%%%%%%%%%%%%%%%%%%%%%%%%%%%%%

\subsection{Piles : implémentation avec un tableau de taille fixe}

\bigskip

Une implémentation avec un tableau de taille fixe impose cette fois une limitation : la pile aura une taille maximale, et on peut refuser l'ajout d'un élément si la pile est déjà pleine.
La structure diffère également du fait que le tableau est alloué une seule fois lors de sa création (voire même lors de la compilation dans le cas statique).

Le schéma suivant présente la structure générale :\\

\begin{center}
\includegraphics[scale=1]{Cours/Piles_5_Tableau_Statique_Structure.png}
\end{center}

\smallskip

On notera cette fois que plusieurs informations distinctes doivent être conservées : l'adresse du tableau, le numéro de case correspondant au sommet de la pile (\textit{head} dans notre cas), la taille du tableau (le nombre maximum d'objets pouvant être stockés), le nombre d'éléments dans le tableau.

Le schéma suivant détaille certaines informations de façon plus explicite :\\

\begin{center}
\includegraphics[scale=1]{Cours/Piles_5_Tableau_Statique_Structure_Detaillee.png}
\end{center}

\smallskip

Dans le cas d'un tableau de taille fixe, le pointeur de sommet ne peut pas utiliser la valeur \TTBF{NULL} comme indicateur de tableau vide, car cette valeur est égale à $ 0 $ (ce qui laisserait à penser que le sommet est effectivement à la case 0).
Plusieurs solutions sont possibles pour indiquer le sommet de la pile et le cas vide :

\begin{itemize}
\item On enregistre dans la structure de la pile une variable servant à compter le nombre d'éléments présents (le sommet peut donc prendre n'importe quelle valeur tant que la pile est vide).
\item On utilise un entier relatif pour indiquer le sommet, et $ -1 $ indique que la pile est vide (l'ajout d'un élément décalera le sommet à $ 0 $, c'est-à-dire la case où sera l'élément).
\item On place le sommet de la pile sur la première case non utilisée, et l'accès au premier élément se fait donc en retirant $ 1 $ au pointeur de sommet (ainsi, un sommet à la case $ 0 $ indique que la pile est vide).
Attention : dans ce cas précis, un tableau plein aura un sommet hors des cases du tableau (il ne faudra donc \textit{jamais} le déréférencer s'il atteinte une telle valeur).
\end{itemize}

\smallskip

L'exemple suivant montre l'évolution d'une pile implémentée avec un tableau fixe au fur et à mesure des ajouts (empiler / \TTBF{PUSH}) et suppressions (dépiler / \TTBF{POP}).\\

\begin{center}
\includegraphics[scale=0.65]{Cours/Piles_6_Tableau_Statique_Usage_pack_1.png}
\end{center}

\begin{center}
\includegraphics[scale=0.65]{Cours/Piles_6_Tableau_Statique_Usage_pack_2.png}
\end{center}

\begin{center}
\includegraphics[scale=0.65]{Cours/Piles_6_Tableau_Statique_Usage_pack_3.png}
\end{center}

\smallskip

Les principales opérations se résument ainsi :
\begin{itemize}
\item Création : on alloue en mémoire le tableau (sauf s'il est statique), et on fixe le sommet de la pile à la valeur prévue pour démarrer ($ -1 $, $ 0 $, ou toute autre valeur choisie) [éventuellement, on met à jour le nombre d'objets dans le tableau en le fixant à $ 0 $].
\item Empiler : si le tableau est plein, on retourne une erreur, sinon, on ajoute un élément, et on décale le sommet de la pile [éventuellement, on met à jour le nombre d'objets dans le tableau].
\item Dépiler : si la pile est vide, on retourne une erreur, sinon, on réduit la valeur du sommet de la pile [éventuellement, on met à jour le nombre d'objets dans le tableau].
\item Vider : on fixe le sommet de la pile à la valeur prévue pour démarrer ($ -1 $, $ 0 $, ou toute autre valeur choisie) [éventuellement, on met à jour le nombre d'objets dans le tableau en le fixant à $ 0 $].
\item Sommet : on renvoie le dernier élément ajouté (cela dépend de comment le sommet a été implémenté !).
\end{itemize}

\bigskip

%%%%%%%%%%%%%%%%%%%%%%%%%%%%%%%%%%%%%%%%%%%%%%%%%%%%%%%%%%%%
%%%%%%%%%%%%%%%%%%%%%%%%%%%%%%%%%%%%%%%%%%%%%%%%%%%%%%%%%%%%
%%%%%%%%%%%%%%%%%%%%%%%%%%%%%%%%%%%%%%%%%%%%%%%%%%%%%%%%%%%%


\subsection{Rappel sur les files}

\bigskip

Les \textbf{files}, ou \textbf{queues} en anglais, sont des structures visant à stocker et rendre les données dans l'ordre d'arrivée.
Une file dispose donc d'une \textit{tête} contenant l'élément le plus ancien (inséré avant tous les autres), et une \textit{queue} contenant l'élément inséré le plus récemment.
Ces structures sont aussi appelées \textbf{FIFO} (\textit{First In First Out}).\\

\begin{center}
\includegraphics[scale=0.75]{Cours/Files_1_Structure_Generale.png}
\end{center}

\smallskip

Deux opérations permettent d'utiliser une file :
\begin{itemize}
\item \TTBF{ENQUEUE} : permettant d'\textit{enfiler} une donnée supplémentaire dans la file
\item \TTBF{DEQUEUE} : permettant de \textit{défiler} une donnée depuis la file
\end{itemize}
On ajoute donc une donnée en l'enfilant avec un \TTBF{ENQUEUE}, celle-ci se retrouve en \textit{queue} de file, c'est-à-dire au fond de la file.
On récupère une donnée en défilant avec un \TTBF{DEQUEUE}, celle-ci se trouvait en \textit{tête} de file, c'est-à-dire qu'elle attendait son tour depuis son insertion.
On accède donc aux éléments dans l'ordre d'arrivée.\\

Voici un exemple où l'on crée une file, puis on enfile successivement $ 42 $, $ 5 $, et $ 13 $, puis, on défile une fois (pour récupérer $ 42 $), et enfin, on enfile successivement $ 37 $, $ 10 $, $ 24 $, et $ 8 $.\\

\begin{center}
\includegraphics[scale=0.65]{Cours/Files_2_Structure_Generale_Usage_pack_1.png}
\end{center}

\begin{center}
\includegraphics[scale=0.65]{Cours/Files_2_Structure_Generale_Usage_pack_2.png}
\end{center}

\begin{center}
\includegraphics[scale=0.65]{Cours/Files_2_Structure_Generale_Usage_pack_3.png}
\end{center}

\smallskip

Les files, et surtout le respect de l'ordre d'arrivée des objets, sont couramment utilisés : mise en attente de personnes face à des guichets (voitures à un péage, clients face à une caisse, etc).

En informatique, on utilisera les files pour stocker temporairement et traiter les requêtes dans leur ordre d'arrivée.
Dans le cas des \textit{schedulers} (ordonnanceurs) visant à déterminer quel processus exécuter sur le c\oe{}ur d'un processus, on ajoute parfois une priorité à chaque objet de la file.
Ceci implique de mettre à jour l'ordre des objets dans la file lors de certains évènements (par exemple lorsque l'on enfile ou défile un élément, ou après un certain temps).\\

Afin d'implémenter une file, il est donc nécessaire d'avoir un espace de stockage ordonné (un tableau numéroté ou une liste chaînée), et deux indicateurs pour l'élément tête de file et l'élément en queue de file.
Nous allons maintenant voir comment implémenter une file avec des listes chaînées et un tableau de taille fixe.

\bigskip

%%%%%%%%%%%%%%%%%%%%%%%%%%%%%%%%%%%%%%%%%%%%%%%%%%%%%%%%%%%%
%%%%%%%%%%%%%%%%%%%%%%%%%%%%%%%%%%%%%%%%%%%%%%%%%%%%%%%%%%%%
%%%%%%%%%%%%%%%%%%%%%%%%%%%%%%%%%%%%%%%%%%%%%%%%%%%%%%%%%%%%

\subsection{Files : implémentation avec un tableau de taille fixe}

\bigskip

Une implémentation avec un tableau de taille fixe impose cette fois une limitation : la file aura une taille maximale, et on peut refuser l'ajout d'un élément si la file est déjà pleine.
La structure diffère également du fait que le tableau est alloué une seule fois lors de sa création (voire même lors de la compilation dans le cas statique).

Le schéma suivant présente la structure générale :\\

\begin{center}
\includegraphics[scale=1]{Cours/Files_5_Tableau_Statique_Structure.png}
\end{center}

\smallskip

On notera cette fois que plusieurs informations distinctes doivent être conservées : l'adresse du tableau, le numéro de case correspondant à la tête de la file (\textit{head}), le numéro de case correspondant à la queue de la file (\textit{tail}), la taille du tableau (le nombre maximum d'objets pouvant être stockés), le nombre d'éléments dans le tableau.

Le schéma suivant détaille certaines informations de façon plus explicite :\\

\begin{center}
\includegraphics[scale=1]{Cours/Files_5_Tableau_Statique_Structure_Detaillee_1.png}
\end{center}

\smallskip

Une façon d'implémenter le tableau permet de supposer que la case $ 0 $ contiendra toujours l'élément de tête.
Dans ce cas très précis, on peut donc se passer de la variable de tête, et au contraire, s'appuyer sur la variable donnant le nombre d'éléments dans le tableau pour savoir si la file est vide ou non.
Cette implémentation ressemble donc à cela :\\

\begin{center}
\includegraphics[scale=1]{Cours/Files_5_Tableau_Statique_Structure_Detaillee_2.png}
\end{center}

\smallskip

Dans le cas d'un tableau de taille fixe, les pointeurs de tête et de queue ne peuvent pas utiliser la valeur \TTBF{NULL} comme indicateur de tableau vide, car cette valeur est égale à $ 0 $ (ce qui laisserait à penser que la queue est effectivement à la case 0).
Plusieurs solutions sont possibles pour indiquer les cases où se trouvent la tête et la queue de la file, ainsi que le cas vide :

\begin{itemize}
\item On enregistre dans la structure de la file une variable servant à compter le nombre d'éléments présents (la queue peut donc prendre n'importe quelle valeur tant que la file est vide).
\item On utilise un entier relatif pour indiquer la position, et $ -1 $ indique que la file est vide (l'ajout d'un élément décalera la tête et la queue à $ 0 $, c'est-à-dire la case où sera l'élément).
\item On place la queue de la file sur la première case non utilisée, et l'accès au dernier élément se fait donc en retirant $ 1 $ au pointeur de sommet (ainsi, une queue à la case $ 0 $ indique que la file est vide).
Attention : dans ce cas précis, un tableau plein aura un sommet hors des cases du tableau (il ne faudra donc \textit{jamais} le déréférencer s'il atteinte une telle valeur).
\item On représente complètement différemment la file : on décale les pointeurs de tête et de queue au fur et à mesure des insertions et suppressions ($ +1 $ / $ -1 $). Le pointeur de queue peut passer de la dernière case à la première, car les éléments actuellement dans la file se trouvent uniquement entre la tête et la queue.
Vider ce tableau revient uniquement à passer les pointeurs à la valeur $ -1 $.
Attention : dans ce cas précis, il est nécessaire de faire très attention à l'ordre de lecture des éléments entre la tête et la queue.
\end{itemize}

\smallskip

L'exemple suivant montre l'évolution d'une file implémentée avec un tableau fixe au fur et à mesure des ajouts (enfiler / \TTBF{ENQUEUE}) et suppressions (défiler / \TTBF{DEQUEUE}).\\

\begin{center}
\includegraphics[scale=0.65]{Cours/Files_6_Tableau_Statique_Usage_pack_1.png}
\end{center}

\begin{center}
\includegraphics[scale=0.65]{Cours/Files_6_Tableau_Statique_Usage_pack_2.png}
\end{center}

\smallskip

Dans les schémas suivants, deux versions sont présentées :
\begin{itemize}
\item celui de gauche présente la version standard où la tête ne bouge pas (il est nécessaire de décaler l'ensemble des éléments du tableau dès que l'on défile),
\item celui de droite présente la version où seuls les pointeurs de tête et de queue sont décalés (il faut faire attention à l'ordre de lecture entre la tête et la file, et s'appuyer sur les modulos).
\end{itemize}

\begin{center}
\includegraphics[scale=0.65]{Cours/Files_6_Tableau_Statique_Usage_pack_2_double.png}
\end{center}

\begin{center}
\includegraphics[scale=0.65]{Cours/Files_6_Tableau_Statique_Usage_pack_3.png}
\end{center}

\begin{center}
\includegraphics[scale=0.65]{Cours/Files_6_Tableau_Statique_Usage_pack_4.png}
\end{center}

\begin{center}
\includegraphics[scale=0.65]{Cours/Files_6_Tableau_Statique_Usage_pack_5.png}
\end{center}

\smallskip

Dans ce dernier cas, lorsque le tableau de gauche était complètement rempli, on a décalé le pointeur de queue au début du tableau, simplement en utilisant un modulo de la taille du tableau.
Il est important de respecter l'ordre de lecture : on lit bel et bien depuis le pointeur de tête, en effectuant un $ +1 $ (lecture vers la gauche) tout en appliquant un modulo de la taille du tableau au résultat, jusqu'à atteindre le pointeur de queue.

On peut également comprendre qu'il y a eu un dépassement en comparant la position du pointeur de tête et de queue : la différence entre leurs positions est négative ! \linebreak
($ pos. tail - pos. head = 0 - 1 = -1$)

\bigskip

Cette autre façon de gérer la file est un tout petit peu plus complexe dans l'écriture des algorithmes de gestion, mais elle évite énormément de réécritures dans le tableau/en mémoire (inutile de lire une case, l'écrire ailleurs, et recommencer ainsi de suite lors de chaque \textit{dequeue}).
Pour ce TP, vous êtes libres de choisir l'implémentation que vous souhaitez réaliser.

\bigskip

Les principales opérations se résument ainsi :
\begin{itemize}
\item Création : on alloue en mémoire le tableau (sauf s'il est statique), et on fixe la tête et la queue de la file à la valeur prévue pour démarrer ($ -1 $, $ 0 $, ou toute autre valeur choisie) [éventuellement, on met à jour le nombre d'objets dans le tableau en le fixant à $ 0 $].
\item Enfiler : si le tableau est plein, on retourne une erreur, sinon, on ajoute le nouvel élément à gauche de la position du pointeur de queue, et on décale la queue de la file d'un cran à gauche [éventuellement, on met à jour le nombre d'objets dans le tableau].
Autre version : on ajoute le nouvel élément dans la case à gauche de la position du pointeur de queue modulo la taille du tableau, et on décale la queue de la file.
\item Défiler : si la file est vide, on retourne une erreur, sinon, on décale l'ensemble des éléments vers la droite [éventuellement, on met à jour le nombre d'objets dans le tableau].
Autre version : on décale la tête de la file d'un cran à gauche.
\item Vider : on fixe les pointeurs de tête et de queue à la valeur prévue pour démarrer ($ -1 $, $ 0 $, ou toute autre valeur choisie) [éventuellement, on met à jour le nombre d'objets dans le tableau en le fixant à $ 0 $].
\item Tête : on renvoie le contenu de la case du pointeur de tête de file (le prochain élément qui sera défilé).
\item Queue : on renvoie le contenu de la case du pointeur de queue de file (le dernier élément qui sera défilé).
\end{itemize}


\setlength{\parindent}{\defaultparindent}


\newpage

%%%%%%%%%%%%%%%
%% EXERCICES %%
%%%%%%%%%%%%%%%

%% Exercice 1
\section{Exercice 1 - Bibliothèques statique et dynamique}

\vspace*{0.7cm}

%% Exercice 1

%\ExoSpecs{\TTBF{CalculTVA.sh}}{\TTBF{\RenduDir/src/exo1/}}{750}{640}{\TTBF{write}}
\ExoSpecsCustom{\TTBF{exo1\_fun.php} [my\_Calculette(int, int, string)]}{\TTBF{\RenduDir/src/exo1/}}{}{}{Fonctions recommandées}{\TTBF{(Bases PHP)}, \TTBF{(Maths PHP)}, \TTBF{return}}

\vspace*{0.7cm}

\noindent \ExoObjectif{Le but de l'exercice est de créer une mini calculatrice en PHP.}

\bigskip

\noindent Vous devez écrire une fonction nommée \TTBF{my\_Calculette} qui prendra trois paramètres (deux nombres, puis l'opérateur), et renverra le résultat de l'opération désignée ou un message d'erreur.

\noindent Vous devez implémenter les 5 opérations suivantes : l'addition (symbole \TTBF{+}), la soustraction (symbole \TTBF{-}), la multiplication (lettre \TTBF{*}), la division (symbole \TTBF{/}), et le reste de la division euclidienne (symbole  \TTBF{\%}).

\noindent \`A la fin du calcul, votre fonction doit renvoyer le résultat.

\bigskip

\lstset{language=html}
\begin{lstlisting}[frame=single,title={Cas général PHP}]
<textarea cols="80" rows="25" readonly="readonly">
<?php
  require_once("exo1_fun.php");
  $my_text = my_Calculette(42, 38, "+");
  echo($my_text);
?>
</textarea>
\end{lstlisting}

\lstset{language=html}
\begin{lstlisting}[frame=single,title={Cas général PHP exécuté}]
<textarea cols="80" rows="25" readonly="readonly">
80</textarea>
\end{lstlisting}

\bigskip

\noindent Les deux premiers paramètres doivent être des entiers, et le troisième doit être une chaîne de caractères. Si ça n'est pas le cas, vous devez renvoyer le texte suivant.

\bigskip

\noindent \TTBF{Incorrect\textvisiblespace parameters\textvisiblespace type}

\bigskip

\lstset{language=html}
\begin{lstlisting}[frame=single,title={Cas d'erreur 1 : mauvais paramètres}]
<textarea cols="80" rows="25" readonly="readonly">
<?php
  require_once("exo1_fun.php");
  $my_text = my_Calculette("+", 38, 42);
  echo($my_text);
?>
</textarea>
\end{lstlisting}

\lstset{language=html}
\begin{lstlisting}[frame=single,title={Cas d'erreur 1 exécuté}]
<textarea cols="80" rows="25" readonly="readonly">
Incorrect parameters type</textarea>
\end{lstlisting}

\bigskip

\noindent Si le troisième paramètre donné n'est ni un \TTBF{+}, ni un \TTBF{-}, ni un \TTBF{*}, ni un \TTBF{/}), ni un \TTBF{\%}, alors vous devez renvoyer le message d'erreur suivant.

\bigskip

\noindent \TTBF{Unknown\textvisiblespace operator}

\bigskip

\lstset{language=html}
\begin{lstlisting}[frame=single,title={Cas d'erreur 2 : opérateur inconnu}]
<textarea cols="80" rows="25" readonly="readonly">
<?php
  require_once("exo1_fun.php");
  $my_text = my_Calculette(42, 38, "A");
  echo($my_text);
?>
</textarea>
\end{lstlisting}

\lstset{language=html}
\begin{lstlisting}[frame=single,title={Cas d'erreur 2 exécuté}]
<textarea cols="80" rows="25" readonly="readonly">
Unknown operator</textarea>
\end{lstlisting}

\bigskip

\noindent Si le deuxième paramètre donné à la division est 0, vous devez renvoyer le message d'erreur suivant.

\bigskip

\noindent \TTBF{Division\textvisiblespace by\textvisiblespace 0\textvisiblespace is\textvisiblespace forbidden}

\bigskip

\lstset{language=html}
\begin{lstlisting}[frame=single,title={Cas d'erreur 3 : division par 0}]
<textarea cols="80" rows="25" readonly="readonly">
<?php
  require_once("exo1_fun.php");
  $my_text = my_Calculette(42, 0, "/");
  echo($my_text);
?>
</textarea>
\end{lstlisting}

\lstset{language=html}
\begin{lstlisting}[frame=single,title={Cas d'erreur 3 exécuté}]
<textarea cols="80" rows="25" readonly="readonly">
Division by 0 is forbidden</textarea>
\end{lstlisting}

\bigskip

%\noindent Si le deuxième paramètre donné au modulo est 0, vous devez renvoyer le message d'erreur suivant.
%
%\bigskip
%
%\noindent \TTBF{Modulo\textvisiblespace 0\textvisiblespace is\textvisiblespace forbidden}
%
%\bigskip
%
%\lstset{language=php}
%\begin{lstlisting}[frame=single,title={Cas d'erreur 3 : division par 0}]
%<textarea cols="80" rows="25" readonly="readonly">
%<?php
%  require_once("exo1_fun.php");
%  $my_text = my_Calculette(42, 0, "%");
%  echo($my_text);
%?>
%</textarea>
%\end{lstlisting}
%
%\lstset{language=php}
%\begin{lstlisting}[frame=single,title={Cas d'erreur 3 exécuté}]
%<textarea cols="80" rows="25" readonly="readonly">
%Modulo 0 is forbidden</textarea>
%\end{lstlisting}
%
%\bigskip

\noindent Si plusieurs des problèmes précédents sont rencontrés simultanément, vous devez les gérer dans cet ordre de priorité : le problème de type en priorité, l'opérateur inconnu en second, et la division par 0 en dernier.

%\noindent Si plusieurs des problèmes précédents sont rencontrés simultanément, vous devez les gérer dans cet ordre de priorité : le problème de type en priorité, l'opérateur inconnu en second, et enfin le modulo ou la division par 0 en dernier.

\bigskip

\lstset{language=html}
\begin{lstlisting}[frame=single,title={Cas d'erreurs}]
<textarea cols="80" rows="25" readonly="readonly">
<?php
  require_once("exo1_fun.php");
  $my_text = my_Calculette("A", 42, 0);
  echo($my_text);
?>
</textarea>
\end{lstlisting}

\lstset{language=html}
\begin{lstlisting}[frame=single,title={Cas d'erreurs exécuté}]
<textarea cols="80" rows="25" readonly="readonly">
Incorrect parameters type</textarea>
\end{lstlisting}

\lstset{language=html}
\begin{lstlisting}[frame=single,title={Cas d'erreurs}]
<textarea cols="80" rows="25" readonly="readonly">
<?php
  require_once("exo1_fun.php");
  $my_text = my_Calculette(42, 0, "A");
  echo($my_text);
?>
</textarea>
\end{lstlisting}

\lstset{language=html}
\begin{lstlisting}[frame=single,title={Cas d'erreurs exécuté}]
<textarea cols="80" rows="25" readonly="readonly">
Unknown operator</textarea>
\end{lstlisting}


\newpage

%% Exercice 2
\section{Exercice 2 - Pile avec liste chaînée}

\vspace*{0.7cm}

%% Exercice 2

%\ExoSpecs{\TTBF{CalculTVA.sh}}{\TTBF{\RenduDir/src/exo1/}}{750}{640}{\TTBF{write}}
%\ExoSpecsCustom{\TTBF{my\_calc}}{\TTBF{\RenduDir/src/my\_calc.c}}{750}{640}{Commandes autorisées}{\TTBF{sh}, \TTBF{echo}, \TTBF{exit}}
\ExoSpecsSimple{\TTBF{my\_calc.c}}{\TTBF{\RenduDir/src/my\_calc.c}}{750}{640}

\vspace*{0.7cm}

\noindent \ExoObjectif{Le but de l'exercice est de créer une mini calculatrice en C.}

\bigskip

\noindent Vous devez écrire un programme qui prendra trois paramètres (deux nombres, puis l'opérateur), et affichera le résultat de l'opération désignée.

\noindent Vous devez implémenter les 5 opérations suivantes : l'addition (symbole \TTBF{+}), la soustraction (symbole \TTBF{-}), la multiplication (lettre \TTBF{x}), la division entière (symbole \TTBF{/}), et le reste de la division euclidienne (symbole \TTBF{\%}).

\noindent \`A la fin du calcul, votre programme doit renvoyer 0.

\bigskip

\lstset{language=sh}
\begin{lstlisting}[frame=single,title={Cas général}]
$ ./my_calc 1 3 +
4
$ echo $?
0
$ ./my_calc 144 362 -
-218
$ echo $?
0
$ ./my_calc 6 7 x
42
$ echo $?
0
$ ./my_calc 13 3 /
4
$ echo $?
0
$ ./my_calc 69 2 %
1
$ echo $?
0
\end{lstlisting}

\bigskip

\noindent Si des paramètres manquent, vous devez écrire le message suivant, et renvoyer 1.

\bigskip

\noindent \TTBF{Not\textvisiblespace enough\textvisiblespace parameters.}

\noindent \TTBF{Usage:\textvisiblespace ./my\_calc\textvisiblespace number\textvisiblespace number\textvisiblespace operator}

\noindent \TTBF{Operator\textvisiblespace might\textvisiblespace be\textvisiblespace :\textvisiblespace +\textvisiblespace -\textvisiblespace x\textvisiblespace /\textvisiblespace \%}

\bigskip

\lstset{language=sh}
\begin{lstlisting}[frame=single,title={Cas d'erreur 1 : pas assez de paramètres}]
$ ./my_calc
Not enough parameters.
Usage: ./my_calc number number operator
Operator might be : + - x / %
$ echo $?
1
$ ./my_calc 0 4
Not enough parameters.
Usage: ./my_calc number number operator
Operator might be : + - x / %
$ echo $?
1
\end{lstlisting}

\bigskip

\noindent Si des paramètres sont en trop, vous devez écrire le message suivant, et renvoyer 2.

\bigskip

\noindent \TTBF{Too\textvisiblespace much\textvisiblespace parameters.}

\noindent \TTBF{Usage:\textvisiblespace ./my\_calc\textvisiblespace number\textvisiblespace number\textvisiblespace operator}

\noindent \TTBF{Operator\textvisiblespace might\textvisiblespace be\textvisiblespace :\textvisiblespace +\textvisiblespace -\textvisiblespace x\textvisiblespace /\textvisiblespace \%}

\bigskip

\lstset{language=sh}
\begin{lstlisting}[frame=single,title={Cas d'erreur 2 : trop de paramètres}]
$ ./my_calc 0 4 x 45
Too much parameters.
Usage: ./my_calc number number operator
Operator might be : + - x / %
$ echo $?
2
\end{lstlisting}

\bigskip

\noindent Si l'opérateur n'est pas placé après les nombres/en dernier/à la troisième position, vous devez indiquer le message suivant, et renvoyer 3.

\bigskip

\noindent \TTBF{Wrong\textvisiblespace parameters.}

\noindent \TTBF{Usage:\textvisiblespace ./my\_calc\textvisiblespace number\textvisiblespace number\textvisiblespace operator}

\noindent \TTBF{Operator\textvisiblespace might\textvisiblespace be\textvisiblespace :\textvisiblespace +\textvisiblespace -\textvisiblespace x\textvisiblespace /\textvisiblespace \%}

\bigskip

\lstset{language=sh}
\begin{lstlisting}[frame=single,title={Cas d'erreur 3 : pas les bons paramètres}]
$ ./my_calc x x x
Wrong parameters.
Usage: ./my_calc number number operator
Operator might be : + - x / %
$ echo $?
3
$ ./my_calc 1 2 3
Wrong parameters.
Usage: ./my_calc number number operator
Operator might be : + - x / %
$ echo $?
3
$ ./my_calc 1 + 2
Wrong parameters.
Usage: ./my_calc number number operator
Operator might be : + - x / %
$ echo $?
3
\end{lstlisting}

\bigskip

\noindent Si le deuxième paramètre donné à la division est 0, vous devez renvoyer 4 et écrire le message d'erreur suivant.

\bigskip

\noindent \TTBF{Division\textvisiblespace by\textvisiblespace 0\textvisiblespace is\textvisiblespace forbidden.}

\bigskip

\lstset{language=sh}
\begin{lstlisting}[frame=single,title={Cas d'erreur 4 : division par 0}]
$ ./my_calc 1 0 /
Division by 0 is forbidden.
$ echo $?
4
\end{lstlisting}

\bigskip

\noindent Si plusieurs des problèmes précédents sont rencontrés simultanément, vous devez les gérer dans cet ordre de priorité : le manque de paramètres est affiché en priorité, l'excès de paramètres est affiché en seconde priorité, le mauvais ordre/mauvaise nature de paramètres en troisième, et la division par 0 en dernier.

\bigskip

\lstset{language=sh}
\begin{lstlisting}[frame=single,title={Cas d'erreurs : ordre des erreurs}]
$ ./my_calc 1 0
Not enough parameters.
Usage: ./my_calc number number operator
Operator might be : + - x / %
$ echo $?
1
$ ./my_calc 1 0 0 /
Too much parameters.
Usage: ./my_calc number number operator
Operator might be : + - x / %
$ echo $?
2
$ ./my_calc 1 / 0
Wrong parameters.
Usage: ./my_calc number number operator
Operator might be : + - x / %
$ echo $?
3
$ ./my_calc 1 0 /
Division by 0 is forbidden.
$ echo $?
4
\end{lstlisting}

\bigskip

\noindent Les paramètres donnés lors des tests/de la correction seront toujours des nombres ou des opérateurs.
Vous n'avez pas à gérer les cas où des caractères sont donnés en paramètres.

\bigskip

\begin{RedBoxTitle}{ATTENTION}
    Il est formellement interdit d'utiliser le programme \TTBF{bc} ou tout autre programme ou bibliothèque de calcul.
%    (se référer à la section "commandes autorisées")
\end{RedBoxTitle}

\newpage

%% Exercice 3
\section{Exercice 3 - Pile avec tableau}

\vspace*{0.7cm}

%% Exercice 3

%\ExoSpecs{\TTBF{CalculTVA.sh}}{\TTBF{\RenduDir/src/exo1/}}{750}{640}{\TTBF{write}}
\ExoSpecsCustom{\TTBF{stack\_linked.c}}{\TTBF{\RenduDir/src/}}{750}{640}{Fonctions autorisées}{\TTBF{malloc(3)}, \TTBF{free(3)}, \TTBF{printf(3)}}

\vspace*{0.7cm}

\noindent \ExoObjectif{Le but de l'exercice est d'implémenter une des structures de base : les piles à base de listes chaînées.}

\bigskip

%\noindent Les fonctions demandées dans cet exercice devront se trouver dans une bibliothèque nommée \TTBF{libmystack}.
%Après un appel à la commande \texttt{make} à la racine du projet, il faut que votre chaîne de compilation produise à la racine de votre projet une version statique de la bibliothèque (qui se nommera \TTBF{libmystack.a}) ainsi qu'une version dynamique de la bibliothèque (qui se nommera \TTBF{libmystack.so}).
%
%\bigskip

\noindent Vous devez écrire plusieurs fonctions permettant de gérer des piles à base de listes chaînées, c'est-à-dire des piles exploitant des listes à base de pointeurs.
Un fichier \TTBF{stack\_linked.h} contenant toutes les fonctions exportables à implémenter vous est fourni en annexe.
Vous devrez réutiliser la structure \TTBF{list\_linked} telle quelle issue du fichier \TTBF{list\_linked.h} : \textbf{vous ne devez pas créer de nouvelle structure}.
La liste chaînée ainsi représentée contient deux champs exactement : \textit{elt} (l'élément inséré) et \textit{next} (pointeur vers le maillon suivant de la chaîne).

\smallskip

%\noindent Conceptuellement, les fonctions manipulant des piles de type \TTBF{stack\_ll*} devront pouvoir gérer ces 3 cas :

%\bigskip

%\begin{center}
%\includegraphics[scale=0.85]{Cours/Piles_Implementation_LL.png}
%\end{center}

\bigskip
%\newpage

\noindent Vous devez implémenter les fonctions suivantes :

\bigskip
%\medskip

\lstset{language=C}
%\begin{lstlisting}[frame=single,title={Liste des fonctions pour une pile avec liste chaînée}]
\begin{lstlisting}[frame=single]
list_linked *push_stack_linked(list_linked *list,
                               int elt);
list_linked *pop_stack_linked(list_linked *list);

int length_stack_linked(list_linked *list);
int is_empty_stack_linked(list_linked *list);

void print_stack_linked(list_linked *list);

int get_head_stack_linked(list_linked *list);

int clear_stack_linked(list_linked *list);
\end{lstlisting}


\clearpage


\subsubsection*{\TTBF{list\_linked *push\_stack\_linked(list\_linked *list, int elt)}}

\noindent Cette fonction empile un nouvel élément, c'est-à-dire qu'elle ajoute un élément en tête de la pile, c'est-à-dire qu'elle ajoute un élément en première position de la liste chaînée donnée en paramètre.

\smallskip

\noindent Si la pile est vide, il faut y créer un premier élément.

\noindent Si l'élément \textit{elt} donné en paramètre est inférieur à $ 1 $, la fonction doit renvoyer un pointeur \TTBF{NULL} sans rien modifier dans la liste.

\smallskip

\noindent La fonction doit retourner un pointeur vers la tête de liste, ou vers l'éventuelle nouvelle tête de liste si celle-ci a été modifiée.
%
En cas d'erreur (pas assez de mémoire), cette fonction doit renvoyer un pointeur \TTBF{NULL}.
%\textit{Vous n'avez pas à allouer d'espace pour stocker l'élément \TTBF{elt} : c'est à l'utilisateur de votre bibliothèque de le faire.}

%\smallskip
%
%\noindent Si plusieurs cas d'erreur se produisent simultanément, leur gestion doit se faire dans cet ordre précisément : le test de la valeur de l'élément, puis en dernier la mémoire insuffisante.

\bigskip


\subsubsection*{\TTBF{list\_linked *pop\_stack\_linked(list\_linked *list)}}

\noindent Cette fonction dépile l'élément en tête, c'est-à-dire qu'elle supprime l'élément en première position de la liste chaînée donnée en paramètre.

\smallskip

\noindent Si la liste donnée en paramètre est vide, cette fonction doit renvoyer un pointeur \TTBF{NULL}.

\smallskip

\noindent La fonction doit retourner un pointeur vers la tête de liste, ou vers l'éventuelle nouvelle tête de liste si celle-ci a été modifiée.

\bigskip


\subsubsection*{\TTBF{int length\_stack\_linked(list\_linked *list)}}

\noindent Cette fonction renvoie la taille de la pile donnée en paramètre, c'est-à-dire le nombre d'éléments présents dans la liste.

\smallskip

\noindent Si la liste donnée en paramètre est \TTBF{NULL}, la fonction doit renvoyer $ 0 $.

\bigskip


\subsubsection*{\TTBF{int is\_empty\_stack\_linked(list\_linked *list)}}

\noindent Cette fonction teste si la pile est vide ou non.

\smallskip

\noindent Si la liste est vide, c'est-à-dire si le pointeur est \TTBF{NULL}, il faut renvoyer $ 1 $ (l'équivalent de \textit{true} en C), sinon, si la liste n'est pas vide, il faut renvoyer $ 0 $ (l'équivalent de \textit{faux} en C).

\bigskip


\subsubsection*{\TTBF{void print\_stack\_linked(list\_linked *list)}}

\noindent Cette procédure affiche les éléments de la pile depuis la tête.
Chaque élément doit être suivi d'un retour à la ligne.

\smallskip

\noindent Si la liste donnée en paramètre est vide, la procédure ne fait rien.

\noindent Le format attendu est le suivant :

\bigskip

\noindent \TTBF{\textit{elt}\textbackslash{}n}

\bigskip

\noindent Ce qui donnerait cet affichage pour la liste suivante :

\clearpage

\begin{table}[ht!]
  \centering
  \begin{minipage}{0.45\textwidth}
    \centering

\lstset{language=sh}
\begin{lstlisting}[frame=single]
$ ./liste_ll_example1
42
21
8
24
64
$
\end{lstlisting}

  \end{minipage}
  \hfillx
  \begin{minipage}{0.45\textwidth}
    \centering

\begin{tabular}{m{1cm} C{0.5cm} C{0.5cm} C{0.5cm} C{0.5cm} C{0.5cm} }
pos & 1 & 2 & 3 & 4 & 5 \\
\end{tabular}

\begin{tabular}{m{1cm}|C{0.5cm}|C{0.5cm}|C{0.5cm}|C{0.5cm}|C{0.5cm}|}
\cline{2-6}
elt & 42 & 21 & 8 & 24 & 64 \\
\cline{2-6}
\end{tabular}

  \end{minipage}
\end{table}

\vspace*{-0.5cm}


\subsubsection*{\TTBF{int get\_head\_stack\_linked(list\_linked *list)}}

\noindent Cette fonction renvoie l'élément en tête de la pile, c'est-à-dire celui à la première position.

\smallskip

\noindent Si la liste donnée en paramètre est \TTBF{NULL}, la fonction doit renvoyer $ -1 $.

\bigskip


\subsubsection*{\TTBF{int clear\_stack\_linked(list\_linked *list)}}

\noindent Cette fonction vide la pile de tous ses éléments, puis renvoie le nombre d'éléments qui ont été supprimés.

\smallskip

\noindent Si la liste donnée en paramètre est \TTBF{NULL}, la fonction doit renvoyer $ 0 $.


\newpage

%% Exercice 4
\section{Exercice 4 - Suite de tests}

\vspace*{0.7cm}

%% Exercice 4

%\ExoSpecs{\TTBF{CalculTVA.sh}}{\TTBF{\RenduDir/src/exo1/}}{750}{640}{\TTBF{write}}
\ExoSpecsCustom{\TTBF{exo4\_fun.php} [my\_TicTacToe(array)]}{\TTBF{\RenduDir/src/exo4/}}{}{}{Fonctions recommandées}{\TTBF{(Bases PHP)}, \TTBF{return}}

\vspace*{0.7cm}

\noindent \ExoObjectif{Le but de l'exercice est de déterminer quel est le gagnant d'une partie de morpion (ou \textit{Tic Tac Toe} en angais).}

\bigskip

\noindent Les règles sont relativement simples : chaque joueur place un \TTBF{X} ou un \TTBF{O} à chaque tour dans un tableau de 3 cases sur 3 cases, le premier joueur alignant trois fois son symbole a gagné (une ligne verticale, horizontale, ou en diagonale).
Il est relativement simple de faire un match nul (aucun joueur n'arrive à aligner trois fois son symbole).

\lstset{language=sh}
\begin{lstlisting}[frame=single,title={Exemple de jeu où le joueur X a gagné par la diagonale}]
X O X
O X O
X O O
\end{lstlisting}

\noindent Vous devez écrire une fonction nommée \TTBF{my\_TicTacToe} qui prendra en paramètre un tableau de tableaux.
La fonction renverra tout d'abord le tableau de jeu, puis elle renverra le gagnant.

\bigskip

\noindent Exemple d'entrée :

\lstset{language=php}
\begin{lstlisting}[frame=single,title={Exemple de tableau de jeu (exo4\_data.php)}]
$line1 = array("X", "O", "X");
$line2 = array("O", "X", "O");
$line3 = array("X", "O", "O");

$game[] = $line1;
$game[] = $line2;
$game[] = $line3;
\end{lstlisting}

\lstset{language=php}
\begin{lstlisting}[frame=single,title={Appel de la fonction}]
<textarea cols="80" rows="25" readonly="readonly">
<?php
  require_once("exo4_data.php");
  require_once("exo4_fun.php");
  $my_text = my_TicTacToe($game);
  echo($my_text);
?>
</textarea>
\end{lstlisting}

\bigskip

\noindent Si un joueur est gagnant, vous devez l'indiquer avec les chaînes de caractères :

\bigskip

\noindent \TTBF{Player\textvisiblespace X\textvisiblespace won}

\noindent \TTBF{Player\textvisiblespace O\textvisiblespace won}

\bigskip

\noindent Si aucun joueur n'est gagnant, vous devez l'indiquer ainsi :

\bigskip

\noindent \TTBF{It's\textvisiblespace a\textvisiblespace draw}

\bigskip

\lstset{language=php}
\begin{lstlisting}[frame=single,title={Cas général gagnant}]
<textarea cols="80" rows="25" readonly="readonly">
XOX
OXO
XOO
Player X won</textarea>
\end{lstlisting}

\lstset{language=php}
\begin{lstlisting}[frame=single,title={Cas général en match nul}]
<textarea cols="80" rows="25" readonly="readonly">
OXX
XXO
OOX
It's a draw</textarea>
\end{lstlisting}

\bigskip

\noindent Un cas d'erreur doit être géré avant tout affichage.
Si le tableau contient des caractères autres que \TTBF{X} ou \TTBF{O}, il faut indiquer le message suivant :

\bigskip

\noindent \TTBF{Incorrect\textvisiblespace game}

\bigskip

\lstset{language=php}
\begin{lstlisting}[frame=single,title={Exemple de tableau de jeu incorrect (exo4\_data.php)}]
$line1 = array("X", "O", "A");
$line2 = array("O", "X", "O");
$line3 = array("O", "O", "O");

$game[] = $line1;
$game[] = $line2;
$game[] = $line3;
\end{lstlisting}

\lstset{language=php}
\begin{lstlisting}[frame=single,title={Cas d'erreur}]
<textarea cols="80" rows="25" readonly="readonly">
Incorrect game</textarea>
\end{lstlisting}

\bigskip

\begin{RedBoxTitle}{ATTENTION}
    Les retours à la ligne ne doivent pas être faits avec la balise \TTBF{"<br />"}, mais avec \TTBF{"\textbackslash n"}.
    (se référer à la section \hyperref[sec:AideMemoire]{Aide Mémoire})
\end{RedBoxTitle}


\newpage
%% Exercice 5
\section{Exercice 5 - Pile avec tableau statique (bonus)}

\vspace*{0.7cm}

%% Exercice 5

%\ExoSpecs{\TTBF{CalculTVA.sh}}{\TTBF{\RenduDir/src/exo1/}}{750}{640}{\TTBF{write}}
\ExoSpecsCustom{\TTBF{stack\_static.c}}{\TTBF{\RenduDir/src/}}{750}{640}{Fonctions autorisées}{\TTBF{write(2)}}

\vspace*{0.7cm}

\noindent \ExoObjectif{Le but de l'exercice est d'implémenter une pile en C utilisant un tableau statique et sans jamais modifier le tas du processus (c'est-à-dire sans utiliser \texttt{malloc(3)}, \texttt{mmap(2)}, ou tout autre appel système ou fonction modifiant le tas ou réservant des pages mémoire lors de l'exécution).}

\bigskip

\noindent Les fonctions demandées dans cet exercice devront également se trouver dans la bibliothèque nommée \TTBF{libmystack}.

\bigskip

\noindent Vous devez écrire plusieurs fonctions permettant de créer, utiliser, vider, et libérer une pile.
Un fichier \TTBF{stack\_static.h} contenant toutes les fonctions exportables à implémenter vous est fourni en annexe.
Vous devez déclarer une structure \TTBF{stack\_static} et l'ajouter dans \TTBF{stack\_static.h}.
N'oubliez pas qu'un tableau statique est généré par le compilateur : vous devrez indiquer dans la constante de pré-compilation \TTBF{STACK\_STATIC\_MAX\_LEN} la taille maximum.
La pile étant statique, il vous faudra déclarer une variable globale statique nommée \TTBF{g\_stack\_static} qui pointera vers la structure (elle-même déclarée dans la variable globale nommée \TTBF{g\_s\_stack\_static}).
Pour les premières étapes, vous devrez implémenter une version simplifiée de la pile qui ne prend en charge que des entiers positifs.

\noindent \textit{Attention : étant donné qu'il s'agit d'une version statique, vous ne devez \textbf{JAMAIS} utiliser \TTBF{malloc}, \TTBF{free}, ou toute autre fonction ou appel système visant à réserver de la mémoire.}

%\noindent Vous devez implémenter les 11 fonctions suivantes :
%\begin{itemize}
%\item \TTBF{void stack\_static\_create(void)}
%\item \TTBF{void stack\_static\_delete(void)}
%\item \TTBF{int stack\_static\_length(void)}
%\item \TTBF{int stack\_static\_max\_length(void)
%\item \TTBF{int stack\_static\_push(int elt)}
%\item \TTBF{int stack\_static\_pop(void)}
%\item \TTBF{int stack\_static\_head(void)}
%\item \TTBF{int stack\_static\_clear(void)}
%\item \TTBF{int stack\_static\_is\_empty(void)}
%\item \TTBF{int stack\_static\_search(int elt)}
%\item \TTBF{void stack\_static\_reverse(void)}
%\item \TTBF{void stack\_static\_print(void)}
%\end{itemize}

\noindent Vous devez implémenter les fonctions suivantes :

\bigskip

\lstset{language=C}
\begin{lstlisting}[frame=single,title={Liste des fonctions pour une pile avec liste chaînée}]
void stack_static_create(void);
void stack_static_delete(void);

int stack_static_length(void);
int stack_static_max_length(void);

int stack_static_push(int elt);
int stack_static_pop(void);
int stack_static_head(void);

int stack_static_clear(void);
int stack_static_is_empty(void);

int stack_static_search(int elt);
void stack_static_reverse(void);
void stack_static_print(void);
\end{lstlisting}


\subsubsection*{\TTBF{void stack\_static\_create(void)}}

\noindent Cette fonction initialise les valeurs de la structure.


\subsubsection*{\TTBF{void stack\_static\_delete(void)}}

\noindent Cette fonction vide une pile de l'ensemble de ses éléments.


\subsubsection*{\TTBF{int stack\_static\_max\_length(void)}}

\noindent Cette fonction renvoie la longueur du tableau contenant la pile.


\subsubsection*{\TTBF{int stack\_static\_length(void)}}

\noindent Cette fonction renvoie la longueur de la pile (c'est-à-dire le nombre d'éléments actuellement dans la pile).


\subsubsection*{\TTBF{int stack\_static\_push(int elt)}}

\noindent Cette fonction empile un élément dans une pile, c'est-à-dire qu'elle ajoute un élément au sommet.
En cas de succès, la fonction renvoie $ 0 $.
Si le nombre donné en paramètre est inférieur à $ 0 $, la fonction renvoie $ -4 $.
Si le tableau est déjà plein, la fonction renvoie $ -3 $.


\subsubsection*{\TTBF{int stack\_static\_pop(void)}}

\noindent Cette fonction dépile un élément d'une pile, c'est-à-dire qu'elle supprime l'élément au sommet.
En cas de succès, la fonction renvoie $ 0 $.
Si la pile est vide, la fonction renvoie $ -2 $.


\subsubsection*{\TTBF{int stack\_static\_head(void)}}

\noindent Cette fonction renvoie l'élément au sommet de la pile.
Si la pile est vide, la fonction renvoie $ -2 $.


\subsubsection*{\TTBF{int stack\_static\_clear(void)}}

\noindent Cette fonction vide une pile de l'ensemble de ses éléments, sans détruire la structure de la pile.
La fonction renvoie le nombre d'éléments supprimés de la mémoire.
Si la pile est vide, la fonction renvoie $ 0 $.


\subsubsection*{\TTBF{int stack\_static\_is\_empty(void)}}

\noindent Cette fonction teste si une pile est vide ou non.
Si la pile est vide, la fonction renvoie $ 1 $.
Si la pile n'est pas vide, la fonction renvoie $ 0 $.


\subsubsection*{\TTBF{int stack\_static\_search(int elt)}}

\noindent Cette fonction recherche un élément dans la pile et renvoie sa position dans le tableau.
La première position est celle où l'élément le plus ancien a été placé (c'est-à-dire le fond de la pile), cette position sera numérotée $ 0 $.
Si l'élément n'est pas trouvé, la fonction renvoie $ -4 $.


\subsubsection*{\TTBF{void stack\_static\_reverse(void)}}

\noindent Cette fonction inverse la position de tous les éléments de la pile.
Le premier élément devient le dernier, l'avant dernier devient le deuxième, etc.
\textit{Attention : vous ne devez pas utiliser de \TTBF{malloc} ou de tableau temporaire pour effectuer cette fonction.}

\subsubsection*{\TTBF{void stack\_static\_print(void)}}

\noindent Cette fonction affiche le contenu de la pile.
Le format d'affichage attendu implique d'afficher un seul élément par ligne, suivi d'un retour à la ligne.
L'élément en tête de pile sera affiché en premier.
Si la pile est vide, seul un retour à la ligne est affiché.
\textit{Attention, en version statique, vous devrez utiliser \texttt{write(2)} et non pas \texttt{printf(3)}.}

\bigskip

\lstset{language=sh}
\begin{lstlisting}[frame=single,title={Exemple d'affichage du cas normal : pile contenant 42, 5, 13}]
$ ./my_stack_static
42
5
13

$
\end{lstlisting}

\bigskip

\lstset{language=sh}
\begin{lstlisting}[frame=single,title={Exemple d'affichage d'une pile vide}]
$ ./my_stack_static

$
\end{lstlisting}


\end{document}
