%% Exercice 5

%\ExoSpecs{\TTBF{CalculTVA.sh}}{\TTBF{\RenduDir/src/exo1/}}{750}{640}{\TTBF{write}}
\ExoSpecsCustom{\TTBF{stack\_static.c}}{\TTBF{\RenduDir/src/}}{750}{640}{Fonctions autorisées}{\TTBF{write(2)}}

\vspace*{0.7cm}

\noindent \ExoObjectif{Le but de l'exercice est d'implémenter une pile en C utilisant un tableau statique et sans jamais modifier le tas du processus (c'est-à-dire sans utiliser \texttt{malloc(3)}, \texttt{mmap(2)}, ou tout autre appel système ou fonction modifiant le tas ou réservant des pages mémoire lors de l'exécution).}

\bigskip

\noindent Les fonctions demandées dans cet exercice devront également se trouver dans la bibliothèque nommée \TTBF{libmystack}.

\bigskip

\noindent Vous devez écrire plusieurs fonctions permettant de créer, utiliser, vider, et libérer une pile.
Un fichier \TTBF{stack\_static.h} contenant toutes les fonctions exportables à implémenter vous est fourni en annexe.
Vous devez déclarer une structure \TTBF{stack\_static} et l'ajouter dans \TTBF{stack\_static.h}.
N'oubliez pas qu'un tableau statique est généré par le compilateur : vous devrez indiquer dans la constante de pré-compilation \TTBF{STACK\_STATIC\_MAX\_LEN} la taille maximum.
La pile étant statique, il vous faudra déclarer une variable globale statique nommée \TTBF{g\_stack\_static} qui pointera vers la structure (elle-même déclarée dans la variable globale nommée \TTBF{g\_s\_stack\_static}).
Pour les premières étapes, vous devrez implémenter une version simplifiée de la pile qui ne prend en charge que des entiers positifs.

\noindent \textit{Attention : étant donné qu'il s'agit d'une version statique, vous ne devez \textbf{JAMAIS} utiliser \TTBF{malloc}, \TTBF{free}, ou toute autre fonction ou appel système visant à réserver de la mémoire.}

%\noindent Vous devez implémenter les 11 fonctions suivantes :
%\begin{itemize}
%\item \TTBF{void stack\_static\_create(void)}
%\item \TTBF{void stack\_static\_delete(void)}
%\item \TTBF{int stack\_static\_length(void)}
%\item \TTBF{int stack\_static\_max\_length(void)
%\item \TTBF{int stack\_static\_push(int elt)}
%\item \TTBF{int stack\_static\_pop(void)}
%\item \TTBF{int stack\_static\_head(void)}
%\item \TTBF{int stack\_static\_clear(void)}
%\item \TTBF{int stack\_static\_is\_empty(void)}
%\item \TTBF{int stack\_static\_search(int elt)}
%\item \TTBF{void stack\_static\_reverse(void)}
%\item \TTBF{void stack\_static\_print(void)}
%\end{itemize}

\noindent Vous devez implémenter les fonctions suivantes :

\bigskip

\lstset{language=C}
\begin{lstlisting}[frame=single,title={Liste des fonctions pour une pile avec liste chaînée}]
void stack_static_create(void);
void stack_static_delete(void);

int stack_static_length(void);
int stack_static_max_length(void);

int stack_static_push(int elt);
int stack_static_pop(void);
int stack_static_head(void);

int stack_static_clear(void);
int stack_static_is_empty(void);

int stack_static_search(int elt);
void stack_static_reverse(void);
void stack_static_print(void);
\end{lstlisting}


\subsubsection*{\TTBF{void stack\_static\_create(void)}}

\noindent Cette fonction initialise les valeurs de la structure.


\subsubsection*{\TTBF{void stack\_static\_delete(void)}}

\noindent Cette fonction vide une pile de l'ensemble de ses éléments.


\subsubsection*{\TTBF{int stack\_static\_max\_length(void)}}

\noindent Cette fonction renvoie la longueur du tableau contenant la pile.


\subsubsection*{\TTBF{int stack\_static\_length(void)}}

\noindent Cette fonction renvoie la longueur de la pile (c'est-à-dire le nombre d'éléments actuellement dans la pile).


\subsubsection*{\TTBF{int stack\_static\_push(int elt)}}

\noindent Cette fonction empile un élément dans une pile, c'est-à-dire qu'elle ajoute un élément au sommet.
En cas de succès, la fonction renvoie $ 0 $.
Si le nombre donné en paramètre est inférieur à $ 0 $, la fonction renvoie $ -4 $.
Si le tableau est déjà plein, la fonction renvoie $ -3 $.


\subsubsection*{\TTBF{int stack\_static\_pop(void)}}

\noindent Cette fonction dépile un élément d'une pile, c'est-à-dire qu'elle supprime l'élément au sommet.
En cas de succès, la fonction renvoie $ 0 $.
Si la pile est vide, la fonction renvoie $ -2 $.


\subsubsection*{\TTBF{int stack\_static\_head(void)}}

\noindent Cette fonction renvoie l'élément au sommet de la pile.
Si la pile est vide, la fonction renvoie $ -2 $.


\subsubsection*{\TTBF{int stack\_static\_clear(void)}}

\noindent Cette fonction vide une pile de l'ensemble de ses éléments, sans détruire la structure de la pile.
La fonction renvoie le nombre d'éléments supprimés de la mémoire.
Si la pile est vide, la fonction renvoie $ 0 $.


\subsubsection*{\TTBF{int stack\_static\_is\_empty(void)}}

\noindent Cette fonction teste si une pile est vide ou non.
Si la pile est vide, la fonction renvoie $ 1 $.
Si la pile n'est pas vide, la fonction renvoie $ 0 $.


\subsubsection*{\TTBF{int stack\_static\_search(int elt)}}

\noindent Cette fonction recherche un élément dans la pile et renvoie sa position dans le tableau.
La première position est celle où l'élément le plus ancien a été placé (c'est-à-dire le fond de la pile), cette position sera numérotée $ 0 $.
Si l'élément n'est pas trouvé, la fonction renvoie $ -4 $.


\subsubsection*{\TTBF{void stack\_static\_reverse(void)}}

\noindent Cette fonction inverse la position de tous les éléments de la pile.
Le premier élément devient le dernier, l'avant dernier devient le deuxième, etc.
\textit{Attention : vous ne devez pas utiliser de \TTBF{malloc} ou de tableau temporaire pour effectuer cette fonction.}

\subsubsection*{\TTBF{void stack\_static\_print(void)}}

\noindent Cette fonction affiche le contenu de la pile.
Le format d'affichage attendu implique d'afficher un seul élément par ligne, suivi d'un retour à la ligne.
L'élément en tête de pile sera affiché en premier.
Si la pile est vide, seul un retour à la ligne est affiché.
\textit{Attention, en version statique, vous devrez utiliser \texttt{write(2)} et non pas \texttt{printf(3)}.}

\bigskip

\lstset{language=sh}
\begin{lstlisting}[frame=single,title={Exemple d'affichage du cas normal : pile contenant 42, 5, 13}]
$ ./my_stack_static
42
5
13

$
\end{lstlisting}

\bigskip

\lstset{language=sh}
\begin{lstlisting}[frame=single,title={Exemple d'affichage d'une pile vide}]
$ ./my_stack_static

$
\end{lstlisting}
