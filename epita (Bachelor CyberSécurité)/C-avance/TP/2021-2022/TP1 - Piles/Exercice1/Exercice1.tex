%% Exercice 1

%\ExoSpecs{\TTBF{CalculTVA.sh}}{\TTBF{\RenduDir/src/exo1/}}{750}{640}{\TTBF{write}}
\ExoSpecsCustom{\TTBF{Makefile}}{\TTBF{\RenduDir/}}{750}{640}{Outils recommandés}{\TTBF{gcc(1)}, \TTBF{ar(1)}}

\vspace*{0.7cm}

\noindent \ExoObjectif{Le but de l'exercice est de faire fonctionner l'ensemble de votre projet avec un makefile simple, et de produire une bibliothèque statique ainsi qu'une dynamique.}

\bigskip

\noindent Une bibliothèque est un ensemble de fonctions et procédures prêtes à être utilisées (ayant éventuellement des variables statiques, ou encore faisant appel à \texttt{malloc(3)}) par un utilisateur ne connaissant pas leur implémentation.
L'utilisateur développe et écrit son propre programme (utilisant un \texttt{main} pour s'exécuter) et il peut éventuellement s'appuyer sur des bibliothèques (n'ayant pas de \texttt{main}) pour réutiliser des implémentations existantes.

\bigskip

\noindent Les exercices suivants vont vous conduire à produire des fonctions manipulant des structures, afin d'offrir un service à des utilisateurs : une pile et son interface pour l'exploiter (une \texttt{API}).
Vous devez donc écrire le (ou les) \textit{makefile(s)} de votre projet et un fichier \texttt{configure} (éventuellement vide) afin de produire une bibliothèque statique nommée \TTBF{libmystack.a} et une bibliothèque dynamique nommée \TTBF{libmystack.so}.

\bigskip

\noindent Le \textit{makefile} que vous allez écrire rendra votre projet complet et autonome.
Pour cela, vous devez faire en sorte que plusieurs \textit{cibles} soient présentent dans le makefile (celles-ci sont rappelées dans le paragraphe suivant).
%Afin de rendre le projet parfaitement complet et autonome, vous devez écrire un \textit{makefile} répondant aux cibles indiquées dans les consignes générales (celles-ci sont rappelées dans le paragraphe suivant).

\bigskip

\noindent Plusieurs stratégies existent pour compiler avec des makefiles, l'une d'entre elle consiste à placer un makefile par dossier afin que le makefile principal appelle les suivants avec la bonne règle (par exemple : le makefile principal va appeler le makefile du dossier \textit{src} pour compiler le projet, mais le makefile principal appellera celui du dossier \textit{check} lorsque l'on demandera d'exécuter la suite de tests).

Pour ce premier contact avec les makefiles, vous pouvez vous contenter d'un seul makefile à la racine exécutant des lignes de compilation toutes prêtes sans aucune réécriture ou extension.

\bigskip

\noindent Afin de générer des bibliothèques statiques et dynamiques, vous devez compiler vos fichiers pour produire des fichiers \textit{objets} (des fichiers \TTBF{.o}), mais vous ne devez pas laisser l'étape d'\textit{édition de liens} classique se faire.
À la place, la cible \textit{static} de votre makefile devra produire une bibliothèque statique nommée \TTBF{libmystack.a}, et la cible \textit{shared} devra produire une bibliothèque dynamique nommée \TTBF{libmystack.so}.

\bigskip

\noindent Pour générer une bibliothèque statique (\textit{static library} en anglais), on utilise la commande \TTBF{ar} qui créée des \textit{archives}.
Pour produire la bibliothèque statique \textit{libtest.a} à partir des fichiers \textit{test1.c} et \textit{file.c}, on utilisera ces commandes :\\

\begin{tabular}{l}
\TTBF{cc -c test1.c file.c}\\
\TTBF{ar cr libtest.a test1.o file.o}\\
\end{tabular}

\bigskip

\noindent Pour générer une bibliothèque dynamique (\textit{shared library} en anglais), on utilise l'option \TTBF{-shared} du compilateur.
Pour produire la bibliothèque dynamique \textit{libtest.so} à partir des fichiers \textit{test1.c} et \textit{file.c}, on utilisera ces commandes :\\

\begin{tabular}{l}
\TTBF{cc -c test1.c file.c} \\
\TTBF{cc test1.o file.o -shared -o libtest.so} \\
\end{tabular}

\bigskip

La première commande génère des fichiers objets, et la deuxième les réunit dans un seul fichier.
Il arrivera sur certains systèmes qu'il soit nécessaire d'ajouter l'option \TTBF{-fpic} ou \TTBF{-fPIC} (\textit{Position Independent Code}) pour générer des bibliothèques dynamiques.
Gardez en tête que \textit{toutes} les bibliothèques \textbf{doivent} être préfixées par un \textit{\texttt{lib}} dans le nom des fichiers \TTBF{.a} et \TTBF{.so} générés.
Mais, lorsque l'on utiliser une bibliothèque dans un projet, on doit ignorer ce préfixe lors de la compilation et l'édition de liens.

\bigskip

Pour utiliser une bibliothèque dynamique sur votre système, plusieurs méthodes existent selon le système où vous vous trouvez.\\

\begin{itemize}
\item L'une d'entre elle consiste à ajouter le dossier contenant la bibliothèque à la variable d'environnement \TTBF{LD\_LIBRARY\_PATH} (comme pour la variable \TTBF{PATH}).
Selon votre système, il peut s'agir de la variable d'environnement \TTBF{DYLD\_LIBRARY\_PATH}, \TTBF{LIBPATH}, ou encore \TTBF{SHLIB\_PATH}.
Pour ajouter le dossier courant comme dossier contenant des bibliothèques dynamiques en plus d'autres dossiers, vous pouvez taper :\\
\TTBF{export LD\_LIBRARY\_PATH=.:/usr/lib:/usr/local/lib}
\\

\item Une autre méthode consiste à modifier le fichier \TTBF{/etc/ls.so.conf} et/ou ajouter un fichier contenant le chemin vers votre bibliothèque dans \TTBF{/etc/ld.so.conf.d/} puis à exécuter \TTBF{/sbin/ldconfig} pour recharger les dossiers à utiliser.\\

\item Enfin, vous pouvez demander les droits administrateur et installer votre bibliothèque dans \TTBF{/usr/lib} ou \TTBF{/usr/local/lib}.\\
\end{itemize}

\bigskip

\noindent \'Evidemment, à long terme, l'objectif est d'avoir une bibliothèque installée dans le répertoire \TTBF{/usr/lib}.
Mais, lors du développement, il est préférable d'utiliser la variable \TTBF{LD\_LIBRARY\_PATH}.
Pour vous assurer du fonctionnement de votre exécutable, vous pouvez utiliser \TTBF{ldd(1)} avec le chemin complet vers un exécutable.
\TTBF{ldd} vous indiquera quelles bibliothèques sont requises, et où celui-ci les trouve.
Si l'une d'entre elle n'est pas trouvée, \TTBF{ldd} vous en informera :\\

\TTBF{ldd /usr/bin/ls}

\TTBF{ldd my\_main}


\bigskip

Pour rappel voici les cibles demandées pour le \TTBF{Makefile} principal à la racine de votre projet :

\bigskip

\begin{tabular}{l p{13cm}}
\texttt{all} & \textit{[Première règle]} lance la règle \texttt{libmystack}.\\
\texttt{clean} & supprime tous les fichiers temporaires et ceux créés par le compilateur.\\
\texttt{dist} & crée une archive propre, valide, et répondant aux exigences de rendu.\\
\texttt{distclean} & lance la règle \texttt{clean}, puis supprime les binaires et bibliothèques.\\
\texttt{check} & lance le(s) script(s) de test.\\
\texttt{libmystack} & lance les règles \texttt{shared} et \texttt{static} \\
\texttt{shared} & compile l'ensemble du projet avec les options de compilations exigées et génère une bibliothèque dynamique.\\
\texttt{static} & compile l'ensemble du projet avec les options de compilations exigées et génère une bibliothèque statique.\\
\end{tabular}
