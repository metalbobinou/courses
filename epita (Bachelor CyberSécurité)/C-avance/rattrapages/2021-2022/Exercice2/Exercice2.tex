%% Exercice 2

%\ExoSpecs{\TTBF{CalculTVA.sh}}{\TTBF{\RenduDir/src/exo1/}}{750}{640}{\TTBF{write}}
\ExoSpecsCustom{\TTBF{linkedlist.c}}{\TTBF{\RenduDir/src/}}{750}{640}{Fonctions autorisées}{\TTBF{malloc(3)}, \TTBF{free(3)}, \TTBF{memcpy(3)}, \TTBF{printf(3)}}

\vspace*{0.7cm}

\noindent \ExoObjectif{Le but de l'exercice est d'implémenter une liste chaînée en C.}

\bigskip

\noindent Les fonctions demandées dans cet exercice devront se trouver dans une bibliothèque nommée \TTBF{libmylinkedlist}.
Après un appel à la commande \texttt{make} à la racine du projet, il faut que votre chaîne de compilation produise à la racine de votre projet une version statique de la bibliothèque (qui se nommera \TTBF{libmylinkedlist.a}) ainsi qu'une version dynamique de la bibliothèque (qui se nommera \TTBF{libmylinkedlist.so}).

\bigskip

\noindent Vous devez écrire plusieurs fonctions permettant de créer, utiliser, vider, et libérer une liste.
Un fichier \TTBF{linkedlist.h} contenant toutes les fonctions exportables à implémenter vous est fourni en annexe.
Vous devez déclarer une structure \TTBF{my\_linkedlist} et l'ajouter dans \TTBF{linkedlist.h}.
N'oubliez pas de déclarer également une structure qui contiendra les éléments de la liste chaînée.
Pour les premières étapes, vous devrez implémenter une version simplifiée de la file qui ne prend en charge que des entiers positifs.

\bigskip

\noindent Conceptuellement, les fonctions manipulant des files de type \TTBF{my\_linkedlist*} devront pouvoir gérer ces 3 cas :

\bigskip

\begin{center}
%\includegraphics[scale=0.85]{Cours/Linked_List.png}
\end{center}

%\bigskip
\newpage

%\noindent Vous devez implémenter les 11 fonctions suivantes :
%\begin{itemize}
%\item \TTBF{my\_linkedlist *ll\_create(void)}
%\item \TTBF{void ll\_delete(my\_linkedlist *ll)}
%\item \TTBF{int ll\_length(my\_linkedlist *ll)}
%\item \TTBF{int ll\_head(my\_linkedlist *ll)}
%\item \TTBF{int ll\_get(int pos, my\_linkedlist *ll)}
%\item \TTBF{int ll\_tail(my\_linkedlist *ll)}
%\item \TTBF{int ll\_clear(my\_linkedlist *ll)}
%\item \TTBF{int ll\_is\_empty(my\_linkedlist *ll)}
%\item \TTBF{int ll\_insert(int elt, int pos, my\_linkedlist *ll)}
%\item \TTBF{int ll\_replace(int elt, int pos, my\_linkedlist *ll)}
%\item \TTBF{int ll\_remove(int pos, my\_linkedlist *ll)}
%\item \TTBF{int ll\_search(int elt, my\_linkedlist *ll)}
%\item \TTBF{my\_linkedlist *ll\_reverse(my\_linkedlist *ll)}
%\item \TTBF{void ll\_print(my\_linkedlist *ll)}
%\end{itemize}

\noindent Vous devez implémenter les fonctions suivantes :

\bigskip

\lstset{language=C}
\begin{lstlisting}[frame=single,title={Liste des fonctions pour une liste chaînée}]
my_linkedlist *ll_create(void);
void ll_delete(my_linkedlist *ll);

int ll_length(my_linkedlist *ll);

int ll_head(my_linkedlist *ll);
int ll_tail(my_linkedlist *ll);
int ll_get(int pos, my_linkedlist *ll);

int ll_clear(my_linkedlist *ll);
int ll_is_empty(my_linkedlist *ll);

int ll_insert(int elt, int pos, my_linkedlist *ll);
int ll_replace(int elt, int pos, my_linkedlist *ll);
int ll_remove(int pos, my_linkedlist *ll);

int ll_search(int elt, my_linkedlist *ll);
my_linkedlist *ll_reverse(my_linkedlist *ll);
void ll_print(my_linkedlist *ll);
\end{lstlisting}


\subsubsection*{\TTBF{my\_linkedlist *ll\_create(void)}}

\noindent Cette fonction crée une liste vide.
En cas d'erreur (pas assez de mémoire), elle renvoie un pointeur \TTBF{NULL}.


\subsubsection*{\TTBF{void ll\_delete(my\_linkedlist *ll)}}

\noindent Cette fonction vide une liste de l'ensemble de ses éléments, et détruit la structure restante.
Si le paramètre donné est \TTBF{NULL}, la fonction ne fait rien.


\subsubsection*{\TTBF{int ll\_length(my\_linkedlist *ll)}}

\noindent Cette fonction renvoie la longueur de la liste (c'est-à-dire le nombre d'éléments actuellement dans la liste).
Si le paramètre donné est \TTBF{NULL}, la fonction renvoie $ -1 $.


\subsubsection*{\TTBF{int ll\_head(my\_linkedlist *ll)}}

\noindent Cette fonction renvoie l'élément en tête de liste (position 0).
Si la liste donnée en paramètre est \TTBF{NULL}, la fonction renvoie $ -1 $.
Si la liste donnée en paramètre est vide, la fonction renvoie $ -2 $.


\subsubsection*{\TTBF{int ll\_tail(my\_linkedlist *ll)}}

\noindent Cette fonction renvoie l'élément en queue de liste (dernière position).
Si la liste donnée en paramètre est \TTBF{NULL}, la fonction renvoie $ -1 $.
Si la liste donnée en paramètre est vide, la fonction renvoie $ -2 $.


\subsubsection*{\TTBF{int ll\_get(int pos, my\_linkedlist *ll)}}

\noindent Cette fonction renvoie l'élément stocké à la position \textit{pos} de la liste. %%%%%%%%%%%%%%%%%%%%%%%%%%%
Si la liste donnée en paramètre est \TTBF{NULL}, la fonction renvoie $ -1 $. %%%%%%%%%%%%%%%%%%%%
Si la liste donnée en paramètre est vide, la fonction renvoie $ -2 $. %%%%%%%%%%%%%%%%%%
Si la position n'est pas trouvée, la fonction renvoie $ -4 $. %%%%%%%%%%%%%%%%%%


\subsubsection*{\TTBF{int ll\_clear(my\_linkedlist *ll)}}

\noindent Cette fonction vide une liste de l'ensemble de ses éléments, sans détruire la structure de la liste.
La fonction renvoie le nombre d'éléments supprimés de la mémoire.
Si le paramètre donné est \TTBF{NULL}, la fonction renvoie $ -1 $.
Si la liste donnée en paramètre est vide, la fonction renvoie $ 0 $.


\subsubsection*{\TTBF{int ll\_is\_empty(my\_linkedlist *ll)}}

\noindent Cette fonction teste si une liste est vide ou non.
Si la liste est vide, la fonction renvoie $ 1 $.
Si la liste n'est pas vide, la fonction renvoie $ 0 $.
Si la liste donnée en paramètre est \TTBF{NULL}, la fonction renvoie $ -1 $.


\subsubsection*{\TTBF{int ll\_insert(int elt, int pos, my\_linkedlist *ll)}}

\noindent Cette fonction ajoute un élément dans la liste à l'emplacement \textit{pos}.
L'ancien élément qui était présent à cette position est décalé d'un cran en arrière (vers la queue).
La première position est celle où l'élément le plus ancien a été placé (c'est-à-dire la tête de la liste), cette position sera numérotée $ 0 $.
Si l'emplacement n'existe pas et est positif, on ajoute l'élément en queue.
Si l'emplacement n'existe pas et est négatif, on ajoute l'élément en tête.
Si le nombre donné en paramètre est inférieur à $ 0 $, la fonction renvoie $ -4 $.
Si la liste donnée en paramètre est \TTBF{NULL}, la fonction renvoie $ -1 $.
%Si la liste donnée en paramètre est vide, la fonction renvoie $ -2 $.
S'il y a un problème de mémoire, la fonction renvoie $ -3 $.
En cas de succès, la fonction renvoie $ 0 $.


\subsubsection*{\TTBF{int ll\_replace(int elt, int pos, my\_linkedlist *ll)}}

\noindent Cette fonction remplace un élément dans la liste à l'emplacement \textit{pos}, et renvoie l'élément qui était présent à cet endroit.
La première position est celle où l'élément le plus ancien a été placé (c'est-à-dire la tête de la liste), cette position sera numérotée $ 0 $.
Si l'emplacement n'existe pas et est positif, on remplace l'élément en queue.
Si l'emplacement n'existe pas et est négatif, on remplace l'élément en tête.
Si le nombre donné en paramètre est inférieur à $ 0 $, la fonction renvoie $ -4 $.
%En cas de succès, la fonction renvoie $ 0 $.
Si la liste donnée en paramètre est \TTBF{NULL}, la fonction renvoie $ -1 $.
Si la liste donnée en paramètre est vide, la fonction renvoie $ -2 $.


\subsubsection*{\TTBF{int ll\_remove(int pos, my\_linkedlist *ll)}}

\noindent Cette fonction supprime l'élément stocké à la position \textit{pos} de la liste. %%%%%%%%%%%%%%%%%%%%%%%%%%%
Si la suppression a réussi, la fonction renvoie $ 0 $. %%%%%%%%%%%%%%
Si la liste donnée en paramètre est \TTBF{NULL}, la fonction renvoie $ -1 $. %%%%%%%%%%%%%%%%%%%%
Si la liste donnée en paramètre est vide, la fonction renvoie $ -2 $. %%%%%%%%%%%%%%%%%%
Si la position n'est pas trouvée, la fonction renvoie $ -4 $. %%%%%%%%%%%%%%%%%%


\subsubsection*{\TTBF{int ll\_search(int elt, my\_linkedlist *ll)}}

\noindent Cette fonction recherche un élément dans la liste et renvoie sa position.
La première position est celle où l'élément le plus ancien a été placé (c'est-à-dire la tête de la liste), cette position sera numérotée $ 0 $.
Si l'élément n'est pas trouvé, la fonction renvoie $ -4 $.
Si la liste donnée en paramètre est \TTBF{NULL}, la fonction renvoie $ -1 $.


\subsubsection*{\TTBF{my\_linkedlist *ll\_reverse(my\_linkedlist *ll)}}

\noindent Cette fonction inverse la position de tous les éléments de la liste.
Le premier élément devient le dernier, l'avant dernier devient le deuxième, etc.
En cas de succès, la fonction renvoie le pointeur vers l'éventuelle nouvelle adresse en mémoire de la structure de la liste inversée.
En cas de problème mémoire, on renvoie \TTBF{NULL}, et l'ancienne liste doit rester à son ancienne adresse mémoire sans subir la moindre modification.
Si la liste donnée en paramètre est \TTBF{NULL}, la fonction renvoie \TTBF{NULL}.


\subsubsection*{\TTBF{void ll\_print(my\_linkedlist *ll)}}

\noindent Cette fonction affiche le contenu de la liste.
Le format d'affichage attendu implique d'afficher un seul élément par ligne, suivi d'un retour à la ligne.
L'élément en tête de file sera affiché en premier.
Si la liste donnée en paramètre est vide, seul un retour à la ligne est affiché.
Si la liste donnée en paramètre est \TTBF{NULL}, rien n'est affiché.

\bigskip

%\lstset{language=sh}
\lstset{style=sh}
\begin{lstlisting}[frame=single,title={Exemple d'affichage du cas normal : liste chaînée contenant 42, 5, 13 dans cet ordre précis}]
%*\LSTPrompt*) ./my_linked_list
42
5
13

%*\LSTPrompt*)
\end{lstlisting}

\bigskip

\lstset{style=sh}
\begin{lstlisting}[frame=single,title={Exemple d'affichage d'une liste chainée vide}]
%*\LSTPrompt*) ./my_linked_list

%*\LSTPrompt*)
\end{lstlisting}

\bigskip

\lstset{style=sh}
\begin{lstlisting}[frame=single,title={Exemple d'affichage d'un pointeur NULL}]
%*\LSTPrompt*) ./my_linked_list
%*\LSTPrompt*)
\end{lstlisting}
