\documentclass[11pt,a4paper]{article}
\usepackage[utf8]{inputenc}
\usepackage[french]{babel}
\usepackage[T1]{fontenc}

\usepackage{amsmath}
\usepackage{amsfonts}
\usepackage{amssymb}

\newcommand{\NomAuteur}{Fabrice BOISSIER}
\newcommand{\TitreMatiere}{Algorithmique - Premiers Pas}
\newcommand{\NomUniv}{EPITA - Bachelor Cyber Sécurité}
\newcommand{\NiveauUniv}{CYBER1}
\newcommand{\NumGroupe}{CYBER1}
\newcommand{\AnneeUniv}{2022-2023}
\newcommand{\DateExam}{mai 2023}
\newcommand{\TypeExam}{Rattrapage}
\newcommand{\TitreExam}{\TitreMatiere}
\newcommand{\DureeExam}{1h30}
\newcommand{\MyWaterMark}{\AnneeUniv} % Watermark de protection

% Ajout de mes classes & definitions
\usepackage{MetalExam} % Appelle un .sty

% "Tableau" et pas "Table"
\addto\captionsfrench{\def\tablename{Tableau}}

%%%%%%%%%%%%%%%%%%%%%%%
%Header
%%%%%%%%%%%%%%%%%%%%%%%
\lhead{\TypeExam}							%Gauche Haut
\chead{\NomUniv}							%Centre Haut
\rhead{\NumGroupe}							%Droite Haut
\lfoot{\DateExam}							%Gauche Bas
\cfoot{\thepage{} / \pageref*{LastPage}}	%Centre Bas
\rfoot{\texttt{\TitreMatiere}}				%Droite Bas

%%%%%

\usepackage{tabularx}

\newlength{\LabelWidth}%
%\setlength{\LabelWidth}{1.3in}%
\setlength{\LabelWidth}{1cm}%
%\settowidth{\LabelWidth}{Employee E-mail:}%  Specify the widest text here.

% Optional first parameter here specifies the alignment of
% the text within the \makebox.  Default is [l] for left
% alignment. Other options are [r] and [c] for right and center
\newcommand*{\AdjustSize}[2][l]{\makebox[\LabelWidth][#1]{#2}}%


\definecolor{mGreen}{rgb}{0,0.6,0}
\definecolor{mGray}{rgb}{0.5,0.5,0.5}
\definecolor{mPurple}{rgb}{0.58,0,0.82}
\definecolor{backgroundColour}{rgb}{0.95,0.95,0.92}

\lstdefinestyle{CStyle}{
    backgroundcolor=\color{backgroundColour},
    commentstyle=\color{mGreen},
    keywordstyle=\color{magenta},
    numberstyle=\tiny\color{mGray},
    stringstyle=\color{mPurple},
    basicstyle=\footnotesize,
    breakatwhitespace=false,
    breaklines=true,
    captionpos=b,
    keepspaces=true,
    numbers=left,
    numbersep=5pt,
    showspaces=false,
    showstringspaces=false,
    showtabs=false,
    tabsize=2,
    language=C
}


\hyphenation{op-tical net-works SIGKILL}


\begin{document}

%\MakeExamTitleDuree     % Pour afficher la duree
\MakeExamTitle                   % Ne pas afficher la duree

%% \MakeStudentName    %% A reutiliser sur chaque nouvelle page

\bigskip
%\bigskip

Vous devez respecter les consignes suivantes, sous peine de 0 :

\begin{itemize}
\item Lisez le sujet en entier avec attention
\item Répondez sur le sujet
\item Ne détachez pas les agrafes du sujet
\item \'Ecrivez lisiblement vos réponses (si nécessaire en majuscules)
\item Vous devez écrire dans le langage algorithmique classique ou en C (donc pas de Python ou autre)
\item Ne trichez pas
\end{itemize}

%\bigskip

\vfillFirst

% Questions cours
\section{Questions (5 points)}

\subsection{(2,5 points) Sélectionnez les conditions vraies pour A = 4 et B = 7 : }

% Compter pour chaque réponse 0,5 points : si bonne ou pas bonne
\bigskip

\begin{itemize}
  \item[\checkmark] (A < B) et ((B - 3) >= A) % Vrai
  \item[\checkmark] (A >= B) ou ((B - 3) <= A) % Vrai
  \item[\CaseCoche] ((A + 3) > B) et ((B - 2) < (A + 2)) % Faux
  \item[\checkmark] ((A + 3) <= B) et ((B - 2) < (A + 2)) % Vrai
  \item[\CaseCoche] ((A + 3) <= (B - 1)) ou ((B + 3) > (A + 6)) % Faux
\end{itemize}


%\bigskip


\subsection{(2,5 points) Quelles sont les caractéristiques de cet algorithme : }

\bigskip

\begin{lstlisting}[style=algorithmique]
algorithme fonction CalculXYZ : entier
  parametres locaux
    entier    x, y, z
debut
  si (y == 1)
    retourne (1)
  sinon
    si ((x %*\texttt{\%}*) y) == 0)
      retourne (y + CalculXYZ(x, (y - 1), z))
    sinon
      retourne (CalculXYZ(x, (y - 1), z))
    fin si
  fin si
fin algorithme fonction CalculXYZ \end{lstlisting}

\begin{itemize}
  \item[\checkmark] Il est récursif \\
  \item[\CaseCoche] Il est même récursif terminal \\
  \item[\checkmark] Il s'agit d'une fonction \\
  \item[\CaseCoche] Il s'agit d'une procédure \\
\end{itemize}

%%%%%%%%%%%%%%%%%%%%%%%%%%%%%%%%%%%%%%%%%%%%%%%%%%%%%%%%%%%%%%%%

\clearpage

\section{Algorithmes (15 points)}

\subsection{(3 points) \'Ecrivez une fonction \og \textit{SommeNInt} \fg{} récursive calculant la somme des N premiers entiers. (Vous décrirez d'abord le ou les cas d'arrêts, puis le cas général) }

\bigskip

\begin{center}
Explications du/des cas d'arrêts, puis du cas général :
\GrilleReponseN{6}

\bigskip

Algorithme :
\GrilleReponseN{15}
\end{center}

\smallskip
%\medskip
%\bigskip

\clearpage

\subsection{(3 points) \'Ecrivez une fonction \og \textit{strlen} \fg{} récursive renvoyant la taille d'une chaîne caractères.  (Vous décrirez d'abord le ou les cas d'arrêts, puis le cas général) }

%\smallskip
\bigskip

\begin{center}
Explications du/des cas d'arrêts, puis du cas général :
\GrilleReponseN{6}

\bigskip

Algorithme :
\GrilleReponseN{16}
\end{center}


%\clearpage


\subsection{(3 points) \'Ecrivez une fonction \og \textit{MedianeTab} \fg{} calculant la médiane d'un tableau trié d'entiers. (Vous décrirez d'abord le fonctionnement général) }

%\smallskip

\noindent Pour rappel, la médiane est le nombre au centre d'une distribution triée. Si le tableau a un nombre paire de cases, vous ferez la moyenne des deux éléments centraux.

%\smallskip

\begin{table}[h!]
  \centering
  \begin{minipage}{0.4\textwidth}
    \centering
    \begin{tabular}{| c | c | c | c | c |}
      \hline
      0 & 3 & 9 & 10 & 11 \\
      \hline
    \end{tabular}
%    \caption{Tableau d'exécution pas à pas}
%    \label{table-somme-n-premiers-entiers-execution}

  \smallskip

  La médiane de ce tableau est $ 9 $
  \end{minipage}
    \hfillx
    \begin{minipage}{0.4\textwidth}
    \centering
    \begin{tabular}{| c | c | c | c |}
      \hline
      0 & 9 & 10 & 11 \\
      \hline
    \end{tabular}
%    \caption{Tableau d'exécution pas à pas}
%    \label{table-somme-n-premiers-entiers-execution}

  \smallskip

  La médiane de ce tableau est $ 9,5 $

  $ (9 + 10) / 2 = 9,5 $
  \end{minipage}
%  \caption{Algorithme de la somme des N premiers entiers}
%  \label{somme-n-premiers-entiers}
\end{table}

\begin{center}
Explications du fonctionnement :
\GrilleReponseN{6}

Algorithme :
\GrilleReponseN{14}
\end{center}


%\clearpage


\subsection{(3 points) \'Ecrivez une procédure \og \textit{MoyenneTab} \fg{} itérative affichant la moyenne des éléments d'un tableau d'entiers. (Vous décrirez d'abord le fonctionnement général) }

%\bigskip

\begin{center}
Explications du fonctionnement :
\GrilleReponseN{6.5}

Algorithme :
\GrilleReponseN{16}
\end{center}


%\clearpage


\subsection{(3 points) \'Ecrivez une fonction \og \textit{TabToInt} \fg{} transformant un tableau d'entiers en un unique entier (chaque case contient un nombre positif mais inférieur à 10). (Vous décrirez d'abord le fonctionnement général) }

%\medskip

\begin{center}
  \begin{tabular}{| c | c | c | c |}
    \hline
    4 & 0 & 2 & 3 \\
    \hline
  \end{tabular}

  \smallskip

  Ce tableau doit devenir $ 4023 $
\end{center}

%\bigskip

\begin{center}
Explications du fonctionnement :
\GrilleReponseN{6}

Algorithme :
\GrilleReponseN{15}
\end{center}
% tabtoint(tab, len, acc, mul)
% si len == 0
%  retourner (acc + (mul * tab[0]))
% sinon
%  retourner (f(tab, len - 1, acc + (mul * tab[len]), mul * 10)

\end{document}
