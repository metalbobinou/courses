\documentclass[11pt,a4paper]{article}
\usepackage[utf8]{inputenc}
\usepackage[french]{babel}
\usepackage[T1]{fontenc}

\usepackage{amsmath}
\usepackage{amsfonts}
\usepackage{amssymb}

\newcommand{\NomAuteur}{Fabrice BOISSIER}
\newcommand{\TitreMatiere}{Algorithmique - Premiers Pas}
\newcommand{\NomUniv}{EPITA - Bachelor Cyber Sécurité}
\newcommand{\NiveauUniv}{CYBER1}
\newcommand{\NumGroupe}{CYBER1}
\newcommand{\AnneeUniv}{2022-2023}
\newcommand{\DateExam}{Septembre 2022}
\newcommand{\TypeExam}{QCM 2}
\newcommand{\TitreExam}{\TitreMatiere}
\newcommand{\DureeExam}{20 min}
\newcommand{\MyWaterMark}{\AnneeUniv} % Watermark de protection

% Ajout de mes classes & definitions
\usepackage{MetalExam} % Appelle un .sty

% "Tableau" et pas "Table"
\addto\captionsfrench{\def\tablename{Tableau}}

%%%%%%%%%%%%%%%%%%%%%%%
%Header
%%%%%%%%%%%%%%%%%%%%%%%
\lhead{\TypeExam}							%Gauche Haut
\chead{\NomUniv}							%Centre Haut
\rhead{\NumGroupe}							%Droite Haut
\lfoot{\DateExam}							%Gauche Bas
\cfoot{\thepage{} / \pageref*{LastPage}}	%Centre Bas
\rfoot{\texttt{\TitreMatiere}}				%Droite Bas

%%%%%

\usepackage{tabularx}

\newlength{\LabelWidth}%
%\setlength{\LabelWidth}{1.3in}%
\setlength{\LabelWidth}{1cm}%
%\settowidth{\LabelWidth}{Employee E-mail:}%  Specify the widest text here.

% Optional first parameter here specifies the alignment of
% the text within the \makebox.  Default is [l] for left
% alignment. Other options are [r] and [c] for right and center
\newcommand*{\AdjustSize}[2][l]{\makebox[\LabelWidth][#1]{#2}}%


\definecolor{mGreen}{rgb}{0,0.6,0}
\definecolor{mGray}{rgb}{0.5,0.5,0.5}
\definecolor{mPurple}{rgb}{0.58,0,0.82}
\definecolor{backgroundColour}{rgb}{0.95,0.95,0.92}

\lstdefinestyle{CStyle}{
    backgroundcolor=\color{backgroundColour},
    commentstyle=\color{mGreen},
    keywordstyle=\color{magenta},
    numberstyle=\tiny\color{mGray},
    stringstyle=\color{mPurple},
    basicstyle=\footnotesize,
    breakatwhitespace=false,
    breaklines=true,
    captionpos=b,
    keepspaces=true,
    numbers=left,
    numbersep=5pt,
    showspaces=false,
    showstringspaces=false,
    showtabs=false,
    tabsize=2,
    language=C
}


\hyphenation{op-tical net-works SIGKILL}


\begin{document}

% \MakeExamTitleDuree     % Pour afficher la duree
\MakeExamTitle                   % Ne pas afficher la duree

%% \MakeStudentName    %% A reutiliser sur chaque nouvelle page


% \setcounter{section}{1}  % Ne pas faire une liste de 0.1, 0.2, ... mais 1.1, 1.2, ...

\renewcommand{\thesubsection}{\arabic{subsection}} % Subsection sans 0.x, mais juste x

% \newcommand{\thesubsection}{\thesection.\arabic{subsection}} % Subsection avec numero section (0.x)



\subsection{(4 points) Indiquer si les algorithmes suivants sont récursifs terminaux ou non }

\begin{table}[htb!]
  \centering
  \begin{minipage}{0.50\textwidth}
    \centering
% %*   *)
  \begin{lstlisting}[style=algorithmique]
algorithme fonction Calcul1
  parametres locaux
    entier    a, b
debut
  si (a == 0)
    retourne (1)
  sinon
    var = a + b
    retourne (Calcul1(a - 1, b + var))
fin algorithme fonction Calcul1
  \end{lstlisting}

  \begin{itemize}
    \item[\CaseCoche] Récursif simple \\
    \item[\CaseCoche] Récursif terminal \\
  \end{itemize}

  \end{minipage}
  \hfillx
  \begin{minipage}{0.5\textwidth}
    \centering
% %*   *)
  \begin{lstlisting}[style=algorithmique]
algorithme fonction Calcul2
  parametres locaux
    entier    a, b
debut
  si (a == 0)
    retourne (b)
  sinon
    var = a + b
    retourne (Calcul2(a - 1, b + var))
fin algorithme fonction Calcul2
  \end{lstlisting}

  \begin{itemize}
    \item[\CaseCoche] Récursif simple \\
    \item[\CaseCoche] Récursif terminal \\
  \end{itemize}

  \end{minipage}
\end{table}


\bigskip


\subsection{(4 points) Cochez la (ou les) affirmation(s) vraie(s) : }

\begin{itemize}
  \item[\CaseCoche] Un tableau de taille N démarrant à l'index 0 a sa dernière case à l'index (N - 1) \\
  \item[\CaseCoche] Un tableau de taille N démarrant à l'index 0 a sa dernière case à l'index N \\
  \item[\CaseCoche] Un tableau de taille N démarrant à l'index 1 a sa dernière case à l'index (N - 1) \\
  \item[\CaseCoche] Un tableau de taille N démarrant à l'index 1 a sa dernière case à l'index N \\
\end{itemize}


\bigskip


\subsection{(4 points) Cochez la (ou les) affirmation(s) vraie(s) : }

\begin{itemize}
  \item[\CaseCoche] Les récursions successives peuvent facilement entraîner un dépassement de la pile d'appels \\
  \item[\CaseCoche] Les itérations successives peuvent facilement entraîner un dépassement de la pile d'appels \\
  \item[\CaseCoche] Les accumulateurs sont obligatoires dans les algorithmes récursifs \\
  \item[\CaseCoche] Les algorithmes itératifs terminaux ont aussi besoin d'un accumulateur \\
\end{itemize}


%\bigskip
\newpage

\subsection{(4 points) Ces deux algorithmes génèrent-ils les mêmes résultats s'ils sont exécutés avec les mêmes paramètres ? }

\begin{table}[ht!]
  \centering
  \begin{minipage}{0.5\textwidth}
    \centering
% %*   *)
  \begin{lstlisting}[style=algorithmique]
algorithme fonction Algo1Rec
  parametres locaux
    entier    a, b
debut
  si (a == 0)
    retourne (0)
  sinon
    retourne (Algo1Rec(a - 1, b) + b)
fin algorithme fonction Calcul1 \end{lstlisting}
  \end{minipage}
  \hfillx
  \begin{minipage}{0.5\textwidth}
    \centering
% %*   *)
  \begin{lstlisting}[style=algorithmique]
algorithme fonction Algo1Iter
  parametres locaux
    entier    a, b
debut
  var1 = 0
  var2 = 0
  var3 = 0
  tant que (var1 < a)
    var3 = (var3 + b) + var1
    var2 = var2 + var1
    var1 = var1 + 1
  fin tant que
  retourne (var3 - var2)
fin algorithme fonction Algo1Iter \end{lstlisting}
  \end{minipage}
\end{table}

\smallskip

\begin{itemize}
  \item[\CaseCoche] Oui \\
  \item[\CaseCoche] Non \\
\end{itemize}


\bigskip


\subsection{(4 points) Quelle(s) condition(s) est (ou sont) considérée(s) comme vraie(s) pour A = vrai et B = faux ? }

\begin{itemize}
  \item[\CaseCoche] ((non A) et (non B)) et (A et (non B)) \\
  \item[\CaseCoche] ((non A) et    B)    ou (A et    B)    \\
  \item[\CaseCoche] ((non A) et (non B)) ou (A et (non B)) \\
  \item[\CaseCoche] ((non A) ou (non B)) et (A ou (non B)) \\
\end{itemize}


\bigskip


\subsection{[BONUS] (0 point) Le Magicien des Ténèbres est : }

\begin{itemize}
  \item[\CaseCoche] L'évolution XY d'Ectoplasma dans Pokémon \\
  \item[\CaseCoche] L'ennemi principal de la 1\up{ère} saison de Digimon (Digimon Adventure) \\
  \item[\CaseCoche] Le leader du Gang du Latex dans Medabots \\
  \item[\CaseCoche] La carte préférée de Yûgi Muto (Yu-Gi-Oh!) \\
  \item[\CaseCoche] Votre camarade écrivant des algorithmes parfaits la nuit lors de coupures de courant \\
\end{itemize}

\end{document}
