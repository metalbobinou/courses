\documentclass[11pt,a4paper]{article}
%\documentclass[11pt,a4paper,twoside]{article}
\usepackage[utf8]{inputenc}
\usepackage[french]{babel}
\usepackage[T1]{fontenc}

\usepackage{amsmath}
\usepackage{amsfonts}
\usepackage{amssymb}

\newcommand{\TitreMatiere}{Architecture des Ordinateurs}
\newcommand{\TitreSeance}{Opérations Binaires}
\newcommand{\SousTitreSeance}{$ + $ $ - $ $ \times $ $ \div $ binaires}
\newcommand{\DateCours}{Septembre 2024}
\newcommand{\AnneeScolaire}{2024-2025}
\newcommand{\Organisation}{EPITA}
\newcommand{\NomAuteurA}{Fabrice BOISSIER}
\newcommand{\MailAuteurA}{fabrice.boissier@epita.fr}
\newcommand{\NomAuteurB}{ }
\newcommand{\MailAuteurB}{ }
\newcommand{\DocKeywords}{Architecture ; Opérations Binaires ; Addition Binaire ; Soustraction Binaire ; Multiplication Binaire ; Division Binaire}
\newcommand{\DocLangue}{fr} % "en", "fr", ...

\usepackage{MetalCourseBooklet}

% Babel ne traduit pas toujours bien les tableaux et autres
\renewcommand*\frenchfigurename{%
    {\scshape Figure}%
}
\renewcommand*\frenchtablename{%
    {\scshape Tableau}%
}

% Ne pas afficher le numéro de la légende sur tableaux et figures
\captionsetup{format=sanslabel}


\usepackage{xlop}  % Ajout des jolies divisions posées :   \opdiv{25}{7}  \opidiv{25}{7}
%\usepackage{pstricks}  % style pour xlop

\begin{document}

\EncadreTitre

\bigskip


%\begin{center}
%\begin{tabular}{p{5cm} p{11cm}}
%\textbf{Commandes étudiées :} & \texttt{sh}, \texttt{bash}, \texttt{man}, \texttt{ls}, \texttt{mkdir}, \texttt{touch}, \texttt{chmod}, \texttt{mv}, \texttt{rm}, \texttt{rmdir}, \texttt{cat}, \texttt{file}, \texttt{which}, \texttt{which}\\
%
%\textbf{Builtins étudiées :} & \texttt{pwd}, \texttt{cd}, \texttt{exit}, \texttt{logout}, \texttt{echo}, \texttt{umask}, \texttt{type}, \texttt{>}, \texttt{>{}>}, \texttt{<}, \texttt{<{}<}, \texttt{|}\\
%
%\textbf{Notions étudiées :} & Shell, Manuels, Fichiers, Répertoires, Droits, Redirections\\
%\end{tabular}
%\end{center}

\bigskip


Ce document a pour objectif de vous familiariser avec les opérations binaires.

%\medskip
%
%Les opérations seront réalisées exclusivement sur des valeurs non signées.

\bigskip

\vfillFirst

%%%%%%%%%%%%%%%%%%%%%%%%%%%%%%%%%%%%%%%%%%%%%%%%%%%%%%%%%%%%%%%%%%%%%%%%%%%%%

\section{Addition}

\medskip

L'addition en binaire se déroule exactement comme les additions base 10, à la différence qu'une retenue est créée dès que l'on dépasse \og 1 \fg{}.

\medskip

Ainsi, pour effectuer l'addition de \og $ \% \, 1101 $ \fg{} (13) et \og $ \% \, 0110 $ \fg{} (6), on fera :

\begin{center}

\begin{table}[ht!]
  \centering
  \begin{minipage}{0.15\textwidth}

\par\vspace{3\oplineheight}
\phantom{\oplput(0,3){\tiny }}
\oplput(1,2){1}\oplput(2,2){1}\oplput(3,2){0}\oplput(4,2){1}
\oplput(1,1){0}\oplput(2,1){1}\oplput(3,1){1}\oplput(4,1){0}
\oplput(-0.25,1.05){$+$}
\ophline(0,0.8){5}
\oplput(4,0){1}

  \end{minipage}
  \hfillx
  \begin{minipage}{0.1\textwidth}

$ \Rightarrow $

  \end{minipage}
  \hfillx
  \begin{minipage}{0.15\textwidth}

\par\vspace{3\oplineheight}
\phantom{\oplput(0,3){\tiny }}
\oplput(1,2){1}\oplput(2,2){1}\oplput(3,2){0}\oplput(4,2){1}
\oplput(1,1){0}\oplput(2,1){1}\oplput(3,1){1}\oplput(4,1){0}
\oplput(-0.25,1.05){$+$}
\ophline(0,0.8){5}
\oplput(3,0){1}\oplput(4,0){1}

  \end{minipage}
  \hfillx
  \begin{minipage}{0.1\textwidth}

$ \Rightarrow $

  \end{minipage}
  \hfillx
  \begin{minipage}{0.15\textwidth}

\par\vspace{3\oplineheight}
\oplput(1,3){\tiny \tinycircled{1}}
\oplput(1,2){1}\oplput(2,2){1}\oplput(3,2){0}\oplput(4,2){1}
\oplput(1,1){0}\oplput(2,1){1}\oplput(3,1){1}\oplput(4,1){0}
\oplput(-0.25,1.05){$+$}
\ophline(0,0.8){5}
\oplput(2,0){0}\oplput(3,0){1}\oplput(4,0){1}

  \end{minipage}
  \hfillx
  \begin{minipage}{0.1\textwidth}

$ \Rightarrow $

  \end{minipage}
  \hfillx
  \begin{minipage}{0.15\textwidth}

\par\vspace{3\oplineheight}
\oplput(0,3){\tiny \tinycircled{1}}\oplput(1,3){\tiny \tinycircled{1}}
\oplput(1,2){1}\oplput(2,2){1}\oplput(3,2){0}\oplput(4,2){1}
\oplput(1,1){0}\oplput(2,1){1}\oplput(3,1){1}\oplput(4,1){0}
\oplput(-0.25,1.05){$+$}
\ophline(0,0.8){5}
\oplput(1,0){0}\oplput(2,0){0}\oplput(3,0){1}\oplput(4,0){1}

  \end{minipage}
  \hfillx
  \begin{minipage}{0.1\textwidth}

$ \Rightarrow $

  \end{minipage}
  \hfillx
  \begin{minipage}{0.15\textwidth}

\par\vspace{3\oplineheight}
\oplput(0,3){\tiny \tinycircled{1}}\oplput(1,3){\tiny \tinycircled{1}}
\oplput(1,2){1}\oplput(2,2){1}\oplput(3,2){0}\oplput(4,2){1}
\oplput(1,1){0}\oplput(2,1){1}\oplput(3,1){1}\oplput(4,1){0}
\oplput(-0.25,1.05){$+$}
\ophline(0,0.8){5}
\oplput(0,0){1}\oplput(1,0){0}\oplput(2,0){0}\oplput(3,0){1}\oplput(4,0){1}

  \end{minipage}
\end{table}

\end{center}

On en déduit donc que : $ 1101_{2} + 0110_{2} = 10011_{2} $ (19)

\smallskip

Du fait de la base 2, on notera que dans le cas où un nombre est constitué de plusieurs \og 1 \fg{} de suite, la retenue est propagée jusqu'au premier \og 0 \fg{} trouvé.

\bigskip

%\vfillFirst
\vfillLast

%%%%%%%%%%%%%%%%%%%%%%%%%%%%%%%%%%%%%%%%%%%%%%%%%%%%%%%%%%%%%%%%%%%%%%%%%%%%%

\section{Multiplication}

\medskip

La multiplication s'applique exactement comme son équivalent décimale.
Il faut noter que les additions successives qui en résultent peuvent être nombreuses.
Il est donc préférable de placer le nombre le plus petit en dessous du plus grand.

\medskip

N'oubliez pas que multiplier un nombre par 2, c'est le décaler vers la gauche d'un bit en ajoutant un 0.
Ainsi, multiplier un nombre par une puissance de 2 (2, 4, 8, ...), c'est le décaler vers la gauche de plusieurs bits.

\medskip

Ainsi, pour effectuer la multiplication de \og $ \% \, 1101 $ \fg{} (13) par \og $ \% \, 110 $ \fg{} (6), on fera :

\begin{center}

\begin{table}[ht!]
  \centering
  \begin{minipage}{0.15\textwidth}

\par\vspace{6\oplineheight}
\oplput(3,5){1}\oplput(4,5){1}\oplput(5,5){0}\oplput(6,5){1}
\oplput(4,4){1}\oplput(5,4){1}\oplput(6,4){0}
\oplput(-0.25,4.05){$\times$}
\ophline(0,3.8){7}
\phantom{\oplput(3,3){0}\oplput(4,3){0}\oplput(5,3){0}\oplput(6,3){0}}
\phantom{\oplput(2,2){1}\oplput(3,2){1}\oplput(4,2){0}\oplput(5,2){1}\oplput(6,2){.}}
\phantom{\oplput(1,1){1}\oplput(2,1){1}\oplput(3,1){0}\oplput(4,1){1}\oplput(5,1){.}\oplput(6,1){.}}
\phantom{\oplput(-0.25,1.05){$+$}}
\phantom{\ophline(0,0.8){7}}
\phantom{\oplput(6,0){.}}

  \end{minipage}
  \hfillx
  \begin{minipage}{0.1\textwidth}
    \centering

$ \Rightarrow $

  \end{minipage}
  \hfillx
  \begin{minipage}{0.15\textwidth}

\par\vspace{6\oplineheight}
\oplput(3,5){1}\oplput(4,5){1}\oplput(5,5){0}\oplput(6,5){1}
\oplput(4,4){1}\oplput(5,4){1}\oplput(6,4){0}
\oplput(-0.25,4.05){$\times$}
\ophline(0,3.8){7}
\oplput(3,3){0}\oplput(4,3){0}\oplput(5,3){0}\oplput(6,3){0}
\phantom{\oplput(2,2){1}\oplput(3,2){1}\oplput(4,2){0}\oplput(5,2){1}\oplput(6,2){.}}
\phantom{\oplput(1,1){1}\oplput(2,1){1}\oplput(3,1){0}\oplput(4,1){1}\oplput(5,1){.}\oplput(6,1){.}}
\phantom{\oplput(-0.25,1.05){$+$}}
\phantom{\ophline(0,0.8){7}}
\phantom{\oplput(6,0){.}}

  \end{minipage}
  \hfillx
  \begin{minipage}{0.1\textwidth}
    \centering

$ \Rightarrow $

  \end{minipage}
  \hfillx
  \begin{minipage}{0.15\textwidth}

\par\vspace{6\oplineheight}
\oplput(3,5){1}\oplput(4,5){1}\oplput(5,5){0}\oplput(6,5){1}
\oplput(4,4){1}\oplput(5,4){1}\oplput(6,4){0}
\oplput(-0.25,4.05){$\times$}
\ophline(0,3.8){7}
\oplput(3,3){0}\oplput(4,3){0}\oplput(5,3){0}\oplput(6,3){0}
\oplput(2,2){1}\oplput(3,2){1}\oplput(4,2){0}\oplput(5,2){1}\oplput(6,2){.}
\phantom{\oplput(1,1){1}\oplput(2,1){1}\oplput(3,1){0}\oplput(4,1){1}\oplput(5,1){.}\oplput(6,1){.}}
\phantom{\oplput(-0.25,1.05){$+$}}
\phantom{\ophline(0,0.8){7}}
\phantom{\oplput(6,0){.}}

  \end{minipage}
  \hfillx
  \begin{minipage}{0.1\textwidth}
    \centering

$ \Rightarrow $

  \end{minipage}
  \hfillx
  \begin{minipage}{0.15\textwidth}

\par\vspace{6\oplineheight}
\oplput(3,5){1}\oplput(4,5){1}\oplput(5,5){0}\oplput(6,5){1}
\oplput(4,4){1}\oplput(5,4){1}\oplput(6,4){0}
\oplput(-0.25,4.05){$\times$}
\ophline(0,3.8){7}
\oplput(3,3){0}\oplput(4,3){0}\oplput(5,3){0}\oplput(6,3){0}
\oplput(2,2){1}\oplput(3,2){1}\oplput(4,2){0}\oplput(5,2){1}\oplput(6,2){.}
\oplput(1,1){1}\oplput(2,1){1}\oplput(3,1){0}\oplput(4,1){1}\oplput(5,1){.}\oplput(6,1){.}
\phantom{\oplput(-0.25,1.05){$+$}}
\phantom{\ophline(0,0.8){7}}
\phantom{\oplput(6,0){.}}

  \end{minipage}
  \hfillx
  \begin{minipage}{0.1\textwidth}
    \centering

$ \Rightarrow $

  \end{minipage}
  \hfillx
  \begin{minipage}{0.15\textwidth}

\par\vspace{6\oplineheight}
\oplput(3,5){1}\oplput(4,5){1}\oplput(5,5){0}\oplput(6,5){1}
\oplput(4,4){1}\oplput(5,4){1}\oplput(6,4){0}
\oplput(-0.25,4.05){$\times$}
\ophline(0,3.8){7}
\oplput(3,3){0}\oplput(4,3){0}\oplput(5,3){0}\oplput(6,3){0}
\oplput(2,2){1}\oplput(3,2){1}\oplput(4,2){0}\oplput(5,2){1}\oplput(6,2){0}
\oplput(1,1){1}\oplput(2,1){1}\oplput(3,1){0}\oplput(4,1){1}\oplput(5,1){0}\oplput(6,1){0}
\oplput(-0.25,1.05){$+$}
\ophline(0,0.8){7}
\oplput(0,0){1}\oplput(1,0){0}\oplput(2,0){0}\oplput(3,0){1}\oplput(4,0){1}\oplput(5,0){1}\oplput(6,0){0}
\oplput(0,3){\tiny \tinycircled{1}}\oplput(1,3){\tiny \tinycircled{1}}

  \end{minipage}
\end{table}

\end{center}

On en déduit donc que : $ 1101_{2} \times 110_{2} = 1001110_{2} $ (78)

%\vfillLast

%\bigskip
\clearpage

%%%%%%%%%%%%%%%%%%%%%%%%%%%%%%%%%%%%%%%%%%%%%%%%%%%%%%%%%%%%%%%%%%%%%%%%%%%%%

\section{Soustraction}

\medskip

Plusieurs techniques de soustractions binaires existent : la conversion en passant par le complément à deux, ou l'opération bit à bit similaire à l'addition.


\vfillFirst

\subsection{Soustraction par complément à 2}

\smallskip

Pour connaitre la différence entre deux nombres binaires positifs il suffit simplement de passer le plus petit en son équivalent négatif en représentation signée, c'est-à-dire effectuer le complément à 2, puis d'additionner le résultat \textit{sans tenir compte de la retenue finale}.

\medskip

Par exemple, pour effectuer la soustraction entre $ \% \, 1100 $ (11) et $ \% \, 1001 $ (9) :


\begin{center}

\begin{table}[ht!]
  \centering
  \begin{minipage}{0.15\textwidth}

\par\vspace{3\oplineheight}
\phantom{\oplput(1,3){\tiny }}
\oplput(1,2){1}\oplput(2,2){1}\oplput(3,2){0}\oplput(4,2){0}
\oplput(1,1){1}\oplput(2,1){0}\oplput(3,1){0}\oplput(4,1){1}
\oplput(-0.25,1.05){$-$}
\ophline(0,0.8){5}
\phantom{\oplput(3,0){ }}

  \end{minipage}
  \hfillx
  \begin{minipage}{0.3\textwidth}

\begin{center}
Complément à 2 de $ \% \, 1001 $ :

\medskip

\begin{tabular}{ m{1cm}  c c c c  m{1cm} }
 &  \TTBF{1} & \TTBF{0} & \TTBF{0} & \TTBF{1}  & \\
\end{tabular}

\smallskip

\textit{(complément à 1)}

\smallskip

\begin{tabular}{ m{1cm}  c c c c  m{1cm} }
 &  \TTBF{0} & \TTBF{1} & \TTBF{1} & \TTBF{0}  & \\
\end{tabular}

\smallskip

\textit{(ajout de 1)}

\smallskip

\begin{tabular}{ m{1cm}  c c c c  m{1cm} }
 &  \TTBF{0} & \TTBF{1} & \TTBF{1} & \TTBF{1}  & \\
\end{tabular}
\end{center}

  \end{minipage}
  \hfillx
  \begin{minipage}{0.1\textwidth}

\phantom{ }

  \end{minipage}
  \hfillx
  \begin{minipage}{0.15\textwidth}

\par\vspace{3\oplineheight}
\phantom{\oplput(1,3){\tiny }}
\oplput(1,2){1}\oplput(2,2){1}\oplput(3,2){0}\oplput(4,2){0}
\oplput(1,1){0}\oplput(2,1){1}\oplput(3,1){1}\oplput(4,1){1}
\oplput(-0.25,1.05){$+$}
\ophline(0,0.8){5}
\oplput(4,0){1}

  \end{minipage}
  \hfillx
  \begin{minipage}{0.1\textwidth}

\phantom{ } $ \Rightarrow $

  \end{minipage}
  \hfillx
  \begin{minipage}{0.15\textwidth}

\par\vspace{3\oplineheight}
\phantom{\oplput(1,3){\tiny }}
\oplput(1,2){1}\oplput(2,2){1}\oplput(3,2){0}\oplput(4,2){0}
\oplput(1,1){0}\oplput(2,1){1}\oplput(3,1){1}\oplput(4,1){1}
\oplput(-0.25,1.05){$+$}
\ophline(0,0.8){5}
\oplput(3,0){1}\oplput(4,0){1}

  \end{minipage}
  \hfillx
  \begin{minipage}{0.1\textwidth}

\phantom{ } $ \Rightarrow $

  \end{minipage}
  \hfillx
  \begin{minipage}{0.15\textwidth}

\par\vspace{3\oplineheight}
\oplput(1,3){\tiny \tinycircled{1}}
\oplput(1,2){1}\oplput(2,2){1}\oplput(3,2){0}\oplput(4,2){0}
\oplput(1,1){0}\oplput(2,1){1}\oplput(3,1){1}\oplput(4,1){1}
\oplput(-0.25,1.05){$+$}
\ophline(0,0.8){5}
\oplput(2,0){0}\oplput(3,0){1}\oplput(4,0){1}

  \end{minipage}
  \hfillx
  \begin{minipage}{0.1\textwidth}

$ \Rightarrow $

  \end{minipage}
  \hfillx
  \begin{minipage}{0.15\textwidth}

\par\vspace{3\oplineheight}
\oplput(0,3){\tiny \textcolor{red}{\tinycircled{1}}}\oplput(1,3){\tiny \tinycircled{1}}
\oplput(1,2){1}\oplput(2,2){1}\oplput(3,2){0}\oplput(4,2){0}
\oplput(1,1){0}\oplput(2,1){1}\oplput(3,1){1}\oplput(4,1){1}
\oplput(-0.25,1.05){$+$}
\ophline(0,0.8){5}
\oplput(0,0){\textcolor{red}{\textbf{1}}}\oplput(1,0){0}\oplput(2,0){0}\oplput(3,0){1}\oplput(4,0){1}

  \end{minipage}
\end{table}

\end{center}

L'addition donne donc : $ 1100_{2} + 0111_{2} = 10011_{2} $

\smallskip

Mais en retirant la retenue finale, on obtient : $ \text{\sout{1}}0011_{2} $

\medskip

On en déduit donc que : $ 1100_{2} - 1001_{2} = 0011_{2} $ (3)

\bigskip

Notez bien que pour des nombres de grandeurs très différentes, il faut d'abord les aligner sur le même nombre de bits \textit{avant} de faire le complément à 2.
Exemple :

$ 120 - 3 = 111 \, 1000_{2} - 11_{2} = 111 \, 1000_{2} - 000 \, 0011_{2} \Rightarrow 111 \, 1000_{2} + 111 \, 1101_{2} \Rightarrow \text{\textcolor{red}{\sout{1}}}111 \, 0101_{2} \, (117) $


\vfillLast

%\bigskip
\clearpage

%%%%%%%%%%%%%%%%%%%%%%%%%%%%%%%%%%%%%%%

\subsection{Soustraction bit à bit}

Dans la soustraction bit à bit, on va effectuer une différence entre chaque bit, et éventuellement propager une retenue vers la gauche.
Plusieurs cas peuvent se présenter :

\medskip

\begin{center}
\rule{1.0\linewidth}{0.75pt}

\begin{table}[ht!]
  \centering
  \begin{minipage}{0.15\textwidth}

\par\vspace{3\oplineheight}
\phantom{\oplput(1,2.25){0}}\oplput(2,2.25){0}
\phantom{\oplput(1,1){0}}\oplput(2,1){0}
\oplput(-0.25,1.05){$-$}
\ophline(0,0.8){3}
\oplput(2,0){0}

\medskip

$ \Rightarrow 0 $

  \end{minipage}
  \hfillx
  \vrule\begin{minipage}{0.05\textwidth}

  \end{minipage}
  \hfillx
  \begin{minipage}{0.15\textwidth}

\par\vspace{3\oplineheight}
\phantom{\oplput(1,2.25){0}}\oplput(2,2.25){1}
\phantom{\oplput(1,1){0}}\oplput(2,1){1}
\oplput(-0.25,1.05){$-$}
\ophline(0,0.8){3}
\oplput(2,0){0}

\medskip

$ \Rightarrow 0 $

  \end{minipage}
  \hfillx
  \vrule\begin{minipage}{0.05\textwidth}

  \end{minipage}
  \hfillx
  \begin{minipage}{0.15\textwidth}

\par\vspace{3\oplineheight}
\phantom{\oplput(1,2.25){0}}\oplput(2,2.25){1}
\phantom{\oplput(1,1){0}}\oplput(2,1){0}
\oplput(-0.25,1.05){$-$}
\ophline(0,0.8){3}
\oplput(2,0){1}

\medskip

$ \Rightarrow 1 $

  \end{minipage}
  \hfillx
  \vrule\begin{minipage}{0.05\textwidth}

  \end{minipage}
  \hfillx
  \begin{minipage}{0.15\textwidth}

\par\vspace{3\oplineheight}
\phantom{\oplput(1,2.25){0}}\oplput(1.5,2.25){\scriptsize \textcolor{red}{1}}\oplput(2,2.25){0}
\oplput(-0.25,1.05){$-$}
\phantom{\oplput(1,1){0}}\oplput(0.75,1.75){\tiny \textcolor{red}{\tinycircled{1}}}\oplput(2,1){1}
\ophline(0,0.8){3}
\oplput(1,0){\textcolor{gray(x11gray)}{1}}\oplput(2,0){1}

\medskip

\textcolor{gray(x11gray)}{$ \mathit{\Rightarrow -1} $}

  \end{minipage}
\end{table}

%%%%%%%%%%%%%%%%%%%%%%%%%%%%

\rule{1.0\linewidth}{0.75pt}

\begin{table}[ht!]
  \centering
  \begin{minipage}{0.15\textwidth}

\par\vspace{3\oplineheight}
\phantom{\oplput(1,2.25){0}}\oplput(2,2.25){0}\oplput(3,2.25){.}
\oplput(-0.25,1.05){$-$}
\phantom{\oplput(1,1){0}}\oplput(2,1){1}\oplput(1.75,1.75){\tiny \textcolor{blue}{\tinycircled{1}}}\oplput(3,1){.}
\ophline(0,0.8){4}
\phantom{\oplput(1,0){.}\oplput(2,0){.}}\oplput(3,0){.}

  \end{minipage}
  \hfillx
  \begin{minipage}{0.1\textwidth}

\phantom{ } $ \Rightarrow $

  \end{minipage}
  \hfillx
  \begin{minipage}{0.15\textwidth}

\par\vspace{3\oplineheight}
\phantom{\oplput(1,2.25){0}}\oplput(2,2.25){0}\oplput(3,2.25){.}
\oplput(-0.25,1.05){$-$}
\phantom{\oplput(1,1){0}}\oplput(1.75,1.25){\scriptsize 1}\oplput(2.25,1.25){\scriptsize 0}\oplput(3,1){.}
\ophline(0,0.8){4}
\phantom{\oplput(1,0){.}\oplput(2,0){.}}\oplput(3,0){.}

  \end{minipage}
  \hfillx
  \begin{minipage}{0.1\textwidth}

\phantom{ } $ \Rightarrow $

  \end{minipage}
  \hfillx
  \begin{minipage}{0.15\textwidth}

\par\vspace{3\oplineheight}
\phantom{\oplput(1,2.25){0}}\oplput(1.5,2.25){\scriptsize \textcolor{red}{1}}\oplput(2,2.25){0}\oplput(3,2.25){.}
\oplput(-0.25,1.05){$-$}
\phantom{\oplput(1,1){0}}\oplput(0.75,1.75){\tiny \textcolor{red}{\tinycircled{1}}}\oplput(1.75,1.25){\scriptsize 1}\oplput(2.25,1.25){\scriptsize 0}\oplput(3,1){.}
\ophline(0,0.8){4}
\oplput(1,0){\textcolor{gray(x11gray)}{1}}\oplput(2,0){0}\oplput(3,0){.}

  \end{minipage}
  \hfillx
  \begin{minipage}{0.1\textwidth}

\phantom{ } $ \Rightarrow $

  \end{minipage}
  \hfillx
  \begin{minipage}{0.15\textwidth}

\par\vspace{3\oplineheight}
\phantom{\oplput(1,2.25){0}}\oplput(1.75,2){\scriptsize 1}\oplput(2.25,2){\scriptsize 0}\oplput(3,2.25){.}
\oplput(-0.25,1.05){$-$}
\phantom{\oplput(1,1){0}}\oplput(0.75,1.75){\tiny \textcolor{red}{\tinycircled{1}}}\oplput(1.75,1.25){\scriptsize 1}\oplput(2.25,1.25){\scriptsize 0}\oplput(3,1){.}
\ophline(0,0.8){4}
\oplput(1,0){\textcolor{gray(x11gray)}{1}}\oplput(2,0){0}\oplput(3,0){.}

  \end{minipage}
  \hfillx
  \begin{minipage}{0.1\textwidth}

\textcolor{gray(x11gray)}{$ \mathit{\Rightarrow -2} $}

  \end{minipage}
\end{table}

%%%%%%%%%%%%%%%%%%%%%%%%%%%%

\rule{1.0\linewidth}{0.75pt}

\begin{table}[ht!]
  \centering
  \begin{minipage}{0.15\textwidth}

\par\vspace{3\oplineheight}
\phantom{\oplput(1,2.25){0}}\oplput(2,2.25){1}\oplput(3,2.25){.}
\oplput(-0.25,1.05){$-$}
\phantom{\oplput(1,1){0}}\oplput(2,1){1}\oplput(1.75,1.75){\tiny \textcolor{blue}{\tinycircled{1}}}\oplput(3,1){.}
\ophline(0,0.8){4}
\phantom{\oplput(1,0){.}\oplput(2,0){.}}\oplput(3,0){.}

  \end{minipage}
  \hfillx
  \begin{minipage}{0.1\textwidth}

\phantom{ } $ \Rightarrow $

  \end{minipage}
  \hfillx
  \begin{minipage}{0.15\textwidth}

\par\vspace{3\oplineheight}
\phantom{\oplput(1,2.25){0}}\oplput(2,2.25){1}\oplput(3,2.25){.}
\oplput(-0.25,1.05){$-$}
\phantom{\oplput(1,1){0}}\oplput(1.75,1.25){\scriptsize 1}\oplput(2.25,1.25){\scriptsize 0}\oplput(3,1){.}
\ophline(0,0.8){4}
\phantom{\oplput(1,0){.}\oplput(2,0){.}}\oplput(3,0){.}

  \end{minipage}
  \hfillx
  \begin{minipage}{0.1\textwidth}

\phantom{ } $ \Rightarrow $

  \end{minipage}
  \hfillx
  \begin{minipage}{0.15\textwidth}

\par\vspace{3\oplineheight}
\phantom{\oplput(1,2.25){0}}\oplput(1.5,2.25){\scriptsize \textcolor{red}{1}}\oplput(2,2.25){1}\oplput(3,2.25){.}
\oplput(-0.25,1.05){$-$}
\phantom{\oplput(1,1){0}}\oplput(0.75,1.75){\tiny \textcolor{red}{\tinycircled{1}}}\oplput(1.75,1.25){\scriptsize 1}\oplput(2.25,1.25){\scriptsize 0}\oplput(3,1){.}
\ophline(0,0.8){4}
\oplput(1,0){\textcolor{gray(x11gray)}{1}}\oplput(2,0){1}\oplput(3,0){.}

  \end{minipage}
  \hfillx
  \begin{minipage}{0.1\textwidth}

\phantom{ } $ \Rightarrow $

  \end{minipage}
  \hfillx
  \begin{minipage}{0.15\textwidth}

\par\vspace{3\oplineheight}
\phantom{\oplput(1,2.25){0}}\oplput(1.75,2){\scriptsize 1}\oplput(2.25,2){\scriptsize 1}\oplput(3,2.25){.}
\oplput(-0.25,1.05){$-$}
\phantom{\oplput(1,1){0}}\oplput(0.75,1.75){\tiny \textcolor{red}{\tinycircled{1}}}\oplput(1.75,1.25){\scriptsize 1}\oplput(2.25,1.25){\scriptsize 0}\oplput(3,1){.}
\ophline(0,0.8){4}
\oplput(1,0){\textcolor{gray(x11gray)}{1}}\oplput(2,0){1}\oplput(3,0){.}

  \end{minipage}
  \hfillx
  \begin{minipage}{0.1\textwidth}

\textcolor{gray(x11gray)}{$ \mathit{\Rightarrow -1} $}

  \end{minipage}
\end{table}

\rule{1.0\linewidth}{0.75pt}
\end{center}

\vfillFirst

%%%%%%%%%%%%%%%%%%%%%%%%%%%%%%%%%%%%%%%%%%%%%%%%%%%%%%%%

\medskip

Pour illustrer ces cas, voici un exemple simple avec la soustraction entre $ \% 1100 $ (12) et $ \% 0111 $ (7) :

\begin{center}

\begin{table}[ht!]
  \centering
  \begin{minipage}{0.15\textwidth}

\par\vspace{3\oplineheight}
\phantom{\oplput(0,3){\scriptsize }}
\oplput(1,2.25){1}\oplput(2,2.25){1}\oplput(3,2.25){0}\oplput(4,2.25){0}
\oplput(-0.25,1.05){$-$}
\oplput(1,1){0}\oplput(2,1){1}\oplput(3,1){1}\oplput(4,1){1}
\ophline(0,0.8){5}
\phantom{\oplput(4,0){ }}

  \end{minipage}
  \hfillx
  \begin{minipage}{0.1\textwidth}
  \centering

$ \Rightarrow $

  \end{minipage}
  \hfillx
  \begin{minipage}{0.15\textwidth}

\par\vspace{3\oplineheight}
\phantom{\oplput(0,3){\scriptsize }}
\oplput(1,2.25){1}\oplput(2,2.25){1}\oplput(3,2.25){0}\oplput(3.6,2.25){\scriptsize \textcolor{red}{1}}\oplput(4,2.25){0}
\oplput(-0.25,1.05){$-$}
\oplput(1,1){0}\oplput(2,1){1}\oplput(2.75,1.75){\tiny \textcolor{red}{\tinycircled{1}}}\oplput(3,1){1}\oplput(4,1){1}
\ophline(0,0.8){5}
\oplput(4,0){1}

  \end{minipage}
  \hfillx
  \begin{minipage}{0.1\textwidth}
  \centering

$ \Rightarrow $

  \end{minipage}
  \hfillx
  \begin{minipage}{0.15\textwidth}

\par\vspace{3\oplineheight}
\phantom{\oplput(0,3){\scriptsize }}
\oplput(1,2.25){1}\oplput(2,2.25){1}\oplput(2.6,2.25){\scriptsize \textcolor{blue}{1}}\oplput(3,2.25){0}\oplput(3.6,2.25){\scriptsize \textcolor{red}{1}}\oplput(4,2.25){0}
\oplput(-0.25,1.05){$-$}
\oplput(1,1){0}\oplput(1.75,1.75){\tiny \textcolor{blue}{\tinycircled{1}}}\oplput(2,1){1}\oplput(2.75,1.75){\tiny \textcolor{red}{\tinycircled{1}}}\oplput(3,1){1}\oplput(4,1){1}
\ophline(0,0.8){5}
\oplput(3,0){0}\oplput(4,0){1}

  \end{minipage}
  \hfillx
  \begin{minipage}{0.1\textwidth}
  \centering

$ \Rightarrow $

  \end{minipage}
  \hfillx
  \begin{minipage}{0.15\textwidth}

\par\vspace{3\oplineheight}
\phantom{\oplput(0,3){\scriptsize }}
\oplput(1,2.25){1}\oplput(1.6,2.25){\scriptsize \textcolor{green(htmlcssgreen)}{1}}\oplput(2,2.25){1}\oplput(2.6,2.25){\scriptsize \textcolor{blue}{1}}\oplput(3,2.25){0}\oplput(3.6,2.25){\scriptsize \textcolor{red}{1}}\oplput(4,2.25){0}
\oplput(-0.25,1.05){$-$}
\oplput(0.75,1.75){\tiny \textcolor{green(htmlcssgreen)}{\tinycircled{1}}}\oplput(1,1){0}\oplput(1.75,1.75){\tiny \textcolor{blue}{\tinycircled{1}}}\oplput(2,1){1}\oplput(2.75,1.75){\tiny \textcolor{red}{\tinycircled{1}}}\oplput(3,1){1}\oplput(4,1){1}
\ophline(0,0.8){5}
\oplput(2,0){1}\oplput(3,0){0}\oplput(4,0){1}

  \end{minipage}
  \hfillx
  \begin{minipage}{0.1\textwidth}
  \centering

$ \Rightarrow $

  \end{minipage}
  \hfillx
  \begin{minipage}{0.15\textwidth}

\par\vspace{3\oplineheight}
\phantom{\oplput(0,3){\scriptsize }}
\oplput(1,2.25){1}\oplput(1.6,2.25){\scriptsize \textcolor{green(htmlcssgreen)}{1}}\oplput(2,2.25){1}\oplput(2.6,2.25){\scriptsize \textcolor{blue}{1}}\oplput(3,2.25){0}\oplput(3.6,2.25){\scriptsize \textcolor{red}{1}}\oplput(4,2.25){0}
\oplput(-0.25,1.05){$-$}
\oplput(0.75,1.75){\tiny \textcolor{green(htmlcssgreen)}{\tinycircled{1}}}\oplput(1,1){0}\oplput(1.75,1.75){\tiny \textcolor{blue}{\tinycircled{1}}}\oplput(2,1){1}\oplput(2.75,1.75){\tiny \textcolor{red}{\tinycircled{1}}}\oplput(3,1){1}\oplput(4,1){1}
\ophline(0,0.8){5}
\oplput(1,0){0}\oplput(2,0){1}\oplput(3,0){0}\oplput(4,0){1}

  \end{minipage}
\end{table}

\end{center}

\vspace*{-0.5cm}

On en déduit donc que : $ 1100_{2} - 0111_{2} = 0101_{2} $ (5)

\medskip

%%%%%%%%%%%%%%%%%%%%%%%%%%%%%%%%%%%%%%%%%%%%%%%%%%%%%%%%

\bigskip

Voici un autre exemple plus complet avec la soustraction entre $ \% 11 \, 1001 $ (57) et $ \% 01 \, 1101 $ (29) :

\begin{center}

\begin{table}[ht!]
  \centering
  \begin{minipage}{0.15\textwidth}

\par\vspace{3\oplineheight}
\phantom{\oplput(0,3){\scriptsize }}
\oplput(1,2.25){1}\oplput(2,2.25){1}\oplput(3,2.25){1}\oplput(4,2.25){0}\oplput(5,2.25){0}\oplput(6,2.25){1}
\oplput(-0.25,1.05){$-$}
\oplput(1,1){0}\oplput(2,1){1}\oplput(3,1){1}\oplput(4,1){1}\oplput(5,1){0}\oplput(6,1){1}
\ophline(0,0.8){7}
\phantom{\oplput(6,0){ }}

  \end{minipage}
  \hfillx
  \begin{minipage}{0.1\textwidth}
  \centering

$ \Rightarrow $

  \end{minipage}
  \hfillx
  \begin{minipage}{0.15\textwidth}

\par\vspace{3\oplineheight}
\phantom{\oplput(0,3){\scriptsize }}
\oplput(1,2.25){1}\oplput(2,2.25){1}\oplput(3,2.25){1}\oplput(4,2.25){0}\oplput(5,2.25){0}\oplput(6,2.25){1}
\oplput(-0.25,1.05){$-$}
\oplput(1,1){0}\oplput(2,1){1}\oplput(3,1){1}\oplput(4,1){1}\oplput(5,1){0}\oplput(6,1){1}
\ophline(0,0.8){7}
\oplput(6,0){0}

  \end{minipage}
  \hfillx
  \begin{minipage}{0.1\textwidth}
  \centering

$ \Rightarrow $

  \end{minipage}
  \hfillx
  \begin{minipage}{0.15\textwidth}

\par\vspace{3\oplineheight}
\phantom{\oplput(0,3){\scriptsize }}
\oplput(1,2.25){1}\oplput(2,2.25){1}\oplput(3,2.25){1}\oplput(4,2.25){0}\oplput(5,2.25){0}\oplput(6,2.25){1}
\oplput(-0.25,1.05){$-$}
\oplput(1,1){0}\oplput(2,1){1}\oplput(3,1){1}\oplput(4,1){1}\oplput(5,1){0}\oplput(6,1){1}
\ophline(0,0.8){7}
\oplput(5,0){0}\oplput(6,0){0}

  \end{minipage}
  \hfillx
  \begin{minipage}{0.1\textwidth}
  \centering

$ \Rightarrow $

  \end{minipage}
  \hfillx
  \begin{minipage}{0.15\textwidth}

\par\vspace{3\oplineheight}
\phantom{\oplput(0,3){\scriptsize }}
\oplput(1,2.25){1}\oplput(2,2.25){1}\oplput(3,2.25){1}\oplput(3.5,2.25){\scriptsize \textcolor{red}{1}}\oplput(4,2.25){0}\oplput(5,2.25){0}\oplput(6,2.25){1}
\oplput(-0.25,1.05){$-$}
\oplput(1,1){0}\oplput(2,1){1}\oplput(2.75,1.75){\tiny \textcolor{red}{\tinycircled{1}}}\oplput(3,1){1}\oplput(4,1){1}\oplput(5,1){0}\oplput(6,1){1}
\ophline(0,0.8){7}
\oplput(4,0){.}\oplput(5,0){0}\oplput(6,0){0}

  \end{minipage}
  \hfillx
  \begin{minipage}{0.1\textwidth}
  \centering

$ \rightarrow $

  \end{minipage}
  \hfillx
  \begin{minipage}{0.15\textwidth}

\par\vspace{3\oplineheight}
\phantom{\oplput(0,3){\scriptsize }}
\oplput(1,2.25){1}\oplput(2,2.25){1}\oplput(3,2.25){1}\oplput(3.5,2.25){\scriptsize \textcolor{red}{1}}\oplput(4,2.25){0}\oplput(5,2.25){0}\oplput(6,2.25){1}
\oplput(-0.25,1.05){$-$}
\oplput(1,1){0}\oplput(2,1){1}\oplput(2.75,1.75){\tiny \textcolor{red}{\tinycircled{1}}}\oplput(3,1){1}\oplput(4,1){1}\oplput(5,1){0}\oplput(6,1){1}
\ophline(0,0.8){7}
\oplput(4,0){1}\oplput(5,0){0}\oplput(6,0){0}

  \end{minipage}
  \hfillx
  \begin{minipage}{0.1\textwidth}
  \centering

$ \Rightarrow $

  \end{minipage}
\end{table}

%%%%%%%%%%%

\begin{table}[ht!]
  \centering
  \begin{minipage}{0.1\textwidth}
  \centering

$ \Rightarrow $

  \end{minipage}
  \hfillx
  \begin{minipage}{0.15\textwidth}

\par\vspace{3\oplineheight}
\phantom{\oplput(0,3){\scriptsize }}
\oplput(1,2.25){1}\oplput(2,2.25){1}\oplput(2.5,2.25){\scriptsize \textcolor{blue}{1}}\oplput(3,2.25){1}\oplput(3.5,2.25){\scriptsize \textcolor{red}{1}}\oplput(4,2.25){0}\oplput(5,2.25){0}\oplput(6,2.25){1}
\oplput(-0.25,1.05){$-$}
\oplput(1,1){0}\oplput(1.75,1.75){\tiny \textcolor{blue}{\tinycircled{1}}}\oplput(2,1){1}\oplput(2.75,1.75){\tiny \textcolor{red}{\tinycircled{1}}}\oplput(3,1){1}\oplput(4,1){1}\oplput(5,1){0}\oplput(6,1){1}
\ophline(0,0.8){7}
\oplput(3,0){.}\oplput(4,0){1}\oplput(5,0){0}\oplput(6,0){0}

  \end{minipage}
  \hfillx
  \begin{minipage}{0.1\textwidth}
  \centering

$ \rightarrow $

  \end{minipage}
  \hfillx
  \begin{minipage}{0.15\textwidth}

\par\vspace{3\oplineheight}
\phantom{\oplput(0,3){\scriptsize }}
\oplput(1,2.25){1}\oplput(2,2.25){1}\oplput(2.5,2.25){\scriptsize \textcolor{blue}{1}}\oplput(3,2.25){1}\oplput(3.5,2.25){\scriptsize \textcolor{red}{1}}\oplput(4,2.25){0}\oplput(5,2.25){0}\oplput(6,2.25){1}
\oplput(-0.25,1.05){$-$}
\oplput(1,1){0}\oplput(1.75,1.75){\tiny \textcolor{blue}{\tinycircled{1}}}\oplput(2,1){1}\oplput(2.75,1.75){\tiny \textcolor{red}{\tinycircled{1}}}\oplput(3,1){1}\oplput(4,1){1}\oplput(5,1){0}\oplput(6,1){1}
\ophline(0,0.8){7}
\oplput(3,0){1}\oplput(4,0){1}\oplput(5,0){0}\oplput(6,0){0}

  \end{minipage}
  \hfillx
  \begin{minipage}{0.1\textwidth}
  \centering

$ \Rightarrow $

  \end{minipage}
  \hfillx
  \begin{minipage}{0.15\textwidth}

\par\vspace{3\oplineheight}
\phantom{\oplput(0,3){\scriptsize }}
\oplput(1,2.25){1}\oplput(1.5,2.25){\scriptsize \textcolor{green(htmlcssgreen)}{1}}\oplput(2,2.25){1}\oplput(2.5,2.25){\scriptsize \textcolor{blue}{1}}\oplput(3,2.25){1}\oplput(3.5,2.25){\scriptsize \textcolor{red}{1}}\oplput(4,2.25){0}\oplput(5,2.25){0}\oplput(6,2.25){1}
\oplput(-0.25,1.05){$-$}
\oplput(0.75,1.75){\tiny \textcolor{green(htmlcssgreen)}{\tinycircled{1}}}\oplput(1,1){0}\oplput(1.75,1.75){\tiny \textcolor{blue}{\tinycircled{1}}}\oplput(2,1){1}\oplput(2.75,1.75){\tiny \textcolor{red}{\tinycircled{1}}}\oplput(3,1){1}\oplput(4,1){1}\oplput(5,1){0}\oplput(6,1){1}
\ophline(0,0.8){7}
\oplput(2,0){.}\oplput(3,0){1}\oplput(4,0){1}\oplput(5,0){0}\oplput(6,0){0}

  \end{minipage}
  \hfillx
  \begin{minipage}{0.1\textwidth}
  \centering

$ \rightarrow $

  \end{minipage}
  \hfillx
  \begin{minipage}{0.15\textwidth}

\par\vspace{3\oplineheight}
\phantom{\oplput(0,3){\scriptsize }}
\oplput(1,2.25){1}\oplput(1.5,2.25){\scriptsize \textcolor{green(htmlcssgreen)}{1}}\oplput(2,2.25){1}\oplput(2.5,2.25){\scriptsize \textcolor{blue}{1}}\oplput(3,2.25){1}\oplput(3.5,2.25){\scriptsize \textcolor{red}{1}}\oplput(4,2.25){0}\oplput(5,2.25){0}\oplput(6,2.25){1}
\oplput(-0.25,1.05){$-$}
\oplput(0.75,1.75){\tiny \textcolor{green(htmlcssgreen)}{\tinycircled{1}}}\oplput(1,1){0}\oplput(1.75,1.75){\tiny \textcolor{blue}{\tinycircled{1}}}\oplput(2,1){1}\oplput(2.75,1.75){\tiny \textcolor{red}{\tinycircled{1}}}\oplput(3,1){1}\oplput(4,1){1}\oplput(5,1){0}\oplput(6,1){1}
\ophline(0,0.8){7}
\oplput(2,0){1}\oplput(3,0){1}\oplput(4,0){1}\oplput(5,0){0}\oplput(6,0){0}

  \end{minipage}
  \hfillx
  \begin{minipage}{0.1\textwidth}
  \centering

$ \Rightarrow $

  \end{minipage}
  \hfillx
  \begin{minipage}{0.15\textwidth}

\par\vspace{3\oplineheight}
\phantom{\oplput(0,3){\scriptsize }}
\oplput(1,2.25){1}\oplput(1.5,2.25){\scriptsize \textcolor{green(htmlcssgreen)}{1}}\oplput(2,2.25){1}\oplput(2.5,2.25){\scriptsize \textcolor{blue}{1}}\oplput(3,2.25){1}\oplput(3.5,2.25){\scriptsize \textcolor{red}{1}}\oplput(4,2.25){0}\oplput(5,2.25){0}\oplput(6,2.25){1}
\oplput(-0.25,1.05){$-$}
\oplput(0.75,1.75){\tiny \textcolor{green(htmlcssgreen)}{\tinycircled{1}}}\oplput(1,1){0}\oplput(1.75,1.75){\tiny \textcolor{blue}{\tinycircled{1}}}\oplput(2,1){1}\oplput(2.75,1.75){\tiny \textcolor{red}{\tinycircled{1}}}\oplput(3,1){1}\oplput(4,1){1}\oplput(5,1){0}\oplput(6,1){1}
\ophline(0,0.8){7}
\oplput(1,0){0}\oplput(2,0){1}\oplput(3,0){1}\oplput(4,0){1}\oplput(5,0){0}\oplput(6,0){0}

  \end{minipage}
\end{table}

\end{center}

\vspace*{-0.5cm}

On en déduit donc que : $ 11 \, 1001_{2} - 01 \, 1101_{2} = 01 \, 1100_{2} $ (28)

%\medskip
\bigskip

Cette technique a l'avantage de donner le résultat exact immédiatement, mais au prix de nombreuses retenues à propager.

\vfillLast

%\bigskip
\clearpage

%%%%%%%%%%%%%%%%%%%%%%%%%%%%%%%%%%%%%%%%%%%%%%%%%%%%%%%%%%%%%%%%%%%%%%%%%%%%%

\section{Division}

\medskip

La division binaire se déroule comme une division décimale, néanmoins, il faut garder en tête que les nombres ne sont constitués que de 0 et de 1, c'est-à-dire qu'il faut appliquer plusieurs décalages d'affilés sur le nombre divisé et donc ajouter plusieurs 0 de suite au dividende.

%\medskip
%
%On notera que diviser par une puissance de 2 (2, 4, 8, ...) implique de simplement décaler la virgule vers la gauche.

\medskip

Ainsi, pour diviser \og $ \% \, 10 \, 1101 $ \fg{} (45) par \og $ \% \, 11 $ \fg{} (3), on fera :

\begin{center}

\begin{table}[ht!]
  \centering
  \begin{minipage}{0.15\textwidth}

\par\vspace{6\oplineheight}
\oplput(0,5){1}\oplput(1,5){0}\oplput(2,5){1}\oplput(3,5){1}\oplput(4,5){0}\oplput(5,5){1}
\oplput(7,5){1}\oplput(8,5){1}
%\opvline(6.35,5.75){6}
\opvline(6.35,6){6}
\ophline(6.35,4.75){4}

  \end{minipage}
  \hfillx
  \begin{minipage}{0.1\textwidth}
    \centering

$ \Rightarrow $

  \end{minipage}
  \hfillx
  \begin{minipage}{0.15\textwidth}

\par\vspace{6\oplineheight}
\oplput(0,5){\textcolor{red}{1}}\oplput(1,5){0}\oplput(2,5){1}\oplput(3,5){1}\oplput(4,5){0}\oplput(5,5){1}
\oplput(7,5){1}\oplput(8,5){1}
%\opvline(6.35,5.75){6}
\opvline(6.35,6){6}
\ophline(6.35,4.75){4}
\oplput(7,4){\textcolor{red}{0}}

  \end{minipage}
  \hfillx
  \begin{minipage}{0.1\textwidth}
    \centering

$ \Rightarrow $

  \end{minipage}
  \hfillx
  \begin{minipage}{0.15\textwidth}

\par\vspace{6\oplineheight}
\oplput(0,5){\textcolor{red}{1}}\oplput(1,5){\textcolor{red}{0}}\oplput(2,5){1}\oplput(3,5){1}\oplput(4,5){0}\oplput(5,5){1}
\oplput(7,5){1}\oplput(8,5){1}
%\opvline(6.35,5.75){6}
\opvline(6.35,6){6}
\ophline(6.35,4.75){4}
\oplput(7,4){0}\oplput(8,4){\textcolor{red}{0}}

  \end{minipage}
  \hfillx
  \begin{minipage}{0.1\textwidth}
    \centering

$ \Rightarrow $

  \end{minipage}
\end{table}


\begin{table}[ht!]
  \centering
  \begin{minipage}{0.15\textwidth}

\par\vspace{6\oplineheight}
\oplput(0,5){\textcolor{red}{1}}\oplput(1,5){\textcolor{red}{0}}\oplput(2,5){\textcolor{red}{1}}\oplput(3,5){1}\oplput(4,5){0}\oplput(5,5){1}
\oplput(7,5){1}\oplput(8,5){1}
%\opvline(6.35,5.75){6}
\opvline(6.35,6){6}
\ophline(6.35,4.75){4}
\oplput(7,4){0}\oplput(8,4){0}\oplput(9,4){\textcolor{red}{1}}

  \end{minipage}
  \hfillx
  \begin{minipage}{0.1\textwidth}
    \centering

$ \rightarrow $

  \end{minipage}
  \hfillx
  \begin{minipage}{0.15\textwidth}

\par\vspace{6\oplineheight}
\oplput(0,5){\textcolor{red}{1}}\oplput(1,5){\textcolor{red}{0}}\oplput(2,5){\textcolor{red}{1}}\oplput(3,5){1}\oplput(4,5){0}\oplput(5,5){1}
\oplput(7,5){1}\oplput(8,5){1}
%\opvline(6.35,5.75){6}
\opvline(6.35,6){6}
\ophline(6.35,4.75){4}
\oplput(7,4){0}\oplput(8,4){0}\oplput(9,4){\textcolor{red}{1}}
\oplput(1,4){1}\oplput(2,4){1}
\oplput(-1.25,4.05){$-$}
\ophline(-1,3.8){4}

  \end{minipage}
  \hfillx
  \begin{minipage}{0.1\textwidth}
    \centering

$ \rightarrow $

  \end{minipage}
  \hfillx
  \begin{minipage}{0.15\textwidth}

\par\vspace{6\oplineheight}
\oplput(0,5){\textcolor{red}{1}}\oplput(1,5){\textcolor{red}{0}}\oplput(2,5){\textcolor{red}{1}}\oplput(3,5){1}\oplput(4,5){0}\oplput(5,5){1}
\oplput(7,5){1}\oplput(8,5){1}
%\opvline(6.35,5.75){6}
\opvline(6.35,6){6}
\ophline(6.35,4.75){4}
\oplput(7,4){0}\oplput(8,4){0}\oplput(9,4){\textcolor{red}{1}}
\oplput(1,4){1}\oplput(2,4){1}
\oplput(-1.25,4.05){$-$}
\ophline(-1,3.8){4}
\oplput(0,3){0}\oplput(1,3){1}\oplput(2,3){0}

  \end{minipage}
  \hfillx
  \begin{minipage}{0.1\textwidth}
    \centering

$ \rightarrow $

  \end{minipage}
\end{table}


\begin{table}[ht!]
  \centering
  \begin{minipage}{0.15\textwidth}

\par\vspace{6\oplineheight}
\oplput(0,5){\textcolor{red}{1}}\oplput(1,5){\textcolor{red}{0}}\oplput(2,5){\textcolor{red}{1}}\oplput(3,5){1}\oplput(4,5){0}\oplput(5,5){1}
\oplput(7,5){1}\oplput(8,5){1}
%\opvline(6.35,5.75){6}
\opvline(6.35,6){6}
\ophline(6.35,4.75){4}
\oplput(7,4){0}\oplput(8,4){0}\oplput(9,4){\textcolor{red}{1}}
\oplput(1,4){1}\oplput(2,4){0}

  \end{minipage}
  \hfillx
  \begin{minipage}{0.1\textwidth}
    \centering

$ \Rightarrow $

  \end{minipage}
  \hfillx
  \begin{minipage}{0.15\textwidth}

\par\vspace{6\oplineheight}
\oplput(0,5){1}\oplput(1,5){0}\oplput(2,5){1}\oplput(3,5){1}\oplput(4,5){0}\oplput(5,5){1}
\oplput(7,5){1}\oplput(8,5){1}
%\opvline(6.35,5.75){6}
\opvline(6.35,6){6}
\ophline(6.35,4.75){4}
\oplput(7,4){0}\oplput(8,4){0}\oplput(9,4){1}\oplput(10,4){\textcolor{red}{1}}
\oplput(1,4){\textcolor{red}{1}}\oplput(2,4){\textcolor{red}{0}}\oplput(3,4){\textcolor{red}{1}}

  \end{minipage}
  \hfillx
  \begin{minipage}{0.1\textwidth}
    \centering

$ \rightarrow $

  \end{minipage}
  \hfillx
  \begin{minipage}{0.15\textwidth}

\par\vspace{6\oplineheight}
\oplput(0,5){1}\oplput(1,5){0}\oplput(2,5){1}\oplput(3,5){1}\oplput(4,5){0}\oplput(5,5){1}
\oplput(7,5){1}\oplput(8,5){1}
%\opvline(6.35,5.75){6}
\opvline(6.35,6){6}
\ophline(6.35,4.75){4}
\oplput(7,4){0}\oplput(8,4){0}\oplput(9,4){1}\oplput(10,4){\textcolor{red}{1}}
\oplput(1,4){\textcolor{red}{1}}\oplput(2,4){\textcolor{red}{0}}\oplput(3,4){\textcolor{red}{1}}
\oplput(2,3){1}\oplput(3,3){1}
\oplput(-0.25,3.05){$-$}
\ophline(0,2.8){4}
\oplput(1,2){0}\oplput(2,2){1}\oplput(3,2){0}

  \end{minipage}
  \hfillx
  \begin{minipage}{0.1\textwidth}
    \centering

$ \rightarrow $

  \end{minipage}
\end{table}

\begin{table}[ht!]
  \centering
  \begin{minipage}{0.15\textwidth}

\par\vspace{6\oplineheight}
\oplput(0,5){1}\oplput(1,5){0}\oplput(2,5){1}\oplput(3,5){1}\oplput(4,5){0}\oplput(5,5){1}
\oplput(7,5){1}\oplput(8,5){1}
%\opvline(6.35,5.75){6}
\opvline(6.35,6){6}
\ophline(6.35,4.75){4}
\oplput(7,4){0}\oplput(8,4){0}\oplput(9,4){1}\oplput(10,4){\textcolor{red}{1}}
\oplput(1,4){\textcolor{red}{1}}\oplput(2,4){\textcolor{red}{0}}\oplput(3,4){\textcolor{red}{1}}
\oplput(2,3){1}\oplput(3,3){0}

  \end{minipage}
  \hfillx
  \begin{minipage}{0.1\textwidth}
    \centering

$ \Rightarrow $

  \end{minipage}
  \hfillx
  \begin{minipage}{0.15\textwidth}

\par\vspace{6\oplineheight}
\oplput(0,5){1}\oplput(1,5){0}\oplput(2,5){1}\oplput(3,5){1}\oplput(4,5){0}\oplput(5,5){1}
\oplput(7,5){1}\oplput(8,5){1}
%\opvline(6.35,5.75){6}
\opvline(6.35,6){6}
\ophline(6.35,4.75){4}
\oplput(7,4){0}\oplput(8,4){0}\oplput(9,4){1}\oplput(10,4){1}\oplput(11,4){\textcolor{red}{1}}
\oplput(1,4){1}\oplput(2,4){0}\oplput(3,4){1}
\oplput(2,3){\textcolor{red}{1}}\oplput(3,3){\textcolor{red}{0}}\oplput(4,3){\textcolor{red}{0}}

  \end{minipage}
  \hfillx
  \begin{minipage}{0.1\textwidth}
    \centering

$ \rightarrow $

  \end{minipage}
  \hfillx
  \begin{minipage}{0.15\textwidth}

\par\vspace{6\oplineheight}
\oplput(0,5){1}\oplput(1,5){0}\oplput(2,5){1}\oplput(3,5){1}\oplput(4,5){0}\oplput(5,5){1}
\oplput(7,5){1}\oplput(8,5){1}
%\opvline(6.35,5.75){6}
\opvline(6.35,6){6}
\ophline(6.35,4.75){4}
\oplput(7,4){0}\oplput(8,4){0}\oplput(9,4){1}\oplput(10,4){1}\oplput(11,4){\textcolor{red}{1}}
\oplput(1,4){1}\oplput(2,4){0}\oplput(3,4){1}
\oplput(2,3){\textcolor{red}{1}}\oplput(3,3){\textcolor{red}{0}}\oplput(4,3){\textcolor{red}{0}}
\oplput(3,2){1}\oplput(4,2){1}
\oplput(0.75,2.05){$-$}
\ophline(1,1.8){4}
\oplput(2,1){0}\oplput(3,1){0}\oplput(4,1){1}

  \end{minipage}
  \hfillx
  \begin{minipage}{0.1\textwidth}
    \centering

$ \rightarrow $

  \end{minipage}
\end{table}

\begin{table}[ht!]
  \centering
  \begin{minipage}{0.2\textwidth}

\par\vspace{6\oplineheight}
\oplput(0,5){1}\oplput(1,5){0}\oplput(2,5){1}\oplput(3,5){1}\oplput(4,5){0}\oplput(5,5){1}
\oplput(7,5){1}\oplput(8,5){1}
%\opvline(6.35,5.75){6}
\opvline(6.35,6){6}
\ophline(6.35,4.75){4}
\oplput(7,4){0}\oplput(8,4){0}\oplput(9,4){1}\oplput(10,4){1}\oplput(11,4){\textcolor{red}{1}}
\oplput(1,4){1}\oplput(2,4){0}\oplput(3,4){1}
\oplput(2,3){\textcolor{red}{1}}\oplput(3,3){\textcolor{red}{0}}\oplput(4,3){\textcolor{red}{0}}
\oplput(4,2){1}

  \end{minipage}
  \hfillx
  \begin{minipage}{0.1\textwidth}
    \centering

$ \Rightarrow $

  \end{minipage}
  \hfillx
  \begin{minipage}{0.25\textwidth}

\par\vspace{6\oplineheight}
\oplput(0,5){1}\oplput(1,5){0}\oplput(2,5){1}\oplput(3,5){1}\oplput(4,5){0}\oplput(5,5){1}
\oplput(7,5){1}\oplput(8,5){1}
%\opvline(6.35,5.75){6}
\opvline(6.35,6){6}
\ophline(6.35,4.75){4}
\oplput(7,4){0}\oplput(8,4){0}\oplput(9,4){1}\oplput(10,4){1}\oplput(11,4){1}\oplput(12,4){\textcolor{red}{1}}
\oplput(1,4){1}\oplput(2,4){0}\oplput(3,4){1}
\oplput(2,3){1}\oplput(3,3){0}\oplput(4,3){0}
\oplput(4,2){\textcolor{red}{1}}\oplput(5,2){\textcolor{red}{1}}

  \end{minipage}
  \hfillx
  \begin{minipage}{0.1\textwidth}
    \centering

$ \Rightarrow $

  \end{minipage}
  \hfillx
  \begin{minipage}{0.2\textwidth}

\par\vspace{6\oplineheight}
\oplput(0,5){1}\oplput(1,5){0}\oplput(2,5){1}\oplput(3,5){1}\oplput(4,5){0}\oplput(5,5){1}
\oplput(7,5){1}\oplput(8,5){1}
%\opvline(6.35,5.75){6}
\opvline(6.35,6){6}
\ophline(6.35,4.75){4}
\oplput(7,4){0}\oplput(8,4){0}\oplput(9,4){1}\oplput(10,4){1}\oplput(11,4){1}\oplput(12,4){1}
\oplput(1,4){1}\oplput(2,4){0}\oplput(3,4){1}
\oplput(2,3){1}\oplput(3,3){0}\oplput(4,3){0}
\oplput(4,2){1}\oplput(5,2){1}
\oplput(5,1){\textcolor{red}{0}}

  \end{minipage}
  \hfillx
  \begin{minipage}{0.1\textwidth}
    \centering

\phantom{$ \Rightarrow $}

  \end{minipage}
\end{table}

\end{center}

On en déduit donc que : $ 10 \, 1101_{2} \div 11_{2} = 1111_{2} $ (15)

\medskip

On notera que diviser par une puissance de 2 (2, 4, 8, ...) implique de simplement décaler la virgule vers la gauche.

%\bigskip

%%%%%%%%%%%%%%%%%%%%%%%%%%%%%%%%%%%%%%%%%%%%%%%%%%%%%%%%%%%%%%%%%%%%

\vfillFirst

\vfillLast

\begin{center}
\textit{Ce document et ses illustrations ont été réalisés par Fabrice BOISSIER en septembre 2024}

\textit{(dernière mise à jour en octobre 2024)}
\end{center}

\end{document}
