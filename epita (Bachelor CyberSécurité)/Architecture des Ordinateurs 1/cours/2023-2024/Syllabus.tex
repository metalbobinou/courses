\documentclass[11pt,a4paper]{article}
\usepackage[utf8]{inputenc}
\usepackage[french]{babel}
\usepackage[T1]{fontenc}

\usepackage{amsmath}
\usepackage{amsfonts}
\usepackage{amssymb}

\newcommand{\TitreMatiere}{Architecture des Ordinateurs 1}
\newcommand{\TitreSeance}{Architecture des Ordinateurs 1}
\newcommand{\NumeroTD}{Syllabus}
\newcommand{\DateCours}{Septembre 2023}
\newcommand{\AnneeScolaire}{2023-2024}
\newcommand{\Organisation}{EPITA}
\newcommand{\NomAuteurA}{Fabrice BOISSIER}
\newcommand{\MailAuteurA}{fabrice.boissier@epita.fr}
\newcommand{\NomAuteurB}{ }
\newcommand{\MailAuteurB}{ }
\newcommand{\DocKeywords}{Algorithmique}
\newcommand{\DocLangue}{fr} % "en", "fr", ...

\usepackage{MetalCourseBooklet}

% Babel ne traduit pas toujours bien les tableaux et autres
\renewcommand*\frenchfigurename{%
    {\scshape Figure}%
}
\renewcommand*\frenchtablename{%
    {\scshape Tableau}%
}

% Ne pas afficher le numéro de la légende sur tableaux et figures
\captionsetup{format=sanslabel}


\begin{document}

\EncadreTitre

\bigskip


%\begin{center}
%\begin{tabular}{p{5cm} p{11cm}}
%\textbf{Commandes étudiées :} & \texttt{sh}, \texttt{bash}, \texttt{man}, \texttt{ls}, \texttt{mkdir}, \texttt{touch}, \texttt{chmod}, \texttt{mv}, \texttt{rm}, \texttt{rmdir}, \texttt{cat}, \texttt{file}, \texttt{which}, \texttt{which}\\
%
%\textbf{Builtins étudiées :} & \texttt{pwd}, \texttt{cd}, \texttt{exit}, \texttt{logout}, \texttt{echo}, \texttt{umask}, \texttt{type}, \texttt{>}, \texttt{>{}>}, \texttt{<}, \texttt{<{}<}, \texttt{|}\\
%
%\textbf{Notions étudiées :} & Shell, Manuels, Fichiers, Répertoires, Droits, Redirections\\
%\end{tabular}
%\end{center}

\bigskip


Ce document a pour objectif de lister les notions à voir / vues dans le cours.

\bigskip


\begin{enumerate}
% 1
\item Introduction Générale + Composants d'un ordinateur \textit{[21/09/2023]}\\
  TD :
  \begin{itemize}
%  \item Cours Magistral/Slides :
  \item Historique informatique et ordinateurs
  \item Description CPU, RAM, Carte Mère, ...
  \item Explications PC, north Bridge/South Bridge, ...
  \item 3 Bus (données, adresses, contrôle)
  \item Lecture/Ecriture en mémoire
  \item Utilité des interruptions
  \end{itemize}

\medskip

% 2
\item Conversion Binaire/Hexadécimal non-signés \textit{[28/09/2023]}\\
  TD :
  \begin{itemize}
  \item Puissances de 2
  \item Représentation des entiers en binaire
  \item Conversions décimaux vers/depuis binaires non-signés
  \item Symboles et valeurs hexadécimales
  \item Conversions décimaux vers/depuis hexadécimaux non-signés
  \item Calculs de tête avec les puissances de 2
  \item Vérifications rapides (bit de parité, puissances de 2, ...)
%  \item Conversions signés négatifs
  \end{itemize}

\medskip

% 3
\item Conversions Binaire/Hexadécimal signés \textit{[12/10/2023]}\\
  TD :
  \begin{itemize}
  \item Rappel conversions non-signés
  \item Complément à 2
  \item Complément à 1
  \item Conversions négatifs décimaux vers/depuis binaires signés
  \item Logique pour étendre la représentation des nombres binaires signés
  \item Conversions négatifs décimaux vers/depuis hexdécimaux signés
  \end{itemize}

\medskip

% 4
\item Code Gray et BCD \textit{[19/10/2023]}\\
  TD :
  \begin{itemize}
  \item Rappel conversions non-signés et signés
  \item Logique générale du Code Gray / Binaire réfléchi
  \item Conversions Binaire vers Gray (formule mathématique)
  \item Rappel opérateur XOR
  \item Conversions Binaire vers Gray (formule bit à bit)
  \item Conversions Gray vers Binaire (formule bit à bit)
  \item BCD / Code 8421
  \item Code 2421
  \end{itemize}

%\medskip
\clearpage

% 5
\item Représentation d'une structure en mémoire \textit{[09/11/2023]}\\
  TD :
  \begin{itemize}
  \item Rappel types algorithmiques
  \item Code ASCII et binaire/hexadécimal
  \item Structures et représentation réelle des types en mémoire
  \item \textit{padding} et \textit{\_\_attribute\_\_((packed))} en C
  \end{itemize}

\medskip

% 6
\item Flottants conversions $ 10 \Rightarrow 2 $ \textit{[16/11/2023]}\\
  TD :
  \begin{itemize}
  \item Flottants : difficulté de représentation
  \item Standard pour flottants : IEEE 754
  \item Cas spéciaux représentation IEEE 754 (zéros, infinis, NaN, normalisés/dénormalisés)
  \item Conversions décimaux vers IEEE 754 normalisés
  \end{itemize}

\medskip

% 7
\item Flottants conversions conversions $ 2 \Rightarrow 10 $, dénormalisés \textit{[23/11/2023]}\\ % $ 2 \Leftrightarrow 10 $
  TD :
  \begin{itemize}
  \item Rappel conversions décimaux vers IEEE 754 normalisés
  \item Conversions IEEE 754 normalisés vers décimaux
  \item Conversions dénormalisés
  \end{itemize}

\medskip

% 8
\item Portes Logiques \& Fonctions Logiques \textit{[30/11/2023]}\\
  TD :
  \begin{itemize}
  \item Notations NOT, AND, OR, NAND, NOR, XOR (symboles et portes logiques)
  \item Conjonction, Disjonction, Implication, \'Equivalence
  \item Contradiction \& Tautologie
  \item Tables de vérité
  \item Formules logiques
  \item Loi de De Morgan
  \item Règle d'assemblages des portes
  \end{itemize}

\medskip

% 9
\item Simplification des formules, Minterm, Maxterm, et Tableaux de Karnaugh \textit{[14/12/2023]}\\
  TD :
  \begin{itemize}
  \item Difficultés à simplifier des formules logiques
  \item Minterm
  \item Maxterm
  \item Tableaux de Karnaugh (autre méthode de simplification)
  \end{itemize}

\medskip

% 10
\item ??? \textit{[21/12/2023]}\\
  TD :
  \begin{itemize}
  \item ???
  \end{itemize}

\medskip


\end{enumerate}

\bigskip

Mode d'évaluation :

\begin{itemize}
\item 2 examens papiers (80\% note finale)
\item 2 QCM (20\% note finale)
\end{itemize}




\bigskip

\vfillFirst

\vfillLast


\begin{center}
\textit{Ce document et ses illustrations ont été réalisés par Fabrice BOISSIER en novembre 2023}
\end{center}

\end{document}
