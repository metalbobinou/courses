\documentclass[11pt,a4paper]{article}
\usepackage[utf8]{inputenc}
\usepackage[french]{babel}
\usepackage[T1]{fontenc}

\usepackage{amsmath}
\usepackage{amsfonts}
\usepackage{amssymb}

\newcommand{\NomAuteur}{Fabrice BOISSIER}
\newcommand{\TitreMatiere}{Architecture des Ordinateurs}
\newcommand{\NomUniv}{EPITA - Bachelor Cyber Sécurité}
\newcommand{\NiveauUniv}{CYBER1}
\newcommand{\NumGroupe}{CYBER1}
\newcommand{\AnneeUniv}{2022-2023}
\newcommand{\DateExam}{Novembre 2022}
\newcommand{\TypeExam}{QCM}
\newcommand{\TitreExam}{\TitreMatiere}
\newcommand{\DureeExam}{20 min}
\newcommand{\MyWaterMark}{\AnneeUniv} % Watermark de protection

% Ajout de mes classes & definitions
\usepackage{MetalExam} % Appelle un .sty

% "Tableau" et pas "Table"
\addto\captionsfrench{\def\tablename{Tableau}}

%%%%%%%%%%%%%%%%%%%%%%%
%Header
%%%%%%%%%%%%%%%%%%%%%%%
\lhead{\TypeExam}							%Gauche Haut
\chead{\NomUniv}							%Centre Haut
\rhead{\NumGroupe}							%Droite Haut
\lfoot{\DateExam}							%Gauche Bas
\cfoot{\thepage{} / \pageref*{LastPage}}	%Centre Bas
\rfoot{\texttt{\TitreMatiere}}				%Droite Bas

%%%%%

\usepackage{tabularx}

\newlength{\LabelWidth}%
%\setlength{\LabelWidth}{1.3in}%
\setlength{\LabelWidth}{1cm}%
%\settowidth{\LabelWidth}{Employee E-mail:}%  Specify the widest text here.

% Optional first parameter here specifies the alignment of
% the text within the \makebox.  Default is [l] for left
% alignment. Other options are [r] and [c] for right and center
\newcommand*{\AdjustSize}[2][l]{\makebox[\LabelWidth][#1]{#2}}%


\definecolor{mGreen}{rgb}{0,0.6,0}
\definecolor{mGray}{rgb}{0.5,0.5,0.5}
\definecolor{mPurple}{rgb}{0.58,0,0.82}
\definecolor{backgroundColour}{rgb}{0.95,0.95,0.92}

\lstdefinestyle{CStyle}{
    backgroundcolor=\color{backgroundColour},
    commentstyle=\color{mGreen},
    keywordstyle=\color{magenta},
    numberstyle=\tiny\color{mGray},
    stringstyle=\color{mPurple},
    basicstyle=\footnotesize,
    breakatwhitespace=false,
    breaklines=true,
    captionpos=b,
    keepspaces=true,
    numbers=left,
    numbersep=5pt,
    showspaces=false,
    showstringspaces=false,
    showtabs=false,
    tabsize=2,
    language=C
}


\hyphenation{op-tical net-works SIGKILL}


\begin{document}

% \MakeExamTitleDuree     % Pour afficher la duree
\MakeExamTitle                   % Ne pas afficher la duree

%% \MakeStudentName    %% A reutiliser sur chaque nouvelle page


% \setcounter{section}{1}  % Ne pas faire une liste de 0.1, 0.2, ... mais 1.1, 1.2, ...

\renewcommand{\thesubsection}{\arabic{subsection}} % Subsection sans 0.x, mais juste x

% \newcommand{\thesubsection}{\thesection.\arabic{subsection}} % Subsection avec numero section (0.x)


%\vfillFirst

\subsection{(4 points) Cochez les valeurs qui ne sont pas des puissances de 2 : }

\begin{table}[h!]
  \centering
  \begin{minipage}{0.45\textwidth}
\begin{itemize}
  \item[\CaseCoche] 1      \phantom{(} \\
  \item[\CaseCoche] 16     \phantom{(} \\
  \item[\checkmark] 12     \phantom{(} \\
  \item[\checkmark] 514    \phantom{(} \\
  \item[\checkmark] 1020   \phantom{(} \\
\end{itemize}
  \end{minipage}
  \hfillx
  \begin{minipage}{0.45\textwidth}
    \centering
\begin{itemize}
  \item[\CaseCoche] 8192   \phantom{(} \\
  \item[\checkmark] 8194   \phantom{(} \\
  \item[\checkmark] 2046   \phantom{(} \\
  \item[\checkmark] 2044   \phantom{(} \\
  \item[\checkmark] 258    \phantom{(} \\
\end{itemize}
  \end{minipage}
%  \caption{Algorithme de la somme des N premiers entiers}
%  \label{somme-n-premiers-entiers}
\end{table}


\bigskip


\subsection{(4 points) Combien fait en décimal : $ \% \; 0110 \; 1011 $ }

\begin{table}[h!]
  \centering
  \begin{minipage}{0.45\textwidth}
\begin{itemize}
  \item[\CaseCoche] -20    \phantom{(} \\
  \item[\CaseCoche] -21    \phantom{(} \\
  \item[\CaseCoche] -107   \phantom{(} \\
  \item[\CaseCoche] -148   \phantom{(} \\
  \item[\CaseCoche] -149   \phantom{(} \\
\end{itemize}
  \end{minipage}
  \hfillx
  \begin{minipage}{0.45\textwidth}
    \centering
\begin{itemize}
  \item[\CaseCoche] 20     \phantom{(} \\
  \item[\CaseCoche] 21     \phantom{(} \\
  \item[\checkmark] 107    \phantom{(} \\  %%%
  \item[\CaseCoche] 148    \phantom{(} \\
  \item[\CaseCoche] 149    \phantom{(} \\
\end{itemize}
  \end{minipage}
%  \caption{Algorithme de la somme des N premiers entiers}
%  \label{somme-n-premiers-entiers}
\end{table}


\bigskip


\subsection{(4 points) Combien fait en décimal : $ \$ \; \text{AB} $ }
% 1010 1011
%  43 - 128 = ???
%  101 0100
%  101 0101 = -85

\begin{table}[h!]
  \centering
  \begin{minipage}{0.45\textwidth}
\begin{itemize}
  \item[\CaseCoche] -43    \phantom{(} \\
  \item[\CaseCoche] -84    \phantom{(} \\  
  \item[\checkmark] -85    \phantom{(} \\ %%%
  \item[\CaseCoche] -171   \phantom{(} \\
  \item[\CaseCoche] -172   \phantom{(} \\
\end{itemize}
  \end{minipage}
  \hfillx
  \begin{minipage}{0.45\textwidth}
    \centering
\begin{itemize}
  \item[\CaseCoche] 43     \phantom{(} \\
  \item[\CaseCoche] 84     \phantom{(} \\
  \item[\CaseCoche] 85     \phantom{(} \\
  \item[\checkmark] 171    \phantom{(} \\ %%%
  \item[\CaseCoche] 172    \phantom{(} \\
\end{itemize}
  \end{minipage}
%  \caption{Algorithme de la somme des N premiers entiers}
%  \label{somme-n-premiers-entiers}
\end{table}


\bigskip


\subsection{(4 points) Cochez la (ou les) affirmation(s) vraie(s) : }

\begin{itemize}
  \item[\checkmark] Certains processeurs 32 bits peuvent exécuter des instructions 16 bits \\
  \item[\checkmark] Les instructions de processeurs RISC sont de tailles fixes \\
  \item[\checkmark] Les instructions de processeurs CISC sont de tailles variables \\
  \item[\checkmark] Les bus d'adresses et de données peuvent être de tailles différentes \\
  \item[\checkmark] Les principaux étages du pipeline d'un processeur sont : \\
  Fetch, Decode, Execute, Memory R/W, Write Back \\
  \item[\checkmark] Il existe différents types de registres dans les processeurs \\
  (certains gèrent les flottants, d'autres des valeurs entières, des flags, ...) \\
  \item[\checkmark] On compare des valeurs en testant l'état de flags \\
\end{itemize}


\bigskip


\subsection{(4 points) Que peut-on dire de cette valeur lorsqu'elle est interprétée comme un flottant IEEE 754 en simple précision : $ \$ \; 8\text{F} \; 81 \; 00 \; 01 $ }

\begin{table}[h!]
  \centering
  \begin{minipage}{0.45\textwidth}
\begin{itemize}
  \item[\CaseCoche] - 0            \phantom{(} \\
  \item[\CaseCoche] - $ \infty $   \phantom{(} \\  
  \item[\CaseCoche] - NaN          \phantom{(} \\
\end{itemize}
  \end{minipage}
  \hfillx
  \begin{minipage}{0.45\textwidth}
    \centering
\begin{itemize}
  \item[\CaseCoche] + 0           \phantom{(} \\
  \item[\CaseCoche] + $ \infty $  \phantom{(} \\
  \item[\CaseCoche] + NaN         \phantom{(} \\
\end{itemize}
  \end{minipage}
%  \caption{Algorithme de la somme des N premiers entiers}
%  \label{somme-n-premiers-entiers}
\end{table}

\begin{center}
\checkmark

{ \large Rien de tout cela : c'est un nombre flottant normalisé ou dénormalisé }
\end{center}

\bigskip


\subsection{[BONUS] (0 point) Pour obtenir des points supplémentaires en cours d'archi et d'algo, il faut... }

\begin{itemize}
  \item[\CaseCoche] avoir les compétences exceptionnelles de négociation du premier rang \\
  \item[\CaseCoche] ostensiblement jouer pendant le cours malgré les rappels \\
  \item[\CaseCoche] se dévouer à coder le Quick Sort au tableau \\
  \item[\checkmark] travailler, réviser le soir, et dormir la nuit. \\
  \item[\CaseCoche] envoyer des mèmes à l'enseignant \\
  \item[\CaseCoche] dormir en cours... ZZZzzz... \\
  \item[\CaseCoche] \phantom{autre :} \\
\end{itemize}

\end{document}
