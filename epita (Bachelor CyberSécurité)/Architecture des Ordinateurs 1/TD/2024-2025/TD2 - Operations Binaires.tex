\documentclass[11pt,a4paper]{article}
\usepackage[utf8]{inputenc}
\usepackage[french]{babel}
\usepackage[T1]{fontenc}

\usepackage{amsmath}
\usepackage{amsfonts}
\usepackage{amssymb}

\newcommand{\TitreMatiere}{Architecture des Ordinateurs 1}
\newcommand{\TitreSeance}{TD2 - Opérations Binaires}
\newcommand{\SousTitreSeance}{$ + $ $ - $ $ \times $ $ \div $ binaires}
\newcommand{\DateCours}{Septembre 2024}
\newcommand{\AnneeScolaire}{2024-2025}
\newcommand{\Organisation}{EPITA}
\newcommand{\NomAuteurA}{Fabrice BOISSIER}
\newcommand{\MailAuteurA}{fabrice.boissier@epita.fr}
\newcommand{\NomAuteurB}{ }
\newcommand{\MailAuteurB}{ }
\newcommand{\DocKeywords}{Architecture ; Entiers Non-Signés ; Binaire ; Hexadécimal ; Opérations ; Addition ; Soustraction ; Multiplication ; Division}
\newcommand{\DocLangue}{fr} % "en", "fr", ...

\usepackage{MetalQuickLabs}

% Babel ne traduit pas toujours bien les tableaux et autres
\renewcommand*\frenchfigurename{%
    {\scshape Figure}%
}
\renewcommand*\frenchtablename{%
    {\scshape Tableau}%
}

% Ne pas afficher le numéro de la légende sur tableaux et figures
\captionsetup{format=sanslabel}


\begin{document}

\EncadreTitre

\bigskip


%\begin{center}
%\begin{tabular}{p{5cm} p{11cm}}
%\textbf{Commandes étudiées :} & \texttt{sh}, \texttt{bash}, \texttt{man}, \texttt{ls}, \texttt{mkdir}, \texttt{touch}, \texttt{chmod}, \texttt{mv}, \texttt{rm}, \texttt{rmdir}, \texttt{cat}, \texttt{file}, \texttt{which}, \texttt{which}\\
%
%\textbf{Builtins étudiées :} & \texttt{pwd}, \texttt{cd}, \texttt{exit}, \texttt{logout}, \texttt{echo}, \texttt{umask}, \texttt{type}, \texttt{>}, \texttt{>{}>}, \texttt{<}, \texttt{<{}<}, \texttt{|}\\
%
%\textbf{Notions étudiées :} & Shell, Manuels, Fichiers, Répertoires, Droits, Redirections\\
%\end{tabular}
%\end{center}

\bigskip


Ce document a pour objectif de vous familiariser avec les opérations effectuées sur des nombres représentés au format binaire.

\bigskip

%%%%%%%%%%%%%%%%%%%%%%%%%%%%%%%%%%%%%%%%%%%%%%%%

\section{Additions}

%%%%%%%%%%%%%%%%%%%

\medskip

Effectuez les additions suivantes :

\bigskip

\vfillFirst

\centerline{
\begin{tabular}{ | c c c | C{6cm} |}
\hline
 \multicolumn{3}{| c |}{\multirow{2}{*}{Opération}} & \multirow{2}{*}{Résultat} \\
 & & & \\
\hline
\multirow{2}{*}{$ \Scale[1.2]{ \% \, 0110 \, 1101 } $} & \multirow{2}{*}{$ \Scale[1.2]{ + } $} & \multirow{2}{*}{$ \Scale[1.2]{ \% \, 0010 \, 1111 } $} & \\
 & & & \\
\hline
\multirow{2}{*}{$ \Scale[1.2]{ \% \, 0100 \, 0110 } $} & \multirow{2}{*}{$ \Scale[1.2]{ + } $} & \multirow{2}{*}{$ \Scale[1.2]{ \% \, 0110 \, 1001 } $} & \\
 & & & \\
\hline
\multirow{2}{*}{$ \Scale[1.2]{ \% \, 1001 \, 0101 } $} & \multirow{2}{*}{$ \Scale[1.2]{ + } $} & \multirow{2}{*}{$ \Scale[1.2]{ \% \, 0011 \, 0110 } $} & \\
 & & & \\
\hline
\multirow{2}{*}{$ \Scale[1.2]{ \% \, 1010 \, 1010 } $} & \multirow{2}{*}{$ \Scale[1.2]{ + } $} & \multirow{2}{*}{$ \Scale[1.2]{ \% \, 0100 \, 0011 } $} & \\
 & & & \\
\hline
\multirow{2}{*}{$ \Scale[1.2]{ \% \, 1011 \, 1100 } $} & \multirow{2}{*}{$ \Scale[1.2]{ + } $} & \multirow{2}{*}{$ \Scale[1.2]{ \% \, 0111 \, 1011 } $} & \\
 & & & \\
\hline
\multirow{2}{*}{$ \Scale[1.2]{ \% \, 1101 \, 1110 } $} & \multirow{2}{*}{$ \Scale[1.2]{ + } $} & \multirow{2}{*}{$ \Scale[1.2]{ \% \, 1101 \, 0111 } $} & \\
 & & & \\
\hline
\hline
\multirow{2}{*}{$ \Scale[1.2]{ \$ \, \text{1289} \, \text{FABB} } $} & \multirow{2}{*}{$ \Scale[1.2]{ + } $} & \multirow{2}{*}{$ \Scale[1.2]{ \$ \, \text{1A32} \, \text{14F8} } $} & \\
 & & & \\
\hline
\multirow{2}{*}{$ \Scale[1.2]{ \$ \, \text{1418} \, \text{3945} } $} & \multirow{2}{*}{$ \Scale[1.2]{ + } $} & \multirow{2}{*}{$ \Scale[1.2]{ \$ \, \text{B0B0} \, \text{ABCD} } $} & \\
 & & & \\
\hline
\multirow{2}{*}{$ \Scale[1.2]{ \$ \, \text{1A2B} \, \text{C8D9} } $} & \multirow{2}{*}{$ \Scale[1.2]{ + } $} & \multirow{2}{*}{$ \Scale[1.2]{ \$ \, \text{4273} \, \text{AE10} } $} & \\
 & & & \\
\hline
\multirow{2}{*}{$ \Scale[1.2]{ \$ \, \text{42FE} \, \text{BA17} } $} & \multirow{2}{*}{$ \Scale[1.2]{ + } $} & \multirow{2}{*}{$ \Scale[1.2]{ \$ \, \text{89CD} \, \text{67E3} } $} & \\
 & & & \\
\hline
\multirow{2}{*}{$ \Scale[1.2]{ \$ \, \text{BABA} \, \text{2A2A} } $} & \multirow{2}{*}{$ \Scale[1.2]{ + } $} & \multirow{2}{*}{$ \Scale[1.2]{ \$ \, \text{3264} \, \text{2A2A} } $} & \\
 & & & \\
\hline
\multirow{2}{*}{$ \Scale[1.2]{ \$ \, \text{1789} \, \text{1871} } $} & \multirow{2}{*}{$ \Scale[1.2]{ + } $} & \multirow{2}{*}{$ \Scale[1.2]{ \$ \, \text{ABCD} \, \text{EF68} } $} & \\
 & & & \\
\hline
\end{tabular}
}

\vfillLast

%\bigskip
\clearpage

%%%%%%%%%%%%%%%%%%%%%%%%%%%%%%%%%%%%%%%%%%%%%%%%

\section{Multiplications}

%%%%%%%%%%%%%%%%%%%

\medskip

Effectuez les multiplications suivantes :

\bigskip

\vfillFirst

\centerline{
\begin{tabular}{ | c c c | C{6cm} |}
\hline
 \multicolumn{3}{| c |}{\multirow{2}{*}{Opération}} & \multirow{2}{*}{Résultat} \\
 & & & \\
\hline
\multirow{2}{*}{$ \Scale[1.2]{ \% \, 0001 \, 1101 } $} & \multirow{2}{*}{$ \Scale[1.2]{ \times } $} & \multirow{2}{*}{$ \Scale[1.2]{ \% \, 0010 } $} & \\
 & & & \\
\hline
\multirow{2}{*}{$ \Scale[1.2]{ \% \, 0001 \, 1101 } $} & \multirow{2}{*}{$ \Scale[1.2]{ \times } $} & \multirow{2}{*}{$ \Scale[1.2]{ \% \, 0011 } $} & \\
 & & & \\
\hline
\multirow{2}{*}{$ \Scale[1.2]{ \% \, 0010 \, 0101 } $} & \multirow{2}{*}{$ \Scale[1.2]{ \times } $} & \multirow{2}{*}{$ \Scale[1.2]{ \% \, 0110 } $} & \\
 & & & \\
\hline
\multirow{2}{*}{$ \Scale[1.2]{ \% \, 1010 \, 1010 } $} & \multirow{2}{*}{$ \Scale[1.2]{ \times } $} & \multirow{2}{*}{$ \Scale[1.2]{ \% \, 1010 } $} & \\
 & & & \\
\hline
\multirow{2}{*}{$ \Scale[1.2]{ \% \, 0011 \, 1101 } $} & \multirow{2}{*}{$ \Scale[1.2]{ \times } $} & \multirow{2}{*}{$ \Scale[1.2]{ \% \, 1011 } $} & \\
 & & & \\
\hline
\multirow{2}{*}{$ \Scale[1.2]{ \% \, 0101 \, 1110 } $} & \multirow{2}{*}{$ \Scale[1.2]{ \times } $} & \multirow{2}{*}{$ \Scale[1.2]{ \% \, 0111 } $} & \\
 & & & \\
\hline
\hline
\multirow{2}{*}{$ \Scale[1.2]{ \$ \, \text{FABB} } $} & \multirow{2}{*}{$ \Scale[1.2]{ \times } $} & \multirow{2}{*}{$ \Scale[1.2]{ \$ \, \text{000F} } $} & \\
 & & & \\
\hline
\multirow{2}{*}{$ \Scale[1.2]{ \$ \, \text{3945} } $} & \multirow{2}{*}{$ \Scale[1.2]{ \times } $} & \multirow{2}{*}{$ \Scale[1.2]{ \$ \, \text{0023} } $} & \\
 & & & \\
\hline
\multirow{2}{*}{$ \Scale[1.2]{ \$ \, \text{4EDF} } $} & \multirow{2}{*}{$ \Scale[1.2]{ \times } $} & \multirow{2}{*}{$ \Scale[1.2]{ \$ \, \text{01AE} } $} & \\
 & & & \\
\hline
\multirow{2}{*}{$ \Scale[1.2]{ \$ \, \text{BA13} } $} & \multirow{2}{*}{$ \Scale[1.2]{ \times } $} & \multirow{2}{*}{$ \Scale[1.2]{ \$ \, \text{00D2} } $} & \\
 & & & \\
\hline
\multirow{2}{*}{$ \Scale[1.2]{ \$ \, \text{2A2A} } $} & \multirow{2}{*}{$ \Scale[1.2]{ \times } $} & \multirow{2}{*}{$ \Scale[1.2]{ \$ \, \text{A2A2} } $} & \\
 & & & \\
\hline
\multirow{2}{*}{$ \Scale[1.2]{ \$ \, \text{FEFE} } $} & \multirow{2}{*}{$ \Scale[1.2]{ \times } $} & \multirow{2}{*}{$ \Scale[1.2]{ \$ \, \text{00C4} } $} & \\
 & & & \\
\hline
\end{tabular}
}

\vfillLast

%\bigskip
\clearpage

%%%%%%%%%%%%%%%%%%%%%%%%%%%%%%%%%%%%%%%%%%%%%%%%

\section{Soustractions}

%%%%%%%%%%%%%%%%%%%

\medskip

Effectuez les soustractions suivantes :

\bigskip

\vfillFirst

\centerline{
\begin{tabular}{ | c c c | C{6cm} |}
\hline
 \multicolumn{3}{| c |}{\multirow{2}{*}{Opération}} & \multirow{2}{*}{Résultat} \\
 & & & \\
\hline
\multirow{2}{*}{$ \Scale[1.2]{ \% \, 1010 \, 1101 } $} & \multirow{2}{*}{$ \Scale[1.2]{ - } $} & \multirow{2}{*}{$ \Scale[1.2]{ \% \, 0010 \, 1111 } $} & \\
 & & & \\
\hline
\multirow{2}{*}{$ \Scale[1.2]{ \% \, 1100 \, 0110 } $} & \multirow{2}{*}{$ \Scale[1.2]{ - } $} & \multirow{2}{*}{$ \Scale[1.2]{ \% \, 0110 \, 1001 } $} & \\
 & & & \\
\hline
\multirow{2}{*}{$ \Scale[1.2]{ \% \, 1001 \, 0101 } $} & \multirow{2}{*}{$ \Scale[1.2]{ - } $} & \multirow{2}{*}{$ \Scale[1.2]{ \% \, 0011 \, 0110 } $} & \\
 & & & \\
\hline
\multirow{2}{*}{$ \Scale[1.2]{ \% \, 1010 \, 1010 } $} & \multirow{2}{*}{$ \Scale[1.2]{ - } $} & \multirow{2}{*}{$ \Scale[1.2]{ \% \, 0100 \, 0011 } $} & \\
 & & & \\
\hline
\multirow{2}{*}{$ \Scale[1.2]{ \% \, 1011 \, 1100 } $} & \multirow{2}{*}{$ \Scale[1.2]{ - } $} & \multirow{2}{*}{$ \Scale[1.2]{ \% \, 0111 \, 1011 } $} & \\
 & & & \\
\hline
\multirow{2}{*}{$ \Scale[1.2]{ \% \, 1101 \, 1110 } $} & \multirow{2}{*}{$ \Scale[1.2]{ - } $} & \multirow{2}{*}{$ \Scale[1.2]{ \% \, 1101 \, 0111 } $} & \\
 & & & \\
\hline
\hline
\multirow{2}{*}{$ \Scale[1.2]{ \$ \, \text{1A2B} \, \text{3C4D} } $} & \multirow{2}{*}{$ \Scale[1.2]{ - } $} & \multirow{2}{*}{$ \Scale[1.2]{ \$ \, \text{1234} \, \text{DCBA} } $} & \\
 & & & \\
\hline
\multirow{2}{*}{$ \Scale[1.2]{ \$ \, \text{BABA} \, \text{B0B0} } $} & \multirow{2}{*}{$ \Scale[1.2]{ - } $} & \multirow{2}{*}{$ \Scale[1.2]{ \$ \, \text{ADAD} \, \text{1337} } $} & \\
 & & & \\
\hline
\multirow{2}{*}{$ \Scale[1.2]{ \$ \, \text{B1B1} \, \text{1E48} } $} & \multirow{2}{*}{$ \Scale[1.2]{ - } $} & \multirow{2}{*}{$ \Scale[1.2]{ \$ \, \text{4273} \, \text{AE10} } $} & \\
 & & & \\
\hline
\multirow{2}{*}{$ \Scale[1.2]{ \$ \, \text{AE86} \, \text{3008} } $} & \multirow{2}{*}{$ \Scale[1.2]{ - } $} & \multirow{2}{*}{$ \Scale[1.2]{ \$ \, \text{A380} \, \text{B747} } $} & \\
 & & & \\
\hline
\multirow{2}{*}{$ \Scale[1.2]{ \$ \, \text{18E9} \, \text{2A2A} } $} & \multirow{2}{*}{$ \Scale[1.2]{ - } $} & \multirow{2}{*}{$ \Scale[1.2]{ \$ \, \text{07DE} \, \text{A2A2} } $} & \\
 & & & \\
\hline
\multirow{2}{*}{$ \Scale[1.2]{ \$ \, \text{FFFF} \, \text{EEEE} } $} & \multirow{2}{*}{$ \Scale[1.2]{ - } $} & \multirow{2}{*}{$ \Scale[1.2]{ \$ \, \text{ABCD} \, \text{EF68} } $} & \\
 & & & \\
\hline
\end{tabular}
}

\vfillLast

%\bigskip
\clearpage

%%%%%%%%%%%%%%%%%%%%%%%%%%%%%%%%%%%%%%%%%%%%%%%%

\section{Divisions}

%%%%%%%%%%%%%%%%%%%

\medskip

Effectuez les divisions suivantes :

\bigskip

\vfillFirst

\centerline{
\begin{tabular}{ | c c c | C{4cm} | C{2cm} | }
\hline
 \multicolumn{3}{| c |}{\multirow{2}{*}{Opération}} & \multirow{2}{*}{Dividende} & \multirow{2}{*}{Reste} \\
 & & & & \\
\hline
\multirow{2}{*}{$ \Scale[1.2]{ \% \, 0110 \, 1101 } $} & \multirow{2}{*}{$ \Scale[1.2]{ \div } $} & \multirow{2}{*}{$ \Scale[1.2]{ \% \, 0010 } $} & & \\
 & & & & \\
\hline
\multirow{2}{*}{$ \Scale[1.2]{ \% \, 0110 \, 1101 } $} & \multirow{2}{*}{$ \Scale[1.2]{ \div } $} & \multirow{2}{*}{$ \Scale[1.2]{ \% \, 0011 } $} & & \\
 & & & & \\
\hline
\multirow{2}{*}{$ \Scale[1.2]{ \% \, 1001 \, 0101 } $} & \multirow{2}{*}{$ \Scale[1.2]{ \div } $} & \multirow{2}{*}{$ \Scale[1.2]{ \% \, 0110 } $} & & \\
 & & & & \\
\hline
\multirow{2}{*}{$ \Scale[1.2]{ \% \, 1010 \, 1010 } $} & \multirow{2}{*}{$ \Scale[1.2]{ \div } $} & \multirow{2}{*}{$ \Scale[1.2]{ \% \, 1010 } $} & & \\
 & & & & \\
\hline
\multirow{2}{*}{$ \Scale[1.2]{ \% \, 1011 \, 1100 } $} & \multirow{2}{*}{$ \Scale[1.2]{ \div } $} & \multirow{2}{*}{$ \Scale[1.2]{ \% \, 1011 } $} & & \\
 & & & & \\
\hline
\multirow{2}{*}{$ \Scale[1.2]{ \% \, 1101 \, 1110 } $} & \multirow{2}{*}{$ \Scale[1.2]{ \div } $} & \multirow{2}{*}{$ \Scale[1.2]{ \% \, 0111 } $} & & \\
 & & & & \\
\hline
%\hline
%\multirow{2}{*}{$ \Scale[1.2]{ \$ \, \text{1289} \, \text{FABB} } $} & \multirow{2}{*}{$ \Scale[1.2]{ \div } $} & \multirow{2}{*}{$ \Scale[1.2]{ \$ \, \text{002A} } $} & & \\
% & & & & \\
%\hline
%\multirow{2}{*}{$ \Scale[1.2]{ \$ \, \text{1418} \, \text{3945} } $} & \multirow{2}{*}{$ \Scale[1.2]{ \div } $} & \multirow{2}{*}{$ \Scale[1.2]{ \$ \, \text{00AD} } $} & & \\
% & & & & \\
%\hline
%\multirow{2}{*}{$ \Scale[1.2]{ \$ \, \text{1A2B} \, \text{C8D9} } $} & \multirow{2}{*}{$ \Scale[1.2]{ \div } $} & \multirow{2}{*}{$ \Scale[1.2]{ \$ \, \text{0234} } $} & & \\
% & & & & \\
%\hline
%\multirow{2}{*}{$ \Scale[1.2]{ \$ \, \text{42FE} \, \text{BA17} } $} & \multirow{2}{*}{$ \Scale[1.2]{ \div } $} & \multirow{2}{*}{$ \Scale[1.2]{ \$ \, \text{12DE} } $} & & \\
% & & & & \\
%\hline
%\multirow{2}{*}{$ \Scale[1.2]{ \$ \, \text{BABA} \, \text{2A2A} } $} & \multirow{2}{*}{$ \Scale[1.2]{ \div } $} & \multirow{2}{*}{$ \Scale[1.2]{ \$ \, \text{AEF4} } $} & & \\
% & & & & \\
%\hline
%\multirow{2}{*}{$ \Scale[1.2]{ \$ \, \text{1789} \, \text{1871} } $} & \multirow{2}{*}{$ \Scale[1.2]{ \div } $} & \multirow{2}{*}{$ \Scale[1.2]{ \$ \, \text{ABCD} } $} & & \\
% & & & & \\
%\hline
\end{tabular}
}

\vfillLast

\bigskip

%%%%%%%%%%%%%%%%%%%%%%%%%%%%%%%%%%%%%%%%%%%%%%%%%%%%%%%%%%%%%%%%%%%%%%%%%%%%%%%%

%\bigskip
%
%\vfillFirst
%
%\vfillLast

\begin{center}
\textit{Ce document et ses illustrations ont été réalisés par Fabrice BOISSIER en octobre 2024}

\textit{(dernière mise à jour octobre 2024)}
\end{center}

\end{document}
