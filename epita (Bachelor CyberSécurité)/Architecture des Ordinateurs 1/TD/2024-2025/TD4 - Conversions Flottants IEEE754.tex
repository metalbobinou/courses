\documentclass[11pt,a4paper]{article}
\usepackage[utf8]{inputenc}
\usepackage[french]{babel}
\usepackage[T1]{fontenc}

\usepackage{amsmath}
\usepackage{amsfonts}
\usepackage{amssymb}

\newcommand{\TitreMatiere}{Architecture des Ordinateurs 1}
\newcommand{\TitreSeance}{TD4 - Conversions Flottants}
\newcommand{\SousTitreSeance}{IEEE 754}
\newcommand{\DateCours}{Novembre 2024}
\newcommand{\AnneeScolaire}{2024-2025}
\newcommand{\Organisation}{EPITA}
\newcommand{\NomAuteurA}{Fabrice BOISSIER}
\newcommand{\MailAuteurA}{fabrice.boissier@epita.fr}
\newcommand{\NomAuteurB}{ }
\newcommand{\MailAuteurB}{ }
\newcommand{\DocKeywords}{Architecture ; Binaire ; Hexadécimal ; Conversion Flottants ; Flottants ; IEEE754 ; IEEE 754 ; Virgule Flottante}
\newcommand{\DocLangue}{fr} % "en", "fr", ...

\usepackage{MetalQuickLabs}

% Babel ne traduit pas toujours bien les tableaux et autres
\renewcommand*\frenchfigurename{%
    {\scshape Figure}%
}
\renewcommand*\frenchtablename{%
    {\scshape Tableau}%
}

% Ne pas afficher le numéro de la légende sur tableaux et figures
\captionsetup{format=sanslabel}


\begin{document}

\EncadreTitre

\bigskip


%\begin{center}
%\begin{tabular}{p{5cm} p{11cm}}
%\textbf{Commandes étudiées :} & \texttt{sh}, \texttt{bash}, \texttt{man}, \texttt{ls}, \texttt{mkdir}, \texttt{touch}, \texttt{chmod}, \texttt{mv}, \texttt{rm}, \texttt{rmdir}, \texttt{cat}, \texttt{file}, \texttt{which}, \texttt{which}\\
%
%\textbf{Builtins étudiées :} & \texttt{pwd}, \texttt{cd}, \texttt{exit}, \texttt{logout}, \texttt{echo}, \texttt{umask}, \texttt{type}, \texttt{>}, \texttt{>{}>}, \texttt{<}, \texttt{<{}<}, \texttt{|}\\
%
%\textbf{Notions étudiées :} & Shell, Manuels, Fichiers, Répertoires, Droits, Redirections\\
%\end{tabular}
%\end{center}

\bigskip


Ce document a pour objectif de vous familiariser avec les conversions entre plusieurs bases pour les flottants en respectant la norme IEEE 754.

\bigskip

Pour rappel, les flottants IEEE 754 peuvent être convertis depuis la base 10 vers du binaire sur 32 bits, 64 bits, ou quelques autres formats.
Dans ce document, nous nous concentrerons sur les formats 32 bits et 64 bits.

\bigskip

%%%%%%%%%%%%%%%%%%%%%%%%%%%%%%%%%%%%%%%%%%%%%%%%

\section{Standards IEEE 754}

%%%%%%%%%%%%%%%%%%%

\medskip

Rappelez la taille de chacune des parties représentant un flottant IEEE 754 :

\medskip

\begin{center}

Test.

\end{center}

%%%%%%%%%%%%%%%%%%%%%%%%%%%%%%%%%%%%%%%%%%%%%%%%

\section{Conversions flottants IEEE 754 vers la base 10}

%%%%%%%%%%%%%%%%%%%

\medskip

Convertissez ces nombres depuis le format IEEE 754 vers la base 10.

\medskip

\begin{center}
\centerline{
\begin{tabular}{ | C{3.5cm} | C{3.5cm} | }
\hline
IEEE 754 & Base 10 \\
\hline
 \multirow{2}{*}{$ \Scale[1.25]{ \$ \, \text{0101} \, \text{1101} } $} & \\
 & \\
\hline
\end{tabular}
}
\end{center}

%%%%%%%%%%%%%%%%%%%%%%%%%%%%%%%%%%%%%%%%%%%%%%%%

\section{Conversions base 10 vers flottants IEEE 754}

%%%%%%%%%%%%%%%%%%%

\medskip

Convertissez ces nombres depuis le format IEEE 754 vers la base 10.

\medskip

\begin{center}
\centerline{
\begin{tabular}{ | C{3.5cm} | C{3.5cm} | }
\hline
Base 10 & IEEE 754 \\
\hline
 \multirow{2}{*}{$ \Scale[1.25]{ 42,42 } $} & \\
 & \\
\hline
\end{tabular}
}
\end{center}

%%%%%%%%%%%%%%%%%%%%%%%%%%%%%%%%%%%%%%%%%%%%%%%%%%%%%%%%%%%%%%%%%%%%%%%%%%%%%%%%%%%%%%%%%%%%%%%%

\bigskip

\vfillFirst

\vfillLast

\begin{center}
\textit{Ce document et ses illustrations ont été réalisés par Fabrice BOISSIER en octobre 2024}

%\textit{(dernière mise à jour octobre 2024)}
\end{center}

\end{document}
