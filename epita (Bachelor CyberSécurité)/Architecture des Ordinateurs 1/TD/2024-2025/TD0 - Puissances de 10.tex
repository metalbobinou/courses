\documentclass[11pt,a4paper]{article}
\usepackage[utf8]{inputenc}
\usepackage[french]{babel}
\usepackage[T1]{fontenc}

\usepackage{amsmath}
\usepackage{amsfonts}
\usepackage{amssymb}

\newcommand{\TitreMatiere}{Architecture des Ordinateurs 1}
\newcommand{\TitreSeance}{TD0 - Puissances de 10}
\newcommand{\SousTitreSeance}{Rappels}
\newcommand{\DateCours}{Septembre 2024}
\newcommand{\AnneeScolaire}{2024-2025}
\newcommand{\Organisation}{EPITA}
\newcommand{\NomAuteurA}{Fabrice BOISSIER}
\newcommand{\MailAuteurA}{fabrice.boissier@epita.fr}
\newcommand{\NomAuteurB}{Rodrigue MALEOMBHO}
\newcommand{\MailAuteurB}{ }
\newcommand{\DocKeywords}{Architecture ; Puissances 10}
\newcommand{\DocLangue}{fr} % "en", "fr", ...

\usepackage{MetalQuickLabs}

% Babel ne traduit pas toujours bien les tableaux et autres
\renewcommand*\frenchfigurename{%
    {\scshape Figure}%
}
\renewcommand*\frenchtablename{%
    {\scshape Tableau}%
}

% Ne pas afficher le numéro de la légende sur tableaux et figures
\captionsetup{format=sanslabel}


\begin{document}

\EncadreTitre

\bigskip


%\begin{center}
%\begin{tabular}{p{5cm} p{11cm}}
%\textbf{Commandes étudiées :} & \texttt{sh}, \texttt{bash}, \texttt{man}, \texttt{ls}, \texttt{mkdir}, \texttt{touch}, \texttt{chmod}, \texttt{mv}, \texttt{rm}, \texttt{rmdir}, \texttt{cat}, \texttt{file}, \texttt{which}, \texttt{which}\\
%
%\textbf{Builtins étudiées :} & \texttt{pwd}, \texttt{cd}, \texttt{exit}, \texttt{logout}, \texttt{echo}, \texttt{umask}, \texttt{type}, \texttt{>}, \texttt{>{}>}, \texttt{<}, \texttt{<{}<}, \texttt{|}\\
%
%\textbf{Notions étudiées :} & Shell, Manuels, Fichiers, Répertoires, Droits, Redirections\\
%\end{tabular}
%\end{center}

\bigskip


Ce document a pour objectif de vous rappeler le fonctionnement des puissances, et en particulier des puissances de 10.

%\bigskip
%
%Pour rappel, on peut additioner/soustraire les exposants lorsqu'ils sont appliqués au même nombre :
%
%\smallskip
%
%$ 10^{3} \times 10^{2} = (10 \times 10 \times 10) \times (10 \times 10) = 10^{3 + 2} = 10^{5} $
%
%\smallskip
%
%$ \frac{10^{3}}{10^{2}} = 10^{3} \div 10^{2} = 10^{3} \times 10^{-2} = 10^{3 - 2} = 10^{1} $
%
%\smallskip
%
%$ (10^{3})^{2} = 10^{3} \times 10^{3} = 10^{3 \times 2} = 10^{6} $

\bigskip

%%%%%%%%%%%%%%%%%%%%%%%%%%%%%%%%%%%%%%

\section{Puissances de 10}

%%%%%%%%%%%%%%%%%%%

\subsection{Rappels multiplications}

\medskip

Réécrivez les formules suivantes sous la forme d'une puissance de 10.

\bigskip

%\centerline{
\begin{tabular}{ l l  C{1.75cm}  l l  C{1.75cm}  l l }
1. & $ \Scale[1.5]{ \frac{10^{-2}}{10^{-5}} = } $ & &  9. & $ \Scale[1.25]{ 10^{4} \times 10^{0} = } $  & & 17. & $ \Scale[1.5]{ \frac{10^{-3}}{10^{-1}} = } $ \\
 & & & & & & & \\
2. & $ \Scale[1.25]{ 10^{-6} \times 10^{-4} = } $ & & 10. & $ \Scale[1.25]{ 10^{-1} \times 10^{2} = } $ & & 18. & $ \Scale[1.25]{ (10^{0})^{-4} = } $ \\
 & & & & & & & \\
3. & $ \Scale[1.5]{ \frac{10^{-3}}{10^{-4}} = } $ & & 11. & $ \Scale[1.25]{ (10^{-2})^{3} = } $         & & 19. & $ \Scale[1.25]{ 10^{-4} \times 10^{5} = } $ \\
 & & & & & & & \\
4. & $ \Scale[1.25]{ 10^{-6} \times 10^{4} = } $  & & 12. & $ \Scale[1.25]{ (10^{1})^{1} = } $          & & 20. & $ \Scale[1.5]{ \frac{10^{-6}}{10^{4}} = } $ \\
 & & & & & & & \\
5. & $ \Scale[1.5]{ \frac{10^{5}}{10^{-2}} = } $  & & 13. & $ \Scale[1.25]{ 10^{1} \times 10^{-4} = } $ & & 21. & $ \Scale[1.25]{ (10^{-1})^{3} = } $ \\
 & & & & & & & \\
6. & $ \Scale[1.25]{ (10^{0})^{-2} = } $          & & 14. & $ \Scale[1.5]{ \frac{10^{-3}}{10^{3}} = } $ & & 22. & $ \Scale[1.5]{ \frac{10^{4}}{10^{-6}} = } $ \\
 & & & & & & & \\
7. & $ \Scale[1.25]{ (10^{1})^{3} = } $           & & 15. & $ \Scale[1.25]{ (10^{0})^{5} = } $          & & 23. & $ \Scale[1.25]{ 10^{4} \times 10^{-1} = } $ \\
 & & & & & & & \\
8. & $ \Scale[1.5]{ \frac{10^{-4}}{10^{5}} = } $  & & 16. & $ \Scale[1.25]{ 10^{0} \times 10^{2} = } $  & & 24. & $ \Scale[1.25]{ (10^{-2})^{5} = } $ \\
\end{tabular}
%}

\bigskip

%%%%%%%%%%%%%%%%%%%

\subsection{Rappels additions}

\medskip

Réécrivez les formules suivantes sous la forme d'une puissance de 10 si c'est possible, sinon sous forme décimale.

\bigskip

%\centerline{
\begin{tabular}{ l l  C{3cm}  l l }
1. & $ \Scale[1.25]{ 10^{2} + 10^{2} = } $                   & &  6. & $ \Scale[1.25]{ 5 \times 10^{1} + 3 \times 10^{2} + 5 \times 10^{1} = } $ \\
 & & & & \\
2. & $ \Scale[1.25]{ 10^{3} + 10^{2} = } $                   & &  7. & $ \Scale[1.25]{ 10^{3} + 10^{2} + 10^{1} = } $ \\
 & & & & \\
3. & $ \Scale[1.25]{ 2 \times 10^{3} + 8 \times 10^{3} = } $ & &  8. & $ \Scale[1.25]{ 4 \times 10^{1} + 6 \times 10^{2} = } $ \\
 & & & & \\
4. & $ \Scale[1.25]{ 4 \times 10^{2} + 6 \times 10^{3} = } $ & &  9. & $ \Scale[1.25]{ (2 \times 10^{1})^{3} = } $ \\
 & & & & \\
5. & $ \Scale[1.25]{ 2 \times 10^{0} + 4 \times 10^{3} = } $ & & 10. & $ \Scale[1.25]{ 10^{3} + 10^{2} + 10^{0} = } $ \\
\end{tabular}
%}


%%%%%%%%%%%%%%%%%%%

\bigskip

\begin{center}
\textit{Ce document et ses illustrations ont été réalisés par Fabrice BOISSIER en septembre 2024}

\textit{Les exercices proviennent en partie de Rodrigue MALEOMBHO}
\end{center}

\end{document}
