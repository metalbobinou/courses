\documentclass[11pt,a4paper]{article}
\usepackage[utf8]{inputenc}
\usepackage[french]{babel}
\usepackage[T1]{fontenc}

\usepackage{amsmath}
\usepackage{amsfonts}
\usepackage{amssymb}

\newcommand{\TitreMatiere}{Architecture des Ordinateurs 1}
\newcommand{\TitreSeance}{TD1 - Conversions des Entiers}
\newcommand{\SousTitreSeance}{Non-Signés et Signés}
\newcommand{\DateCours}{Septembre 2024}
\newcommand{\AnneeScolaire}{2024-2025}
\newcommand{\Organisation}{EPITA}
\newcommand{\NomAuteurA}{Fabrice BOISSIER}
\newcommand{\MailAuteurA}{fabrice.boissier@epita.fr}
\newcommand{\NomAuteurB}{ }
\newcommand{\MailAuteurB}{ }
\newcommand{\DocKeywords}{Architecture ; Conversion Entiers ; Entiers Non-Signés ; Entiers Signés ; Binaire ; Hexadécimal ; Octal}
\newcommand{\DocLangue}{fr} % "en", "fr", ...

\usepackage{MetalQuickLabs}

% Babel ne traduit pas toujours bien les tableaux et autres
\renewcommand*\frenchfigurename{%
    {\scshape Figure}%
}
\renewcommand*\frenchtablename{%
    {\scshape Tableau}%
}

% Ne pas afficher le numéro de la légende sur tableaux et figures
\captionsetup{format=sanslabel}


\begin{document}

\EncadreTitre

\bigskip


%\begin{center}
%\begin{tabular}{p{5cm} p{11cm}}
%\textbf{Commandes étudiées :} & \texttt{sh}, \texttt{bash}, \texttt{man}, \texttt{ls}, \texttt{mkdir}, \texttt{touch}, \texttt{chmod}, \texttt{mv}, \texttt{rm}, \texttt{rmdir}, \texttt{cat}, \texttt{file}, \texttt{which}, \texttt{which}\\
%
%\textbf{Builtins étudiées :} & \texttt{pwd}, \texttt{cd}, \texttt{exit}, \texttt{logout}, \texttt{echo}, \texttt{umask}, \texttt{type}, \texttt{>}, \texttt{>{}>}, \texttt{<}, \texttt{<{}<}, \texttt{|}\\
%
%\textbf{Notions étudiées :} & Shell, Manuels, Fichiers, Répertoires, Droits, Redirections\\
%\end{tabular}
%\end{center}

\bigskip


Ce document a pour objectif de vous familiariser avec les conversions entre plusieurs bases dans le cas des entiers.

\bigskip

La plupart des conversions que nous effectuerons seront entre les bases 2, 8, 10, et 16 pour les entier.

Pour rappel, plusieurs notations pour représenter les bases existent : celles où l'on explicite par un indice en suffixe de nombre dans quelle base celui-ci a été écrit, ou par un caractère en préfixe du nombre.

\medskip

\begin{tabular}{c l}
$ 42_{(10)} $ & indique l'on écrit le nombre \og $42$ \fg{} en base $ 10 $. \\
$ 1010_{(2)} $ & indique que l'on écrit le nombre \og $6$ \fg{} en base $ 2 $. \\
\end{tabular}

\bigskip

Parmi les caractères servant de préfixe, on retrouvera : \og $ \% $ \fg{} \og $ 0o $ \fg{} \og $ \$  $ \fg{}

\smallskip

\begin{tabular}{c l}
$ \% 00101111 $ & le pourcentage indique que l'on manipule un nombre binaire (ici $ 47 $) \\
$ 0o 42 $ & le zéro suivi d'un O minuscule indiquent que l'on manipule un nombre octal (ici $ 34 $) \\
$ \$ 15AB $ & le dollar indique l'on manipule un nombre hexadécimal (ici $ 5547 $) \\
\end{tabular}

\smallskip

Un nombre sans préfixe est considéré par défaut comme un nombre décimal.

\bigskip

%%%%%%%%%%%%%%%%%%%%%%%%%%%%%%%%%%%%%%%%%%%%%%%%

\section{Bornes des Entiers}

%%%%%%%%%%%%%%%%%%%

\medskip

Rappelez tout d'abord les bornes inférieures et supérieures (le plus petit/grand nombre) pour les différents cas : \textit{[au delà de 1 million, arrondissez à un chiffre et une puissance de 10]}

\medskip

\begin{center}
\centerline{
\begin{tabular}{ | c || C{3.5cm} | C{3.5cm} || C{3.5cm} | C{3.5cm} | }
\hline
  &  \multicolumn{2}{ c ||}{entier non-signé}  &  \multicolumn{2}{ c |}{entier signé} \\
\hline
  &  borne inférieure  &  borne supérieure  &  borne inférieure  &  borne supérieure \\
\hline
 \multirow{3}{*}{\begin{minipage}{1.5cm}\centering 4 bits\end{minipage}} & & & & \\
 & & & & \\
% 4 bits & & & & \\
 & & & & \\
\hline
 \multirow{3}{*}{\begin{minipage}{1.5cm}\centering 8 bits\end{minipage}} & & & & \\
 & & & & \\
% 8 bits & & & & \\
 & & & & \\
\hline
 \multirow{3}{*}{\begin{minipage}{1.5cm}\centering 16 bits\end{minipage}} & & & & \\
 & & & & \\
% 16 bits & & & & \\
 & & & & \\
\hline
 \multirow{3}{*}{\begin{minipage}{1.5cm}\centering 24 bits\end{minipage}} & & & & \\
 & & & & \\
% 24 bits & & & & \\
 & & & & \\
\hline
 \multirow{3}{*}{\begin{minipage}{1.5cm}\centering 32 bits\end{minipage}} &
   \multirow{3}{*}{\begin{minipage}{3.5cm}\centering (approx.) {$ - \, 2 \; \times 10^{9} $} \end{minipage}} &
   \multirow{3}{*}{\begin{minipage}{3.5cm}\centering (approx.) {$ + \, 2 \; \times 10^{9} $} \end{minipage}} &
   \multirow{3}{*}{\begin{minipage}{3.5cm}\centering 0 \end{minipage}} &
   \multirow{3}{*}{\begin{minipage}{3.5cm}\centering (approx.) {$   \, 4 \; \times 10^{9} $} \end{minipage}} \\
 & & & & \\
 & & & & \\
% & (approx.) & (approx.) & & (approx.) \\
% 32 bits & & &  0  & \\
% &  $ - \, 2 \; \times 10^{9} $  &  $ + \, 2 \; \times 10^{9} $  & &  $ 4 \; \times 10^{9} $ \\
\hline
 \multirow{3}{*}{\begin{minipage}{1.5cm}\centering 48 bits\end{minipage}} &
   \multirow{3}{*}{\begin{minipage}{3.5cm}\centering (approx.) {$ - \, 1.4 \; \times 10^{14} $} \end{minipage}} &
   \multirow{3}{*}{\begin{minipage}{3.5cm}\centering (approx.) {$ + \, 1.4 \; \times 10^{14} $} \end{minipage}} &
   \multirow{3}{*}{\begin{minipage}{3.5cm}\centering 0 \end{minipage}} &
   \multirow{3}{*}{\begin{minipage}{3.5cm}\centering (approx.) {$   \, 2.8 \; \times 10^{14} $} \end{minipage}} \\
 & & & & \\
 & & & & \\
% & & & & \\
%48 bits &  (approx.) $ - \, 1.4 \; \times 10^{14} $  &  (approx.) $ + \, 1.4 \; \times 10^{14} $  &  0  &  (approx.) $ 2.8 \; \times 10^{14} $ \\
% & & & & \\
\hline
 \multirow{3}{*}{\begin{minipage}{1.5cm}\centering 64 bits\end{minipage}} &
   \multirow{3}{*}{\begin{minipage}{3.5cm}\centering (approx.) {$ - \, 9.2 \; \times 10^{18} $} \end{minipage}} &
   \multirow{3}{*}{\begin{minipage}{3.5cm}\centering (approx.) {$ + \, 9.2 \; \times 10^{18} $} \end{minipage}} &
   \multirow{3}{*}{\begin{minipage}{3.5cm}\centering 0 \end{minipage}} &
   \multirow{3}{*}{\begin{minipage}{3.5cm}\centering (approx.) {$   \, 1.8 \; \times 10^{19} $} \end{minipage}} \\
 & & & & \\
 & & & & \\
% & & & & \\
%64 bits &  (approx.) $ - \, 9.2 \; \times 10^{18} $  &  (approx.) $ + \, 9.2 \; \times 10^{18} $  &  0  &  (approx.) $ 1.8 \; \times 10^{19} $ \\
% & & & & \\
\hline
 \multirow{3}{*}{\begin{minipage}{1.5cm}\centering 128 bits\end{minipage}} &
   \multirow{3}{*}{\begin{minipage}{3.5cm}\centering (approx.) {$ - \, 1.7 \; \times 10^{38} $} \end{minipage}} &
   \multirow{3}{*}{\begin{minipage}{3.5cm}\centering (approx.) {$ + \, 1.7 \; \times 10^{38} $} \end{minipage}} &
   \multirow{3}{*}{\begin{minipage}{3.5cm}\centering 0 \end{minipage}} &
   \multirow{3}{*}{\begin{minipage}{3.5cm}\centering (approx.) {$   \, 3.4 \; \times 10^{38} $} \end{minipage}} \\
 & & & & \\
 & & & & \\
% & & & & \\
%128 bits &  (approx.) $ - \, 1.7 \; \times 10^{38} $  &  (approx.) $ + \, 1.7 \; \times 10^{38} $  &  0  &  (approx.) $ 3.4 \; \times 10^{38} $ \\
% & & & & \\
\hline
\end{tabular}
}
\end{center}

%\bigskip
\clearpage

%%%%%%%%%%%%%%%%%%%%%%%%%%%%%%%%%%%%%%%%%%%%%%%%

\section{Conversions des Entiers avec la Base 2}

%%%%%%%%%%%%%%%%%%%

\subsection{Base 2 \textrightarrow{} Base 10}

\smallskip

Convertissez vers la base 10 ces valeurs binaires représentées sur 8 bits :

\bigskip

\centerline{
\begin{tabular}{ | c || C{2.5cm} | C{2.5cm} || c | C{2.5cm} | C{2.5cm} | }
\hline
 & \multirow{2}{*}{Non-Signé} & \multirow{2}{*}{Signé} & & \multirow{2}{*}{Non-Signé} & \multirow{2}{*}{Signé} \\
 & & & & & \\
\hline
\multirow{2}{*}{$ \Scale[1.25]{ \% \, 0110 \, 1101 } $} & & & \multirow{2}{*}{$ \Scale[1.25]{ \% \, 0101 \, 0101 } $} & & \\
 & & & & & \\
\hline
\multirow{2}{*}{$ \Scale[1.25]{ \% \, 0011 \, 1010 } $} & & & \multirow{2}{*}{$ \Scale[1.25]{ \% \, 0111 \, 1111 } $} & & \\
 & & & & & \\
\hline
\multirow{2}{*}{$ \Scale[1.25]{ \% \, 1101 \, 0111 } $} & & & \multirow{2}{*}{$ \Scale[1.25]{ \% \, 1011 \, 1000 } $} & & \\
 & & & & & \\
\hline
\multirow{2}{*}{$ \Scale[1.25]{ \% \, 1110 \, 0111 } $} & & & \multirow{2}{*}{$ \Scale[1.25]{ \% \, 1100 \, 0011 } $} & & \\
 & & & & & \\
\hline
\multirow{2}{*}{$ \Scale[1.25]{ \% \, 1010 \, 1110 } $} & & & \multirow{2}{*}{$ \Scale[1.25]{ \% \, 1110 \, 1101 } $} & & \\
 & & & & & \\
\hline
\end{tabular}
}

\bigskip

Convertissez vers la base 10 ces valeurs binaires représentées sur 12 bits :

\bigskip

\centerline{
\begin{tabular}{ | c || C{2.5cm} | C{2.5cm} || c | C{2.5cm} | C{2.5cm} | }
\hline
 & \multirow{2}{*}{Non-Signé} & \multirow{2}{*}{Signé} & & \multirow{2}{*}{Non-Signé} & \multirow{2}{*}{Signé} \\
 & & & & & \\
\hline
\multirow{2}{*}{$ \Scale[1.25]{ \% \, 0001 \, 1101 \, 1011 } $} & & & \multirow{2}{*}{$ \Scale[1.25]{ \% \, 0001 \, 0110 \, 1010 } $} & & \\
 & & & & & \\
\hline
\multirow{2}{*}{$ \Scale[1.25]{ \% \, 0011 \, 1101 \, 1011 } $} & & & \multirow{2}{*}{$ \Scale[1.25]{ \% \, 0010 \, 0111 \, 1001 } $} & & \\
 & & & & & \\
\hline
\multirow{2}{*}{$ \Scale[1.25]{ \% \, 0110 \, 0110 \, 1100 } $} & & & \multirow{2}{*}{$ \Scale[1.25]{ \% \, 1101 \, 1011 \, 1101 } $} & & \\
 & & & & & \\
\hline
\multirow{2}{*}{$ \Scale[1.25]{ \% \, 1111 \, 1111 \, 1011 } $} & & & \multirow{2}{*}{$ \Scale[1.25]{ \% \, 1011 \, 0111 \, 1011 } $} & & \\
 & & & & & \\
\hline
\end{tabular}
}

\bigskip

%%%%%%%%%%%%%%%%%%%

\subsection{Base 10 \textrightarrow{} Base 2}

\smallskip

Convertissez ces valeurs vers la base 2 sur 8 bits :

\bigskip

%\centerline{
\begin{tabular}{ l l  C{4.5cm}  l l }
  & $ \Scale[1.25]{ 0_{(10)} = } $   & &  & $ \Scale[1.25]{ -1_{(10)} = } $ \\
 & & & & \\
  & $ \Scale[1.25]{ 42_{(10)} = } $  & &  & $ \Scale[1.25]{ -42_{(10)} = } $ \\
 & & & & \\
  & $ \Scale[1.25]{ 10_{(10)} = } $  & &  & $ \Scale[1.25]{ -10_{(10)} = } $ \\
 & & & & \\
  & $ \Scale[1.25]{ 142_{(10)} = } $ & &  & $ \Scale[1.25]{ -113_{(10)} = } $ \\
 & & & & \\
  & $ \Scale[1.25]{ 143_{(10)} = } $ & &  & $ \Scale[1.25]{ -112_{(10)} = } $ \\
 & & & & \\
  & $ \Scale[1.25]{ 88_{(10)} = } $  & &  & $ \Scale[1.25]{ -88_{(10)} = } $ \\
 & & & & \\
  & $ \Scale[1.25]{ 255_{(10)} = } $ & &  & $ \Scale[1.25]{ -203_{(10)} = } $ \\
 & & & & \\
  & $ \Scale[1.25]{ 203_{(10)} = } $ & &  & $ \Scale[1.25]{ -8_{(10)}  = } $ \\
 & & & & \\
  & $ \Scale[1.25]{ 77_{(10)} = } $  & &  & $ \Scale[1.25]{ -127_{(10)}  = } $ \\
 & & & & \\
  & $ \Scale[1.25]{ 56_{(10)} = } $  & &  & $ \Scale[1.25]{ 127_{(10)}  = } $ \\
\end{tabular}
%}

%\bigskip
\clearpage

%%%%%%%%%%%%%%%%%%%%%%%%%%%%%%%%%%%%%%%%%%%%%%%%

\section{Conversions des Entiers avec la Base 16}

%%%%%%%%%%%%%%%%%%%

\subsection{Base 2 \textrightarrow{} Base 16}

\smallskip

Convertissez ces valeurs binaires vers la base 16 sur 8 bits :

\bigskip

%\centerline{
\begin{tabular}{ l l  C{4.5cm}  l l }
  & $ \Scale[1.25]{ \% \, 0101 \, 1110 = } $  & &  & $ \Scale[1.25]{ \% \, 1101 \, 0111 = } $ \\
 & & & & \\
  & $ \Scale[1.25]{ \% \, 1110 \, 1001 = } $  & &  & $ \Scale[1.25]{ \% \, 0010 \, 1101 = } $ \\
\end{tabular}
%}

%%%%%%%%%%%%%%%%%%%

\subsection{Base 16 \textrightarrow{} Base 2}

\smallskip

Convertissez ces valeurs hexadécimales vers la base 2 sur 16 bits :

\bigskip

%\centerline{
\begin{tabular}{ l l  C{4.5cm}  l l }
  & $ \Scale[1.25]{ \$ \, \text{DEB1} = } $  & &  & $ \Scale[1.25]{ \$ \, \text{CAFE} = } $ \\
 & & & & \\
  & $ \Scale[1.25]{ \$ \, \text{0110} = } $  & &  & $ \Scale[1.25]{ \$ \, \text{1337} = } $ \\
\end{tabular}
%}

%%%%%%%%%%%%%%%%%%%

\subsection{Base 10 \textrightarrow{} Base 16}

\smallskip

Convertissez ces valeurs vers la base 16 sur 8 bits :

\bigskip

%\centerline{
\begin{tabular}{ l l  C{4.5cm}  l l }
  & $ \Scale[1.25]{ 0_{(10)} = } $   & &   & $ \Scale[1.25]{ -1_{(10)} = } $ \\
 & & & & \\
  & $ \Scale[1.25]{ 42_{(10)} = } $  & &   & $ \Scale[1.25]{ -42_{(10)} = } $ \\
 & & & & \\
  & $ \Scale[1.25]{ 68_{(10)} = } $  & &   & $ \Scale[1.25]{ -83_{(10)} = } $ \\
 & & & & \\
%  & $ \Scale[1.25]{ 156_{(10)} = } $ & &   & $ \Scale[1.25]{ -138_{(10)} = } $ \\
% & & & & \\
  & $ \Scale[1.25]{ 224_{(10)} = } $ & &   & $ \Scale[1.25]{ -231_{(10)} = } $ \\
\end{tabular}
%}

\bigskip

Convertissez ces valeurs vers la base 16 sur 16 bits :

\bigskip

%\centerline{
\begin{tabular}{ l l  C{4.5cm}  l l }
  & $ \Scale[1.25]{ 224_{(10)} = } $   & &   & $ \Scale[1.25]{ -231_{(10)} = } $ \\
 & & & & \\
  & $ \Scale[1.25]{ 1283_{(10)} = } $  & &   & $ \Scale[1.25]{ -734_{(10)} = } $ \\
 & & & & \\
  & $ \Scale[1.25]{ 1515_{(10)} = } $  & &   & $ \Scale[1.25]{ -1515_{(10)} = } $ \\
 & & & & \\
%  & $ \Scale[1.25]{ 1789_{(10)} = } $  & &   & $ \Scale[1.25]{ -1234_{(10)} = } $ \\
% & & & & \\
  & $ \Scale[1.25]{ 2468_{(10)} = } $  & &   & $ \Scale[1.25]{ -2000_{(10)} = } $ \\
% & & & & \\
%  & $ \Scale[1.25]{ 3523_{(10)} = } $  & &   & $ \Scale[1.25]{ -1898_{(10)} = } $ \\
\end{tabular}
%}

\bigskip

%\vfillFirst

%%%%%%%%%%%%%%%%%%%

\subsection{Base 16 \textrightarrow{} Base 10}

\smallskip

Convertissez vers la base 10 ces valeurs représentées en hexadécimal sur 16 bits :

\bigskip

\centerline{
\begin{tabular}{ | c || C{2.5cm} | C{2.5cm} || c | C{2.5cm} | C{2.5cm} | }
\hline
 & \multirow{2}{*}{Non-Signé} & \multirow{2}{*}{Signé} & & \multirow{2}{*}{Non-Signé} & \multirow{2}{*}{Signé} \\
 & & & & & \\
\hline
\multirow{2}{*}{$ \Scale[1.25]{ \$ \, \text{1234} } $} & & & \multirow{2}{*}{$ \Scale[1.25]{ \$ \, \text{ABCD} } $} & & \\
 & & & & & \\
\hline
\multirow{2}{*}{$ \Scale[1.25]{ \$ \, \text{4242} } $} & & & \multirow{2}{*}{$ \Scale[1.25]{ \$ \, \text{CAFE} } $} & & \\
 & & & & & \\
\hline
\multirow{2}{*}{$ \Scale[1.25]{ \$ \, \text{DEAD} } $} & & & \multirow{2}{*}{$ \Scale[1.25]{ \$ \, \text{BEEF} } $} & & \\
 & & & & & \\
\hline
\multirow{2}{*}{$ \Scale[1.25]{ \$ \, \text{C700} } $} & & & \multirow{2}{*}{$ \Scale[1.25]{ \$ \, \text{1337} } $} & & \\
 & & & & & \\
\hline
\multirow{2}{*}{$ \Scale[1.25]{ \$ \, \text{FAB4} } $} & & & \multirow{2}{*}{$ \Scale[1.25]{ \$ \, \text{DADA} } $} & & \\
 & & & & & \\
\hline
\end{tabular}
}

%\vfillLast

\clearpage

%%%%%%%%%%%%%%%%%%%%%%%%%%%%%%%%%%%%%%%%%%%%%%%%

\section{Conversions des Entiers avec la Base 8}

%%%%%%%%%%%%%%%%%%%

\subsection{Base X \textrightarrow{} Base 8}

\smallskip

Convertissez ces valeurs vers la base 8 (sur 12 bits dans le cas décimal) :

\bigskip

%\centerline{
\begin{tabular}{ l l  C{4.5cm}  l l }
  & $ \Scale[1.25]{ 42_{(10)} = } $   & &   & $ \Scale[1.25]{ -1_{(10)} = } $ \\
 & & & & \\
  & $ \Scale[1.25]{ 114_{(10)} = } $  & &   & $ \Scale[1.25]{ -42_{(10)} = } $ \\
 & & & & \\
  & $ \Scale[1.25]{ \% \, 1001 \, 1100 = } $  & &   & $ \Scale[1.25]{ \% \, 1110 \, 1101 = } $ \\
 & & & & \\
  & $ \Scale[1.25]{ \% \, 0111 \, 0110 = } $ & &   & $ \Scale[1.25]{ \% \, 1101 \, 0011 = } $ \\
 & & & & \\
  & $ \Scale[1.25]{ \$ \, \text{F1F0} = } $ & &   & $ \Scale[1.25]{ \$ \, \text{BA0B} = } $ \\
\end{tabular}
%}

\bigskip

%%%%%%%%%%%%%%%%%%%

\subsection{Base 8 \textrightarrow{} Base X}

\smallskip

Convertissez ces valeurs depuis la base 8 :

\bigskip

\centerline{
\begin{tabular}{ | c || C{2.5cm} | C{2.5cm} || c | C{2.5cm} | C{2.5cm} | }
\hline
 & \multirow{2}{*}{Base 2} & \multirow{2}{*}{Base 16} & & \multirow{2}{*}{Base 2} & \multirow{2}{*}{Base 16} \\
 & & & & & \\
\hline
\multirow{2}{*}{$ \Scale[1.25]{ 0\text{o} \, 42 } $} & & & \multirow{2}{*}{$ \Scale[1.25]{ 0\text{o} \, 1337 } $} & & \\
 & & & & & \\
\hline
\multirow{2}{*}{$ \Scale[1.25]{ 0\text{o} \, 24 } $} & & & \multirow{2}{*}{$ \Scale[1.25]{ 0\text{o} \, 1516 } $} & & \\
 & & & & & \\
\hline
\multirow{2}{*}{$ \Scale[1.25]{ 0\text{o} \, 63 } $} & & & \multirow{2}{*}{$ \Scale[1.25]{ 0\text{o} \, 2442 } $} & & \\
 & & & & & \\
\hline
\multirow{2}{*}{$ \Scale[1.25]{ 0\text{o} \, 77 } $} & & & \multirow{2}{*}{$ \Scale[1.25]{ 0\text{o} \, 5150 } $} & & \\
 & & & & & \\
\hline
\end{tabular}
}

\bigskip

%%%%%%%%%%%%%%%%%%%

\subsection{Base 8 \textrightarrow{} Base 10}

\smallskip

Convertissez vers la base 10 ces valeurs représentées en octal sur 12 bits :

\bigskip

%\centerline{
\begin{tabular}{ l l  C{4.5cm}  l l }
  & $ \Scale[1.25]{ 0\text{o} \, 0042 = } $   & &   & $ \Scale[1.25]{ 0\text{o} \, 1337 = } $ \\
 & & & & \\
  & $ \Scale[1.25]{ 0\text{o} \, 2442 = } $   & &   & $ \Scale[1.25]{ 0\text{o} \, 1234 = } $ \\
 & & & & \\
  & $ \Scale[1.25]{ 0\text{o} \, 4365 = } $   & &   & $ \Scale[1.25]{ 0\text{o} \, 3280 = } $ \\
 & & & & \\
  & $ \Scale[1.25]{ 0\text{o} \, 2675 = } $   & &   & $ \Scale[1.25]{ 0\text{o} \, 1512 = } $ \\
 & & & & \\
  & $ \Scale[1.25]{ 0\text{o} \, 5150 = } $   & &   & $ \Scale[1.25]{ 0\text{o} \, 5250 = } $ \\
\end{tabular}
%}


%%%%%%%%%%%%%%%%%%%%%%%%%%%%%%%%%%%%%%%%%%%%%%%%%%%%%%%%%%%%%%%%%%%%%%%%%%%%%%%%%%%%%%%%%%%%%%%%

\bigskip

\vfillFirst

\vfillLast

\begin{center}
\textit{Ce document et ses illustrations ont été réalisés par Fabrice BOISSIER en novembre 2022}

\textit{(dernière mise à jour octobre 2024)}
\end{center}

\end{document}
