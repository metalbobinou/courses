\documentclass[11pt,a4paper]{article}
\usepackage[utf8]{inputenc}
\usepackage[french]{babel}
\usepackage[T1]{fontenc}

\usepackage{amsmath}
\usepackage{amsfonts}
\usepackage{amssymb}

\newcommand{\TitreMatiere}{Architecture des Ordinateurs 1}
\newcommand{\TitreSeance}{TD3 - Conversions Autres Codages}
\newcommand{\SousTitreSeance}{Code Gray \& BCD}
\newcommand{\DateCours}{Octobre 2024}
\newcommand{\AnneeScolaire}{2024-2025}
\newcommand{\Organisation}{EPITA}
\newcommand{\NomAuteurA}{Fabrice BOISSIER}
\newcommand{\MailAuteurA}{fabrice.boissier@epita.fr}
\newcommand{\NomAuteurB}{ }
\newcommand{\MailAuteurB}{ }
\newcommand{\DocKeywords}{Architecture ; Conversion Entiers ; Binaire ; Gray ; Code Gray ; BCD ; 8421 ; 2421}
\newcommand{\DocLangue}{fr} % "en", "fr", ...

\usepackage{MetalQuickLabs}

% Babel ne traduit pas toujours bien les tableaux et autres
\renewcommand*\frenchfigurename{%
    {\scshape Figure}%
}
\renewcommand*\frenchtablename{%
    {\scshape Tableau}%
}

% Ne pas afficher le numéro de la légende sur tableaux et figures
\captionsetup{format=sanslabel}


\begin{document}

\EncadreTitre

\bigskip


%\begin{center}
%\begin{tabular}{p{5cm} p{11cm}}
%\textbf{Commandes étudiées :} & \texttt{sh}, \texttt{bash}, \texttt{man}, \texttt{ls}, \texttt{mkdir}, \texttt{touch}, \texttt{chmod}, \texttt{mv}, \texttt{rm}, \texttt{rmdir}, \texttt{cat}, \texttt{file}, \texttt{which}, \texttt{which}\\
%
%\textbf{Builtins étudiées :} & \texttt{pwd}, \texttt{cd}, \texttt{exit}, \texttt{logout}, \texttt{echo}, \texttt{umask}, \texttt{type}, \texttt{>}, \texttt{>{}>}, \texttt{<}, \texttt{<{}<}, \texttt{|}\\
%
%\textbf{Notions étudiées :} & Shell, Manuels, Fichiers, Répertoires, Droits, Redirections\\
%\end{tabular}
%\end{center}

\bigskip


Ce document a pour objectif de vous familiariser avec les conversions en code Gray et en BCD.

\bigskip

Les conversions ne s'intéresseront qu'à des entiers non signés.

Pour rappel, le BCD est par défaut en mode 8421, mais les consignes de ce document vous demanderont aussi de réaliser du 2421.

\bigskip

%%%%%%%%%%%%%%%%%%%%%%%%%%%%%%%%%%%%%%%%%%%%%%%%

\section{Conversion vers le Code Gray \& le BCD}

%%%%%%%%%%%%%%%%%%%

\medskip

Convertissez ces nombres vers du code Gray, du BCD (8421), puis du BCD (2421).

\medskip

\begin{center}
\centerline{
\begin{tabular}{ | c | C{3.5cm} | C{3.5cm} | C{3.5cm} | }
\hline
Nombre & Code Gray & BCD (8421) & BCD (2421) \\
\hline
 \multirow{2}{*}{$ \Scale[1.25]{ 42 } $} & & & \\
 & & & \\
\hline
 \multirow{2}{*}{$ \Scale[1.25]{ 151 } $} & & & \\
 & & & \\
\hline
 \multirow{2}{*}{$ \Scale[1.25]{ 234 } $} & & & \\
 & & & \\
\hline
 \multirow{2}{*}{$ \Scale[1.25]{ 747 } $} & & & \\
 & & & \\
\hline
 \multirow{2}{*}{$ \Scale[1.25]{ 1337 } $} & & & \\
 & & & \\
\hline
 \multirow{2}{*}{$ \Scale[1.25]{ 2934 } $} & & & \\
 & & & \\
\hline
 \multirow{2}{*}{$ \Scale[1.25]{ 3421 } $} & & & \\
 & & & \\
\hline
\end{tabular}
}
\end{center}

%%%%%%%%%%%%%%%%%%%%%%%%%%%%%%%%%%%%%%%%%%%%%%%%

\section{Conversions depuis le Code Gray}

%%%%%%%%%%%%%%%%%%%

\medskip

Convertissez ces nombres depuis le code Gray vers du binaire classique puis vers la base 10.

\medskip

\begin{center}
\centerline{
\begin{tabular}{ | c | C{3.5cm} | C{3.5cm} | }
\hline
Code Gray & Binaire & Base 10 \\
\hline
 \multirow{2}{*}{$ \Scale[1.25]{ \% \, 0101 \, 1101 } $} & & \\
 & & \\
\hline
 \multirow{2}{*}{$ \Scale[1.25]{ \% \, 1001 \, 0111 } $} & & \\
 & & \\
\hline
 \multirow{2}{*}{$ \Scale[1.25]{ \% \, 1101 \, 1011 } $} & & \\
 & & \\
\hline
 \multirow{2}{*}{$ \Scale[1.25]{ \% \, 1010 \, 0101 } $} & & \\
 & & \\
\hline
 \multirow{2}{*}{$ \Scale[1.25]{ \% \, 1011 \, 1011 } $} & & \\
 & & \\
\hline
 \multirow{2}{*}{$ \Scale[1.25]{ \% \, 0111 \, 0111 } $} & & \\
 & & \\
\hline
 \multirow{2}{*}{$ \Scale[1.25]{ \% \, 1111 \, 1111 } $} & & \\
 & & \\
\hline
\end{tabular}
}
\end{center}

%%%%%%%%%%%%%%%%%%%%%%%%%%%%%%%%%%%%%%%%%%%%%%%%

\section{Conversions depuis les BCD}

%%%%%%%%%%%%%%%%%%%

\medskip

Convertissez ces nombres depuis les BCD vers la base 10.

\medskip

\begin{center}
\centerline{
\begin{tabular}{ | c | C{3.5cm} | C{3.5cm} | }
\hline
BCD & (8421) $ \rightarrow $ Base 10 & (2421) $ \rightarrow $ Base 10 \\
\hline
 \multirow{2}{*}{$ \Scale[1.25]{ \% \, 0101 \, 1101 } $} & & \\
 & & \\
\hline
 \multirow{2}{*}{$ \Scale[1.25]{ \% \, 1001 \, 0111 } $} & & \\
 & & \\
\hline
 \multirow{2}{*}{$ \Scale[1.25]{ \% \, 1101 \, 1011 } $} & & \\
 & & \\
\hline
 \multirow{2}{*}{$ \Scale[1.25]{ \% \, 1010 \, 0101 } $} & & \\
 & & \\
\hline
 \multirow{2}{*}{$ \Scale[1.25]{ \% \, 1011 \, 1011 } $} & & \\
 & & \\
\hline
 \multirow{2}{*}{$ \Scale[1.25]{ \% \, 0111 \, 0111 } $} & & \\
 & & \\
\hline
 \multirow{2}{*}{$ \Scale[1.25]{ \% \, 1111 \, 1111 } $} & & \\
 & & \\
\hline
\end{tabular}
}
\end{center}

%%%%%%%%%%%%%%%%%%%%%%%%%%%%%%%%%%%%%%%%%%%%%%%%%%%%%%%%%%%%%%%%%%%%%%%%%%%%%%%%%%%%%%%%%%%%%%%%

\bigskip

\vfillFirst

\vfillLast

\begin{center}
\textit{Ce document et ses illustrations ont été réalisés par Fabrice BOISSIER en octobre 2024}

%\textit{(dernière mise à jour octobre 2024)}
\end{center}

\end{document}
