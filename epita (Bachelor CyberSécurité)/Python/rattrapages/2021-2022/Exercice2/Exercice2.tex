%% Exercice 2

%\ExoSpecs{\TTBF{CalculTVA.sh}}{\TTBF{\RenduDir/src/exo1/}}{750}{640}{\TTBF{write}}
%\ExoSpecsCustom{\TTBF{my\_calc}}{\TTBF{\RenduDir/src/my\_calc.c}}{750}{640}{Commandes autorisées}{\TTBF{sh}, \TTBF{echo}, \TTBF{exit}}
\ExoSpecsSimple{\TTBF{my\_calc.py}}{\TTBF{\RenduDir/src/my\_calc.py}}{750}{640}

\vspace*{0.7cm}

\noindent \ExoObjectif{Le but de l'exercice est de créer une mini calculatrice en Python.}

\bigskip

\noindent Vous devez écrire un programme qui prendra trois paramètres (deux nombres, puis l'opérateur), et affichera le résultat de l'opération désignée.

\noindent Vous devez implémenter les 5 opérations suivantes : l'addition (symbole \TTBF{+}), la soustraction (symbole \TTBF{-}), la multiplication (lettre \TTBF{x}), la division (symbole \TTBF{/}), et le reste de la division euclidienne (symbole \TTBF{\%}).

\noindent \`A la fin du calcul, votre programme doit renvoyer 0.

\bigskip

\lstset{language=sh}
\begin{lstlisting}[frame=single,title={Cas général}]
$ python my_calc.py 1 3 +
4
$ echo $?
0
$ python my_calc.py 144 362 -
-218
$ echo $?
0
$ python my_calc.py 6 7 x
42
$ echo $?
0
$ python my_calc.py 12 3 /
4
$ echo $?
0
$ python my_calc.py 69 2 %
1
$ echo $?
0
\end{lstlisting}

\bigskip

\noindent Si des paramètres manquent, vous devez écrire le message suivant, et renvoyer 1.

\bigskip

\noindent \TTBF{Not\textvisiblespace enough\textvisiblespace parameters.}

\noindent \TTBF{Usage:\textvisiblespace python\textvisiblespace my\_calc.py\textvisiblespace number\textvisiblespace number\textvisiblespace operator}

\noindent \TTBF{Operator\textvisiblespace might\textvisiblespace be\textvisiblespace :\textvisiblespace +\textvisiblespace -\textvisiblespace x\textvisiblespace /\textvisiblespace \%}

\bigskip

\lstset{language=sh}
\begin{lstlisting}[frame=single,title={Cas d'erreur 1 : pas assez de paramètres}]
$ python my_calc.py
Not enough parameters.
Usage: python my_calc.py number number operator
Operator might be : + - x / %
$ echo $?
1
$ python my_calc.py 0 4
Not enough parameters.
Usage: python my_calc.py number number operator
Operator might be : + - x / %
$ echo $?
1
\end{lstlisting}

\bigskip

\noindent Si des paramètres sont en trop, vous devez écrire le message suivant, et renvoyer 2.

\bigskip

\noindent \TTBF{Too\textvisiblespace much\textvisiblespace parameters.}

\noindent \TTBF{Usage:\textvisiblespace python\textvisiblespace my\_calc.py\textvisiblespace number\textvisiblespace number\textvisiblespace operator}

\noindent \TTBF{Operator\textvisiblespace might\textvisiblespace be\textvisiblespace :\textvisiblespace +\textvisiblespace -\textvisiblespace x\textvisiblespace /\textvisiblespace \%}

\bigskip

\lstset{language=sh}
\begin{lstlisting}[frame=single,title={Cas d'erreur 2 : trop de paramètres}]
$ python my_calc.py 0 4 x 45
Too much parameters.
Usage: python my_calc.py number number operator
Operator might be : + - x / %
$ echo $?
2
\end{lstlisting}

\bigskip

\noindent Si l'opérateur n'est pas placé après les nombres/en dernier/à la troisième position, vous devez indiquer le message suivant, et renvoyer 3.

\bigskip

\noindent \TTBF{Wrong\textvisiblespace parameters.}

\noindent \TTBF{Usage:\textvisiblespace python\textvisiblespace my\_calc.py\textvisiblespace number\textvisiblespace number\textvisiblespace operator}

\noindent \TTBF{Operator\textvisiblespace might\textvisiblespace be\textvisiblespace :\textvisiblespace +\textvisiblespace -\textvisiblespace x\textvisiblespace /\textvisiblespace \%}

\bigskip

\lstset{language=sh}
\begin{lstlisting}[frame=single,title={Cas d'erreur 3 : pas les bons paramètres}]
$ python my_calc.py x x x
Wrong parameters.
Usage: python my_calc.py number number operator
Operator might be : + - x / %
$ echo $?
3
$ python my_calc.py 1 2 3
Wrong parameters.
Usage: python my_calc.py number number operator
Operator might be : + - x / %
$ echo $?
3
$ python my_calc.py 1 + 2
Wrong parameters.
Usage: python my_calc.py number number operator
Operator might be : + - x / %
$ echo $?
3
\end{lstlisting}

\bigskip

\noindent Si le deuxième paramètre donné à la division est 0, vous devez renvoyer 4 et écrire le message d'erreur suivant.

\bigskip

\noindent \TTBF{Division\textvisiblespace by\textvisiblespace 0\textvisiblespace is\textvisiblespace forbidden.}

\bigskip

\lstset{language=sh}
\begin{lstlisting}[frame=single,title={Cas d'erreur 4 : division par 0}]
$ python my_calc.py 1 0 /
Division by 0 is forbidden.
$ echo $?
4
\end{lstlisting}

\bigskip

\noindent Si plusieurs des problèmes précédents sont rencontrés simultanément, vous devez les gérer dans cet ordre de priorité : le manque de paramètres est affiché en priorité, l'excès de paramètres est affiché en seconde priorité, le mauvais ordre/mauvaise nature de paramètres en troisième, et la division par 0 en dernier.

\bigskip

\lstset{language=sh}
\begin{lstlisting}[frame=single,title={Cas d'erreurs : ordre des erreurs}]
$ python my_calc.py 1 0
Not enough parameters.
Usage: python my_calc.py number number operator
Operator might be : + - x / %
$ echo $?
1
$ python my_calc.py 1 0 0 /
Too much parameters.
Usage: python my_calc.py number number operator
Operator might be : + - x / %
$ echo $?
2
$ python my_calc.py 1 / 0
Wrong parameters.
Usage: python my_calc.py number number operator
Operator might be : + - x / %
$ echo $?
3
$ python my_calc.py 1 0 /
Division by 0 is forbidden.
$ echo $?
4
\end{lstlisting}

\bigskip

\noindent Les paramètres donnés lors des tests/de la correction seront toujours des nombres ou des opérateurs.
Vous n'avez pas à gérer les cas où des caractères sont donnés en paramètres.

\bigskip

\begin{RedBoxTitle}{ATTENTION}
    Il est formellement interdit d'utiliser le programme \TTBF{bc} ou tout autre programme ou bibliothèque de calcul.
%    (se référer à la section "commandes autorisées")
\end{RedBoxTitle}